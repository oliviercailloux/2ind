\section{More results}
\begin{theorem}
	\label{th:maxDisp}
	Given any profile $\prof$, $\exists x \in H(\prof) \cap \PE \suchthat d(\lprof(x)) ≤ \khalf$.
	That bound is tight, in other words, $\exists \prof \suchthat \forall x \in H(\prof): d(\lprof(x)) ≥ \khalf$.
\end{theorem}
\begin{proof}
	About the first claim, consider the $\khalf + 1$ alternatives having losses for the individual 1 $\lprof(x)_1 ≤ \khalf$.
	At least one of those alternatives will have a loss for the individual 2 within $\intvl{0, \khalf}$ (as the complementary interval of possible losses is $\intvl{\khalf + 1, m - 1}$ which has cardinality $\floor{\frac{m - 1}{2}} < \khalf + 1$). Such alternatives $\set{x \in \allalts \suchthat \lprof(x)_1 ≤ \khalf \land \lprof(x)_2 ≤ \khalf}$ have dispersion $d(\lprof(x)) ≤ \khalf$.
	The best of those for individual $2$, that is, $\argmin_{x \in \allalts \suchthat \lprof(x)_1 ≤ \khalf \land \lprof(x)_2 ≤ \khalf} \lprof(x)_2$, is Pareto-efficient.
	
	About the second claim, write $\allalts = \set{a_1, …, a_m}$, define $S_1$ as the sequence $(a_1, …, a_{\floor{\frac{m + 1}{2}}})$ and $S_2$ as the sequence $(a_{\floor{\frac{m + 1}{2}} + 1}, …, a_m)$, and let $\prof$ contain the preference ordering $(S_1, S_2)$ for individual $1$ and $(S_2, S_1)$ for individual $2$. 
	Alternatives in $S_2$ have dispersion $\floor{\frac{m + 1}{2}}$ (the length of $S_1$) and 
	alternatives in $S_1$ have dispersion $m - (\floor{\frac{m + 1}{2}} + 1) + 1 = (m + 1) - \floor{\frac{m + 1}{2}} - 1 = \ceil{\frac{m + 1}{2}} - 1 = \ceil{\frac{m - 1}{2}}$ (the length of $S_2$).
\end{proof}

\section{Further axiomatizations of the rules}
\subsection{A characterisation of FB based on weight comparison}
For any $w_m, w_i \in [0, 1]$ with $w_m + w_i = 1$, let $f^{w_m, w_i}$ denote the rule that selects the maximal elements according to the disutility of the loss vectors weighted by $w_m$ and $w_i$, as follows: $f^{w_m, w_i}(\prof) = \argmax_{x \in \allalts} -w_m(\min\lprof(x)) - w_i(\max \lprof(x) - \min \lprof(x))$. \commentMN{Why do we have $\max \lprof(x) - \min \lprof(x)$?? What I mean is that the previous expression calls for an explanation. I can add that but it maybe safer to discuss a bit before. Actually, the class of these rules is not immediate to me.}

\begin{remark}
	The currently, and temporarily, adopted notations for $w_m$ and $w_i$ use $m$ and $i$ to stand for \emph{minimal loss aversion} and \emph{inequality aversion}, and these letters in that context have nothing to do with the $m$ and $i$ previously defined for the number of candidates and a generic individual. I leave this is as-is for now and hope this introduces no confusion.\commentMN{What about $w_{ineq}$ rather than $w_i$ and $w_{min}$ for $w_m$?}
\end{remark}

\begin{definition}[Weight-comparing]
	$f$ is weight-comparing iff $\exists w_m, w_i \in [0, 1], w_m + w_i = 1 \suchthat f = f^{w_m, w_i}$.
\end{definition}

Let $\Delta^{x, y}_\mathit{ineq} = (\max \lprof(x) - \min \lprof(x)) - (\max \lprof(y) - \min \lprof(y))$ represent the difference between $x$ and $y$ in terms of inequality of losses. 
(The quantity $\Delta^{x, y}_\mathit{ineq}$ depends on the profile, which is omitted from its notation here.)
That number is (strictly) positive iff $\max \lprof(x) - \min \lprof(x)$ is (strictly) greater than $\max \lprof(y) - \min \lprof(y)$, that is, iff the inequality of $x$ is (strictly) greater than that of $y$, thus, intuitively, iff $f^{w_m, w_i}$ considers $x$ (strictly) worst than $y$ from the point of view of inequality.
Note that $\Delta^{x, y}_\mathit{ineq} = - \Delta^{y, x}_\mathit{ineq}$.

Let $\Delta^{x, y}_\mathit{min} = \min\lprof(x) - \min\lprof(y)$ represent the difference between $x$ and $y$ in terms of their smallest losses. 
That number is (strictly) positive iff $f^{w_m, w_i}$ considers $x$ (strictly) worst than $y$ from the point of view of smallest losses.
Note that $\Delta^{x, y}_\mathit{min} = - \Delta^{y, x}_\mathit{min}$.

\begin{theorem}
	\label{th:wcDelta}
	The rule $f^{w_m, w_i}$ selects the maximal elements of $\succeq$ defined as 
	$x \succeq y ⇔ \frac{w_i}{w_m} \Delta^{y, x}_\mathit{ineq} ≥ \Delta^{x, y}_\mathit{min}$.
\end{theorem}
\begin{proof}
	Define $M_x = \max \lprof(x)$, $m_x = \min \lprof(x)$, and similarly $M_y$ and $m_y$.
	
	By definition, the rule $f^{w_m, w_i}$ selects the maximal elements of $\succeq'$ defined as: 
	\begin{align}
		x \succeq' y & ⇔ -w_m m_x - w_i (M_x - m_x) ≥ -w_m m_y - w_i (M_y - m_y)\\
		& ⇔ w_i [(M_y - m_y) - (M_x - m_x)] ≥ w_m (m_x - m_y)\\
		& ⇔ \frac{w_i}{w_m} \Delta^{y, x}_\mathit{ineq} ≥ \Delta^{x, y}_\mathit{min}.
	\end{align}
	Thus, $\mathbin{\succeq'} = \mathbin{\succeq}$.
\end{proof}

\begin{remark}
	Those rules can also be characterized using only one parameter, as follows: $f^{w_m}(\prof) = f^{w_m, (1 - w_m)}(\prof)$; or as $f^{w_i}(\prof) = f^{(1 - w_i), w_i}(\prof)$.
	
	Such a rule is equally characterized by the tradeoff it is willing to tolerate between $\Delta_\mathit{min}$ and $\Delta_\mathit{ineq}$. E.g. for $w_m = 1/2$, the tradeoff is $\frac{w_m}{1 - w_m} = 1$: one $\Delta_\mathit{min}$ is indifferent to one $\Delta_\mathit{ineq}$.
\end{remark}

\commentMN{I added a proof that Borda is weight-comparing. Are other scoring rules representable via a weight-comparing rule? Should we use weight-comparing or other name?}
\commentOC{Thank you for the proof; it suggests that perhaps it will be good (also wrt your previous remark) to state this “change of basis” argument more generally, as I need it as well for the argument related to Pareto (\cref{th:paretoIneq}). I have attempted this below (\cref{rq:basis}).}
\commentOC{I think (without proof and far from being sure) that most other scoring rules are not w-c rules, as they do not treat losing ranks uniformly across the ranks. Tops is one of the exceptions, see below, perhaps the sole other exception; and this is specific to two individuals as only in this case does Tops equal Plurality.}
\commentOC{The LVC property (\cref{def:lvc}) plus anonymity, perhaps, capture the scoring rules feature (not sure at all); adding an “equal consideration for tradeoffs across ranks” condition (something like Shifting, \cref{def:shifting}) might yield w-c.}
\commentOC{I’m not good with names, I agree that weight comparing is not a good name.}

\begin{remark}[Alternative basis]
	\label{rq:basis}
	If $f$ is a weight-comparing rule with weights $(w_m, w_i)$, it is equally described as a rule that weights minimal and maximal losses with weights $(w'_m, w'_M)$ where $w'_m = w_m - w_i$ and $w'_M = w_i$.
	Indeed, such a rule selects the alternatives that minimize $-w_m (l_m) - w_i (l_M - l_m)$, which equals $-(w_m - w_i) (l_m) - w_i (l_M)$.
	
	However, the relationship and the bounds on those alternate weights for an equivalent class of rules are less nice. Consider the egalitarian rule, \cref{rq:egalitarianism}, for an illustration.
	We have that $w'_M = 1 - \frac{w'_m + 1}{2}$, and the bounds are $w'_m \in [-1, 1]$ and $w'_M \in [0, 1]$.
\end{remark}

\begin{remark}
	Borda satisfies Paretian and weight-comparing and fails SPEL. It uses  $w_m = 2/3$ and $w_i = 1/3$. Indeed, by definition,
	\[\textsc{Borda}(\prof) = \argmin_{\allalts} \sum_N \lprof = \set{x \in \allalts \suchthat \forall y \in \allalts: \sum_N \lprof(x) ≤ \sum_N \lprof(y)}.\]
	
	This makes sense since the loss of an alternative for a individual is equal to $m-$(ranking of the alternative in the preference profile), wheres the usual definition of the Borda score assigned by a individual is that this score equals the ranking of the alternative in the individual's profile. Hence, they are equivalent.
	
	It is Paretian by definition since an alternative $x$ ranked higher by both individuals than an alternative $y$ obtains a higher Borda score. It fails SPEL due to the same example as the one in theorem 4. To see why Borda rule is a weight-comparing rule, take some $f$ with weights $w_i = 1/3$ and $w_m = 2/3$ so that:
\begin{align}f(\prof) &= \argmax_{x \in \allalts} -(2/3)(\min\lprof(x)) - (1/3)(\max \lprof(x) - \min \lprof(x))\\
&\equiv \argmax_{x \in \allalts} -(1/3)(\min\lprof(x)+\max \lprof(x))\\
&\equiv \argmax_{x \in \allalts} -(\sum\lprof(x))\\
&\equiv \argmin_{x \in \allalts} (\sum\lprof(x))=\textsc{Borda}(\prof).\\
\end{align}
\end{remark}

\begin{remark}[Egalitarianism]
	\label{rq:egalitarianism}
	The egalitarian rule that minimizes inequality of losses (disregarding efficiency) satisfies SPEL and weight-comparing and fails Paretianism. It uses $w_i = 1$.
\end{remark}

\commentMN{When $w_m=1$, how is $f$? It selects the max of the minimum loss so quite uninteresting, right?}
\commentOC{Such a rule is extremely averse to min losses, and does not care about inequality. In other words, it minimizes the minimum loss. Thus, it is the rule “Top”, that selects the top alternatives.}

\begin{theorem}[Under w-c, SPEL equivalent to care for equality]
	\label{th:spelEquiv}
	Consider $f$ on variable number of candidates, weight-comparing. Then, $f$ is SPEL iff $\exists w_i ≥ w_m \in [0, 1], w_m + w_i = 1 \suchthat f = f^{w_m, w_i}$.
\end{theorem}
\begin{proof}
	We will need the following fact.
	Given $m \in \N$, $m ≥ 3$ odd, with $\allalts = \set{x, y, a_1, …, a_{m - 2}}$: 
	\begin{equation}
		\label{eq:bigIneq}
		\exists \prof \suchthat x, y \in \PE \land \lprof[\prof](x) = \left(\frac{m - 1}{2}, \frac{m - 1}{2}\right) \land \lprof[\prof](y) = \left(0, \frac{m + 1}{2}\right). 
	\end{equation}
	Letting $S_1$ denote the sequence of $\frac{m - 1}{2}$ alternatives $(y, a_1, …, a_{\frac{m - 1}{2} - 1})$ and $S_2$, the sequence of $\frac{m - 1}{2}$ alternatives $(a_{\frac{m - 1}{2}}, …, a_{m - 2})$, such a profile can be defined as the rankings $(S_1, x, S_2)$ and $(S_2, x, S_1)$. Here is a visual description of it.
	\begin{equation}
		\begin{array}{lllll lllll ll}
			y&a_1&…&a_{\frac{m - 1}{2} - 1}&x&a_{\frac{m - 1}{2}}&a_{\frac{m - 1}{2} + 1}&…&a_{m - 2}\\
			a_{\frac{m - 1}{2}}&a_{\frac{m - 1}{2} + 1}&…&a_{m - 2}&x&y&a_1&…&a_{\frac{m - 1}{2} - 1}\\
		\end{array}.
	\end{equation}
	
	Back to our main claim, given $m \in \N$ and $k \in \intvl{0, m - 1}$, let $\mathcal{P}^{x, k}$ denote the set of profiles $\prof$ where an alternative $x \in \PE$ has losses $(k, k)$. By definition, a rule $f$ is SPEL iff
	$\forall m \in \N, k \in \intvl{0, m - 1}, \prof \in \mathcal{P}^{x, k}$: 
	\begin{equation}
		\label{eq:condNotInF}
		\forall y ≠ x \in \PE: y \notin f(\prof).
	\end{equation}
	When $f$ is weight-comparing, \eqref{eq:condNotInF} is equivalent to:
	\begin{equation}
		\forall y  \in \PE\setminus \{x\}: [-w_m \min \lprof(y) - w_i (\max \lprof(y) - \min \lprof(y)) < -w_m k],
	\end{equation}
	itself equivalent to
	\begin{equation}
		\label{eq:wNotInF}
		\forall y \in \PE\setminus \{x\}: [w_m (k - \min \lprof(y)) < w_i (\max \lprof(y) - \min \lprof(y))].
	\end{equation}
	To conclude, we show that $w_m ≤ w_i$ implies \eqref{eq:wNotInF} and that \eqref{eq:wNotInF} implies $w_m ≤ w_i$.
	
	Observe first that $\forall y,x\in \PE$ with $y ≠ x$,  $k < \max \lprof(y)$ (otherwise, $y$ is Pareto-dominated by $x$); equivalently, $k - \min \lprof(y) < \max \lprof(y) - \min \lprof(y)$. Thus, $w_m ≤ w_i$ implies \eqref{eq:wNotInF}.
	
	Second, considering any $\prof$ satisfying \eqref{eq:bigIneq} and $k = \frac{m - 1}{2}$. The inequality \eqref{eq:wNotInF} implies that any alternative $y$ that satisfies \eqref{eq:bigIneq},  $w_m \left(\frac{m - 1}{2} - 0\right) < w_i \left(\frac{m + 1}{2} - 0\right)$, equivalently, $w_m < w_i \frac{m + 1}{m - 1}$, and because this must be true for all $m$, this implies $w_m ≤ w_i$.
\end{proof}

\begin{theorem}[Paretianism limits inequality consideration]
	\label{th:paretoIneq}
	Consider $f$ on variable number of candidates, weight-comparing. Then, $f$ is Paretian iff $\exists w_m ≥ w_i \in [0, 1], w_m + w_i = 1 \suchthat f = f^{w_m, w_i}$.
\end{theorem}
\begin{proof}
	For the forward direction, consider the losses $(100, 100)$ versus $(0, 99)$, thus in basis (min, ineq), $(100, 0)$ versus $(0, 99)$. The first one will be picked if $-100 w_m > -99 w_i$, which will happen if $w_i > 100/99 w_m$. (The argument can and should be completed by defining a profile where only $a$ and $b$ with losses $(0, 99)$ and $(99, 0)$ are pareto-dominant: on a smaller scale, this is $a, c, d, e, b, x, f, g, h$ and $b, f, g, h, a, x, c, d, e$.)
	
	The backwards direction follows from \cref{rq:basis}: from $w_m ≥ w_i$ follows that $w'_m ≥ 0$, and as $w'_M ≥ 0$ anyway, $f$ will respect dominance of losses (formally, $\lprof(y) > \lprof(x) ⇒ y \notin f(\prof)$).
\end{proof}

\begin{theorem}[FB as weighter]
	\label{th:fbW}
	FB = $f^{\frac{1}{2}, \frac{1}{2}}$.
\end{theorem}
\begin{proof}
	FB selects the alternatives that minimize $\max \lprof$ while $f^{\frac{1}{2}, \frac{1}{2}}$ selects those that maximize $-\frac{1}{2} \min \lprof - \frac{1}{2} (\max \lprof - \min \lprof)$, thus, that maximize $- \max \lprof$.
\end{proof}

\begin{theorem}[FB caract]
	The rule $f$ defined on variable number of candidates is Paretian, SPEL and weight-comparing iff it is FB.
\end{theorem}
\begin{proof}
	We show that the rule $f$ defined on variable number of candidates is Paretian, SPEL and weight-comparing iff it is $f^{\frac{1}{2}, \frac{1}{2}}$, as \cref{th:fbW} then brings the result.
	
	Considering a rule $f$ defined on variable number of candidates and weight-comparing, \cref{th:spelEquiv,th:paretoIneq} apply.
	From \cref{th:spelEquiv}, and assuming $f$ is also SPEL, $\exists w_i ≥ w_m \in [0, 1], w_m + w_i = 1 \suchthat f = f^{w_m, w_i}$.
	From \cref{th:paretoIneq}, and assuming $f$ is also Paretian, $\exists w_m ≥ w_i \in [0, 1], w_m + w_i = 1 \suchthat f = f^{w_m, w_i}$.
	It follows that $f = f^{\frac{1}{2}, \frac{1}{2}}$.
	
	\Cref{th:spelEquiv,th:paretoIneq} also show that the converse holds.
\end{proof}

\begin{remark}
	The rule $f$ with weights $w_{max} = \frac{m - 1}{m}$ and $w_i = \frac{1}{m}$ is, I believe, the subcorrespondence of FB that lexicographically first minimizes the worst loss, then minimizes the best loss. It is weight-comparing when defined on a fixed $m$, but not when defined on variable number of candidates. I think it would be interesting to define a weaker version of weight-comparing to allow for this.\commentMN{I agree} Perhaps some difficulty will stem from the fact that, from LVC, Pareto and a principle of indifference, FB will perhaps prefer the losses (4, 4) to (2, 5) because (4, 4) indifferent to (1, 4), itself Pareto dominating (2, 5).
\end{remark}

\subsection{A characterisation via marginal tradeoffs}
We will need the concept of a tradeoff function, intuitively, a function of the form $t(a, b)$, where $a$ is the ineq level and $b$ is the delta min, which indicates the delta ineq that compensates that delta min.
\begin{definition}[tradeoff function]
	A tradeoff function is a function of the form $t: \N^2 → \N \cup (\N + \frac{1}{2})$, with $t(a, 0) = 0$.
\end{definition}
Tradeoff functions are used to determine tradeoff preferences. To define this concept we first need to generalize the notion of Pareto dominance.
\begin{definition}[min-ineq Pareto relation]
	The min-ineq Pareto relation is a binary relation $R \subseteq \N^2 × \N^2$, defined on the loss tuples using minimal loss and inequality as a basis (from now on called the min-ineq losses), as follows. The min-ineq loss $(m, i)$ wealky Pareto-dominates the min-ineq loss $(m', i')$ iff it has a not worst min loss (that is, $m ≤ m'$) and a not worst ineq (that is, $i ≤ i'$). In other words, the min-ineq Pareto relation coincides with the $≤$ relation over ordered pairs of natural numbers.
\end{definition}
\begin{definition}[tradeoff preference]
	A tradeoff function $t$ determines a corresponding tradeoff preference, $\mathbin{\succeq^t} \subseteq \N^2 × \N^2$, defined on the min-ineq losses, as follows.
	Intuitively, given $d_m \in \N$ (a delta min), the function $t$ indicates that $(m, i)$ is indifferent to $(m, i) + (d_m, t(i, d_m))$.
	Considering in supplement the min-ineq Pareto relation, we obtain our formal definition: $(m, i) \succeq^t (m, i) + (d_m, d_i)$ iff $d_i ≥ t(i, d_m)$.
\end{definition}

\begin{conjecture}[Requirement on $t$]
	The preference $\succeq^t$ corresponding to a given tradeoff function $t$ is acyclic and complete iff $t$ satisfies: $t(i, d_m) < 0 ⇔ d_m > 0$.
\end{conjecture}
\begin{proof}[Intuition for the proof]
	The requirement is a generalization of the requirement that $(1, 3) \succeq (2, 3)$, very intuitively speaking.
	Consider $d_m > 0$.
	$(1, 3) \succeq (1, 3) + (d_m, 0)$ iff $t(i, d_m) < 0 ⇔ d_m > 0$.
\end{proof}

\begin{definition}[tradeoff rule]
	A tradeoff rule $f^t$ is a rule parameterized by a tradeoff function $t$, defined as $f^t(\prof)$ selecting maximal elements according to the relation $\succeq^t$ determined by the tradeoff function. Formally, given $l$ a loss tuple (in the usual basis), define $i(l) = \max l - \min l$ as the inequality of that loss tuple, and given $x, x' \in \allalts$, say that $f^t(\prof)$ wealy prefers $x$ to $x'$ iff $(\min \lprof(x), i(\lprof(x))) \succeq^t (\min \lprof(x), i(\lprof(x)))$, and define $f^t(\prof)$ as the maximal elements of that preference relation.
\end{definition}

These definitions thus lead naturally to the following conjecture.
\begin{conjecture}
	$f^t$ satisfies the natural [min, ineq] Pareto relation (lower is better) iff $t$ satisfies $t(i, d_m) < 0 ⇔ d_m > 0$.
\end{conjecture}
%For this reason, for now on we consider $t$ satisfying $t(i, d_m) < 0 ⇔ d_m > 0$.

\begin{definition}[Shifting]
	\label{def:shifting}
	From $\prof$, define $\prof'$ by adding a new alternative $z$ on top of $i$’s ranking and a new alternative $w$ at bottom of $i$’s ranking, and adding $w$ on top of $\ibar$’s ranking and $z$ at bottom of $\ibar$’s ranking, thereby shifting every previous alternatives one rank down. Shifting requires that $\forall \prof: f(\prof) = f(\prof')$ (unless $z$ or $w$ is taken -- details should be worked up).
\end{definition}

\begin{definition}[Shift up]
	\label{def:shiftUp}
	Consider $\prof$ and let $\set{w, z}$ designate the bottom alternatives (first individual $w$, second one $z$, with possibly $w = z$).
	Define $\prof'$ as $\prof$ where all alternatives are shifted one rank downwards, except that for the first individual, $w$ goes first, and for the last individual, $z$ goes first.
	Then, $[f(\prof') \cap \set{w, z} = \emptyset] ⇒ f(\prof') = f(\prof)$.
\end{definition}

\begin{definition}[Shift down]
	\label{def:shiftDown}
	Consider $\prof$ where the top alternatives $\set{w, z}$ (first individual $w$, second one $z$, with possibly $w = z$) do not win ($f(\prof) \cap \set{w, z} = \emptyset$). 
	Define $\prof'$ as $\prof$ where all alternatives are shifted one rank upwards, except that for the first individual, $w$ goes last, and for the last individual, $z$ goes last.
	Then, $f(\prof') = f(\prof)$.
\end{definition}

\begin{conjecture}
	Shift up implies Shift down.
\end{conjecture}

\begin{definition}[Finite shifting]
	\label{def:finiteShifting}
	$f$ satisfies finite shifting iff it satisfies Shift up (hence, iff it satisfies Shift down and Shift up).
\end{definition}

\begin{definition}[Half shift]
	Consider $\prof$ and let $w$ designate the bottom alternative of individual $i$.
	Define $\prof'$ as $\prof$ except that all alternatives of individual $i$ are shifted one rank downwards, except that $w$ goes first.
	Then, $[f(\prof') \cap \set{w} = \emptyset] ⇒ f(\prof') = f(\prof)$.
\end{definition}

\begin{definition}[Loss vector comparing (LVC)]
	\label{def:lvc}
        Let $\succeq$ be a weak order on $\lvs$. The rule $f^\succeq$ selects all alternatives whose loss vectors are minimal elements of $\restr{{\succeq}}{\lprof(\alts)}$. A rule satisfies LVC iff $\exists {\succeq} \suchthat f = f^\succeq$. (This property is called preorder based in Cailloux and Endriss 2014.)
\end{definition}

\begin{conjecture}
	LVC and shifting implies $\exists t \suchthat f = f^t$.
\end{conjecture}
\begin{proof}[Intuition for the proof]
	Assume that $x$ with losses $(2, 5)$ is preferred (according to $\succeq$) to $y$ with losses $(1, 6)$.
	Thus, in min-ineq basis: $(2, 3) \succeq (1, 5)$.
	We have to prove that then, $(1, 3) \succeq (0, 5)$ [m-i].
	Thus, that $(1, 4) \succeq (0, 5)$ [losses].
	This is clear from shifting.
\end{proof}

\begin{conjecture}
	LVC and finite shifting and Half shift equivalent to w-c.
\end{conjecture}

\begin{conjecture}
	w-c implies finite shifting.
\end{conjecture}
\begin{proof}
	Seems to follow from \cref{th:wcDelta}.
\end{proof}

\begin{conjecture}
	LVC and finite shifting and min-ineq Pareto implies $\exists t \suchthat f = f^t$.
\end{conjecture}
\begin{proof}[Rough sketch of a proof]
	Assume that $x$ with losses $(2, 2)$ is preferred (according to $\succeq$) to $y$ with losses $(1, 3)$.
	Thus, in min-ineq basis: $(2, 0) \succeq (1, 2)$.
	We have to prove that then, $(1, 0) \succeq (0, 2)$ [m-i].
	Thus, that $(1, 1) \succeq (0, 2)$ [losses].
	This is clear from shifting.
	
	More generally.
	Define $x = (2, 0)$ [m-i], and let $y = (1, b)$ be indifferent to it. Then define $t_2$ accordingly. Proceed similarly for other vectors $(2, a)$ indifferent to $(1, b)$, which defines $t_2$ completely. Proceed similarly for $t_m$, for other values of min $m$. Then prove that $t_m$ are equal for all $m$.
\end{proof}

\begin{conjecture}
	Consider $f^t$, SPEL. Then, $t(0, d_m) = inf$. BUT NOT $t(i, d_m) = d_i ≤ d_m$. (Intuitively, inequality matters more than min.)
\end{conjecture}
\begin{conjecture}
	Consider $f^t$, Paretian. Then, $t(i, d_m) = d_i ≥ d_m$. (Intuitively, inequality matters less than min.)
\end{conjecture}

\begin{conjecture}
	LVC and shifting do not imply w-c.
\end{conjecture}
\begin{proof}[Idea of a proof]
%	Assume that $x$ with losses $(2, 5)$ is preferred (according to $\succeq$) to $y$ with losses $(1, 6)$.
%	Then, any pair with $\Delta^{y, x}_\mathit{ineq} = 2$ and $\Delta^{x, y}_\mathit{min} = 1$ will have $x \succeq y$.
%	FURTHERMORE, any pair with $\Delta^{y, x}_\mathit{ineq} = 4$ and $\Delta^{x, y}_\mathit{min} = 2$ will have $x \succeq y$?
%	In fact, any trade of 2 $\Delta^{y, x}_\mathit{ineq}$ for 1 $\Delta^{x, y}_\mathit{min}$ will be accepted.
%	
%	Consider $x$ with losses $(1, 5)$ and $y$ with losses $(0, 6)$. Ineq $x$ is $4$ and ineq $y$ is 6.
%	to prove: $x$ is preferred to $y$.
	
	We could have $(1, 4) > (0, 5) > (0, 6) > (1, 5)$.
\end{proof}

\subsection{Axiomatization of rules other than FB}
$\PVe$ is equivalent to MR

VR is "Borda constrained to MR". Olivier claims that $\VR$ is the biggest rule satisfying both MR and the condition: $x \in f(\prof) ⇒ \sum(\lprof(x)) ≤ \min_{y \in H(\prof)} \sum(\lprof(y))$.

Short listing should be something around “maximax within the first half.” Again a claim of Olivier: Another condition requires that if $\exists y \in H(\prof) \suchthat \min(\lprof(y)) < \min(\lprof(x))$, then $x$ loses. Equivalently, $x \in f(\prof) ⇒ \min(\lprof(x)) ≤ \min_{y \in H(\prof)} \min(\lprof(y))$. The rule $\SL$ is the biggest rule satisfying both MR and that condition. (Note that these two conditions are independent.)

\commentOC{Here is the description of $\SL$, which can be suitably adapted to other rules.} The $\SL$ rule is the rule that selects any alternative that is MR and that is not dominated by an MR alternative in terms of minimal welfare loses. More precisely, define $\prec$ as the following partial order on the set of loss vectors: $(a, b) \prec (c, d) ⇔ [(a, b) ≤^\text{MR} (c, d) \land \min\set{a, b} < \min\set{c, d}] \lor [(a, b) <^\text{MR} (c, d)]$, where $(a, b) ≤^\text{MR} (c, d)$ iff $(a, b)$ is first half. Then, $\SL$ selects $x$ iff it is an undominated element among the loss vectors that exist in $\prof$; formally, $f(\prof) = \set{x \in \allalts \suchthat \nexists y \in \allalts \suchthat y \prec x}$. 
Defining the uncomparability relation $\sim$ as $l_1 \sim l_2$ iff $l_1 \sim^\text{MR} l_2 \land \min l_1 = \min l_2$, we obtain the weak order ${\preceq} = {\prec} \cup {\sim}$, and $f(\prof) = \argmin^{\preceq}_{x \in \allalts} \lprof(x)$.
\commentOC{We could perhaps also describe $\prec$ as a union of relations, one of which is common to all rules.}

\commentRS{Finally can we characterize WMD as a rule?}
