\RequirePackage[l2tabu]{nag}
\documentclass[pagesize, twoside=off, bibliography=totoc, DIV=calc, fontsize=12pt, a4paper]{scrartcl}
\input{preamble/packages}
\input{preamble/redac}
\input{preamble/math_basics}
%Decision Theory
\NewDocumentCommand{\allalts}{}{\mathcal{A}}
\NewDocumentCommand{\allcrits}{}{\mathscr{C}}
\NewDocumentCommand{\alts}{}{A}
\NewDocumentCommand{\dm}{}{i}
\NewDocumentCommand{\allF}{}{\mathscr{F}}
\NewDocumentCommand{\allvoters}{}{\mathscr{N}}
\NewDocumentCommand{\voters}{}{N}
\NewDocumentCommand{\allprofs}{}{\linors^N}
\NewDocumentCommand{\prof}{}{\bm{P}}
\NewDocumentCommand{\lprof}{}{\lambda_{\bm{P}}}
\NewDocumentCommand{\linors}{}{\mathscr{L}(\allalts)}
%Thanks to https://tex.stackexchange.com/q/154549
	%\makeatletter
	%\def\@myRgood@#1#2{\mathrel{R^X_{#2}}}
	%\def\myRgood{\@ifnextchar_{\@myRgood@}{\mathrel{R^X}}}
	%\makeatother
\NewDocumentCommand{\pref}{}{\succ}
\NewDocumentCommand{\prefi}{O{i}}{\succ_{#1}}
\NewDocumentCommand{\prefeqi}{O{i}}{\succeq_{#1}}
\NewDocumentCommand{\prefone}{}{\succ_1}
\NewDocumentCommand{\preftwo}{}{\succ_2}
\NewDocumentCommand{\prefeqone}{}{\succeq_1}
\NewDocumentCommand{\prefeqtwo}{}{\succeq_2}
\NewDocumentCommand{\prefiinv}{O{i}}{\succ_{#1}^{-1}}
\NewDocumentCommand{\ibar}{}{\overline{i}}

\NewDocumentCommand{\lvs}{}{\intvl{0, m - 1}^N}
\NewDocumentCommand{\losses}{}{\intvl{0, m - 1}}
\NewDocumentCommand{\PD}{}{\mathit{PD}(\prof)}
\NewDocumentCommand{\PE}{}{\mathit{PE}(\prof)}

%Rules
\NewDocumentCommand{\rhoP}{}{\rho_{\prof}}
\NewDocumentCommand{\minspread}{O{A}}{\min_{#1}(\sigma \circ \lambda_{\bm{P}})}
\NewDocumentCommand{\mindisp}{O{A}}{\min_{#1}(d \circ \lambda_{\bm{P}})}
\NewDocumentCommand{\FB}{}{\mathit{FB}}
\NewDocumentCommand{\VR}{}{\mathit{VR}}
\NewDocumentCommand{\SL}{}{\mathit{SL}}
\NewDocumentCommand{\PVv}{O{v}}{\mathit{PV}^{#1}}
\NewDocumentCommand{\PVef}{}{\mathit{PV}^{\floor{\frac{m - 1}{2}}}}%f for first
\NewDocumentCommand{\PVes}{}{\mathit{PV}^{\ceil{\frac{m - 1}{2}}}}%s for second
\NewDocumentCommand{\PVe}{}{\mathit{PV^=}}%egalitarian distribution

%Classes
\NewDocumentCommand{\PVcl}{}{\mathcal{PV}}
\NewDocumentCommand{\PVbcl}{}{\mathcal{PV}^b}
\NewDocumentCommand{\PVecl}{}{\mathcal{PV}^=}%egalitarian distribution
\NewDocumentCommand{\PEcl}{}{\mathcal{PE}}
\NewDocumentCommand{\FHcl}{}{\mathcal{FH}}
\NewDocumentCommand{\VCcl}{}{\mathcal{VC}}
\NewDocumentCommand{\VCecl}{}{\mathcal{VC^=}}
\NewDocumentCommand{\ELcl}{}{\mathcal{EL}}
\NewDocumentCommand{\WELcl}{}{\mathcal{WEL}}
\NewDocumentCommand{\PELcl}{}{\mathcal{PEL}}
\NewDocumentCommand{\WPELcl}{}{\mathcal{WPEL}}


\usepackage{bbm}
\usepackage{makecell}

\renewcommand\theadalign{bc}
\renewcommand\theadfont{\bfseries}
\renewcommand\theadgape{\Gape[4pt]}
\renewcommand\cellgape{\Gape[4pt]}

%I find these settings useful in draft mode. Should be removed for final versions.
	%Which line breaks are chosen: accept worse lines, therefore reducing risk of overfull lines. Default = 200.
	%\tolerance=2000
	%Accept overfull hbox up to...
	\hfuzz=2cm
	%Reduces verbosity about the bad line breaks.
	\hbadness 5000
	%Reduces verbosity about the underful vboxes.
	\vbadness=1300

\linenumbers

\title{Answer to the reviewers}
\author[*]{Olivier Cailloux}
\author[**]{Matías Núñez}
\author[*]{M. Remzi Sanver}
\affil[*]{Université Paris-Dauphine, PSL Research University, CNRS, LAMSADE, 75016 Paris, France.}
\affil[**]{CREST, CNRS, École Polytechnique, GENES, ENSAE Paris, Institut Polytechnique de Paris, 91120 Palaiseau, France.}
\date{}

\hypersetup{
	pdfsubject={},
	pdfkeywords={},
}
\usepackage{multirow}
\begin{document}
\maketitle

\section{Introduction}
We wish to thank the editor and both reviewers for their helpful comments. We have revised the paper accordingly. Here are our answers to the reviewers’ comments.

\section{R1}
\begin{enumerate}
  \item We have reorganized the introduction thoroughfully.
  \item Our article aims at filling a gap in the literature as much analysis is devoted to the case of three agents or more and the case of two agents is often neglected. It seems to us that returning to the case of more than two agents would hurt the clarity of the paper.
  \item 

Borda winners (R1 remark 6): the definition is just there already.

\commentOC{R1 remark 9: we chose to keep Def 1 because of the centrality of this notion and because we have now added a remark following R2’s suggestion that refers to this notion. Def 4: similar IMHO. “all the discussion about the equal loss property could be placed in a footnote.”: I ignore which discussion this refers to.} 

\commentOC{R1 comment c: I find this suggestion completely silly. We sometimes use natural language to describe some statements, and sometimes, mathematics, wherever we see fit. For example, from times to times, we use the phrase “in the set”, or “such that”, or “among the worst”. It is not adequate to choose to be only purely formal all the time for some symbols or purely informal.} \commentRS{I agree but let's say this nicely!}\commentMN{This is a very standard referee request. But it's true that he is too tight. We should answer nicely since reports were particularly nice to us!}

\section{R2}
a: I have specified that $\N$ includes zero
\commentOC{R2 c: I believe that she is wrong. We could add this to the text, but is it more helpful than confusing? “Note that this is equivalent because $H(P)$ picks the winners among the first $\khalf + 1 = \khalfAlt + 1$ ranks, whose complement to $m$ is $\floor{\frac{m - 1}{2}}$.” This applies similarly to R2 d.} \commentRS{VR and SL are defined for an odd number of alternatives, no?} \commentMN{Not sure here, it might be worthy to discuss. When looking at our paper, it looks to be defined for any number of alternatives.}

\bibliography{bibliototal}
\end{document}

