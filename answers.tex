\RequirePackage[l2tabu]{nag}
\documentclass[pagesize, twoside=off, bibliography=totoc, DIV=calc, fontsize=12pt, a4paper]{scrartcl}
\input{preamble/packages}
\input{preamble/redac}
\input{preamble/math_basics}
%Decision Theory
\NewDocumentCommand{\allalts}{}{\mathcal{A}}
\NewDocumentCommand{\allcrits}{}{\mathscr{C}}
\NewDocumentCommand{\alts}{}{A}
\NewDocumentCommand{\dm}{}{i}
\NewDocumentCommand{\allF}{}{\mathscr{F}}
\NewDocumentCommand{\allvoters}{}{\mathscr{N}}
\NewDocumentCommand{\voters}{}{N}
\NewDocumentCommand{\allprofs}{}{\linors^N}
\NewDocumentCommand{\prof}{}{\bm{P}}
\NewDocumentCommand{\lprof}{}{\lambda_{\bm{P}}}
\NewDocumentCommand{\linors}{}{\mathscr{L}(\allalts)}
%Thanks to https://tex.stackexchange.com/q/154549
	%\makeatletter
	%\def\@myRgood@#1#2{\mathrel{R^X_{#2}}}
	%\def\myRgood{\@ifnextchar_{\@myRgood@}{\mathrel{R^X}}}
	%\makeatother
\NewDocumentCommand{\pref}{}{\succ}
\NewDocumentCommand{\prefi}{O{i}}{\succ_{#1}}
\NewDocumentCommand{\prefeqi}{O{i}}{\succeq_{#1}}
\NewDocumentCommand{\prefone}{}{\succ_1}
\NewDocumentCommand{\preftwo}{}{\succ_2}
\NewDocumentCommand{\prefeqone}{}{\succeq_1}
\NewDocumentCommand{\prefeqtwo}{}{\succeq_2}
\NewDocumentCommand{\prefiinv}{O{i}}{\succ_{#1}^{-1}}
\NewDocumentCommand{\ibar}{}{\overline{i}}

\NewDocumentCommand{\lvs}{}{\intvl{0, m - 1}^N}
\NewDocumentCommand{\losses}{}{\intvl{0, m - 1}}
\NewDocumentCommand{\PD}{}{\mathit{PD}(\prof)}
\NewDocumentCommand{\PE}{}{\mathit{PE}(\prof)}

%Rules
\NewDocumentCommand{\rhoP}{}{\rho_{\prof}}
\NewDocumentCommand{\minspread}{O{A}}{\min_{#1}(\sigma \circ \lambda_{\bm{P}})}
\NewDocumentCommand{\mindisp}{O{A}}{\min_{#1}(d \circ \lambda_{\bm{P}})}
\NewDocumentCommand{\FB}{}{\mathit{FB}}
\NewDocumentCommand{\VR}{}{\mathit{VR}}
\NewDocumentCommand{\SL}{}{\mathit{SL}}
\NewDocumentCommand{\PVv}{O{v}}{\mathit{PV}^{#1}}
\NewDocumentCommand{\PVef}{}{\mathit{PV}^{\floor{\frac{m - 1}{2}}}}%f for first
\NewDocumentCommand{\PVes}{}{\mathit{PV}^{\ceil{\frac{m - 1}{2}}}}%s for second
\NewDocumentCommand{\PVe}{}{\mathit{PV^=}}%egalitarian distribution

%Classes
\NewDocumentCommand{\PVcl}{}{\mathcal{PV}}
\NewDocumentCommand{\PVbcl}{}{\mathcal{PV}^b}
\NewDocumentCommand{\PVecl}{}{\mathcal{PV}^=}%egalitarian distribution
\NewDocumentCommand{\PEcl}{}{\mathcal{PE}}
\NewDocumentCommand{\FHcl}{}{\mathcal{FH}}
\NewDocumentCommand{\VCcl}{}{\mathcal{VC}}
\NewDocumentCommand{\VCecl}{}{\mathcal{VC^=}}
\NewDocumentCommand{\ELcl}{}{\mathcal{EL}}
\NewDocumentCommand{\WELcl}{}{\mathcal{WEL}}
\NewDocumentCommand{\PELcl}{}{\mathcal{PEL}}
\NewDocumentCommand{\WPELcl}{}{\mathcal{WPEL}}



%I find these settings useful in draft mode. Should be removed for final versions.
	%Which line breaks are chosen: accept worse lines, therefore reducing risk of overfull lines. Default = 200.
	%\tolerance=2000
	%Accept overfull hbox up to...
	\hfuzz=2cm
	%Reduces verbosity about the bad line breaks.
	\hbadness 5000
	%Reduces verbosity about the underful vboxes.
	\vbadness=1300

\title{Answer to the reviewers}
\author[*]{Olivier Cailloux}
\author[**]{Matías Núñez}
\author[*]{M. Remzi Sanver}
\affil[*]{Université Paris-Dauphine, PSL Research University, CNRS, LAMSADE, 75016 Paris, France.}
\affil[**]{CREST, CNRS, École Polytechnique, GENES, ENSAE Paris, Institut Polytechnique de Paris, 91120 Palaiseau, France.}
\date{}

\hypersetup{
	pdfsubject={},
	pdfkeywords={},
}
\begin{document}
\maketitle

\section{Introduction}
We wish to thank the associate editor and the reviewers for their helpful comments. We have revised the paper accordingly and feel that the paper is now more convincing. In the next pages, we detail our answers to the reviewers’ comments and
how we incorporated their remarks in the paper.


\section{Associate Editor}

The associate editor suggests to try to deal with all comments,
suggestions and questions contained in the two reports.


(S)he also writes: In particular, the authors should try to
make some progress in the analysis that Referee 2 proposes at the end of the third paragraph of
his/her report as well as to answer the two questions raised at the end of the report.


While providing an axiomatization of some rules in the current framework is of clear relevance, we consider that this should be the object of future work. Indeed, one of the main points of this work is to highlight that very few papers have considered ordinal social choice between two players from an axiomatic perspective, and it provides the incompatibility between MR and MD as a potential guideline of what cannot be achieved (in line with the usual impossibility results in Social choice). Moreover, it still needs to be determined which rule should be axiomatized. Our results set a divisive line between some existing rules, and this should be considered as the first step towards a better understanding of these rules.




(S)he also makes a list of typos that we have corrected in the paper. Thank you for the suggestions.  






\section{Referee 1}
We thank the reviewer for the positive evaluation and the constructive suggestions.

Here are our replies to the reviewer’s main comments.
\begin{enumerate}[label=({\arabic*})]
  \item We have reorganized the introduction thoroughly.
  \item Our article aims at filling a gap in the literature as much analysis is devoted to the case of three agents or more and the case of two agents is often neglected. It seems to us that returning to the case of more than two agents would hurt the clarity of the paper.
  \item We now present the SCR as a correspondence instead of as a function.
  \item The reviewer is right, the notation $\intvlz{p}$ is simpler than $\intvl{0, p}$ that we used unnecessarily. We have changed the text accordingly.
  \item The notation $\min \max \lprof$ is now in the preliminaries, as suggested. We have also made explicit that the inverse ordering ${\prefiinv}$ is associated to the other individual preferences ${\prefi[\ibar]}$ in Proposition 1.
  \item The definition of Borda winners is given in passing as part of the definition of $\VR$.
  \item We now indicate systematically the preference of each agent in our profiles.
  \item We have completed the sentence.
  \item We chose to keep Definition 1 because of the centrality of this notion and because we have now added a remark following R2’s suggestion that refers to this notion. Moreover, the referee suggested that "Moreover, all the discussion
about the equal loss property could be placed in a footnote". We are not sure which discussion the referee mentions. The notations and definitions dealing with loss are central in the paper so we decided to let them in the main text. 
  \item We have completed the tables to avoid blank entries.
  \item Solved.
  \item Solved.
  \item We have made this consistent.
  \item We have completed the sentence.
\end{enumerate}

About the other comments: we have implemented all the suggested changes, except for comment (c) regarding the use of symbols $\forall$ and $\exists$. Indeed, after careful examination of the paper, we think that the current use of these symbols is clear and consistent enough (and the symbols are very standard in the literature). We hope that you agree with this point. 

\section{Referee 2}
We thank the reviewer for the evaluation and the constructive suggestions.

We agree that the incompatibility between MR and MD can be viewed as one of the main results in the article. However, we think that exploring the relationship between these axioms and most rules that have been considered in the literature about two agents social choice is also significant. It also helps highlighting how these rules differ. Furthermore, we think that it is interesting to observe that no rule that has been considered in the context of two agents satisfies the MD axiom, an observation which covers, perhaps surprisingly, the well-known Fallback Bargaining rule, which could be superficially considered as very egalitarianism-oriented. 

While we agree that an axiomatization is relevant here, we prefer to leave this for future work (see our detailed answer to the Associate editor).

We thank the reviewer for the question about compatibility of $k$-strict Rawlsianism with MD; which has lead to a new remark and footnote (Remark 7 and Footnote 4).


Here are our replies to the reviewer’s main comments.
\begin{enumerate}[label={\alph*})]
  \item We have now specified that $\N$ includes zero.
  \item We have added this observation.
  \item Note that both notions are equivalent. Is is equivalent because $VR$ picks the Borda winners among the first $\khalf + 1 = \khalfAlt + 1$ ranks, whose complement to $m$ is $\floor{\frac{m - 1}{2}}$.” This also applies to R2 d.

  

\end{enumerate}
\end{document}

