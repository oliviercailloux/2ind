\RequirePackage[l2tabu, orthodox]{nag}
\documentclass[version=3.21, pagesize, twoside=off, bibliography=totoc, DIV=calc, fontsize=12pt, a4paper]{scrartcl}
\input{preamble/packages}
\input{preamble/redac}
\input{preamble/math_basics}
%Decision Theory
\NewDocumentCommand{\allalts}{}{\mathcal{A}}
\NewDocumentCommand{\allcrits}{}{\mathscr{C}}
\NewDocumentCommand{\alts}{}{A}
\NewDocumentCommand{\dm}{}{i}
\NewDocumentCommand{\allF}{}{\mathscr{F}}
\NewDocumentCommand{\allvoters}{}{\mathscr{N}}
\NewDocumentCommand{\voters}{}{N}
\NewDocumentCommand{\allprofs}{}{\linors^N}
\NewDocumentCommand{\prof}{}{\bm{P}}
\NewDocumentCommand{\lprof}{}{\lambda_{\bm{P}}}
\NewDocumentCommand{\linors}{}{\mathscr{L}(\allalts)}
%Thanks to https://tex.stackexchange.com/q/154549
	%\makeatletter
	%\def\@myRgood@#1#2{\mathrel{R^X_{#2}}}
	%\def\myRgood{\@ifnextchar_{\@myRgood@}{\mathrel{R^X}}}
	%\makeatother
\NewDocumentCommand{\pref}{}{\succ}
\NewDocumentCommand{\prefi}{O{i}}{\succ_{#1}}
\NewDocumentCommand{\prefeqi}{O{i}}{\succeq_{#1}}
\NewDocumentCommand{\prefone}{}{\succ_1}
\NewDocumentCommand{\preftwo}{}{\succ_2}
\NewDocumentCommand{\prefeqone}{}{\succeq_1}
\NewDocumentCommand{\prefeqtwo}{}{\succeq_2}
\NewDocumentCommand{\prefiinv}{O{i}}{\succ_{#1}^{-1}}
\NewDocumentCommand{\ibar}{}{\overline{i}}

\NewDocumentCommand{\lvs}{}{\intvl{0, m - 1}^N}
\NewDocumentCommand{\losses}{}{\intvl{0, m - 1}}
\NewDocumentCommand{\PD}{}{\mathit{PD}(\prof)}
\NewDocumentCommand{\PE}{}{\mathit{PE}(\prof)}

%Rules
\NewDocumentCommand{\rhoP}{}{\rho_{\prof}}
\NewDocumentCommand{\minspread}{O{A}}{\min_{#1}(\sigma \circ \lambda_{\bm{P}})}
\NewDocumentCommand{\mindisp}{O{A}}{\min_{#1}(d \circ \lambda_{\bm{P}})}
\NewDocumentCommand{\FB}{}{\mathit{FB}}
\NewDocumentCommand{\VR}{}{\mathit{VR}}
\NewDocumentCommand{\SL}{}{\mathit{SL}}
\NewDocumentCommand{\PVv}{O{v}}{\mathit{PV}^{#1}}
\NewDocumentCommand{\PVef}{}{\mathit{PV}^{\floor{\frac{m - 1}{2}}}}%f for first
\NewDocumentCommand{\PVes}{}{\mathit{PV}^{\ceil{\frac{m - 1}{2}}}}%s for second
\NewDocumentCommand{\PVe}{}{\mathit{PV^=}}%egalitarian distribution

%Classes
\NewDocumentCommand{\PVcl}{}{\mathcal{PV}}
\NewDocumentCommand{\PVbcl}{}{\mathcal{PV}^b}
\NewDocumentCommand{\PVecl}{}{\mathcal{PV}^=}%egalitarian distribution
\NewDocumentCommand{\PEcl}{}{\mathcal{PE}}
\NewDocumentCommand{\FHcl}{}{\mathcal{FH}}
\NewDocumentCommand{\VCcl}{}{\mathcal{VC}}
\NewDocumentCommand{\VCecl}{}{\mathcal{VC^=}}
\NewDocumentCommand{\ELcl}{}{\mathcal{EL}}
\NewDocumentCommand{\WELcl}{}{\mathcal{WEL}}
\NewDocumentCommand{\PELcl}{}{\mathcal{PEL}}
\NewDocumentCommand{\WPELcl}{}{\mathcal{WPEL}}


%\input{preamble/draw}

%I find these settings useful in draft mode. Should be removed for final versions.
	%Which line breaks are chosen: accept worse lines, therefore reducing risk of overfull lines. Default = 200.
		\tolerance=2000
	%Accept overfull hbox up to...
		\hfuzz=2cm
	%Reduces verbosity about the bad line breaks.
		\hbadness 5000
	%Reduces verbosity about the underful vboxes.
		\vbadness=1300

\title{Two principles for two-person social choice}
\author{Name}
%\author{Olivier Cailloux}
\affil{Université Paris-Dauphine, PSL Research University, CNRS, LAMSADE, 75016 PARIS, FRANCE\\
%	\href{mailto:olivier.cailloux@dauphine.fr}{olivier.cailloux@dauphine.fr}
}
%\author{Name3}
%\affil{Affil2}
\hypersetup{
	pdfsubject={},
	pdfkeywords={},
}

\begin{document}
\maketitle

\section{Introduction}
\label{sec:intro}
\section{Basic notions and notation}
Let $N = \set{1, 2}$ be the set of two individuals and $\allalts$ be the set of alternatives, with $\card{\allalts} = m\geq 2$. 
Given $i \in N$, let $\ibar \in N \setminus \set{i}$ denote the other individual. Let $\powersetz{\allalts}$ denote the set of non-empty subsets of $\allalts$. Let $\linors$ be the set of linear orders over $\mathcal{A}$. We let ${\pref} \in \linors$ stand for the preference of an individual, written as $\prefi$ when it belongs to $i \in N$. We write $\prof = ({\prefi[1]}, {\prefi[2]}) \in \allprofs$ for a (preference) profile. A social choice rule (SCR) is a function $f: \allprofs → \powersetz{\allalts}$.
Given two SCRs $f$, $f'$, we write $f \subset f'$ to indicate that $f$ is a proper subcorrespondence of $f'$.

An SCR $f$ is anonymous iff $f({\prefi[1]}, {\prefi[2]}) = f({\prefi[2]}, {\prefi[1]})$ for all $({\prefi[1]}, {\prefi[2]}) \in \allprofs$.
An SCR $f$ is neutral iff for all permutations $\sigma$ over $\allalts$ and profile $\prof \in \allprofs$, $\sigma \circ f(\prof) = f(\sigma \circ \prof)$.

Given $j, l \in \N$, let $\intvl{j, l} = [j, l] \cap \N $ denote the interval of integer numbers between $j$ and $l$.
The loss vector of $x$ at $\prof$ is $\lprof(x) \in \intvl{0, m - 1}^N$ that associates to each individual her “loss” resulting of selecting $x$ instead of his favorite alternative, that loss being defined as the number of alternatives that the individual prefers to $x$: $\lprof(x)_i = \card{\set{y \in \allalts \suchthat y \prefi x}}$.
%Let $\bigcup_{k \in \losses}(k^N)$ denote the set of constant loss vectors.
Given two loss vectors $l, l' \in \lvs$, we say that $l$ is strictly smaller than $l'$, $l < l'$, iff $\forall i: l_i ≤ l'_i \land l ≠ l'$. We also write $l ≤ l'$ to denote that $l$ is weakly smaller than $l'$, meaning, strictly smaller or equal. Let $\min_N \lprof(x) = \min_{i \in N} \lprof(x)_i \in \N$ and $\sum_N \lprof(x) = \sum_{i \in N} \lprof(x)_i$ denote, respectively, the minimal loss level and the sum of the losses, considering the loss vector $\lprof(x)$.

Let $\PE = \set{x \in \allalts \suchthat \nexists y \text{ s.t. } \lprof(y) < \lprof(x)}$ be the set of Pareto-efficient alternatives at $\prof$.
Let $\PEcl$ denote the class of SCRs picking only Pareto efficient alternatives ($\forall \prof: f(\prof) \subseteq \PE$).

\section{The first half principle}
\commentRS{We may call this the minimal Rawlsian principle for reasons we discussed with Olivier.}

Given $\prof$ and a loss level $k \in \losses$, define $U(\prof, k) = \set{x \in \allalts \suchthat \lprof(x) ≤ (k, k)}$ as the set of alternatives imposing losses not higher than $k$ for all individuals. We say that such alternatives receive unanimous support at level $k$.
Let $\rhoP = \min \set{k \in \losses \suchthat U(\prof, k) ≠ \emptyset}$ be the least loss level at which some alternative receives unanimous support. By Theorem 1 of \cite{BramsKilgour2001} we have $\forall \prof \in \allprofs$, $\rhoP ≤ \ceil{\frac{m - 1}{2}}$.

Given a profile $\prof \in \allprofs$, let $H(\prof) = \set{x \in \allalts \suchthat \lprof(x) ≤ (\ceil{\frac{m - 1}{2}}, \ceil{\frac{m - 1}{2}})}$ denote the set of alternatives reaching the best half of every individual’s preference. 
In concordance with the ceiling established by Theorem 1 of \cite{BramsKilgour2001}, we use the term “half” to mean the smallest integer $k$ that exceeds $m-k$.

\begin{definition}[First Half (FH)] A SCR satisfies FH if 
	$\forall \prof \in \allprofs,  f(\prof) \subseteq H(\prof)$.
\end{definition}
Let $\FHcl$ denote the class of rules satisfying the First Half (FH) property.

\textbf{Fallback Bargaining} is the SCR $\FB$ that picks all alternatives with unanimous support at $\rhoP$: $\FB(\prof) = U(\prof, \rhoP)$. 

The \textbf{Veto-rank} rule $\VR$ is defined as follows. Each individual vetoes her worst $\floor{\frac{m - 1}{2}}$ alternatives, then the Borda winners among the non vetoed alternatives are picked: $\VR(\prof) = \argmin_{H(\prof)} \sum_N \lprof = \set{x \in H(\prof) \suchthat \forall y \in H(\prof): \sum_N \lprof(x) ≤ \sum_N \lprof(y)}$.

The \textbf{Shortlisting} rule $\SL$ picks the best alternative of individual $1$ that is not among the worst $\floor{\frac{m - 1}{2}}$ alternatives of individual $2$, and the best alternative of $2$ that is not among the worst $\floor{\frac{m - 1}{2}}$ alternatives of $1$. Formally, $\SL(\prof) = \cup_{i \in N} (\argmin_{x \in H^i(\prof)} \lprof(x)_{\ibar})$.

Both $\VR$ and $\SL$ are defined in \cite{Clippel} for $m$ odd only.

The class of \textbf{Pareto-and-veto} rules, $\PVcl$, contains rules parameterized by $v_1, v_2 \in \intvl{0, m - 1}$ with $v_1 + v_2  ≤ m - 1$ where $v_i$ represents the number of alternatives vetoed by individual $i \in N$ (individuals veto the alternatives at the bottom of their preference).
Given $v_i \in \intvl{0, m - 1}$, define $a_i = m - v_i - 1 \in \intvl{0, m - 1}$ as the highest acceptable loss level for individual $i$. For $v=(v_1,v_2)$, the rule $\PVv = \cap_{i \in N}\set{x \in \allalts \suchthat \lprof(x)_i ≤ a_i} \cap \PE$ picks all alternatives in $\PE$ that no individual vetoes. 
The class $\PVcl = \set{\PVv \suchthat v_1, v_2 \in \intvl{0, m - 1} \text{ with } v_1 + v_2 \leq m - 1}$ is the set of those rules, and the class $\PVbcl = \set{\PVv \suchthat v_1, v_2 \in \intvl{0, m - 1} \text{ with } v_1 + v_2 = m - 1}$ is the set of rules where the inequality is binding.

\begin{remark}
    $\FB$, $\VR$, $\SL$ and all SCRs in $\PVcl$ are  neutral.
\end{remark}
\begin{remark}
    $\FB$, $\VR$ and $\SL$ are anonymous, while a SCR in $\PVcl$ is anonymous iff $v_1 = v_2$.
\end{remark}
\begin{remark}\label{th:inFH}
    $\FB, \VR, \SL \in \PEcl \cap \FHcl$.
\end{remark}

We now discuss the relationship of the class $\PVcl$ to the $\FHcl$ property. To this end, we define $\PVe = \PVv[\left(\floor{\frac{m - 1}{2}}, \floor{\frac{m - 1}{2}}\right)]$ as the Pareto-and-veto rule that gives the highest equal veto power to both individuals. 
Thus, under $PV^=$, we have $v_1=v_2= \floor{\frac{m - 1}{2}}$, implying $v_1=v_2=\frac{(m-1)}{2}$ when $m$ is odd and $v_1=v_2= \frac{m}{2}-1$ when $m$ is even. 
Note that $PV^=\in\PVbcl$ iff $m$ is odd.

\begin{proposition}\label{propo:diff}
    $\PVcl ∩ \FHcl = \set{\PVv \in \PVcl \suchthat \forall i: v_i ≥ \floor{\frac{m - 1}{2}}}$.
\end{proposition}
\begin{proof}
	For the “if” part, note that given any profile, the condition $\forall i: v_i ≥ \floor{\frac{m - 1}{2}}$ suffices to guarantee that $\PVv(\prof) \subseteq H(P)$.
	
	To see the “only if” part, consider an arbitrary ordering $\prefi$ over $\allalts$, let $\prefiinv$ denote its inverse, and consider the profile $\prof = (\prefi, \prefiinv)$.
	Observe that $\PVv(\prof)$ will exclusively pick winners in the first half of voter $i$ only if $v_i ≥ \floor{\frac{m - 1}{2}}$.
\end{proof}
A direct implication of Proposition \ref{propo:diff}   is that the set of $\PVcl$ rules satisfying $\FHcl$ is quite reduced, as made explicit by the following remark.
\begin{remark}
    When $\forall i \in \set{1, 2}, v_i ≥ \floor{\frac{m - 1}{2}}$, we have $\abs{v_1 - v_2} ≤ 1$.
	Thus, when $m$ is odd, $\PVcl ∩ \FHcl = \{PV^=\}$, and
	when $m$ is even, $\PVcl ∩ \FHcl = \set{\PVe, \PVv[(\frac{m}{2}, \frac{m}{2} - 1)], \PVv[(\frac{m}{2} - 1, \frac{m}{2})]}$.
\end{remark}

We now establish the relationship of  $\VR, \SL, \FB$ to the class $\PVcl$. Considering two SCRs $f$ and $f'$, let $f \cup f'$ denote the rule $(f \cup f')(\prof) = f(\prof) \cup f'(\prof)$. 
Given any non empty class of SCRs $F$, let $\bigcup F$ denote the maximal (least resolute) SCR that can be formed by unions of rules of $F$.

\begin{proposition}\label{propo:equal}
	$\bigcup(\PEcl \cap \FHcl) = \PVe$.
\end{proposition}
\begin{proof}
    Note that $\bigcup(\PEcl \cap \FHcl)$ is, by definition, the SCR that, for each profile, picks all Pareto alternatives that are in the first half of both individuals’ preferences and only those alternatives. 
    We thus have to show that $\forall \prof: \PVe(\prof) = \PE \cap H(\prof)$. By definition of $\PVe$, it suffices to prove that $\cap_i\set{x \in \allalts \suchthat \lprof(x)_i ≤ m - \floor{\frac{m - 1}{2}} - 1} = H(\prof)$. This in turn follows from the definition of $H(\prof)$.
\end{proof}

The observation below follows from \cref{propo:equal}.
\begin{corollary}\label{th:subPVe}
	A SCR $f \in \PEcl \cap \FHcl$ if and only if $f \subseteq \PVe$.
\end{corollary}

We now establish the relationship between $\FB$, $\VR$ and $\SL$ and show that they can pick disjoint winners.
\begin{proposition}\label{th:different}
	$\exists \prof \suchthat \FB(\prof) \cap \VR(\prof) = \emptyset, \FB(\prof) \cap \SL(\prof) = \emptyset, \VR(\prof) \cap \SL(\prof) = \emptyset$. Furthermore, $\PVe(\prof) ≠ \FB(\prof)$, $\PVe(\prof) ≠ \VR(\prof)$, $\PVe(\prof) ≠ \SL(\prof)$.
\end{proposition}
\begin{proof}
	Consider the following profile $\prof$:
	\begin{equation}
		\label{eq:distinct}
		\begin{array}{llll lll | llll ll}
			a&b&c&d&e&f&g&h&i&j&k&l&m\\
			g&h&i&d&b&j&a&c&e&f&k&l&m\\
		\end{array},
	\end{equation}
	where the first individual prefers $a$ to $b$, $b$ to $c$, etc., and the second individual prefers $g$ to $h$, $h$ to $i$, etc. 
	The bar shows the “half” position.
	The proposition is proven by noting that $\FB(\prof) = \set{d}$, $\VR(\prof) = \set{b}$, $\SL(\prof) = \set{a, g}$ and $\PVe(\prof) = \set{a, b, d, g}$.
\end{proof}

\Cref{th:inFH}, \cref{th:subPVe} and \cref{th:different} lead to the corollary below.
\begin{corollary}
   	$\FB, \VR, \SL \subset \PVe$.
\end{corollary}

\begin{remark}
    The rule $\PVe$ is not merely the union of $\FB$, $\VR$ and $\SL$, as the following profile illustrates.
    \begin{equation}
        \begin{array}{lllll|llll}
                a&b&c&d&e&f&g&h&i\\
                e&c&d&b&a&f&g&h&i\\
        \end{array}.
    \end{equation}
    Here, $\FB(\prof) = \set{c}$, $\VR(\prof) = \set{c}$, $\SL(\prof) = \set{a, e}$ but $\PVe(\prof) = \set{a, b, c, e}$. In fact, no variant of $\VR$ (changing the scoring vector) would elect $b$: its losses are $(1, 3)$ whereas $c$ has losses $(2, 1)$.
\end{remark}
\commentRS{Let's put Section 4 here and see what to do with characterizations and the veto core according to the results we will or will not obtain.}

\subsection{Characterisations}
\commentOC{Attempt for describing formally the useful observation of Matías that $\PVe$ is the FH; SL is max max among the FH; VR maximizes the sum of gains; and FB maximizes the min gain. Mistakes are highly possible.}

In this section, in order to clarify how the four rules that we study here compare, we characterize a rule $f$ by stating conditions under which $f$ is the biggest rule satisfying all those conditions. Those conditions are “necessity” principles, indicating a required condition to be a winner, equivalently, a sufficient condition to lose, thus, they have the form: $f(\prof) \subseteq g(\prof)$ for some $g: \allprofs → \powersetz{\allalts}$, indicating that winning requires to be a member of $g(\prof)$, equivalently, that being not in $g(\prof)$ is sufficient to lose.

The least stringent condition is the FH property: $f(\prof) \subseteq H(\prof)$. The rule $\PVe$ is the biggest rule satisfying FH.

Another condition requires that if $\exists y \in H(\prof) \suchthat \min(\lprof(y)) < \min(\lprof(x))$, then $x$ loses. Equivalently, $x \in f(\prof) ⇒ \min(\lprof(x)) ≤ \min_{y \in H(\prof)} \min(\lprof(y))$. The rule $\SL$ is the biggest rule satisfying both FH and that condition. (Note that these two conditions are independent.)

Similarly, $\VR$ is the biggest rule satisfying both FH and the condition: $x \in f(\prof) ⇒ \sum(\lprof(x)) ≤ \min_{y \in H(\prof)} \sum(\lprof(y))$.

These four rules can also be compared by describing them as follows. \commentOC{Here is the description of $\SL$, which can be suitably adapted to other rules.} The $\SL$ rule is the rule that selects any alternative that is FH and that is not dominated by an FH alternative in terms of minimal welfare loses. More precisely, define $\prec$ as the following partial order on the set of loss vectors: $(a, b) \prec (c, d) ⇔ [(a, b) ≤^\text{FH} (c, d) \land \min\set{a, b} < \min\set{c, d}] \lor [(a, b) <^\text{FH} (c, d)]$, where $(a, b) ≤^\text{FH} (c, d)$ iff $(a, b)$ is first half. Then, $\SL$ selects $x$ iff it is an undominated element among the loss vectors that exist in $\prof$; formally, $f(\prof) = \set{x \in \allalts \suchthat \nexists y \in \allalts \suchthat y \prec x}$. 
Defining the uncomparability relation $\sim$ as $l_1 \sim l_2$ iff $l_1 \sim^\text{FH} l_2 \land \min l_1 = \min l_2$, we obtain the weak order ${\preceq} = {\prec} \cup {\sim}$, and $f(\prof) = \argmin^{\preceq}_{x \in \allalts} \lprof(x)$.
\commentOC{We could perhaps also describe $\prec$ as a union of relations, one of which is common to all rules.}

\subsection{Veto core}
\commentMN{I have been reading Moulin (1983)'s book on veto core correspondence (that is Pareto and veto rule). He mentions that for each outcome in the Pareto and veto rule, there is an order of sincere vetoes that isolates it.
Namely,
Consider an order of players $\{1,2,1,2,1,2,1,2\}$ and the following naive algorithm as follows:\\
\noindent Step 1: player 1 removes his worst alternative in $A$ (denoted $B_1$)\\
\noindent Step 2: player 2 removes his worst alternative in $A\setminus B_1$ (denoted $B_2$)\\
\noindent Step 3: player 1 removes his worst alternative in $A\setminus B_1\cup B_2$ (denoted $B_3$)\\
\noindent Step 4: player 2 removes his worst alternative in $A\setminus B_1\cup B_2 \cup B_3$\\
In the example of Remark 5, this sequence leads to $c$, the $FB$ outcome.
\textbf{Question 1:} Is it always the case that this alternating sequence leads to $FB$?
To obtain $e$ again in the example of Remark 5, the sequence $\{1,1,1,1,2,2,2,2\}$ seems to work. Similarly, to obtain $a$, the sequence $\{2,2,2,2,1,1,1,1\}$ seems to work.
Finally, to isolate $b$, the sequence $\{1,1,1,1,1,1,1,2\}$ seems to work.
This leads me to think that
\textbf{Question 2:} $PV^=$ could be the union of any sequence of length $n-1$. 
\textbf{Question 3:} Similarly, $FB$ consists only of the extreme sequences $\{1,1,1,...,1,2,\ldots,2\}$ and $\{2,2,2,...,2,1,\ldots,1\}$.
\textbf{Question 4:} Which are the sequences that isolate $FB$?
}

Let $s \in \set{1, 2}^{m - 1}$ be a sequence of turns. 
Define a corresponding sequence of subsets of alternatives, $A^s \in \powersetz{\allalts}^m$, as follows: $A^s_1 = \allalts$ and $A^s_{i + 1} = A^s_i \setminus \argmax_{x \in A^s_i} \lprof(x)_{s_i}$. 
Note that $\card{A^s_m} = 1$. 
The \emph{simple veto core} rule $f^s$ is the SCR $f^s(\prof) = A^s_m$ (TODO cite Moulin).
Let $\emptyset ≠ S \subseteq \set{1, 2}^{m - 1}$. Define $f^S = \cup_{s \in S} f^s$.
The set of \emph{veto core} rules is the set $\set{f^S \suchthat \emptyset ≠ S \subseteq \set{1, 2}^{m - 1}}$.

Define $S^= = \set{s \in \set{1, 2}^{m - 1} \suchthat \card{s^{-1}(1)} = \card{s^{-1}(2)}}$.
%%%%%The set of \emph{veto core egalitarian} rules is $\VCecl = \set{f^S \suchthat \emptyset ≠ S \subseteq \set{1, 2}^{m - 1} \land \card{S^{-1}(1)} = \card{S^{-1}(2)}}$.
For $m$ odd, the \emph{veto core egalitarian} rule is $f^{S^=}$.

\begin{proposition}[Might be in Moulin]
    For any sequence of turn $s \in \set{1, 2}^{m - 1}$, $f^s$ is Pareto. 
    Therefore, for any set of sequences $\emptyset ≠ S \subseteq \set{1, 2}^{m - 1}$, $f^S$ is Pareto.
\end{proposition}

\begin{conjecture}
	\label{th:sUnique}
	$\forall m, s, s' \in \set{1, 2}^{m - 1}: s ≠ s' ⇒ f^s ≠ f^{s'}$.
\end{conjecture}
\begin{proof}
	Given a sequence $s$, let $s \circ (+k)$ denote the sequence in $\set{1, 2}^{m - 1 - k}$ that skips the first $k$ elements, thus such that $(s \circ (+k))_i = s_{i + k}$.
	
	Observe first that if $s ≠ s'$, there must be some suffixes of these sequences, of a common length, that give different number of turns to the players (equivalently, to the player $1$).
	Formally: $\exists k \in \intvl{0, m - 2} \suchthat \card{(s \circ (+k))}^{-1}(1) ≠ \card{(s' \circ (+k))}^{-1}(1)$.
	Pick such a $k$ and define $r = (s \circ +k)$ and $r' = (s' \circ +k)$, thus with $r, r' \in \set{1, 2}^{m - 1 - k}$.

	Define $\prof$ by assigning to the voters opposite preferences to the $m - k$ first alternatives and equal preferences to the last $k$ alternatives; more explicitly, ${\prefi[1]} = (a_1, …, a_{m - k}, a_{m - k + 1}, …, a_m)$ and ${\prefi[2]} = (a_{m - k}, …, a_1, a_{m - k + 1}, …, a_m)$. 
	
	Considering $f^s(\prof)$ and $f^{s'}(\prof)$, the first $k$ alternatives “vetoed“ by both $s$ and $s'$ are the $k$ alternatives that the voters agree are the worst, namely, $\set{a_{m - k + 1}, … a_m}$. Considering now any of the rest of the sequences, say (wlog) $r$, we see that the remaining “vetoed” alternatives depends only on the number of turns that each player plays in $r$, namely, $\card{r^{-1}(1)}$ and $\card{r^{-1}(2)}$, as the voters have opposite preferences on the alternatives that have not yet been vetoed after $k$ turns: from the remaining alternatives $\set{a_1, …, a_{m - k}}$, the worst $\card{r^{-1}(i)}$ alternatives from voter $i$ will be vetoed (for $i \in \set{1, 2}$). Thus, $f^s(\prof) = \set{a_{m - k - \card{r^{-1}(1)}}}$ and $f^{s'}(\prof) = \set{a_{m - k - \card{{r'}^{-1}(1)}}}$. These are distinct, as $\card{r^{-1}(1)} ≠ \card{{r'}^{-1}(1)}$.
\end{proof}

\commentOC{What about $S ≠ S'$ but $f^S = f^{S'}$?}

A bijective sequence of $m$ alternatives is a sequence $e \in \allalts^m$ that covers all the alternatives, thus, such that $e_{\intvl{1, m}} = \allalts$.
We can interpret any such bijective sequence $e \in \allalts^m$ as a “veto sequence” and associate to it the sequence of subsets of alternatives, defined, for $0 ≤ j ≤ m - 1$, as $e_{\intvl{1, j}} \subset \allalts$, representing all the alternatives that have been vetoed after turn $j$ (thus with $e_{\intvl{1, 0}} = \emptyset$); and the sequence of complements, defined, for $1 ≤ j ≤ m$, as $\allalts \setminus e_{\intvl{1, j - 1}}$, or equivalently, as $e_{\intvl{j, m}}$, representing the alternatives that remain before turn $j$.

We say that a bijective sequence $e \in \allalts^m$ is a possible elimination sequence for $\prof = \set{(1, {\prefi[1]}), (2, {\prefi[2]})}$ iff $\forall j \in \intvl{1, m - 1}: \exists i \in \set{1, 2} \suchthat e_j = \min_{e_{\intvl{j, m}}} {\prefi}$.

Given a bijective sequence $e \in \allalts^m$ and a preference ${\pref} \in \linors$, we say that the possible vetoes associated to $e$ and $\pref$ are $A^{e, {\pref}} = \set{\min_{e_{\intvl{j, m}}} {\prefi} \suchthat j \in \intvl{1, m - 1}} \subseteq \allalts$.

The following lemma shows, losely speaking, that any profile and possible elimination sequence corresponds to sequences of turns that give to each voter the “responsability of vetoing” some of the alternatives among its possible vetoes.
\begin{lemma}[To be verified]
	\label{th:seqATos}
	Consider a profile $\prof = \set{(1, {\prefi[1]}), (2, {\prefi[2]})}$ and $e \in \allalts^m$, a possible elimination sequence for $\prof$.
	Let $A^{e, {\prefi}}$ be the corresponding possible vetoes for each voter $i \in \set{1, 2}$.
	Pick any two numbers $(k_1, k_2)$ such that $k_1 + k_2 = m - 1$ and $k_i ≥ \card{(A^{e, {\prefi}} \setminus A^{e, {\prefi[\ibar]}})}$.
	
	Then, $\exists s \in \set{1, 2}^{m - 1} \suchthat f^s(\prof) = \set{e_m}$ satisfying $\forall i \in \set{1, 2}: \card{s^{-1}(i)} = k_i$.
\end{lemma}
\begin{proof}
	Observe that $\cup_{i \in \set{1, 2}} A^{e, {\prefi}} = \set{\min_{e_{\intvl{j, m}}} {\prefi} \suchthat i \in \set{1, 2}, j \in \intvl{1, m - 1}}$ (by definition of $A^{e, {\prefi}}$), the latter set being also equal to $e_{\intvl{1, m - 1}} = \allalts \setminus \set{e_m}$, as $e$ is a possible elimination sequence for $\prof$. Also, $\cup_{i \in \set{1, 2}} A^{e, {\prefi}}$ can be written as $(A^{e, {\prefone}} \setminus A^{e, {\preftwo}}) \cup (A^{e, {\preftwo}} \setminus A^{e, {\prefone}}) \cup (A^{e, {\prefone}} \cap A^{e, {\preftwo}})$.
	
	Thus, we can partition $A^{e, {\prefone}} \cup A^{e, {\preftwo}}$ into disjoint sets $K_1, K_2$ such that $(A^{e, {\prefi}} \setminus A^{e, {\prefi[\ibar]}} \subseteq K_i \subseteq A^{e, {\prefi}}$, $\card{K_i} = k_i$ and $K_1 \cup K_2 = e_{\intvl{1, m - 1}}$.
	
	Define, $\forall j \in \intvl{1, m - 1}$: $s_j = i$ iff $e_j \in K_i$.
\end{proof}

For $x \in \allalts$, let ${\prefi}(x) \subseteq \allalts$ designate the strict lower contour set of $x$ and ${\prefeqi}(x) \subseteq \allalts$ designate the lower contour set of $x$ including $x$.
\begin{lemma}
	\label{th:coveringToSeqs}
	Consider $x \in \allalts$, $\prof = \set{(1, {\prefi[1]}), (2, {\prefi[2]})}$.
	Assume that $\cup_i {\prefeqi}(x) = \allalts$ (equivalently, $x \in \PE$).
	
	Define $K_1^u = \allalts \setminus {\prefeqtwo}(x)$.
	Consider any $k_1 \in \intvl{\card{K_1^u}, \card{{\prefone}(x)}}$. Define $k_2 = m - 1 - k_1$.

	Then, $\exists s \in \set{1, 2}^{m - 1} \suchthat f^s(\prof) = \set{x}$ satisfying $\forall i \in \set{1, 2}: \card{s^{-1}(i)} = k_i$.
\end{lemma}
\begin{proof}
	Define $B$ as the bottom $k_1 - \card{K_1^u}$ alternatives from ${\prefone}(x) \setminus K_1^u$.
	Define $K_1 = K_1^u \cup B$.
	Note that $K_1^u \subseteq K_1 \subseteq {\prefone}(x)$ and $\card{K_1} = k_1$.
	Define $K_2 = \allalts \setminus \set{x} \setminus K_1$.
	
	Define $e$ as any sequence that satisfies $\forall i: [y \in K_i ⇒ {\prefi}(y) \text{ is before } y \text{ in } e]$. Apply \cref{th:seqATos}.
\end{proof}

\begin{conjecture}
	\label{th:vce}
	For $m$ odd, $f^{S^=} = \bigcup (\PEcl \cap \FHcl) = \PVe$.
\end{conjecture}
\begin{proof}
	To see that $f^{S^=} \subseteq \bigcup (\PEcl \cap \FHcl)$. Any sequence leads to Pareto efficient (to be detailed). Let $2k+1$ be the number of alternatives. Since any sequence in $S^=$ gives each voter the same number of turns (hence $k$), the outcome is never among the worst $k$ alternatives of each voter. It follows that the outcome is Pareto-efficient and in the first half of each voter's preferences, concluding the proof.
	
	About the other direction. 
	\footnote{Example (Preliminary): abcdefg, gbacdef, b $\Longrightarrow$ 112212.}
	From first half: at least k alternatives in the lower contour set of $x$ for each voter. Apply \cref{th:coveringToSeqs}.
\end{proof}

\begin{conjecture}
	\label{th:vrNotVce}
	$\forall \emptyset ≠ S \subseteq S^=: \VR ≠ f^{S}$.
\end{conjecture}
\begin{proof}
    ?
\end{proof}

\begin{conjecture}
	\label{th:slVce}
	$\SL = f^{\set{(1, 1, …), (2, 2, …)}}$.
\end{conjecture}

\begin{remark}
	\label{th:fbVce}
	It is false that
	$\FB = f^{\set{(1, 2,1,2, …), (2, 1,2,1 …)}}$.
    \begin{equation}
        \begin{array}{lllll}
                a&b&c&d&e\\
                d&c&b&e&a\\
        \end{array}.
    \end{equation}
    $\FB(\prof) = \set{b, c}$ and $f^{\set{(1, 2,1,2, …), (2, 1,2,1 …)}}=\{c\}$.
\end{remark}

\commentOC{Is $\VCecl$ the set of anonymous veto core rules, by chance?}

\commentRS{Below are characterizations}
PV-e iff PE, FH, MAskin monotonicity

I recall from a paper of Vincent that FB is characterized by using a Rawlsian maximin axiom plus maximality 

For SL and VR we should look at Clippel. 

\color{green} MN: no characterization of VR to the best of my knowledge \color{black}

Sertel shows (says Danilo Coelho) that the unanimous compromise set (the FB winners?) is the set of Pareto efficient candidates that maximize the worst-of welfare (welfare = nb alternatives less good than the chosen one). \color{green} MN: This is true, right? \color{black}

\commentOC{Would it be interesting to attempt to characterize scoring rules that attribute an infinitely negative weight to the ranks in the bottom half?}

\section{The equal loss principle}
Given $\prof \in \allprofs$, define $S(\prof) = \set{x \in \allalts \suchthat \lprof(x)_1 = \lprof(x)_2}$ as the set of alternatives that are ranked at the same position by both individuals.
    
\begin{definition}[Weak equal loss (WEL)]
    $\forall \prof \in \allprofs: [S(\prof) ≠ \emptyset] ⇒ f(\prof) \cap S(\prof) ≠ \emptyset$.
\end{definition}

\begin{proposition}
    $\forall m ≥ 3: \PEcl ∩ \WELcl = \emptyset$.
\end{proposition}

Thus, Pareto efficiency is a fortiori incompatible with the following stronger version of the equal loss principle.

\begin{definition}[Equal loss (EL)]
    $\forall \prof \in \allprofs: [S(\prof) ≠ \emptyset] ⇒ f(\prof) \subseteq S(\prof)$.
\end{definition}

We now embed Pareto efficiency into the equal loss requirement. The following property mandates that $f$ picks solely the (unique) Pareto efficient alternative ranked the same by everybody, if there is one.
\begin{definition}[Paretian equal loss (PEL)]
    $\forall \prof \in \allprofs: [S(\prof) \cap PE(\prof) ≠ \emptyset] ⇒ f(\prof) = S(\prof) \cap PE(\prof)$.
\end{definition}

\begin{proposition}
	All rules in PV fail PEL.
\end{proposition}
\begin{proof}
	For $m = 5$, consider the profile $abcde$ and $cbade$. Here, PEL requires to choose uniquely $b$. Consider any $f$ in $\PVcl$. If $b \notin f(\prof)$, the proposition holds. Assume $b \in f(\prof)$. We show that $f(\prof)$ includes another alternative. PV= chooses abc PV(1, 3) picks ab and PV(3, 1) picks bc. PEL picks b.
If PV picks b, it also picks something else.
\end{proof}

\begin{proposition}
	FB satisfies PEL.
\end{proposition}

\begin{definition}[Weak Paretian equal loss (WPEL)]
    $\forall \prof \in \allprofs: [S(\prof) \cap PE(\prof) ≠ \emptyset] ⇒ f(\prof) \cap S(\prof) ≠ \emptyset$.
\end{definition}

\begin{proposition}
	VR and SL fail WPEL.
\end{proposition}
\begin{proof}
	Consider the following profile $\prof$, reused from \cref{eq:distinct}:
	\begin{equation}
		\begin{array}{llll lll | llll ll}
			a&b&c&d&e&f&g&h&i&j&k&l&m\\
			g&h&i&d&b&j&a&c&e&f&k&l&m\\
		\end{array},
	\end{equation}
	where $\VR(\prof) = \set{b}$ and $\SL(\prof) = \set{a, g}$.
PEL requires to choose at least $d$.
\end{proof}

Some rules from the class $\PVcl$ satisfy WPEL. It follows that an adequate refinement of those rules would satisfy PEL. The rule $\PVe$ is among those.
\begin{proposition}
	For any $m ≥ 3$, a PV rule $\PVv[(v_1, v_2)]$ satisfies WPEL iff its veto levels are both at most $\floor{\frac{m}{2}}$, thus, iff $\max_{i \in \set{1, 2}} v_i ≤ \floor{\frac{m}{2}}$.
\end{proposition}
\begin{proof}
	Define $v = \floor{\frac{m}{2}}$.

	For all $\prof \in \allprofs$, if $x \in S(\prof) \cap \PE$, then $x$ is not among the last $v$ ranks, because all alternatives before $x$ for the first voter must be placed after $x$ for the second voter.
	A PV rule with veto parameters at most $v$ will thus pick all such alternatives $S(\prof) \cap \PE$, as required by WPEL.
	
	For the other direction, observe that there exists $\prof \in \allprofs$ such that for some $x \in \allalts$, $x \in S(\prof) \cap \PE$ and $x$ is positioned just better than the last $v$ ranks (thus $\exists \prof \in \allprofs, x \in \allalts \suchthat \forall i \in \set{1, 2}: \lprof(x)_i = \floor{\frac{m - 1}{2}}$, leaving $\ceil{\frac{m - 1}{2}} = v$ positions behind $x$).
	A PV rule such that $\max_{i \in \set{1, 2}} v_i > v$ will thus not include $x$ in the set of winners, hence, the rule will fail WPEL.
\end{proof}

Call the dispersion of a loss vector $l$ at $\prof$ the value $d(l) = \abs{l_1 - l_2}$. 
Thus, $(d \circ \lprof)(x) = \max\lprof(x) - \min\lprof(x)$.
Note that $d \in \Sigma$ and seems to coincide with multiple commonly used spread measures (to be verified). 


\commentOC{I made an attempt below, but the property I defined is always satisfied (because for $m$ fixed, $d \circ \lprof$ is bounded by $m - 1$). I suppose the intuition works only for variable $m$. To be discussed…}
\commentRS{For any k between 1 and m, define the minimal dispersion MD(k) condition as follows: for any x, y that are Pareto, if the dispersion of the loss vector of x exceeds the dispersion of the loss vector of y by k, then x is not elected. Note that MD(k) gets weaker with growing k. MD(1) implies to pick among alternatives with a minimal spread. MD(m-1) only requires that an alternative with extreme spread (i.e., ranked first by a voter and last by another) should not be elected in the presence of a Pareto efficient alternative that is tanked at the same level by both voters. MD(m) is vacuous. Note also that MD(k) implies PEL iff k=1.}
\commentRS{The result here is that the FH principle and MD(k) are compatible iff k is greater than (m/2)-3. (Some more precision is needed for this number). In other words,the FH principle and MD(k) are incompatible iff k is less than than (m/2)-3. This implies that FB, SL, PV, VR all fail MD(k) for small k.}
\commentRS{Let's adapt the definitions and results below accordingly.}

if the dispersion of the loss vector of x exceeds the dispersion of the loss vector of y by k, then x is not elected
\begin{definition}
	A rule $f$ satisfies strong PEL, SPEL, iff $\exists k \suchthat \forall \prof \in \allprofs, x \in \allalts: [\exists y \in \allalts \suchthat (d \circ \lprof)(x) - (d \circ \lprof)(y) ≥ k] ⇒ x \notin f(\prof)$.
\end{definition}

PV, FB, SL and VR, all fail strong PEL. 

In fact, strong PEL and FH are incompatible conditions. 


Can we identify an interesting SCR that satisfies MD(1)?


As Matias suggested, can we use MD(1) to refine VR, SL or FB? Discuss how discriminative the four rules are. All four rules are irresolute. Do they have refinements that are anonymous and neutral? The answer is obviously affirmative for PV (as the remaining three rules do the job) but should be addressed for FB, SL and VR. 
The Theorem, stated at the end of page 4 (\hrefblue{https://link.springer.com/article/10.1007/BF00187429}{Sprumont}), is about refinements of FB. It says that FB admits refinements that satisfy the two conditions of the paper, namely CPB and SO. Moreover, any rule that satisfies CPB and SO is a refinement of FB.

We can refer to this when we address the question of refining FB, SL and VR.

\commentRS{I am not sure we need the four definitions that follow}

\begin{definition}[Paretian compromism compatibility (PCC)]
	Define $\Sigma = \set{\sigma \in (\R^+)^{(\lvs)} \suchthat \sigma^{-1}(0) = \cup_{k \in \losses}(k^N)}$ be the set of all spread measures that attribute the minimal inequality measure to the constant loss vectors and none others.
	\commentOC{Could we perhaps define sigma as a weak order over the loss vectors?}
	$f$ is PCC iff there exists a spread measure $\sigma \in \Sigma$ such that $\forall \prof: f(\prof) \cap \minspread[\PE] ≠ \emptyset$.
\end{definition}

\begin{definition}[Dispersion phobia]
	If $x$ has a strictly higher dispersion than $y$ at $\prof$, then $f$ may not pick $x$.
\end{definition}
\begin{definition}[Loss vector comparing (LVC)]
	Let $\succeq$ be a weak order on $\lvs$. The rule $f^\succeq$ selects all alternatives whose loss vectors are minimal elements of $\restr{{\succeq}}{\lprof(\alts)}$. A rule satisfies LVC iff $\exists {\succeq} \suchthat f = f^\succeq$. (This property is called preorder based in Cailloux and Endriss 2014.)
\end{definition}

\begin{definition}[Dispersion compromism]
	$\forall \prof: f(\prof) \subseteq \mindisp[\PE]$.
\end{definition}



\subsection{Rules we invent}
Anonymized PV rules
PV when the sum of vetoes is less than m-1
\section{Miscellaneous comments}
I proposed: [VR selects equal loss ⇒ VR selects all alternatives with "almost equal loss"], but this is contradicted by: (abc; cba).


Danilo Coelho seems to be interested in implementation of FB and shortlisting and related rules (see the \href{https://www.cmss.auckland.ac.nz/2020/06/03/online-social-choice-seminar-series/}{Online Social Choice Seminar})

\color{green}MN: Yes, I agree. By the way, in his paper, he cites a paper by Clippel with a new characterization of Fallback, we might want to cite it.\color{black}

\bibliography{bibliototal}

\appendix
\section{Various}
I restore here a consideration by Remzi.

\begin{remark}[Consideration]
Now, suppose, under a given sigma, we pick among the PV alternatives the one with the smallest loss spread. For example, at the profile 
\begin{center}
	$
	\begin{array}{cccccc}
		\mathbf{i_1} \quad &a&b&c&d&e\\
		\mathbf{i_2} \quad &d&c&a&b&e\\
	\end{array}
	$
\end{center}

PV would pick a-c and the approach suggest, under a reasonable sigma, would refine this to the singleton c (which is also a refinement of FB).

Question: what is this new rule? It refines PV. It is not FB. At the profile

\begin{center}
	$
	\begin{array}{cccccc}
		\mathbf{i_1} \quad &a&b&c&d&e\\
		\mathbf{i_2} \quad &c&b&a&d&e\\
	\end{array}
	$
\end{center}

PV picks a-b-c, this rule refines it to b (which is the FB outcome) while SL picks a-c.
\end{remark}

\end{document}
