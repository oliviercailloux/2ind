\RequirePackage[l2tabu]{nag}
\documentclass[version=3.21, pagesize, twoside=off, bibliography=totoc, DIV=calc, fontsize=12pt, a4paper]{scrartcl}
%Permits to copy eg x ⪰ y ⇔ v(x) ≥ v(y) from PDF to unicode data, and to search. From pdfTeX users manual. See https://tex.stackexchange.com/posts/comments/1203887.
	\input glyphtounicode
	\pdfgentounicode=1
%Latin Modern has more glyphs than Computer Modern, such as diacritical characters. fntguide commands to load the font before fontenc, to prevent default loading of cmr.
	\usepackage{lmodern}
%Encode resulting accented characters correctly in resulting PDF, permits copy from PDF.
	\usepackage[T1]{fontenc}
%UTF8 seems to be the default in recent TeX installations, but not all, see https://tex.stackexchange.com/a/370280.
	\usepackage[utf8]{inputenc}
%Provides \newunicodechar for easy definition of supplementary UTF8 characters such as → or ≤ for use in source code.
	\usepackage{newunicodechar}
%Text Companion fonts, much used together with CM-like fonts. Provides \texteuro and commands for text mode characters such as \textminus, \textrightarrow, \textlbrackdbl.
	\usepackage{textcomp}
%St Mary’s Road symbol font, used for ⟦ = \llbracket. The \SetSymbolFont command avoids spurious warnings, but also some valid ones, see https://tex.stackexchange.com/a/106719/.
	\usepackage{stmaryrd}\SetSymbolFont{stmry}{bold}{U}{stmry}{m}{n}
%Solves bug in lmodern, https://tex.stackexchange.com/a/261188; probably useful only for unusually big font sizes; and probably better to use exscale instead. Note that the authors of exscale write against this trick.
	%\DeclareFontShape{OMX}{cmex}{m}{n}{
		%<-7.5> cmex7
		%<7.5-8.5> cmex8
		%<8.5-9.5> cmex9
		%<9.5-> cmex10
	%}{}
	%\SetSymbolFont{largesymbols}{normal}{OMX}{cmex}{m}{n}
%More symbols (such as \sum) available in bold version, see https://github.com/latex3/latex2e/issues/71. In article mode (but not in presentation mode), also hides some potentially useful warnings such as when using $\bm{\llbracket}$, see stmaryrd in this document (not sure why).
	\DeclareFontShape{OMX}{cmex}{bx}{n}{%
	   <->sfixed*cmexb10%
	   }{}
	\SetSymbolFont{largesymbols}{bold}{OMX}{cmex}{bx}{n}
%For small caps also in italics, see https://tex.stackexchange.com/questions/32942/italic-shape-needed-in-small-caps-fonts, https://tex.stackexchange.com/questions/284338/italic-small-caps-not-working.
	\usepackage{slantsc}
	\AtBeginDocument{%
		%“Since nearly no font family will contain real italic small caps variants, the best approach is to substitute them by slanted variants.” -- slantsc doc
		%\DeclareFontShape{T1}{lmr}{m}{scit}{<->ssub*lmr/m/scsl}{}%
		%There’s no bold small caps in Latin Modern, we switch to Computer Modern for bold small caps, see https://tex.stackexchange.com/a/22241
		%\DeclareFontShape{T1}{lmr}{bx}{sc}{<->ssub*cmr/bx/sc}{}%
		%\DeclareFontShape{T1}{lmr}{bx}{scit}{<->ssub*cmr/bx/scsl}{}%
	}
%Warn about missing characters.
	\tracinglostchars=2
%Nicer tables: provides \toprule, \midrule, \bottomrule.
	\usepackage{booktabs}
%For new column type X which stretches; can be used together with booktabs, see https://tex.stackexchange.com/a/97137. “tabularx modifies the widths of the columns, whereas tabular* modifies the widths of the inter-column spaces.” Loads array.
	%\usepackage{tabularx}
%math-mode version of "l" column type. Requires \usepackage{array}.
	%\usepackage{array}
	%\newcolumntype{L}{>{$}l<{$}}
%Provides \xpretocmd and loads etoolbox which provides \apptocmd, \patchcmd, \newtoggle… Also loads xparse, which provides \NewDocumentCommand and similar commands intended as replacement of \newcommand in LaTeX3 for defining commands (see https://tex.stackexchange.com/q/98152 and https://github.com/latex3/latex2e/issues/89).
	\usepackage{xpatch}
%for \nexists and because it is a basic package nowadays, see https://tex.stackexchange.com/q/539592/.
	\usepackage{amssymb}
%loads and fixes some bugs in amsmath (a basic, mandatory package nowadays, see Grätzer, More Math Into LaTeX) and provides \DeclarePairedDelimiter. I recommend \begin{equation}, which allows numbering, rather than \[ (and $$ should be avoided), see https://tex.stackexchange.com/questions/503. Relatedly, do not use the displaymath environment: use equation. Do not use the eqnarray environment: use align. This improves spacing. (See l2tabu or amsldoc.)
	\usepackage{mathtools}
%Package frenchb asks to load natbib before babel-french. Package hyperref asks to load natbib before hyperref.
	\usepackage{natbib}

\newtoggle{LCpres}
	\newtoggle{LCart}
	\newtoggle{LCposter}
	\makeatletter
	\@ifclassloaded{beamer}{
		\toggletrue{LCpres}
		\togglefalse{LCart}
		\togglefalse{LCposter}
		\wlog{Presentation mode}
	}{
		\@ifclassloaded{tikzposter}{
			\toggletrue{LCposter}
			\togglefalse{LCpres}
			\togglefalse{LCart}
			\wlog{Poster mode}
		}{
			\toggletrue{LCart}
			\togglefalse{LCpres}
			\togglefalse{LCposter}
			\wlog{Article mode}
		}
	}
	\makeatother%

%Language options ([french, english]) should be on the document level (last is main); except with tikzposter: put [french, english] options next to \usepackage{babel} to avoid warning. beamer uses the \translate command for the appendix: omitting babel results in a warning, see https://github.com/josephwright/beamer/issues/449. Babel also seems required for \refname.
	\iftoggle{LCpres}{
		\usepackage{babel}
	}{
	}
	%\frenchbsetup{AutoSpacePunctuation=false}
%https://ctan.org/pkg/amsmath recommends ntheorem, which supersedes amsthm, which corrects the spacing of proclamations and allows for theoremstyle, but I decided to switch to amsthm with thmtools (mentioned in amsthm doc) because ntheorem “seems essentially unmaintaned and has severe problems”, see https://tex.stackexchange.com/q/535950. Must be loaded after amsmath (from amsthm doc).
		\usepackage{amsthm}
		\usepackage{thmtools}
%listings (1.7) does not allow multi-byte encodings. listingsutf8 works around this only for characters that can be represented in a known one-byte encoding and only for \lstinputlisting. Other workarounds: use literate mechanism; or escape to LaTeX (but breaks alignment).
	%\usepackage{listings}
	%\lstset{tabsize=2, basicstyle=\ttfamily, escapechar=§, literate={é}{{\'e}}1}
%I favor acro over acronym because the former is more recently updated (2018 VS 2015 at time of writing); has a longer user manual (about 40 pages VS 6 pages if not counting the example and implementation parts); has a command for capitalization; and acronym suffers a nasty bug when ac used in section, see https://tex.stackexchange.com/q/103483 (though this might be the fault of the silence package and might be solved in more recent versions, I do not know) and from a bug when used with cleveref, see https://tex.stackexchange.com/q/71364. However, loading it makes compilation time (one pass on this template) go from 0.6 to 1.4 seconds, see https://bitbucket.org/cgnieder/acro/issues/115.
	\usepackage{acro}
	%“All options of acro that have not been mentioned in section 4.1 have to be set up… with… \acsetup{…}” -- acro package doc, cited by the Overleaf support (thanks to them!)
	\acsetup{single}
	\DeclareAcronym{AMCD}{short=AMCD, long={Aide Multicritère à la Décision}}
\DeclareAcronym{AHP}{short=AHP, long={Analytic Hierarchy Process}}
\DeclareAcronym{AR}{short=AR, long={Argumentative Recommender}}
\DeclareAcronym{DA}{short=DA, long={Decision Analysis}}
\DeclareAcronym{DJ}{short=DJ, long={Deliberated Judgment}}
\DeclareAcronym{DM}{short=DM, long={Decision Maker}}
\DeclareAcronym{DP}{short=DP, long={Deliberated Preference}}
\DeclareAcronym{MAVT}{short=MAVT, long={Multiple Attribute Value Theory}}
\DeclareAcronym{MCDA}{short=MCDA, long={Multicriteria Decision Aid}}
\DeclareAcronym{MIP}{short=MIP, long={Mixed Integer Program}}
\DeclareAcronym{SEU}{short=SEU, long={Subjective Expected Utility}}


\iftoggle{LCpres}{
	%I favor fmtcount over nth because it is loaded by datetime anyway; and fmtcount warns about possible conflicts when loaded after nth (“\ordinal already defined use \FCordinal instead”). See also https://english.stackexchange.com/questions/93008.
	\usepackage{fmtcount}
	%For nice input of date of presentation. Must be loaded after the babel package. Has possible problems with srcletter: https://golatex.de/verwendung-von-babel-und-datetime-in-scrlttr2-schlaegt-fehlt-t14779.html.
	\usepackage[nodayofweek]{datetime}
}{
}
%For presentations, Beamer implicitely uses the pdfusetitle option. autonum doc mandates option hypertexnames=false. I want to highlight links only if necessary for the reader to recognize it as a link, to reduce distraction. In presentations, this is already taken care of by beamer (https://tex.stackexchange.com/a/262014). If using colorlinks=true in a presentation, see https://tex.stackexchange.com/q/203056. Crashes the first compilation with tikzposter, just compile again and the problem disappears, see https://tex.stackexchange.com/q/254257.
\makeatletter
\iftoggle{LCpres}{
	\usepackage{hyperref}
}{
	\usepackage[hypertexnames=false, pdfusetitle, linkbordercolor={1 1 1}, citebordercolor={1 1 1}, urlbordercolor={1 1 1}]{hyperref}
	%https://tex.stackexchange.com/a/466235
	\pdfstringdefDisableCommands{%
		\let\thanks\@gobble
	}
}
\makeatother
%urlbordercolor is used both for \url and \doi, which I think shouldn’t be colored, and for \href, thus might want to color manually when required. Requires xcolor.
	\NewDocumentCommand{\hrefblue}{mm}{\textcolor{blue}{\href{#1}{#2}}}
%hyperref doc says: “Package bookmark replaces hyperref’s bookmark organization by a new algorithm (...) Therefore I recommend using this package”.
	\usepackage{bookmark}
%Need to invoke hyperref explicitly to link to line numbers: \hyperlink{lintarget:mylinelabel}{\ref*{lin:mylinelabel}}, with \ref* to disable automatic link. Also see https://tex.stackexchange.com/q/428656 for referencing lines from another document.
	%\usepackage{lineno}
	%\NewDocumentCommand{\llabel}{m}{\hypertarget{lintarget:#1}{}\linelabel{lin:#1}}
	%\setlength\linenumbersep{9mm}
%For complex authors blocks. Seems like authblk wants to be later than hyperref, but sooner than silence. See https://tex.stackexchange.com/q/475513 for the patch to hyperref pdfauthor.
	\ExplSyntaxOn
	\seq_new:N \g_oc_hrauthor_seq
	\NewDocumentCommand{\addhrauthor}{m}{
		\seq_gput_right:Nn \g_oc_hrauthor_seq { #1 }
	}
	\NewExpandableDocumentCommand{\hrauthor}{}{
		\seq_use:Nn \g_oc_hrauthor_seq {,~}
	}
	\ExplSyntaxOff
	{
		\catcode`#=11\relax
		\gdef\fixauthor{\xpretocmd{\author}{\addhrauthor{#2}}{}{}}%
	}
	\iftoggle{LCart}{
		\usepackage{authblk}
		\renewcommand\Affilfont{\small}
		\fixauthor
		\AtBeginDocument{
		    \hypersetup{pdfauthor={\hrauthor}}
		}
	}{
	}
%I do not use floatrow, because it requires an ugly hack for proper functioning with KOMA script (see scrhack doc). Instead, the following command centers all floats (using \centering, as the center environment adds space, http://texblog.net/latex-archive/layout/center-centering/), and I manually place my table captions above and figure captions below their contents (https://tex.stackexchange.com/a/3253).
	\makeatletter
	\g@addto@macro\@floatboxreset\centering
	\makeatother
%Permits to customize enumeration display and references
	%\nottoggle{LCpres}{
		%\usepackage{enumitem} %follow list environments by a string to customize enumeration, example: \begin{description}[itemindent=8em, labelwidth=!] or \begin{enumerate}[label=({\roman*}), ref={\roman*}].
	%}{
	%}
%Provides \Centering, \RaggedLeft, and \RaggedRight and environments Center, FlushLeft, and FlushRight, which allow hyphenation. With tikzposter, seems to cause 1=1 to be printed in the middle of the poster.
	%\usepackage{ragged2e}
%To typeset units by closely following the “official” rules.
	%\usepackage[strict]{siunitx}
%Turns the doi provided by some bibliography styles into URLs.
	\usepackage{doi}
%Makes sure upper case greek letters are italic as well.
	\usepackage{fixmath}
%Provides \mathbb; obsoletes latexsym (see http://tug.ctan.org/macros/latex/base/latexsym.dtx). Relatedly, \usepackage{eucal} to change the mathcal font and \usepackage[mathscr]{eucal} (apparently equivalent to \usepackage[mathscr]{euscript}) to supplement \mathcal with \mathscr. This last option is not very useful as both fonts are similar, and the intent of the authors of eucal was to provide a replacement to mathcal (see doc euscript). Also provides \mathfrak for supplementary letters.
	\usepackage{amsfonts}
%Provides a beautiful (IMHO) \mathscr and really different than \mathcal, for supplementary uppercase letters. But there is no bold version. Alternative: mathrsfs (more slanted), but when used with tikzposter, it warns about size substitution, see https://tex.stackexchange.com/q/495167.
	\usepackage[scr]{rsfso}
%Multiple means to produce bold math: \mathbf, \boldmath (defined to be \mathversion{bold}, see fntguide), \pmb, \boldsymbol (all legacy, from LaTeX base and AMS), \bm (the most recommended one), \mathbold from package fixmath (I don’t see its advantage over \boldsymbol).
%“The \boldsymbol command is obtained preferably by using the bm package, which provides a newer, more powerful version than the one provided by the amsmath package. Generally speaking, it is ill-advised to apply \boldsymbol to more than one symbol at a time.” — AMS Short math guide. “If no bold font appears to be available for a particular symbol, \bm will use ‘poor man’s bold’” — bm. It is “best to load the package after any packages that define new symbol fonts” – bm. bm defines \boldsymbol as synonym to \bm. \boldmath accesses the correct font if it exists; it is used by \bm when appropriate. See https://tex.stackexchange.com/a/10643 and https://github.com/latex3/latex2e/issues/71 for some difficulties with \bm.
	\usepackage{bm}
	\nottoggle{LCpres}{
	%Provides \cref. Unfortunately, cref fails when the language is French and referring to a label whose name contains a colon (https://tex.stackexchange.com/q/83798). Use \cref{sec\string:intro} to work around this. cleveref should go “laster” than hyperref.
		\usepackage[capitalise]{cleveref}
	}{
	}
	\nottoggle{LCposter}{
	%Equations get numbers iff they are referenced. Loading order should be “amsmath → hyperref → cleveref → autonum”, according to autonum doc. Use this in preference to the showonlyrefs option from mathtools, see https://tex.stackexchange.com/q/459918 and autonum doc. See https://tex.stackexchange.com/a/285953 for the etex line. Incompatible with my version of tikzposter (produces “! Improper \prevdepth”). This removes the starred versions, such as equation*. Unfortunately, this prevents using \qedhere in an equation ending a proof, see https://tex.stackexchange.com/q/133358/.
		\expandafter\def\csname ver@etex.sty\endcsname{3000/12/31}\let\globcount\newcount
		\usepackage{autonum}
	}{
	}
%Also loaded by tikz.
	\usepackage{xcolor}
\iftoggle{LCpres}{
	\usepackage{tikz}
	%\usetikzlibrary{babel, matrix, fit, plotmarks, calc, trees, shapes.geometric, positioning, plothandlers, arrows, shapes.multipart}
}{
}
%Vizualization, on top of TikZ
	%\usepackage{pgfplots}
	%\pgfplotsset{compat=1.14}
\usepackage{graphicx}
	\graphicspath{{graphics/}}

%Provides \printlength{length}, useful for debugging.
	%\usepackage{printlen}
	%\uselengthunit{mm}

\iftoggle{LCpres}{
	\usepackage{appendixnumberbeamer}
	%I have yet to see anyone actually use these navigation symbols; let’s disable them
	\setbeamertemplate{navigation symbols}{} 
	\usepackage{preamble/beamerthemeParisFrance}
	\setcounter{tocdepth}{10}
}{
}

%Requires package xcolor.
\definecolor{ao(english)}{rgb}{0.0, 0.5, 0.0}
\NewDocumentCommand{\commentOC}{m}{\textcolor{blue}{\small$\big[$OC: #1$\big]$}}
%Requires package babel and option [french]. According to babel doc, need two braces around \selectlanguage to make the changes really local.
\NewDocumentCommand{\commentOCf}{m}{\textcolor{blue}{{\small\selectlanguage{french}$\big[$OC : #1$\big]$}}}
\NewDocumentCommand{\commentRS}{m}{\textcolor{red}{\small$\big[$RS: #1$\big]$}}
\NewDocumentCommand{\commentMN}{m}{\textcolor{ao(english)}{\small$\big[$MN: #1$\big]$}}

\bibliographystyle{abbrvnat}
\NewDocumentCommand{\possessivecite}{mO{}}{\citeauthor{#1}’s \citeyearpar[#2]{#1}}

%https://tex.stackexchange.com/a/467188, https://tex.stackexchange.com/a/36088 - uncomment if one of those symbols is used.
%\DeclareFontFamily{U} {MnSymbolD}{}
%\DeclareFontShape{U}{MnSymbolD}{m}{n}{
%  <-6> MnSymbolD5
%  <6-7> MnSymbolD6
%  <7-8> MnSymbolD7
%  <8-9> MnSymbolD8
%  <9-10> MnSymbolD9
%  <10-12> MnSymbolD10
%  <12-> MnSymbolD12}{}
%\DeclareFontShape{U}{MnSymbolD}{b}{n}{
%  <-6> MnSymbolD-Bold5
%  <6-7> MnSymbolD-Bold6
%  <7-8> MnSymbolD-Bold7
%  <8-9> MnSymbolD-Bold8
%  <9-10> MnSymbolD-Bold9
%  <10-12> MnSymbolD-Bold10
%  <12-> MnSymbolD-Bold12}{}
%\DeclareSymbolFont{MnSyD} {U} {MnSymbolD}{m}{n}
%\DeclareMathSymbol{\ntriplesim}{\mathrel}{MnSyD}{126}
%\DeclareMathSymbol{\nlessgtr}{\mathrel}{MnSyD}{192}
%\DeclareMathSymbol{\ngtrless}{\mathrel}{MnSyD}{193}
%\DeclareMathSymbol{\nlesseqgtr}{\mathrel}{MnSyD}{194}
%\DeclareMathSymbol{\ngtreqless}{\mathrel}{MnSyD}{195}
%\DeclareMathSymbol{\nlesseqgtrslant}{\mathrel}{MnSyD}{198}
%\DeclareMathSymbol{\ngtreqlessslant}{\mathrel}{MnSyD}{199}
%\DeclareMathSymbol{\npreccurlyeq}{\mathrel}{MnSyD}{228}
%\DeclareMathSymbol{\nsucccurlyeq}{\mathrel}{MnSyD}{229}
%\DeclareFontFamily{U} {MnSymbolA}{}
%\DeclareFontShape{U}{MnSymbolA}{m}{n}{
%  <-6> MnSymbolA5
%  <6-7> MnSymbolA6
%  <7-8> MnSymbolA7
%  <8-9> MnSymbolA8
%  <9-10> MnSymbolA9
%  <10-12> MnSymbolA10
%  <12-> MnSymbolA12}{}
%\DeclareFontShape{U}{MnSymbolA}{b}{n}{
%  <-6> MnSymbolA-Bold5
%  <6-7> MnSymbolA-Bold6
%  <7-8> MnSymbolA-Bold7
%  <8-9> MnSymbolA-Bold8
%  <9-10> MnSymbolA-Bold9
%  <10-12> MnSymbolA-Bold10
%  <12-> MnSymbolA-Bold12}{}
%\DeclareSymbolFont{MnSyA} {U} {MnSymbolA}{m}{n}
%%Rightwards wave arrow: ↝. Alternative: \rightsquigarrow from amssymb, but it’s uglier
%\DeclareMathSymbol{\rightlsquigarrow}{\mathrel}{MnSyA}{160}

%03B3 Greek Small Letter Gamma
\newunicodechar{γ}{\gamma}
%03B4 Greek Small Letter Delta
\newunicodechar{δ}{\delta}
%2115 Double-Struck Capital N
\newunicodechar{ℕ}{\mathbb{N}}
%211D Double-Struck Capital R
\newunicodechar{ℝ}{\mathbb{R}}
%21CF Rightwards Double Arrow with Stroke
\newunicodechar{⇏}{\nRightarrow}
%21D2 Rightwards Double Arrow
\newunicodechar{⇒}{\ensuremath{\Rightarrow}}
%21D4 Left Right Double Arrow
\newunicodechar{⇔}{\Leftrightarrow}
%21DD Rightwards Squiggle Arrow
\newunicodechar{⇝}{\rightsquigarrow}
%2205 Empty Set
\newunicodechar{∅}{\emptyset}
%2212 Minus Sign
\newunicodechar{−}{\ifmmode{-}\else\textminus\fi}
%2227 Logical And
\newunicodechar{∧}{\land}
%2228 Logical Or
\newunicodechar{∨}{\lor}
%2229 Intersection
\newunicodechar{∩}{\cap}
%222A Union
\newunicodechar{∪}{\cup}
%2260 Not Equal To (handy also as text in informal writing)
\newunicodechar{≠}{\ensuremath{\neq}}
%2264 Less-Than or Equal To
\newunicodechar{≤}{\leq}
%2265 Greater-Than or Equal To
\newunicodechar{≥}{\geq}
%2270 Neither Less-Than nor Equal To
\newunicodechar{≰}{\nleq}
%2271 Neither Greater-Than nor Equal To
\newunicodechar{≱}{\ngeq}
%2272 Less-Than or Equivalent To
\newunicodechar{≲}{\lesssim}
%2273 Greater-Than or Equivalent To
\newunicodechar{≳}{\gtrsim}
%2274 Neither Less-Than nor Equivalent To – also, from MnSymbol: \nprecsim, a more exact match to the Unicode symbol; and \npreccurlyeq, too small
\newunicodechar{≴}{\not\preccurlyeq}
%2275 Neither Greater-Than nor Equivalent To
\newunicodechar{≵}{\not\succcurlyeq}
%2279 Neither Greater-Than nor Less-Than – requires MnSymbol; also \nlessgtr from txfonts/pxfonts, \ngtreqless from MnSymbol (but much higher), \ngtrless from MnSymbol (a more exact match to the Unicode symbol); for incomparability (not matching this Unicode symbol), may also consider \ntriplesim from MnSymbol,\nparallelslant from fourier, \between from mathabx, or ⋈
\newunicodechar{≹}{\ngtreqlessslant}
%227A Precedes
\newunicodechar{≺}{\prec}
%227B Succeeds
\newunicodechar{≻}{\succ}
%227C Precedes or Equal To
\newunicodechar{≼}{\preccurlyeq}
%227D Succeeds or Equal To
\newunicodechar{≽}{\succcurlyeq}
%227E Precedes or Equivalent To
\newunicodechar{≾}{\precsim}
%227F Succeeds or Equivalent To
\newunicodechar{≿}{\succsim}
%2280 Does Not Precede
\newunicodechar{⊀}{\nprec}
%2281 Does Not Succeed
\newunicodechar{⊁}{\nsucc}
%2286
\newunicodechar{⊆}{\subseteq}
%22B2 Normal Subgroup Of – using \vartriangleleft from amsfonts, which goes well with \trianglelefteq, \ntriangleright, and so on, also from amsfonts; another possibility is \lhd from latexsym, which seems visually equivalent to \vartriangleleft from amsfonts; latexsym also has ⊴=\unlhd, but doesn’t have a symbol for ⊴. Other related symbols: \triangleleft from latesym package is too small; fdsymbol provides \triangleleft=\medtriangleleft and \vartriangleleft=\smalltriangleleft; MnSymbol provides \medtriangleleft and \vartriangleleft=\lessclosed=\lhd which are smaller than \vartriangleleft from amsfont; \vartriangleleft from mathabx (p. 67), looks different (wider); also \vartriangleleft from boisik (p. 69) looks still different; \vartriangleleft=\lhd from stix are smaller. Oddly enough, \triangleright appears as the LMMathItalic12-Regular font whereas \rhd appears as LASY10 and \vartriangleright appears as MSAM10.
\newunicodechar{⊲}{\vartriangleleft}
%22B3 Contains as Normal Subgroup (also: 25B7 White right-pointing triangle or 25B9 White right-pointing small triangle)
\newunicodechar{⊳}{\vartriangleright}
%22B4 Normal Subgroup of or Equal To
\newunicodechar{⊴}{\trianglelefteq}
%22B5 Contains as Normal Subgroup or Equal To
\newunicodechar{⊵}{\trianglerighteq}
%22C8 Bowtie
\newunicodechar{⋈}{\bowtie}
%22EA Not Normal Subgroup Of
\newunicodechar{⋪}{\ntriangleleft}
%22EB Does Not Contain As Normal Subgroup
\newunicodechar{⋫}{\ntriangleright}
%22EC Not Normal Subgroup of or Equal To
\newunicodechar{⋬}{\ntrianglelefteq}
%22ED Does Not Contain as Normal Subgroup or Equal
\newunicodechar{⋭}{\ntrianglerighteq}
%25A1 White Square
\newunicodechar{□}{\Box}
%27E6 Mathematical Left White Square Bracket – requires stmaryrd (alternative: \text{\textlbrackdbl}, but ugly if used in an italicized text such as a theorem)
\newunicodechar{⟦}{\llbracket}
%27E7 Mathematical Right White Square Bracket
\newunicodechar{⟧}{\rrbracket}
%27FC Long Rightwards Arrow from Bar
\newunicodechar{⟼}{\longmapsto}
%2AB0 Succeeds Above Single-Line Equals Sign
\newunicodechar{⪰}{\succeq}
%301A Left White Square Bracket
\newunicodechar{〚}{\textlbrackdbl}
%301B Right White Square Bracket
\newunicodechar{〛}{\textrbrackdbl}
%→ is defined by default as \textrightarrow, which is invalid in math mode. Same thing for the three other commands. Using \DeclareUnicodeCharacter instead of \newunicodechar because the latter warns about the previous definition.
%→ Rightwards Arrow
\DeclareUnicodeCharacter{2192}{\ifmmode\rightarrow\else\textrightarrow\fi}
%¬ Not Sign
\DeclareUnicodeCharacter{00AC}{\ifmmode\lnot\else\textlnot\fi}
%… Horizontal Ellipsis
\DeclareUnicodeCharacter{2026}{\ifmmode\dots\else\textellipsis\fi}
%× Multiplication Sign
\DeclareUnicodeCharacter{00D7}{\ifmmode\times\else\texttimes\fi}
%Permits to really obtain a straight quote when typing a straight quote; potentially dangerous, see https://tex.stackexchange.com/a/521999
\catcode`\'=\active
\DeclareUnicodeCharacter{0027}{\ifmmode^\prime\else\textquotesingle\fi}


\NewDocumentCommand{\R}{}{ℝ}
\NewDocumentCommand{\N}{}{ℕ}
%\mathscr is rounder than \mathcal.
\NewDocumentCommand{\powerset}{m}{\mathscr{P}(#1)}
%Powerset without zero.
\NewDocumentCommand{\powersetz}{m}{\mathscr{P}^*(#1)}
%https://tex.stackexchange.com/a/45732, works within both \set and \set*, same spacing than \mid (https://tex.stackexchange.com/a/52905).
\NewDocumentCommand{\suchthat}{}{\;\ifnum\currentgrouptype=16 \middle\fi|\;}
%Integer interval.
\NewDocumentCommand{\intvl}{m}{⟦#1⟧}
%Allows for \abs and \abs*, which resizes the delimiters.
\DeclarePairedDelimiter\abs{\lvert}{\rvert}
\DeclarePairedDelimiter\card{\lvert}{\rvert}
\DeclarePairedDelimiter\floor{\lfloor}{\rfloor}
\DeclarePairedDelimiter\ceil{\lceil}{\rceil}
%Perhaps should use U+2016 ‖ DOUBLE VERTICAL LINE here?
\DeclarePairedDelimiter\norm{\lVert}{\rVert}
%From mathtools. Better than using the package braket because braket introduces possibly undesirable space. Then: \begin{equation}\set*{x \in \R^2 \suchthat \norm{x}<5}\end{equation}.
\DeclarePairedDelimiter\set{\{}{\}}
\DeclareMathOperator*{\argmax}{arg\,max}
\DeclareMathOperator*{\argmin}{arg\,min}

%UTR #25: Unicode support for mathematics recommend to use the straight form of phi (by default, given by \phi) rather than the curly one (by default, given by \varphi), and thus use \phi for the mathematical symbol and not \varphi. I however prefer the curly form because the straight form is too easy to mix up with the symbol for empty set.
\let\phi\varphi

%The amssymb solution.
%\NewDocumentCommand{\restr}{mm}{{#1}_{\restriction #2}}
%Another acceptable solution.
%\NewDocumentCommand{\restr}{mm}{{#1|}_{#2}}
%https://tex.stackexchange.com/a/278631; drawback being that sometimes the text collides with the line below.
\NewDocumentCommand\restr{mm}{#1\raisebox{-.5ex}{$|$}_{#2}}


%Decision Theory
\NewDocumentCommand{\allalts}{}{\mathcal{A}}
\NewDocumentCommand{\allcrits}{}{\mathscr{C}}
\NewDocumentCommand{\alts}{}{A}
\NewDocumentCommand{\dm}{}{i}
\NewDocumentCommand{\allF}{}{\mathscr{F}}
\NewDocumentCommand{\allvoters}{}{\mathscr{N}}
\NewDocumentCommand{\voters}{}{N}
\NewDocumentCommand{\allprofs}{}{\linors^N}
\NewDocumentCommand{\prof}{}{\bm{P}}
\NewDocumentCommand{\lprof}{}{\lambda_{\bm{P}}}
\NewDocumentCommand{\linors}{}{\mathscr{L}(\allalts)}
%Thanks to https://tex.stackexchange.com/q/154549
	%\makeatletter
	%\def\@myRgood@#1#2{\mathrel{R^X_{#2}}}
	%\def\myRgood{\@ifnextchar_{\@myRgood@}{\mathrel{R^X}}}
	%\makeatother
\NewDocumentCommand{\pref}{}{\succ}
\NewDocumentCommand{\prefi}{O{i}}{\succ_{#1}}
\NewDocumentCommand{\prefiinv}{O{i}}{\succ_{#1}^{-1}}
\NewDocumentCommand{\ibar}{}{\overline{i}}

\NewDocumentCommand{\lvs}{}{\intvl{0, m - 1}^N}
\NewDocumentCommand{\losses}{}{\intvl{0, m - 1}}
\NewDocumentCommand{\PD}{}{\mathit{PD}(\prof)}
\NewDocumentCommand{\PE}{}{\mathit{PE}(\prof)}

%Rules
\NewDocumentCommand{\rhoP}{}{\rho_{\prof}}
\NewDocumentCommand{\minspread}{O{A}}{\min_{#1}(\sigma \circ \lambda_{\bm{P}})}
\NewDocumentCommand{\mindisp}{O{A}}{\min_{#1}(d \circ \lambda_{\bm{P}})}
\NewDocumentCommand{\FB}{}{\mathit{FB}}
\NewDocumentCommand{\VR}{}{\mathit{VR}}
\NewDocumentCommand{\SL}{}{\mathit{SL}}
\NewDocumentCommand{\PVv}{O{v}}{\mathit{PV}^{#1}}
\NewDocumentCommand{\PVef}{}{\mathit{PV}^{\floor{\frac{m - 1}{2}}}}%f for first
\NewDocumentCommand{\PVes}{}{\mathit{PV}^{\ceil{\frac{m - 1}{2}}}}%s for second
\NewDocumentCommand{\PVe}{}{\mathit{PV^=}}%egalitarian distribution

%Classes
\NewDocumentCommand{\PVcl}{}{\mathcal{PV}}
\NewDocumentCommand{\PVbcl}{}{\mathcal{PV}^b}
\NewDocumentCommand{\PVecl}{}{\mathcal{PV}^=}%egalitarian distribution
\NewDocumentCommand{\PEcl}{}{\mathcal{PE}}
\NewDocumentCommand{\FHcl}{}{\mathcal{FH}}
\NewDocumentCommand{\VCcl}{}{\mathcal{VC}}
\NewDocumentCommand{\VCecl}{}{\mathcal{VC^=}}
\NewDocumentCommand{\ELcl}{}{\mathcal{EL}}


\usepackage{bbm}

%I find these settings useful in draft mode. Should be removed for final versions.
	%Which line breaks are chosen: accept worse lines, therefore reducing risk of overfull lines. Default = 200.
	\tolerance=2000
	%Accept overfull hbox up to...
	\hfuzz=2cm
	%Reduces verbosity about the bad line breaks.
	\hbadness 5000
	%Reduces verbosity about the underful vboxes.
	\vbadness=1300

\title{Two principles for two-person social choice \thanks{Earlier versions of this paper have been presented at CRESE, Université de Franche-Comté and CMSS, the University of Auckland. We thank both institutions. We also thank Miguel Ballester, Sylvain Béal, Danilo Santa Cruz Coelho, John Hillas, Jean-François Laslier, Clemens Puppe, Arkadii Slinko and Yves Sprumont for useful exchanges we had.}}
\author[*]{Olivier Cailloux}
\author[**]{Matías Núñez}
\author[*]{M. Remzi Sanver}
\affil[*]{Université Paris-Dauphine, PSL Research University, CNRS, LAMSADE, 75016 PARIS, FRANCE}
\affil[**]{CREST, CNRS, École Polytechnique, GENES, ENSAE Paris, Institut Polytechnique de Paris, 91120 Palaiseau, France.}
\hypersetup{
	pdfsubject={},
	pdfkeywords={},
}

\begin{document}
\maketitle

\begin{abstract}
    We consider two-person ordinal collective choice from an axiomatic perspective. We identify two principles: minimal Rawlsianism (the chosen alternatives should be in the upper-half of both individuals’ preferences) and the equal loss principle (the chosen alternatives must ensure that both individuals concede “as equally as possible” from their highest ranked alternative). The equal loss principle has variants of different strength, depending on the precise definition of “as equally as possible”. We consider all prominent ordinal two-person social choice rules of the literature and explore which of these principles they satisfy. Moreover, we show that minimal Rawlsianism is logically incompatible with one version of the equal loss principle that we call the minimal dispersion principle. On the other hand, there are social choice rules that satisfy the Rawlsian minimal dispersion principle where the minimal dispersion principle is restricted to alternatives within the upper-half of both individuals’ preferences.
\end{abstract}

\section{Introduction}
\label{sec:intro}
Two-person discrete social choice models allow a specific interpretation of collective decision making: bargaining over a finite set of alternatives. Since the seminal model of \citet{Nash1950}, for a long time, bargaining problems were formulated by assuming a convex set of alternatives. However, there are many instances where bargaining takes place over a finite set of alternatives. Thus, this simplifying assumption of \citeauthor{Nash1950} excludes several real-life situations. 

\Citet{Mariotti1998} is among the first to relax this assumption by characterizing the Nash solution for a finite set of alternatives. His approach is followed by \citet{nagahisa2002axiomatization} who, again in a finite setting, characterize the solution of \citet{kalai1975other}. Both characterizations are built in a cardinal framework. 
 
An ordinal framework of two-person finite bargaining problems is presented by \citet{BramsKilgour2001} who introduce and analyze an ordinal solution, namely \textit{fallback bargaining}, that is based on compromising where each of the two bargainers begins by claiming the best outcome with respect to his ranking of alternatives. When the claims of the two bargainers differ, they continue by falling back, in lockstep, to lower ranked alternatives until a mutually (hence unanimously) agreed outcome is found. A characterization of fallback bargaining is provided by \citet{de2012reason}. As this solution is presented in a model that does not admit a disagreement point, fallback bargaining is rather an arbitration rule in the sense of \citet{Sprumont1993} than being a bargaining solution. An analysis of fallback bargaining in a model with a disagreement point is made by \citet{KibrisSertel2007} who rebaptize the solution as \textit{unanimity compromise} and define several variants of it.\footnote{They also observe that one of these variants, namely the imputational compromise, is the finite version of the equal length rule by \citet{thomson2019equal}. The imputational compromise is further studied by \citet{ConleyWilkie2012}.} In a recent paper, \citet{barbera2022compromising} use the term unanimity compromise in a framework with no disagreement point to refer to fallback bargaining in the sense of \citet{BramsKilgour2001}. We adopt their terminology, to avoid proposing a bargaining solution without a disagreement outcome.


 
As a matter of fact, the compromising approach that underlies unanimity compromise was originally used to design voting rules in settings with more than two individuals with the required support varying from unanimity to simple majority, such as the \textit{Kant-Rawls Social Compromise} by \citet{HurwiczSertel1997} and the \textit{majoritarian compromise} by \citet{sertel1999majoritarian}. It also paved the way to new axioms for social choice, such as \textit{majoritarian approval} and \textit{majoritarian optimality} as well as \textit{efficiency in the degree of compromise} by \citet{ozkal2004efficiency}.
\citet{merlin2019compromise} present a recent comprehensive analysis of voting rules and axioms based on this compromising idea.

A closer look at this compromising idea and in particular at unanimity compromise reveals a principle for two-person social choice. \Citet{Sprumont1993} qualifies arbitration rules that maximize the welfare of the least happy individual as being \textit{Rawlsian}. \citet{congar2012characterization} characterize the Rawlsian principle within the framework of social welfare functions. For social choice rules, \citet{BramsKilgour2001} establish the equivalence between the Rawlsian principle and unanimity compromise. Moreover, they show that every individual ranks a unanimity compromise outcome in the upper-half of his ranking. Thus, at every preference profile, there is an alternative that both individuals rank in the upper-half of their preference. In other words, the least happy individual of the society can always be granted a welfare within the first half of his preference. We qualify a two-person social choice rule that complies with this possibility as \textit{minimally Rawlsian}.%
\footnote{In a two-person collective choice framework with an interpretation that is more specific than ours, \citet{Clippel} call this condition the \textit{minimal satisfaction test}.}

\Citet{cailloux2022compromising} propose a different conception of compromising based on an \textit{equal loss principle} that favors outcomes where every individual concedes as equally as possible from his highest ranked alternative. They show that several two-person social choice rules fail this principle.

Although the minimal Rawlsian and equal loss principles cover many of the two-person social choice rules, the literature is missing an axiomatic analysis of these rules from this perspective, an observation which forms the subject matter of our paper. We consider the following rules, where $m$ is the number of alternatives.
\begin{itemize}
	\item Unanimity compromise, as defined by \citet{BramsKilgour2001} (under the name of fallback bargaining).
	\item The \textit{veto-rank rule} where, for $m$ odd, each individual vetoes $(m - 1) / 2$ alternatives and ranks the remaining $(m+1) / 2$. The outcome is the alternatives with the minimal sum of ranks among those that have not been vetoed.
	\item The \textit{shortlisting rule} where, for $m$ odd, one individual selects her best $(m+1) / 2$ alternatives and the rule picks the best alternative of the other individual out of that shortlist.
	\item The class of \textit{Pareto-and-veto rules} where each individual $i$ vetoes a fixed number $v_i$ of alternatives with $v_1 + v_2$ being lower than $m$. The outcome is the set of Pareto optimal alternatives that are not vetoed. 
\end{itemize}

The veto-rank rule and the shortlisting rule are used for the selection of arbitrators and their strategic aspects are comprehensively analyzed by \citet{Clippel}% 
\footnote{\Citet{Clippel} present the shortlisting mechanism, whose equilibrium outcome corresponds to what we call the shortlisting rule.}. Our class of Pareto-and-veto rules generalizes the Pareto-and-veto rules analyzed by \citet{laslier2021solution} which impose $v_1$ + $v_2$ = $m-1$. These rules we consider cover most of the ordinal two-person social choice rules in the literature. The literature also admits various interesting real-life procedures expressed as extensive form games, such as those in \citet{anbarci1993noncooperative, anbarci2006finite} and \citet{barbera2022compromising}. However, as shown in these papers, the subgame perfect equilibrium outcomes of these games are always among the unanimity compromise alternatives. 

We now summarize our findings. All rules we consider are Paretian. Unanimity compromise, the veto-rank rule and the shortlisting rule are minimally Rawlsian. The class of minimally Rawlsian Pareto-and-veto rules is restricted to those that gives the highest (almost) equal veto power to both individuals. Moreover, when the veto power is equal, the corresponding Pareto-and-veto rule is a super correspondence of every Paretian and minimally Rawlsian social choice rule. Thus, unanimity compromise, the veto-rank rule and the shortlisting rule are all sub correspondences of the Pareto-and-veto rule with the highest equal veto power.

The equal loss principle we consider favors outcomes that have the same rank for both individuals. Without imposing Pareto optimality separately, this principle may lead to Pareto dominated outcomes. Thus, we consider a Paretian version that favors, among the Pareto optimal outcomes, the one that has the same rank for both individuals. Note that such an alternative, if it exists, will be unique. We define two versions of the Paretian equal loss principle, one being stronger than the other. The stronger version requires that the Pareto optimal alternative that has the same rank for both individuals must be uniquely chosen. Under the weaker version it suffices that this alternative be among the outcomes. The veto-rank rule and the shortlisting rule both fail the weak (hence strong) version of the Paretian equal loss principle. While Pareto-and-veto rules that endow individuals with a veto power that does not exceed $\floor{\frac{m}{2}}$ satisfy the weak Paretian equal loss principle, all of them fail the strong version. On the other hand, unanimity compromise satisfies the strong Paretian equal loss principle, thus showing that this principle is compatible with being minimally Rawlsian.
 
Within the spirit of equal loss, we propose the \textit{minimal dispersion principle} as another strengthening of the (weak) Paretian equal loss principle. The \textit{dispersion} of an alternative is the difference between the two ranks at which is it placed at the preferences of the two individuals. The minimal dispersion principle requires that an alternative whose dispersion is minimal must be among the outcomes.%
\footnote{
\label{ft:equalarea}
In a framework with a disagreement outcome, \citet{KibrisSertel2007} argue that the finite version of the \textit{equal area rule} \citep{thomson1994cooperative} minimizes the difference between losses with respect to individually rational alternatives. They also show that this rule differs from their unanimity compromise. In our framework without a disagreement outcome, the equal area rule is equivalent to choose the Pareto optimal alternatives that minimize the difference between losses.}
Not only unanimity compromise fails this principle, but the minimal dispersion principle turns out to be logically incompatible with the minimal Rawlsian principle. As a result, among the social choice rules we consider, the only candidates to satisfy the minimal dispersion principle are the Pareto-and-veto rules that fail the minimal Rawlsian principle. As a matter of fact, those that endow individuals with a veto power that does not exceed a third of the available alternatives turn out to satisfy the minimal dispersion principle. Recall that this upper bound is a half for the satisfaction of the (weaker) Paretian equal loss principle.

Given the incompatibility between the two principles, we introduce the Rawlsian minimal dispersion principle that requires the outcome to contain the Rawlsian alternatives whose loss vectors have minimal dispersion. It turns out that, except the Pareto-and-veto rule that gives the highest equal veto power to both individuals, it is failed by all social choice rules we consider.

\Cref{sec:basic} introduces the basic notions and notation. \Cref{sec:minprinc,sec:eqprinc} are devoted to the minimal Rawlsian and equal loss principles, respectively. \Cref{sec:reconc} introduces the Rawlsian minimal dispersion principle. \Cref{sec:concl} makes some concluding remarks.

\section{Basic notions and notation}
\label{sec:basic}
Let $N = \set{1, 2}$ be a set of two individuals and $\allalts$ be a set of alternatives, with $\card{\allalts} = m\geq 2$. 
Given $i \in N$, let $\ibar \in N \setminus \set{i}$ denote the other individual. Let $\powersetz{\allalts}$ denote the set of non-empty subsets of $\allalts$. Let $\linors$ be the set of linear orders over $\allalts$. We let ${\prefi} \in \linors$ stand for the preference of individual $i \in N$ and $\prof = ({\prefi[1]}, {\prefi[2]}) \in \allprofs$ for a preference profile. 
A social choice rule (SCR) is a function $f: \allprofs → \powersetz{\allalts}$.
Given two SCRs $f$, $f'$, we write $f \subset f'$ to indicate that $f$ is a proper subcorrespondence of $f'$.

A SCR $f$ is anonymous iff $f({\prefi[1]}, {\prefi[2]}) = f({\prefi[2]}, {\prefi[1]})$ for all $({\prefi[1]}, {\prefi[2]}) \in \allprofs$.
A SCR $f$ is neutral iff for all permutations $\sigma$ over $\allalts$ and profile $\prof \in \allprofs$, $\sigma \circ f(\prof) = f(\sigma \circ \prof)$.

Given $p, q \in \R$, let $\intvl{p, q} = [p, q] \cap \N $ denote the interval of integer numbers between $p$ and $q$. The loss of individual $i$ at preference profile $\prof$ for alternative $x$ is defined as the number of alternatives that the individual prefers to $x$: $\lprof(x)_i = \card{\set{y \in \allalts \suchthat y \prefi x}}$.
The loss vector of $x$ at $\prof$, $\lprof(x)=(\lprof(x)_1, \lprof(x)_2) \in \intvl{0, m - 1}^N$, assigns to each individual her loss associated to $x$.

Given two loss vectors $l, l' \in \lvs$, we say that $l$ is weakly smaller than $l'$, $l ≤ l'$, iff $l_i ≤ l'_i$ $\forall i$. We also write $l < l'$ when $l$ is strictly smaller than $l'$, meaning, weakly smaller and different. 
Let $\min \lprof(x) = \min_{i \in N} \lprof(x)_i \in \N$ 
and $\sum \lprof(x) = \sum_{i \in N} \lprof(x)_i$ denote, respectively, the minimal loss and the sum of the losses associated to $\lprof(x)$.

Let $\PE = \set{x \in \allalts \suchthat \nexists y \text{ s.t. } \lprof(y) < \lprof(x)}$ be the set of Pareto efficient alternatives at $\prof$.
We say that $f$ satisfies the Pareto property iff it picks only Pareto efficient alternatives.
Let $\PEcl = \set{f: \allprofs → \powersetz{\allalts} \suchthat \forall \prof: f(\prof) \subseteq \PE}$ denote the class of SCRs satisfying the Pareto property. Similarly, we systematically use calligraphic letters to denote the class of rules satisfying a given property.

In concordance with the ceiling established by Theorem 1 of \citet{BramsKilgour2001}, we use the term “half” to mean the smallest integer $k$ that exceeds $m-k$ and for every profile $\prof \in \allprofs$, we let $H(\prof) = \set{x \in \allalts \suchthat \lprof(x) ≤ (\khalf, \khalf)}$ denote the set of alternatives reaching the best half of every individual’s preference, and $H^i(\prof) = \set{x \in \allalts \suchthat \lprof(x)_i ≤ \khalf}$ those that reach the best half of $i$’s preference. Given $\prof$ and a loss level $k \in \losses$, define $U(\prof, k) = \set{x \in \allalts \suchthat \lprof(x) ≤ (k, k)}$ as the alternatives imposing losses not higher than $k$ for all individuals. 
We say that such alternatives receive unanimous support at level $k$. Let $\rhoP = \min \set{k \in \losses \suchthat U(\prof, k) ≠ \emptyset}$ be the least loss level at which some alternative receives unanimous support.

We now define the SCRs of the literature that we analyze in the paper.
\commentOC{I hope you will not hate me for this late remark, but it suddenly occurs to me that it’s very odd (pun not intended) that we use $\khalf$ throughout instead of the apparently simpler and equivalent $\floor{\frac{m}{2}}$, which has the supplementary advantage of being more obviously a sort of “half”; and whose natural complement to $m$ is the conveniently dual $\ceil{\frac{m}{2}}$ instead of the ugly $\floor{\frac{m - 1}{2}} + 1$. 
If you agree, I’ll take care of trying to change them throughout and check that it does not hurt readability (I expect that it won’t and that it will sometimes help significantly).}\commentMN{Nice remark}

\textbf{Unanimity compromise} is the SCR $\FB$ that picks all alternatives with unanimous support at $\rhoP$: $\FB(\prof) = U(\prof, \rhoP)$. 

The \textbf{veto-rank rule} $\VR$ is defined as follows. Each individual vetoes her worst $\floor{\frac{m - 1}{2}}$ alternatives, then the Borda winners among the non vetoed alternatives are picked: $\VR(\prof) = \argmin_{H(\prof)} \sum \lprof = \set{x \in H(\prof) \suchthat \forall y \in H(\prof): \sum \lprof(x) ≤ \sum \lprof(y)}$.

The \textbf{shortlisting rule} $\SL(\prof) = \cup_{i \in N} (\argmin_{x \in H^i(\prof)} \lprof(x)_{\ibar})$ picks the best alternative of individual $1$ that is not among the worst $\floor{\frac{m - 1}{2}}$ alternatives of individual $2$, and the best alternative of $2$ that is not among the worst $\floor{\frac{m - 1}{2}}$ alternatives of $1$.

Both $\VR$ and $\SL$ are defined in \citet{Clippel} for $m$ odd only.

The class of \textbf{Pareto-and-veto rules}, $\PVcl$, contains rules parametrized by $v_1, v_2 \in \intvl{0, m - 1}$ with $v_1 + v_2 ≤ m - 1$ where $v_i$ represents the number of alternatives vetoed by individual $i \in N$ (individuals veto the alternatives at the bottom of their preference).
Given $v_i \in \intvl{0, m - 1}$, define $a_i = m - v_i - 1 \in \intvl{0, m - 1}$ as the highest acceptable loss level for individual $i$. For $v=(v_1,v_2)$, the rule $\PVv(\prof) = \cap_{i \in N}\set{x \in \allalts \suchthat \lprof(x)_i ≤ a_i} \cap \PE$ picks all Pareto-efficient alternatives that no individual vetoes. 
The class $\PVcl = \set{\PVv \suchthat v_1, v_2 \in \intvl{0, m - 1} \text{ with } v_1 + v_2 \leq m - 1}$ is the set of those rules, and the class $\PVbcl = \set{\PVv \suchthat v_1, v_2 \in \intvl{0, m - 1} \text{ with } v_1 + v_2 = m - 1} \subseteq \PVcl$ is the set of rules where the inequality is binding.

\begin{remark}
	All $\PVv$ rules differ. Formally: $\forall v ≠ v', \PVv ≠ \PVv[v']$. This is readily visible by considering a profile $\prof$ composed of a preference ordering $(a_1, …, a_m)$ and the inverse preference ordering $(a_m, …, a_1)$: $\set{a_{v_2 + 1}, …, a_{m - v_1}} = \PVv(\prof) ≠ \PVv[v'](\prof) = \set{a_{v'_2 + 1}, …, a_{m - v'_1}}$.
\end{remark}

\begin{remark}
	$\FB$, $\VR$, $\SL$ and all SCRs in $\PVcl$ are  neutral.
\end{remark}
\begin{remark}
	$\FB$, $\VR$ and $\SL$ are anonymous, while $\PVv \in \PVcl$ is anonymous iff $v_1 = v_2$.
\end{remark}

\section{The minimal Rawlsian principle}
\label{sec:minprinc}
\begin{definition}[$k$-Rawlsianism] 
	\label{def:kr}
	Given $k \in \intvl{0, m - 1}$, a SCR is $k$-Rawlsian iff it favors alternatives whose losses are within $\intvl{0, k}$ for both individuals. Formally:
	$\forall \prof \in \allprofs,  x, y \in \allalts, \lprof(x) \in \intvl{0, k}^2 \land \lprof(y) \notin \intvl{0, k}^2 ⇒ y \notin f(\prof)$.
\end{definition}

$k$-Rawlsianisn is binding at exactly those profiles where some alternative has its losses in $\intvl{0, k}^2$, and is void at the other profiles. It is thus natural to wonder which values of $k$ make the constraint binding at every profile, so as to guarantee a minimal possible loss to every individual  whatever the profile.
It follows from \citet[Theorem 1]{BramsKilgour2001} that $k$-Rawlsianism is binding at every profile if and only if $k$ is at least “half”.
\begin{theorem}
	$\forall \prof: \exists a \in \allalts \suchthat \lprof(a) \in \intvl{0, k}^2 ⇔ k ≥ \khalf$.
\end{theorem}



\begin{remark}
	A variant of $k$-Rawlsianism that may come to mind is to mandate that the rule be $k$-strict-Rawlsian iff it selects its winners among those alternatives whose losses are within $\intvl{0, k}$ for both individuals.%
	\footnote{We actually thought about that version first. We thank Miguel Ballester for the improved version.}
	Formally:
	$\forall \prof \in \allprofs,  f(\prof) \subseteq \lprofinv(\intvl{0, k} × \intvl{0, k})$.
	By the reasoning above, $\khalf$-Rawlsianism is equivalent to $\khalf$-strict-Rawlsianism, whereas for $k < \khalf$, no rule satisfy $k$-strict-Rawlsianism.
\end{remark}
\commentRS{I propose to make this remark and the thanks to Ballester when we define SRMD.} 
\commentMN{ I'd rather leave it here. }
\commentRS{Let's first agree on the content of the remark. What do we wish to say here? To my understanding, to point of Miguel is the following: using the "natural" half instead of our "artificial" one doesn't matter under SRMD (while it matters in other instances). Am I right, Olivier?} \commentMN{I agree with Remzi's interpretation, even though I did not talk to Miguel}


It follows that the strongest version of $k$-Rawlsianism that is binding at every profile is when $k$ equals $\khalf$. On the other hand, this choice of $k$ reflects a general bound that makes $k$-Rawlsianism systematically constraining while there are several profiles where the minimal loss is lower than $k$. Thus, we qualify $\khalf$-Rawlsianism as “minimal Rawlsianism”, which we formally define as follows.
\begin{definition}[Minimal Rawlsianism (\MRprop)] 
	A SCR $f$ satisfies \MRprop{} iff 
	$\forall \prof \in \allprofs,  f(\prof) \subseteq H(\prof)$.
\end{definition}



\begin{theorem}
	\label{th:inFH}
	$\FB, \VR, \SL \in \PEcl \cap \MRcl$. 
\end{theorem}
 \begin{proof}
	For $\VR$ and $\SL$, the proof follows from the definition of the SCRs since each of them selects only alternatives among the top-half alternatives of both individuals. For $\FB$, Theorem 1 of \citet{BramsKilgour2001} shows that $\forall \prof \in \allprofs$, $\rhoP ≤ \khalf$.
\end{proof} 
   
We now discuss the relationship of the class $\PVcl$ to the \MRprop{} property. To this end, we define $\PVe = \PVv[\left(\floor{\frac{m - 1}{2}}, \floor{\frac{m - 1}{2}}\right)]$ as the Pareto-and-veto rule that gives the highest equal veto power to both individuals. 
Thus, under $PV^=$, we have $v_1=v_2= \floor{\frac{m - 1}{2}}$, implying $v_1=v_2=\frac{m-1}{2}$ when $m$ is odd and $v_1=v_2= \frac{m}{2}-1$ when $m$ is even. 
Note that $PV^=\in\PVbcl$ iff $m$ is odd.

The following result determines which rules in $\PVcl$ satisfy MR.

\begin{theorem}
	\label{th:pvmr}
	$\PVcl ∩ \MRcl = \set{\PVv \in \PVcl \suchthat \forall i: v_i ≥ \floor{\frac{m - 1}{2}}}$.
\end{theorem}
 \begin{proof}
	To show the “if” part, note that given any profile, the condition $\forall i: v_i ≥ \floor{\frac{m - 1}{2}}$ suffices to guarantee that $\PVv(\prof) \subseteq H(P)$. To see the “only if” part, consider an arbitrary ordering $\prefi$ over $\allalts$, let $\prefiinv$ denote its inverse, and consider the profile $\prof = (\prefi, \prefiinv)$.
	Observe that $\PVv(\prof)$ will exclusively pick winners in the first half of individual $i$ only if $v_i ≥ \floor{\frac{m - 1}{2}}$.
\end{proof} 

Note that 
$[\forall i: v_i ≥ \floor{\frac{m - 1}{2}} \land \sum_i v_i ≤ m - 1]$ 
is equivalent to 
$[\exists i \suchthat v_i = \floor{\frac{m - 1}{2}} \land \floor{\frac{m - 1}{2}} ≤ v_{\ibar} ≤ \khalf]$. 
Thus, \cref{th:pvmr} implies that among the class $\PVcl$, when $m$ is odd, only the rule $PV^=$ satisfies MR, and when $m$ is even, only the three rules $\set{\PVe, \PVv[(\frac{m}{2}, \frac{m}{2} - 1)], \PVv[(\frac{m}{2} - 1, \frac{m}{2})]}$ satisfy it.

We now establish the relationship of  $\VR, \SL, \FB$ to the class $\PVcl$. Considering two SCRs $f$ and $f'$, let $f \cup f'$ denote the rule $(f \cup f')(\prof) = f(\prof) \cup f'(\prof)$. 
Given any non empty class of SCRs $F$, let $\bigcup F$ denote the maximal (least resolute) SCR that can be formed by unions of rules of $F$.

\begin{theorem}
	\label{th:equal}
	$\bigcup(\PEcl \cap \MRcl) = \PVe$.
\end{theorem}
\begin{proof}
	Note that $\bigcup(\PEcl \cap \MRcl)$ is, by definition, the SCR that, for each profile, picks all Pareto alternatives that are in the first half of both individuals’ preferences and only those alternatives. 
	We thus have to show that $\forall \prof: \PVe(\prof) = \PE \cap H(\prof)$. By definition of $\PVe$, it suffices to prove that $\cap_i\set{x \in \allalts \suchthat \lprof(x)_i ≤ m - \floor{\frac{m - 1}{2}} - 1} = H(\prof)$. This in turn follows from the definition of $H(\prof)$.
\end{proof}

The observation below follows from \cref{th:equal}.
\begin{corollary}\label{th:subPVe}
	A SCR $f \in \PEcl \cap \MRcl$ if and only if $f \subseteq \PVe$.
\end{corollary}

We now establish the relationship between $\FB$, $\VR$ and $\SL$ and show that they can pick disjoint winners.
\begin{theorem}\label{th:different}
There is some $\prof$ such that $\FB(\prof) \cap \VR(\prof) = \emptyset, \FB(\prof) \cap \SL(\prof) = \emptyset$ and $\VR(\prof) \cap \SL(\prof) = \emptyset$. Furthermore, $\PVe(\prof) ≠ \FB(\prof)$, $\PVe(\prof) ≠ \VR(\prof)$ and $\PVe(\prof) ≠ \SL(\prof)$.
\end{theorem}
\begin{proof}
	Consider the following profile $\prof$:
	\begin{equation}
		\label{eq:distinct}
		\begin{array}{llll lll | llll ll}
			a&b&c&d&e&f&g&h&i&j&k&l&m\\
			g&h&i&d&b&j&a&c&e&f&k&l&m\\
		\end{array},
	\end{equation}
	where the first individual prefers $a$ to $b$, $b$ to $c$, etc., and the second individual prefers $g$ to $h$, $h$ to $i$, etc. 
	The bar shows the “half” position.
	The theorem is proven by noting that $\FB(\prof) = \set{d}$, $\VR(\prof) = \set{b}$, $\SL(\prof) = \set{a, g}$ and $\PVe(\prof) = \set{a, b, d, g}$.
\end{proof}

\Cref{th:inFH}, \cref{th:subPVe} and \cref{th:different} lead to the corollary below.
\begin{corollary}
   	$\FB, \VR, \SL \subset \PVe$.
\end{corollary}

\begin{remark}
	The rule $\PVe$ is not merely the union of $\FB$, $\VR$ and $\SL$, as the following profile illustrates.
	\begin{equation}
		\begin{array}{lllll|llll}
			a&b&c&d&e&f&g&h&i\\
			e&c&d&b&a&f&g&h&i\\
		\end{array}.
	\end{equation}
	Here, $\FB(\prof) = \set{c}$, $\VR(\prof) = \set{c}$, $\SL(\prof) = \set{a, e}$ but $\PVe(\prof) = \set{a, b, c, e}$. In fact, no variant of $\VR$ (changing the scoring vector) would elect $b$: its losses are $(1, 3)$ whereas $c$ has losses $(2, 1)$.
\end{remark}

\section{The equal loss principle}
\label{sec:eqprinc}
Given $\prof \in \allprofs$, define $S(\prof) = \set{x \in \allalts \suchthat \lprof(x)_1 = \lprof(x)_2}$ as the set of alternatives that are ranked at the same position by both individuals.

\begin{definition}[Equal loss (EL)]
	$\forall \prof \in \allprofs: [S(\prof) ≠ \emptyset] ⇒ f(\prof) \cap S(\prof) ≠ \emptyset$.
\end{definition}

The equal loss principle is incompatible with the one of Pareto efficiency as formally stated by the next proposition. 
\begin{theorem}
	$\forall m ≥ 3: \PEcl ∩ \ELcl = \emptyset$.
\end{theorem}

\begin{proof}
Consider the following profile:
	\begin{equation}
		\begin{array}{lll ll}
			a_1&a_2 &a_3 & \ldots &a_m \\
			a_2& a_1 &a_3&\ldots&a_m \\
		\end{array}.
	\end{equation}
  Observe that $a_1$ and $a_2$ are the only Pareto efficient alternatives.
  Thus, all the alternatives that are ranked at the same position by both individuals, namely $\allalts \setminus \set{a_1, a_2}$, are Pareto dominated.
\end{proof}
Thus, Pareto efficiency is a fortiori incompatible with the following stronger version of the equal loss principle.

\begin{definition}[Strong equal loss]
	$\forall \prof \in \allprofs: [S(\prof) ≠ \emptyset] ⇒ f(\prof) \subseteq S(\prof)$.
\end{definition}

We now embed Pareto efficiency into the equal loss requirement by mandating $f$ to pick the (unique) Pareto efficient alternative that is ranked at the same position by both individuals, if there is one.

\begin{definition}[Paretian equal loss (PEL)]
	$\forall \prof \in \allprofs: [S(\prof) \cap \PE ≠ \emptyset] ⇒ f(\prof) \cap S(\prof) \cap \PE ≠ \emptyset$.
\end{definition}

\begin{theorem}
	$\VR$ and $\SL$ fail PEL.
\end{theorem}
\begin{proof}
	Consider the profile $\prof$ stated in \eqref{eq:distinct}, where $\VR(\prof) = \set{b}$ and $\SL(\prof) = \set{a, g}$:
	\begin{equation}
		\begin{array}{llll lll | llll ll}
			a&b&c&d&e&f&g&h&i&j&k&l&m\\
			g&h&i&d&b&j&a&c&e&f&k&l&m\\
		\end{array}.
	\end{equation}
	PEL requires to choose $d$.
\end{proof}

The following result will be useful throughout.
\begin{theorem}
	\label{th:sumlosses} 
	For any $\prof$, if $x \in \PE$ then $\lprof(x)_1+\lprof(x)_2 \leq m-1$.
\end{theorem}
\begin{proof}
Since $x \in \PE$, $\{y\in \mathcal{A}\mid y \succ_1 x\}\cap \{y\in \mathcal{A}\mid y \succ_2 x\} =  \emptyset$, which implies $\#\{y\in \mathcal{A}\mid y \succ_1 x\}+\#\{y\in \mathcal{A}\mid y \succ_2 x\} \leq m - 1$.
\end{proof}

Some rules from the class $\PVcl$ satisfy PEL, $\PVe$ being among those.
\begin{theorem}
	\label{th:pel}	
	For any $m ≥ 3$, a rule $\PVv[(v_1, v_2)] \in \PVcl$ satisfies PEL iff its veto levels are both at most $\floor{\frac{m}{2}}$, thus, iff $\max_{i \in \set{1, 2}} v_i ≤ \frac{m}{2}$.
\end{theorem}
\begin{proof}
	For all $\prof \in \allprofs$, if $x \in S(\prof) \cap \PE$, then $x$ is not among the last $\floor{\frac{m}{2}}$ ranks, as follows from \cref{th:sumlosses}.
	A PV rule with veto parameters at most $\floor{\frac{m}{2}}$ will thus pick all such alternatives $S(\prof) \cap \PE$, as required by PEL.
	
	For the other direction, observe that there exists $\prof \in \allprofs$ such that for some $x \in \allalts$, $x \in S(\prof) \cap \PE$ and $x$ is positioned just better than the last $\floor{\frac{m}{2}}$ ranks (thus $\exists \prof \in \allprofs, x \in \allalts \suchthat \forall i \in \set{1, 2}: \lprof(x)_i = \floor{\frac{m - 1}{2}}$, leaving $\khalf = \floor{\frac{m}{2}}$ positions behind $x$).
	A PV rule such that $\max_{i \in \set{1, 2}} v_i > \floor{\frac{m}{2}}$ will thus not include $x$ in the set of winners, hence, the rule will fail PEL.
\end{proof}

\Cref{th:pel} leads to the following corollary for the binding Pareto-and-veto rules.

\begin{corollary}
	\label{th:pvbpel}
	For any $m ≥ 3$, a rule $\PVv[(v_1, v_2)] \in \PVbcl$ satisfies PEL iff $v_i = \floor{\frac{m - 1}{2}}$ and $v_{\ibar} = \khalf$ for any $\set{i, \ibar} = \set{1, 2}$.
\end{corollary}

The rules that fail PEL will a fortiori fail the following stronger version of the Paretian equal loss property which requires that the Pareto efficient alternative ranked at the same position by both individuals, if it exists, be the unique outcome.

\begin{definition}[Strong Paretian equal loss (SPEL)]
	$\forall \prof \in \allprofs: [S(\prof) \cap \PE ≠ \emptyset] ⇒ f(\prof) = S(\prof) \cap \PE$.
\end{definition}
Thus, $\VR$, $\SL$ and those rules in $\PVcl$ that fail PEL all fail SPEL. Furthermore, as we state and show below, even the rules in $\PVcl$ that satisfy PEL fail SPEL.
\begin{theorem}
	\label{th:PVnotSPEL}
	When $m ≥ 4$, all rules in $\PVcl$ fail SPEL.
\end{theorem}
\begin{proof}
	Let $\allalts = \set{a, b, c, a_4, a_5, …}$.
	Consider the following profile $\prof$: 
	\begin{equation}
		\begin{array}{*6{l}}
			a&b&c&a_4&a_5&…\\
			c&b&a&a_4&a_5&…\\
		\end{array}.
	\end{equation}
	SPEL requires to pick solely $b$.
	To have $f \in \PVcl$ and $f(\prof) = \set{b}$ requires that $v_1 ≥ m - 2$ (for $c \notin f(\prof)$) and $v_2 ≥ m - 2$ (for $a \notin f(\prof)$), which implies $2m - 4 ≤ \sum v_i$. Also, the definition of $\PVcl$ requires that $\sum v_i ≤ m - 1$. And $2m - 4 ≤ m - 1$ is satisfied only when $m ≤ 3$.
\end{proof}

\begin{remark}
When $m = 3$, $\PVv[(1, 1)]$ satisfies SPEL.
\end{remark}

\begin{theorem}
	$\FB$ satisfies SPEL.
\end{theorem}
\begin{proof}
	Pick any profile $\prof$ and any $x \in PE(\prof)$ with $\lprof(x) = (k, k)$ for some $k \in \intvl{0, m - 1}$. Let us show that $\FB(\prof) = \set{x}$. Consider any $y \in \FB(\prof)$. As $\FB$ minimizes the maximal loss, $\max \lprof(y) ≤ \max \lprof(x) = k$. Since $x \in PE(\prof)$, we have $\max \lprof(y) ≥ k$ (otherwise $\min \lprof(y) ≤ \max \lprof(y) < k$ and $y$ Pareto-dominates $x$). Hence, $\max \lprof(y) = k$, thus $x = y$.
\end{proof}

SPEL, failed by all rules we consider but one, allows to distinguish $\FB$ from the rest. Moreover, by satisfying both conditions, $\FB$ establishes the compatibility between SPEL and MR. However, as discussed below, this compatibility vanishes when another stronger version of PEL is adopted. 

Call the dispersion of a loss vector $l$ at $\prof$ the value $d(l) = \abs{l_1 - l_2}$. 
Thus, $(d \circ \lprof)(x) = \max\lprof(x) - \min\lprof(x)$.
We show in \cref{sec:spreads} that $d$ coincides with several commonly used spread measures, 

Given a profile $\prof \in \allprofs$, define $\min_{\PE} (d \circ \lprof)$ as the minimal dispersion obtained by loss vectors of Paretian alternatives in that profile, and $\argmin_{\PE} (d \circ \lprof)$ as the Paretian alternatives whose loss vectors have minimal dispersion among Paretian alternatives.

Define the minimal dispersion (MD) condition as follows.

\begin{definition}[Minimal dispersion]
	$f(\prof) \cap \argmin_{\PE} (d \circ \lprof) ≠ \emptyset$, $\forall \prof \in \allprofs$.
\end{definition}
MD requires the outcome to contain some Paretian alternatives whose loss vectors have minimal dispersion. As such, MD is another strengthening of PEL while it is logically independent of SPEL. Nevertheless, although there are rules that satisfy both SPEL and MR, MD turns out to be logically incompatible with MR. 
The next theorem will permit to establish the relationship of MD to the rules we consider.

\begin{theorem}
	\label{th:caractEmpty}
	Given $m ≥ 3$, $\forall t \in \intvl{0, m - 1}$, the following statements are logically equivalent: 
	\begin{enumerate}
		\item \label{it:tbound} $t ≥ \frac{2m - 4}{3}$
		\item \label{it:Pt} $\forall \prof: \lprofinv(\intvl{0, t} × \intvl{0, t}) \cap \argmin_{\PE} (d \circ \lprof) ≠ \emptyset$
		\item \label{it:Ptbig2} $\forall \prof: \lprofinv(\intvl{0, t} × \intvl{0, m - 1}) \cap \argmin_{\PE} (d \circ \lprof) ≠ \emptyset$
		\item \label{it:Ptbig1} $\forall \prof: \lprofinv(\intvl{0, m - 1} × \intvl{0, t}) \cap \argmin_{\PE} (d \circ \lprof) ≠ \emptyset$.
	\end{enumerate}
\end{theorem}

\begin{proof}
	For the claim $\ref{it:tbound} ⇒ 2$, let us consider any $t ≥ \frac{2m - 4}{3}$ and any profile $\prof$, and let us show that some alternative lies in $\lprofinv(\intvl{0, t} × \intvl{0, t}) \cap \argmin_{\PE} (d \circ \lprof)$.
	
	Consider any $x \in \argmin_{\PE} (d \circ \lprof)$, and let $i$ denote any individual such that $\max \lprof(x) = \lprof(x)_i$. 
	Observe that if $\lprof(x)_i ≤ t$, then also $\lprof(x)_{\ibar} ≤ t$ and thus $x \in \lprofinv(\intvl{0, t} × \intvl{0, t}) \cap \argmin_{\PE} (d \circ \lprof)$, and the proof is done. Thus, assume that $\lprof(x)_i > t$.
%	The alternative $x$ thus has $k$ alternatives better than it for the individual $i$, and therefore must have at least those $k$ alternatives worst than it for the other individual (otherwise it is Pareto-dominated by at least one of those $k$ alternatives). Thus $\lprof(x)_{\ibar} ≤ m - 1 - k$. In other words, the losses of $x$ are $k$ and at most $m - 1 - k$, with $k$ the highest of these two losses. 
	
%	If [equivalently, ], then 
	Let $A = \set{a \suchthat \lprof(a)_i ≤ t}$ designate the $t + 1$ top alternatives for $i$. 
	Define $y = \argmin_A \lprof(.)_{\ibar}$ as the best alternative for individual $\ibar$ among $A$. 
%	By construction thus, $\forall a ≠ y \in A: \lprof(y)_{\ibar} < \lprof(a)_{\ibar}$, hence, $\lprof(y)_{\ibar} ≤ m - 1 - \card{(A \setminus \set{y})} = m - 1 - t$.
	
	Towards showing that $y \in \lprofinv(\intvl{0, t} × \intvl{0, t}) \cap \argmin_{\PE} (d \circ \lprof)$, observe that $y \in \PE$: for $i$, only the alternatives in $A$ may be better than $y$, and those are worst than $y$ for $\ibar$ by definition of $y$.
	
	Since $\lprof(x)_i > t$ and $y \in A$, we see that $\lprof(y)_i < \lprof(x)_i$.
	It follows that $\lprof(x)_{\ibar} < \lprof(y)_{\ibar}$, otherwise $y$ Pareto-dominates $x$; whence $\lprof(y)_i < \lprof(y)_{\ibar}$ otherwise $\lprof(x)_{\ibar} < \lprof(y)_{\ibar} ≤ \lprof(y)_i < \lprof(x)_i$ and $d(\lprof(y)) < d(\lprof(x))$, contradicting $x \in \argmin_{\PE} (d \circ \lprof)$.
	
	 Observe that as $t \in \N$, $t ≥ \frac{2m - 4}{3} ⇒ t ≥ \ceil{\frac{2m - 4}{3}}$, and when $m ≥ 3$, $\ceil{\frac{2m - 4}{3}} ≥ \khalf$; whence $t ≥ \khalf$.
	 \footnote{When $m$ odd, $m ≥ 5 ⇒ \frac{2m - 4}{3} ≥ \frac{m - 1}{2} = \khalf$ and when $m$ is even, $m ≥ 8 ⇒ \frac{2m - 4}{3} ≥ \frac{m}{2} = \khalf$; when $m = 3$, $\ceil{\frac{2m - 4}{3}} = \ceil{\frac{2}{3}} = 1 = \khalf$; when $m = 4$, $\ceil{\frac{2m - 4}{3}} = \ceil{\frac{4}{3}} = 2 = \khalf$; when $m = 6$, $\ceil{\frac{2m - 4}{3}} = \ceil{\frac{8}{3}} = 3 = \khalf$.} 
	
	By construction, $\forall a ≠ y \in A: \lprof(y)_{\ibar} < \lprof(a)_{\ibar}$, hence, $\lprof(y)_{\ibar} ≤ m - 1 - \card{(A \setminus \set{y})} = m - 1 - t$.
	Also, $t ≥ \khalf$ thus $m - 1 - t ≤ m - 1 - \khalf = \floor{\frac{m - 1}{2}} ≤ t$.
	These two facts, together with $\lprof(y)_i < \lprof(y)_{\ibar}$, establish that $\lprof(y)_i < \lprof(y)_{\ibar}≤ m - 1 - t ≤ t$, which yields that $y \in \lprofinv(\intvl{0, t} × \intvl{0, t})$ and that $d(\lprof(y)) ≤ m - 1 - t$. 
	
	Observe now that $\min \lprof(x) ≤ m - 1 - \max \lprof(x)$, by \cref{th:sumlosses}. This yields $d(\lprof(x)) = \max \lprof(x) - \min\lprof(x) ≥ \max \lprof(x) - (m - 1 - \max \lprof(x)) = 2 \lprof(x)_i - m + 1 ≥ 2 (t + 1) - m + 1$.
	
	To conclude, note that the condition $t ≥ \frac{2m - 4}{3}$ is equivalent to $t + 1 ≥ \frac{2m - 1}{3}$, itself equivalent to $2 (t + 1) - m + 1 ≥ m - t - 1$, hence $d(\lprof(x)) ≥ m - t - 1 ≥ d(\lprof(y))$, which implies that $y \in \argmin_{\PE} (d \circ \lprof)$.
	
%	\commentOC{It would be more elegant to show that any other alternative than $y$ has either $\max\lprof(x) ≤ t$ or $d(x) ≥ d(y)$.}
	
	The claims $\ref{it:Pt} ⇒ \ref{it:Ptbig2}$ and $\ref{it:Pt} ⇒ \ref{it:Ptbig1}$ and are seen to hold simply thanks to set inclusion.
	
	Turning now to the claim $\ref{it:Ptbig2} ⇒ \ref{it:tbound}$, assume, considering the contrapositive, that $t < \frac{2m - 4}{3}$, and let us define a profile $\prof$ such that $\lprofinv(\intvl{0, t} × \intvl{0, m - 1}) \cap \argmin_{\PE} (d \circ \lprof) = \emptyset$. (That proof applies equally to the claim $\ref{it:Ptbig1} ⇒ \ref{it:tbound}$ by inverting the role of individuals 1 and 2.)
	
	Observing that $m = t + 1 + (m - t - 2) + 1$, let us name the alternatives $\set{a_1, …, a_{t + 1}, c_1, …, c_{m - t - 2}, x}$.
	Define the sequences of alternatives $A = (a_1, …, a_{t + 1})$ and $C = (c_1, …, c_{m - t - 2})$. 
	Define the preference of individual $1$ as $(A, x, C)$ and the preference of individual $2$ as $(C, x, A)$.

	Observe that as $\lprof(x)_1 = t + 1$, $x \notin \lprofinv(\intvl{0, t} × \intvl{0, m - 1})$, and $x \in \PE$, so the claim will be proven by showing that $\forall y \in A \cup C: d(\lprof(y)) > d(\lprof(x))$, so that $\forall y \in A \cup C: y \notin \argmin_{\PE} (d \circ \lprof)$.
	
	Note that $\card{A} - \card{C} ≥ 0$ as $\card{A} - \card{C} = t + 1 - (m - t - 2) = 2t + 3 - m ≥ 2\khalf + 3 - m ≥ 2 \frac{m - 1}{2} + 3 - m = 2$.
	Thus, $d(\lprof(x)) = \card{A} - \card{C} = 2t + 3 - m$.
	
	Considering $a_i \in A$, $\lprof(a_i) = (i - 1, \card{C} + 1 + i - 1)$ thus $d(\lprof(a_i)) = m - t - 1$.
	It follows that $d(\lprof(a_i)) > d(\lprof(x))$, equivalently, $m - t - 1 > 2t + 3 - m$, as $3t < 2m - 4$ by hypothesis.
	
	And considering $c_i \in C$, $\lprof(c_i) = (\card{A} + 1 + (i - 1), i - 1)$ thus $d(\lprof(c_i)) = t + 2$.
	Using $\card{A} ≥ \card{C}$, it follows that $d(\lprof(c_i)) = \card{A} + 1 ≥ \card{C} + 1 = d(\lprof(a_i)) > d(\lprof(x))$, thus $d(\lprof(c_i)) > d(\lprof(x))$.
\end{proof}

\begin{remark}
	The equivalence $1 ⇔ 2$ says that in order ensure that there is an alternative within the first $t$ alternatives for both players that minimizes dispersion among the Pareto ones, it is necessary and sufficient to set $t$ at least $\frac{2m - 4}{3}$. In fact, by the anonymity of $\argmin_{\PE} (d \circ \lprof)$, ensuring this existence for the first $t$ alternatives for both players is equivalent to ensure it for the first $t$ alternatives for one of the players, as expressed by the equivalences $2 ⇔ 3$ and $2 ⇔ 4$. Thus, considering $1 ⇔ 3$ or $1 ⇔ 4$,  in order to ensure  that an alternative that minimizes dispersion among the Pareto ones exists within the first $t$ alternatives for one of the players, it is necessary and sufficient to set $t$ at least $\frac{2m - 4}{3}$.
\end{remark}

The next results follow from \cref{th:caractEmpty}.

\begin{theorem}
	\label{th:profsIncompat}
	$\forall \prof: H(P) \cap \argmin_{\PE} (d \circ \lprof) ≠ \emptyset ⇔ m ≤ 6 \lor m = 8$
\end{theorem}
\begin{proof}
	When $m = 2$, $H(\prof) = \lprofinv(\intvl{0, 1} × \intvl{0, 1}) = \prof$ thus $H(\prof) \cap \argmin_{\PE} (d \circ \lprof) ≠ \emptyset$.
	
	When $m ≥ 3$, fix $t = \khalf$ and use \cref{th:caractEmpty} to obtain:
	\begin{equation}
		m ≤ \frac{3 \khalf + 4}{2} ⇔ \forall \prof: H(P) \cap \argmin_{\PE} (d \circ \lprof) ≠ \emptyset.
	\end{equation}
	The left hand side is equivalent, when $m$ is odd, to $m ≤ \frac{3m + 5}{4}$ thus $m ≤ 5$, in other words, $m \in \set{3, 5}$, and when $m$ is even, to $m ≤ \frac{3m + 8}{4}$ thus $m ≤ 8$, that is, $m \in \set{4, 6, 8}$.
\end{proof}

\begin{theorem}
	\label{th:noMRMD}
	$[\exists f \in \MRcl \cap \MDcl] ⇔ [m ≤ 6 \lor m = 8]$.
\end{theorem}
\begin{proof}
	It follows from \cref{th:profsIncompat} that when $m ≤ 6$ or $m = 8$, the SCR $H(\prof) \cap \argmin_{\PE} (d \circ \lprof)$ is well-defined and satisfies MR and MD by construction (\cref{th:caractEmpty} does not apply for $m = 2$; in that case,   also holds); and when $m = 7$ or $m ≥ 9$, there is no $f \in \MRcl$ that satisfies MD.
\end{proof}
 
The minimal Rawlsian and minimal dispersion principles being logically incompatible, the SCRs that satisfy MR (namely $\FB$, $\VR$, $\SL$ and the rules in $\PVcl$ identified by \cref{th:pvmr}) all fail MD. 

\begin{remark}
	As $\FB$ has a strong egalitarian flavor, it may be surprising that it fails $\MD$. The following profile $\prof$ (built using the proof of \cref{th:caractEmpty}) illustrates this failure for $m = 7$.
	\begin{equation}
		\begin{array}{*7{l}}
			y& a_1 & a_2 & a_3 & x & c_1 & c_2\\
			c_1& c_2 & x & y & a_1 & a_2 & a_3\\
		\end{array}.
	\end{equation}
	Observe that $\FB(\prof) = \set{y}$ while the $\MD$ principle mandates $x$ to be picked.
\end{remark}

Since \cref{th:pvmr} shows that most rules in $\PVcl$ fail MR, the following result determines the subclass of $\PVcl$ satisfying MD. 

\begin{theorem}
	\label{th:pvMD}
	For $m ≥ 3$, $\PVv[(v_1, v_2)]$ satisfies MD iff $\max_{i \in \set{1, 2}} v_i ≤ \frac{m + 1}{3}$.
\end{theorem}
\begin{proof}
	Define $t = \min_{i \in \set{1, 2}} (m - 1 - v_i) \in \N$.
	Observe that $\max_{i \in \set{1, 2}} v_i ≤ \frac{m + 1}{3} ⇔ \forall i \in \set{1, 2}: v_i ≤ \frac{m + 1}{3} ⇔ \forall i \in \set{1, 2}: m - 1 - v_i ≥ \frac{2m - 4}{3} ⇔ t ≥ \frac{2m - 4}{3}$.
	
	If $t = m - 1 - v_1$, define $S = \intvl{0, t} × \intvl{0, m - 1}$, otherwise (implying that $t = m - 1 - v_2$), define $S = \intvl{0, m - 1} × \intvl{0, t}$.
	By definition, $\forall \prof: \PVv[(v_1, v_2)](\prof) = \lprofinv(\intvl{0, m - 1 - v_1} × \intvl{0, m - 1 - v_2}) \cap \PE$.
	It follows that $\forall \prof$:
	\begin{equation}
		\lprofinv(\intvl{0, t} × \intvl{0, t}) \cap \PE \subseteq \PVv[(v_1, v_2)](\prof)
	\end{equation} 
	and
	\begin{equation}
		\PVv[(v_1, v_2)](\prof) \subseteq \lprofinv(S) \cap \PE,
	\end{equation}
	therefore (intersecting all sets with $\argmin_{\PE} (d \circ \lprof)$ and using $\PE \cap \argmin_{\PE} (d \circ \lprof) = \argmin_{\PE} (d \circ \lprof)$):
	\begin{equation}
		\lprofinv(\intvl{0, t} × \intvl{0, t}) \cap \argmin_{\PE} (d \circ \lprof) \subseteq \PVv[(v_1, v_2)](\prof) \cap \argmin_{\PE} (d \circ \lprof)
	\end{equation} 
	and
	\begin{equation}
		\PVv[(v_1, v_2)](\prof) \cap \argmin_{\PE} (d \circ \lprof) \subseteq \lprofinv(S) \cap \argmin_{\PE} (d \circ \lprof).
	\end{equation}
	
	It follows from \cref{th:caractEmpty} that $t ≥ \frac{2m - 4}{3} ⇔ \forall \prof: \lprofinv(\intvl{0, t} × \intvl{0, t}) \cap \argmin_{\PE} (d \circ \lprof) ≠ \emptyset ⇔ \forall \prof: \lprofinv(S) \cap \argmin_{\PE} (d \circ \lprof) ≠ \emptyset$, and thus $\max_{i \in \set{1, 2}} v_i ≤ \frac{m + 1}{3} ⇔ t ≥ \frac{2m - 4}{3} ⇔ \forall \prof: \PVv[(v_1, v_2)](\prof) \cap \argmin_{\PE} (d \circ \lprof) ≠ \emptyset$.
%	Assume that $\max_{i \in \set{1, 2}} v_i ≤ \frac{m + 1}{3}$ (thus $t ≥ \frac{2m - 4}{3}$).
%	Note that 
%	We can apply \cref{th:caractEmpty} to obtain $\lprofinv(\intvl{0, t} × \intvl{0, t}) \cap \argmin_{\PE} (d \circ \lprof) ≠ \emptyset$, whence $\lprofinv(\intvl{0, t} × \intvl{0, t}) \cap \PE \cap \argmin_{\PE} (d \circ \lprof) ≠ \emptyset$ and thus $\PVv[(v_1, v_2)](\prof) \cap \argmin_{\PE} (d \circ \lprof) ≠ \emptyset$.
%
%	Assume that $\max_{i \in \set{1, 2}} v_i > \frac{m + 1}{3}$ (thus $t < \frac{2m - 4}{3}$).
%	By \cref{th:caractEmpty}, $\exists \prof \suchthat \lprofinv(\intvl{0, t} × \intvl{0, m - 1}) \cap \argmin_{\PE} (d \circ \lprof) = \emptyset$.
%	It follows from $\PVv[(v_1, v_2)](\prof) \subseteq \lprofinv(\intvl{0, t} × \intvl{0, m - 1})$ that $\PVv[(v_1, v_2)](\prof) \cap \argmin_{\PE} (d \circ \lprof) = \emptyset$.
\end{proof}
\commentRS{A point that Miguel suggested: Does Theorem 15 pave the way to show that one can always ensure a dispersion that is no more than a third of m?} \commentMN{Good point indeed. To be more precise, you mean that : for any profile, one can always find some alternative with a dispersion of at most $m/3$. Is that it?  Olivier, do you some counterexample to that? } \commentRS{Yes, this is what I mean.} \commentMN{This is probably true. Since $FH$ is never empty, the dispersion is at most $m/2$. Now, how to reduce the upper-bound? Should we try a contradiction? I think that this is very interesting. }
\begin{remark}
   As MD implies PEL, the class of PV rules that satisfy MD is a subclass of those that satisfy PEL. This relationship can be more precisely observed by comparing \cref{th:pel} and \cref{th:pvMD}. 
\end{remark}

The final part of this section further analyzes the relationship between the minimal dispersion principle and Pareto-and-veto rules, beyond the proximity suggested by \cref{th:pvMD}. 

\begin{remark}
	\Cref{th:pvMD} shows that PV rules with $\max_i v_i ≤ \frac{m + 1}{3}$ satisfy MD. One may wonder whether these rules also satisfy a stronger version of MD that mandates picking \emph{all} minimal dispersion alternatives (thus $\argmin_{\PE} (d \circ \lprof) \subseteq f(\prof)$). This is false. To illustrate this, for $m = 5$ consider the rule $PV^{(2, 2)}$ and the profile $\prof$.
	\begin{equation}
		\begin{array}{*5{l}}
			a& b & c & d & e \\
			c& d & e & b & a\\
		\end{array}.
	\end{equation}
	Here, $\argmin_{\PE} (d \circ \lprof) = \set{b, c}$ while $PV^{(2, 2)}(\prof) = \set{c}$.
\end{remark}

Define the rule $\MD(\prof) = \argmin_{\PE}(d \circ \lprof)$ which selects each Pareto efficient alternative that minimizes the dispersion. As mentioned in \cref{ft:equalarea}, this rule is the finite version of \possessivecite{thomson1994cooperative} equal area rule in our framework with no disagreement outcome. 

\begin{remark}
	As already mentioned, the rules $\FB$, $\VR$ and $\SL$ fail the MD principle, therefore, are distinct from the rule $\MD$. Also,
the rule $\MD$ is not in $\PVcl$: $\MD$ satisfies SPEL while no rule in $\PVcl$ does (for $m ≥ 4$), as shown by \cref{th:PVnotSPEL}.
	It follows that satisfying MD does not imply being a Pareto-and-veto rule.
\end{remark}

\section{Reconciling the two principles}
\label{sec:reconc}
Given the incompatibility between MR and MD, one can attempt to reconcile the two principles by imposing minimal dispersion over the alternatives that are minimally Rawlsian. 

Given a profile, let us call “Rawlsian minimal dispersion (RMD) alternatives” those that minimize dispersion among the Paretian alternatives within the first half. As every profile admits RMD alternatives, we can define the RMD principle as the requirement that the social choice always contains an RMD alternative while remaining within the first half.
\begin{definition}[Rawlsian minimal dispersion]
	$\forall \prof \in \allprofs:
	f(\prof) \cap \argmin_{H(\prof) \cap \PE} (d \circ \lprof) ≠ \emptyset$
	and $f(\prof) \subseteq H(\prof)$.
\end{definition}

RMD strengthens MR. On the other hand, the relationship between RMD and MD depends on $m$: when $m$ is small, RMD is stronger than MD while when $m$ is large, no rule can satisfy both RMD and MD (because there are profiles where all Paretian dispersion minimizers are out of the first half, by \cref{th:profsIncompat}).
Summarizing, when $m$ is small enough ($m ≤ 6$ or $m = 8$), and only then, first, RMD implies MD, and second, some rules are both RMD and MD.
\begin{theorem}
	\label{th:RMDMD}
	$[m ≤ 6 \lor m = 8] ⇔ \RMDcl \subseteq \MDcl ⇔ \RMDcl \cap \MDcl ≠ \emptyset$.
\end{theorem}
\begin{proof}
	First consider $[m ≤ 6 \lor m = 8] ⇒ \RMDcl \subseteq \MDcl$. \Cref{th:profsIncompat} indicates that $\forall \prof: H(\prof) \cap \argmin_{\PE} (d \circ \lprof) ≠ \emptyset ⇔ m ≤ 6 \lor m = 8$. 
	And $[H(\prof) \cap \argmin_{\PE} (d \circ \lprof) ≠ \emptyset] ⇒ [\argmin_{H(\prof) \cap \PE} (d \circ \lprof) \subseteq \argmin_{\PE} (d \circ \lprof)]$.
	Given $f \in \RMDcl$, $f(\prof) \cap \argmin_{H(\prof) \cap \PE} (d \circ \lprof) ≠ \emptyset$, thus, when $m ≤ 6 \lor m = 8$, $f(\prof) \cap \argmin_{\PE} (d \circ \lprof) ≠ \emptyset$.
	
	Second, observe that $\RMDcl \subseteq \MDcl ⇒ \RMDcl \cap \MDcl ≠ \emptyset$.
	
	To conclude, we prove that $\RMDcl \cap \MDcl ≠ \emptyset ⇒ [m ≤ 6 \lor m = 8]$.
	As $\RMDcl \subseteq \MRcl$, we have $[\RMDcl \cap \MDcl ≠ \emptyset] ⇒ [\MRcl \cap \MDcl ≠ \emptyset]$. And from \cref{th:noMRMD}, the latter implies $[m ≤ 6 \lor m = 8]$.
\end{proof}
	
\begin{theorem}
	\label{th:compatRMD}
	 $\FB$, $\VR$, $\SL$ fail RMD, and $\PVcl \cap \RMDcl = \set{\PVe}$.
\end{theorem}
\begin{proof}
	Consider the following profile $\prof$ with 11 alternatives, $\set{a, b, c, d, e, f, g, h,\allowbreak x, y, z}$ (the bar indicates the “half” position).
	\begin{equation}
		\begin{array}{*6{l}|ll}
			x&a&b&c&d&y&w&\ldots \\
			e&f&g&y&x&h&w&\ldots
		\end{array}.
	\end{equation}

$H(\prof)=\{x, y\}$ which are both Pareto efficient. For a SCR $f$ to satisfy RMD, we must have $y\in   f(\prof)$. As $\FB(\prof)$ = $\VR(\prof)=\{x\}$, both rules fail RMD. 

To see that $\SL$ fails RMD, consider the following profile $\prof$ with $\SL(\prof)=\set{a, c}$ while RMD requires $b \in \SL(\prof)$. 

	\begin{equation}
		\begin{array}{lll|ll}
			a&b&c&d&e \\
			c&b&a&d&e
		\end{array}.
	\end{equation}

We now turn to the rules in $\PVcl$. As RMD implies MR, $\PVcl ∩ \RMDcl \subseteq \PVcl ∩ \MRcl$, and by \cref{th:pvmr}, $\PVcl ∩ \MRcl = \set{\PVe}$ when $m$ is odd and $\PVcl ∩ \MRcl = \set{\PVe, \PVv[(\frac{m}{2}, \frac{m}{2} - 1)], \PVv[(\frac{m}{2} - 1, \frac{m}{2})]}$ when $m$ is even.
Thus, we need only prove that for $m$ even, $\PVv[(\frac{m}{2}, \frac{m}{2} - 1)]$ and $\PVv[(\frac{m}{2} - 1, \frac{m}{2})]$ fail RMD. To see this, consider the profile $\prof$
	\begin{equation}
		\begin{array}{llll}
			a&b&c&d \\
			d&c&a&b
		\end{array}
	\end{equation}
where $\PVv[(\frac{m}{2}, \frac{m}{2} - 1)](\prof)=\{a\}$ while RMD requires $c \in \PVv[(\frac{m}{2}, \frac{m}{2} - 1)](\prof)$. A similar argument shows that $\PVv[(\frac{m}{2} - 1, \frac{m}{2})]$ fails RMD.

By \cref{th:equal}, $\PVe(\prof) = H(\prof) \cap \PE$, thus, it satisfies RMD.
\end{proof}

$\PVe$ satisfies the RMD principle at the expense of resoluteness, since it picks every Pareto efficient alternative within the first half. It is thus natural to seek for a stronger principle that imposes more resoluteness. To this end, we define the Strong RMD principle, which requires to pick only RMD alternatives.
\begin{definition}[Strong Rawlsian minimal dispersion]
	$\forall \prof \in \allprofs:
	f(\prof) \subseteq \argmin_{H(\prof) \cap \PE} (d \circ \lprof)$.
\end{definition}

The following theorem is the counterpart to \cref{th:RMDMD} for SRMD.
\begin{theorem}
	$[m ≤ 6 \lor m = 8] ⇔ \SRMDcl \subseteq \MDcl ⇔ \SRMDcl \cap \MDcl ≠ \emptyset$.
\end{theorem}
\begin{proof}
	First, it follows from $[m ≤ 6 \lor m = 8] ⇒ \RMDcl \subseteq \MDcl$ (by \cref{th:RMDMD}) and $\SRMDcl \subseteq \RMDcl$ that $[m ≤ 6 \lor m = 8] ⇒ \SRMDcl \subseteq \MDcl$. 
	
	Second, observe that $\SRMDcl \subseteq \MDcl ⇒ \SRMDcl \cap \MDcl ≠ \emptyset$.
	
	To conclude, we prove that $\SRMDcl \cap \MDcl ≠ \emptyset ⇒ [m ≤ 6 \lor m = 8]$.
	As $\SRMDcl \subseteq \MRcl$, we have $[\SRMDcl \cap \MDcl ≠ \emptyset] ⇒ [\MRcl \cap \MDcl ≠ \emptyset]$. And from \cref{th:noMRMD}, the latter implies $[m ≤ 6 \lor m = 8]$.
\end{proof}
	
It also follows from \cref{th:equal} that $\PVe$ fails Strong RMD. Thus, no rule that we have examined so far satisfy Strong RMD.
\commentOC{We might want to choose the name once we understand the rule a bit better, perhaps.}
Define the Strong RMD rule $f(\prof) = \argmin_{H(\prof) \cap \PE} (d \circ \lprof)$ as the least resolute rule satisfying Strong RMD. 
The Strong RMD rule (or SMRD) might appear as a reasonable compromise to satisfy the two main concepts exposed here (besides Pareto), that of being minimally Rawlsian and that of favoring small dispersion alternatives. It is also reasonably resolute.
\begin{theorem}
	The Strong RMD rule $f$ selects at most two alternatives, i.e., $\forall \prof \in \allprofs: 1 ≤ \card{f(\prof)} ≤ 2$.
\end{theorem}
\begin{proof}
	Consider any $x \in \argmin_{H(\prof) \cap \PE} (d \circ \lprof)$ and any $i \suchthat \min \lprof(x) = \lprof(x)_i$. Consider any $y \in \argmin_{H(\prof) \cap \PE} (d \circ \lprof)$. We first prove that $\min \lprof(y) = \lprof(y)_i$ iff $y = x$. Equivalently, we prove that $\forall a \in H(\prof): \lprof(a)_i ≠ \lprof(x)_i ⇒ [a \notin \argmin_{H(\prof) \cap \PE} (d \circ \lprof) \lor \min \lprof(a) ≠ \lprof(a)_i]$. Indeed, consider any $a \in H(\prof) \suchthat \lprof(a)_i < \lprof(x)_i$, then $\lprof(a)_{\ibar} > \lprof(x)_{\ibar}$ (because $x \in \PE$), so $\lprof(a)_i < \lprof(x)_i ≤ \lprof(x)_{\ibar} < \lprof(a)_{\ibar}$ and $(d \circ \lprof)(a) > (d \circ \lprof)(x)$; and considering now any $a \in H(\prof) \suchthat \lprof(a)_i > \lprof(x)_i$, then $\lprof(a)_{\ibar} < \lprof(x)_{\ibar}$ is required for $a \in \PE$, and if $\min \lprof(a) = \lprof(a)_i$ then $\lprof(x)_i < \lprof(a)_i ≤ \lprof(a)_{\ibar} < \lprof(x)_{\ibar}$ and $(d \circ \lprof)(a) < (d \circ \lprof)(x)$, contradicting $x \in \argmin_{H(\prof) \cap \PE} (d \circ \lprof)$, so that again $\min \lprof(a) ≠ \lprof(a)_i$.
	
	To conclude, observe that if $\argmin_{H(\prof) \cap \PE} (d \circ \lprof)$ contain at least two elements (say, $x$ and $y$), one of them, say, $x$, has $\min \lprof(x) = \lprof(x)_i$, the other one has $\min \lprof(y) = \lprof(y)_{\ibar}$, and thus there cannot be a third element $z$ in $\argmin_{H(\prof) \cap \PE} (d \circ \lprof)$ as neither $\min \lprof(z) = \lprof(z)_i$ nor $\min \lprof(z) = \lprof(z)_{\ibar}$ is possible, by the above result.
\end{proof}




\color{red} NEW VERSION:More resoluteness can be achieved by adding a requirement that among the Pareto efficient alternatives in the FH. Define to that effect the \emph{resolute RMD} rule as follows. This rule makes a selection of the strong RMD alternatives, by picking the one(s) with the highest sum of losses: thus $f(\prof) = \argmax_{a \in SMRD(\prof)} \lprof(a)_1+\lprof(a)_2 $. 

\color{black}


\color{blue}OLD VERSION:More resoluteness can be achieved by adding a requirement that among the alternatives in $\argmin_{H(\prof) \cap \PE} (d \circ \lprof)$, the rule selects those that dominate considering the anonymous set of losses (thus $\set{\lprof(x)_1, \lprof(x)_2}$), so that for example in $\prof = \set{(1, (a, b, c, d, e, f)), (2, (c, ., ., b, ., .))}$, $c$ with losses $\set{0, 2}$ will be favored over $b$ with losses $\set{1, 3}$. It associates to each Paretian alternative $a \in \PE$ in a profile $\prof$ having losses $\lprof(a)$ the score (or “badness”) $s_{\prof}(a) = md(\lprof(a)) + \min\lprof(a) + m^2 \ind_{a \notin H(\prof)}$, where $m = \card{\allalts}$ and $d(l) = \max l - \min l$ as previously and $\ind_{a \notin H(\prof)}$ equals $1$ iff $a \notin H(\prof)$ and $0$ otherwise. The rule picks as winners the Paretian alternatives with minimal score, thus $f(\prof) = \argmin_{a \in \PE} s_{\prof}(a)$.\color{black}

As it satisfies Strong RMD, the resolute RMD picks at most two alternatives.
To conclude this section, we present a characterization of this rule. The first property, called preorder based in Cailloux and Endriss 2014, mandates that the rule decides who wins according to the losses only, and following a fixed weak order over losses.
\begin{definition}[Loss vector comparing (LVC)]
	\label{def:lvc}
	Let $\preceq$ be a weak order (a reflexive, transitive and complete binary relation) on $\lvs$. The rule $f^\preceq$ selects all alternatives whose losses are minimal elements of $\restr{{\preceq}}{\lprof(\alts)}$. A rule satisfies LVC iff $\exists {\preceq} \suchthat f = f^\preceq$.
\end{definition}
Given a weak order $\preceq$ and two losses $l, l'$, let $l \sim l'$ stand for $l \preceq l' \land l' \preceq l$.

\begin{theorem}
	\label{th:caractRMD}
	A rule $f$ is the resolute RMD rule iff it satisfies anonymity, Strong RMD and LVC.
\end{theorem}

\section{Concluding remarks}
\label{sec:concl}
Axiomatic analysis of social choice rules with or without strategic concerns presents two strands of the literature that complement each other. This complementarity appears less balanced in two-person collective choice problems where there is a clear focus on a strategic analysis that usually adopts subgame perfect equilibrium as the solution concept.%
\footnote{While we will not restate here the relevant papers already cited in the introduction, we wish to remark that the focus on subgame perfect equilibrium can be explained by the classical result of \citet{HurwiczSchmeidler78} and \citet{Maskin99} on the impossibility of Nash implementing Pareto efficient two-person social choice rules.} 

The richness of the non-strategic axiomatic analysis of collective choice with three or more individuals is accompanied by a wealthy list of conceived social choice rules. On the other hand, for two individuals, it is hard to name a prominent social choice rule beyond unanimity compromise, the shortlisting rule, the veto-rank rule, and the class of Pareto-and-veto rules. Moreover, these social choice rules are conceived under different interpretations of the two-person collective choice model, thus being analyzed from somewhat different perspectives.

We bring a consideration based on a common interpretation and perspective. The axiomatic analysis we propose is free of strategic concerns and relies on two basic principles that we identify: the minimally Rawlsian principle and the equal loss principle. These two principles that emerge from the existing literature exhibit an incompatibility. More precisely, no minimally Rawlsian social choice rule can minimize the dispersion of the loss vector.
\Cref{fig:props} summarizes our findings. It should be noted that the axioms listed on \Cref{fig:props} do not characterize the SCRs that we discuss. This is somehow expected, as these axioms impose conditions at a given problem without alluding to the consequences when the given problem is modified, i.e., they are “punctual” in the sense of \citet{thomson2012axiomatics}.% 
\footnote{We thank Sylvain Béal for this comment.}

\begin{table}
	\begin{tabular}{cl*{5}{c}}
		\toprule
		& MR & PEL & SPEL & MD & RMD\\
		\midrule
		$FB$ & ✓ & ✓ & ✓\\
		$VR$ & ✓ & \\
		$SL$ & ✓ & \\
		$\PVcl^=$ & ✓ & \\
		$\PVcl \suchthat \max v_i ≤ \frac{m}{2}$ & & ✓\\
		$\PVcl \suchthat \max v_i ≤ \frac{m + 1}{3}$ & & ✓ & & ✓\\
		$\PVe$ & ✓ & & & & ✓ \\
		\bottomrule
	\end{tabular}
	\caption{Summary of the results. A tick means that all members of the corresponding class of rules satisfy the corresponding property (for $m ≥ 4$). The notation $\PVcl^=$ designates the class referred to by \cref{th:pvmr}.}
	\label{fig:props}
\end{table}

In front of this incompatibility, the literature seems to favor the minimally Rawlsian principle, as unanimity compromise, the shortlisting rule and the veto-rank rule are minimally Rawlsian. Moreover, within the class of Pareto-and-veto rules, perhaps the most prominent ones, namely those which give both individuals the highest equal or almost equal veto power are minimally Rawlsian. By the established incompatibility, these rules cannot minimize loss dispersion but, except the Pareto-and-veto rule that gives the highest equal veto power, they do not even minimize loss dispersion among the Paretian alternatives that are ranked within the first half of both individuals.

Among the rules we consider, those that minimize loss dispersion are the Pareto-and-veto rules that give each individual a veto power that does not exceed a third of the total number of alternatives. These rules will typically make coarse choices with several tied alternatives. Aiming at more refined outcomes that minimize loss dispersion, one can directly pick at every preference profile the Pareto efficient alternatives that minimize loss dispersion or the Pareto efficient alternatives within the first half of both individuals that minimize loss dispersion. Seemingly none of these two rules are considered in the literature and their properties are not known, which suggest that there is room to conceive and analyze new social choice rules.
\commentRS{We should add to the conclusion our findings about the strong RMD rule. We should compare it to the VR rule. Should we mention the rule proposed by Hillas?}\commentMN{Yes, The Hillas rule is a good idea.}

\appendix

\section{Spread measures and minimal dispersion}
\label{sec:spreads}
Recall that we call the dispersion of a loss vector $l$ at $\prof$ the value $d(l) = \abs{l_1 - l_2}$. 

We reuse the following definitions from \citet{cailloux2022compromising}, letting $n$ denote the number of individuals, $l \in \R^n$ denote a generalized loss tuple and $\bar{l} = \sum_{i = 1}^n l_i / n$ denote the arithmetic mean of the losses:
\begin{itemize}
	\item the mean absolute difference $\sigma_{mad}(l)= \frac{1}{n^2} \sum_{i = 1}^n\sum_{j = 1}^n \abs{l_i - l_j}$;
	\item the average absolute deviation $\sigma_{ad}(l)= \frac{\sum_{i = 1}^n \abs{l_i - \bar{l}}}{n}$;
	\item the standard deviation $\sigma_{sd}(l)= \sqrt{\frac{\sum_{i = 1}^n(l_i - \bar{l})^2}{n}}$;
	\item the Gini coefficient $\sigma_{G}(l)= \frac{\sum_{i = 1}^n\sum_{j = 1}^n \abs{l_i - l_j}}{2 n \sum_{i = 1}^n l_i}$.
\end{itemize} 

When $n = 2$ as in our case, it is plain that $\forall i \in \set{1, 2}: \abs{l_i - \bar{l}} = \frac{\abs{l_1 - l_2}}{2}$; $\sum_{i = 1}^n\sum_{j = 1}^n \abs{l_i - l_j} = 2 \abs{l_1 - l_2}$; $\forall l \in \R^n: \sigma_{mad}(l) = \sigma_{ad}(l) = \sigma_{sd}(l) = \frac{d(l)}{2}$ and $\sigma_G(l) = \frac{\abs{l_1 - l_2}}{2 (l_1 + l_2)}$. 
Thus, $\sigma_{mad}$, $\sigma_{ad}$ and $\sigma_{sd}$ coincide with $d$, but $\sigma_G$ does not. 
For example, the Gini coefficient considers $(49, 51)$ as less unequal than $(0, 1)$ whereas $d$ orders these tuples reversely.

\section{Characterization of resolute RMD}
\begin{lemma}
	\label{th:inZ}
	Given a subset of alternatives $X \subseteq \allalts$, let $S_X$ denote an linear ordering of all the alternatives in $X$.
	Given a Paretian and MR rule $f$ and any profile $\prof$ and $1 ≤ k ≤ \khalf$,
	if there exists a partition (a disjoint complete cover) of the alternatives into sets $A, B, C$ and $Z$ of respective sizes $k - 1$, $k - 1$, $m - 2 k$ and $2$ and orderings $S_{A \cup Z}$, $S_{B \cup Z}$, $S_A$, $S_B$ and $S_C$ such that $\prof$ contains the two following preferences (associated to individuals $(1, 2)$ or $(2, 1)$):
	\begin{equation}
		\begin{array}{*{3}c}
			S_{A \cup Z} & S_C & S_B \\
			S_{B \cup Z} & S_C & S_A
		\end{array},
	\end{equation}
	then $f(\prof) \subseteq Z$.
\end{lemma}

\begin{proof}
	We see first that $C \cap f(\prof) = \emptyset$ because $\forall c \in C, c \notin \PE$, which, as $f$ is Paretian, excludes $c \in f(\prof)$.
	
	We see then that $(A \cup B) \cap f(\prof) = \emptyset$ because  and $\forall z \in A \cup B, \max \lprof(z) ≥ (k - 1 + 2) + (m - 2 k) = m + 1 - k = \khalf + \floor{\frac{m - 1}{2}} + 2 - k$ and with $k ≤ \khalf$, $\max \lprof(z) ≥ \floor{\frac{m - 1}{2}} + 2 = \ceil{\frac{m - 2}{2}} + 2 = \ceil{\frac{m}{2}} + 1 ≥ \khalf + 1 > \khalf$, thus $z \notin H(\prof)$, which, as $f$ is MR, excludes $z \in f(\prof)$.
\end{proof}
	
\begin{lemma}
	\label{th:precLVC}
	Consider any Paretian, MR and LVC rule $f$ and any $\preceq$ such that $f^\preceq = f$. Then, $\forall l, l' \in \intvl{0, \khalf}^N \suchthat [\forall i \in N: l(i) < l'(i)]: l \prec l'$.
\end{lemma}
\begin{proof}
	Name the alternatives $a_1, …, a_{\max l' - 1}, b_1, …, b_{\max l' - 1}, x, y, c_1, …, \allowbreak{} c_{m - 2 \max l'}$.

	Pick any $i \in \argmin l'$, and recall that by hypothesis, $l(i) < l'(i)$; also, $l(\ibar) < \max l'$ as $l(\ibar) < l'(\ibar) ≤ \max l'$.
	Define the sequences $A_1 = (a_1, …, a_{l(i)})$, $A_2 = (a_{l(i) + 1}, …, a_{l'(i) - 1})$, $A_3 = (a_{l'(i)}, …, a_{\max l' - 1})$, $B_1 = (b_1, …, b_{l(\ibar)})$, $B_2 = (b_{l(\ibar) + 1}, …, b_{\max l' - 1})$, $C = (c_1, …, c_{m - 2 \max l'})$, and consider the profile $\prof$ defined as:
	\begin{equation}
		\begin{array}{*{5}c}
			i &\mapsto & (A_1, x, A_2, y, A_3) & C & (B_1, B_2) \\
			\ibar &\mapsto & (B_1, x, B_2, y) & C & (A_1, A_2, A_3)
		\end{array}.
	\end{equation}
%	The layout of $\prof$ reflects the fact that the sequences $(A_1, A_2, A_3)$ and $(B_1, B_2)$ (hence, $(A_1, x, A_2, y, A_3)$ and $(B_1, x, B_2, y)$) have equal lengths (the latter having length $\max l' + 1$).
	
	We see from \cref{th:inZ} that $f(\prof) \subseteq \set{x, y}$.
	
	As $y \notin \PE$, we obtain $f(\prof) = \set{x}$ and therefore $f^\preceq(\prof) = \set{x}$, thus $\lprof(x) \prec \lprof(y)$.
	
	Observe, to conclude, that $\lprof(x)_i = l(i)$, $\lprof(x)_{\ibar} = l(\ibar)$, $\lprof(y)_i = l'(i)$ and $\lprof(y)_{\ibar} = \max l' = l'(\ibar)$, thus, $\lprof(x) = l$ and $\lprof(y) = l'$.
\end{proof}

\begin{remark}
	A superficial look at the LVC property may produce the impression that Paretianism of an LVC rule $f^\preceq$ unconditionally lead to $\preceq$ respecting dominance of the losses (that is, $l ≤ l' ⇒ l \preceq l'$). This is false. For example, with $m = 3$, define the equivalence classes of losses $C_1 = \set{(0, 0), (0, 1), (1, 0)}$, $C_2 = \set{(1, 2), (2, 1), (2, 2)}$, $C_3 = \set{(1, 1)}$ and define $\preceq$ as considering $C_1$ smaller than $C_2$ smaller than $C_3$ (thus $((0, 0) \sim (0, 1) \sim (1, 0)) \prec ((1, 2) \sim (2, 1) \sim (2, 2)) \prec (1, 1)$). The relation $\preceq$ does not respect dominance of the losses ($(1, 1) < (2, 2)$ though $(1, 1) \npreceq (2, 2)$), but the corresponding rule $f^\preceq$ is Paretian: if some alternatives have losses $(0, 0)$, $(0, 1)$ or $(1, 0)$ in $\prof$, they will be elected, satisfying Paretianism; otherwise (thus if $\prof$ contains no alternatives with losses among $C_1$) then it contains the losses $\set{(0, 2), (2, 0), (1, 1)}$ where no Pareto dominance occurs.
\end{remark}

\begin{lemma}
	\label{th:indiffLVC}
	Consider any anonymous, Paretian, MR and LVC rule $f$ and any $\preceq$ such that $f^\preceq = f$. Then, $\forall l, l' \in \intvl{0, \khalf}^N \suchthat [l(1) = l'(2) \land l(2) = l'(1)]: l \sim l'$.
\end{lemma}
\begin{proof}
	If $l(1) = l(2)$ then $l = l'$ and the conclusion follows from $\preceq$ being reflexive.
	Thus, assume that $l(1) ≠ l(2)$, equivalently, that $\min l < \max l$.
	Also note that $\max l = \max l'$, whence $1 ≤ \max l' ≤ \khalf$.
	
	Name the alternatives $a_1, …, a_{\max l' - 1}, b_1, …, b_{\max l' - 1}, x, y, c_1, …, \allowbreak{} c_{m - 2 \max l'}$.  %Note that this is feasible as $0 ≤ \min l < \max l' ≤ \khalf = \floor{\frac{m}{2}}$, whence $1 ≤ \max l'$ and $2 \max l' ≤ 2 \floor{\frac{m}{2}} ≤ m$ which leads to $2 (\max l' - 1) ≤ m - 2$.
	Define the (possibly empty) sequences $A_1 = (a_1, …, a_{\min l})$, $A_2 = (a_{\min l + 1}, …, a_{\max l' - 1})$, $B_1 = (b_1, …, b_{\min l})$, $B_2 = (b_{\min l + 1}, …, b_{\max l' - 1})$ and $C = (c_1, …, c_{m - 2 \max l'})$.
 	Pick $i \in \argmin l$, equivalently, $\ibar \in \argmax l$,
	and define the profile $\prof$ as:
	\begin{equation}
		\begin{array}{*{25}{l}}
			i &\mapsto & A_1 & x & A_2 & y & C & B_1 & B_2 \\
			\ibar &\mapsto & B_1 & y &B_2 & x & C & A_1 & A_2
		\end{array}.
	\end{equation}
	
	We see from \cref{th:inZ} that $f(\prof) \subseteq \set{x, y}$, and $f = f^\preceq$, thus, $x \in f(\prof) ⇔ \lprof(x) \preceq \lprof(y)$ and $y \in f(\prof) ⇔ \lprof(y) \preceq \lprof(x)$.
	
	Observe also that $\lprof(x)_i = \min l = \min l' = l(i)$, $\lprof(x)_{\ibar} = \max l = \max l' = l(\ibar)$, $\lprof(y)_i = l'(i)$ and $\lprof(y)_{\ibar} = \max l' = l'(\ibar)$, thus, $\lprof(x) = l$ and $\lprof(y) = l'$, which yields $x \in f(\prof) ⇔ l \preceq l'$ and $y \in f(\prof) ⇔ l' \preceq l$.
%	Observe also that $\lprof(x)(N) = \lprof(y)(N) = \set{\min l, \max l}$, thus, $\lprof(\set{x, y}) = \set{l, l'}$.
	
	Consider now the profile $\prof'$ associating to each individual the preference of the other individual in $\prof$, as follows:
	\begin{equation}
		\begin{array}{*{25}{l}}
			i &\mapsto & B_1 & y &B_2 & x & C & A_1 & A_2 \\
			\ibar &\mapsto & A_1 & x & A_2 & y & C & B_1 & B_2
		\end{array}.
	\end{equation}
	The argument above applies, mutatis mutandis, and yields that $x \in f(\prof') ⇔ \lprof[\prof'](x) \preceq \lprof[\prof'](y)$ and $y \in f(\prof') ⇔ \lprof[\prof'](y) \preceq \lprof[\prof'](x)$.
	Also, $\lprof[\prof'](x) = l'$ and $\lprof[\prof'](y) = l$, which yields $x \in f(\prof') ⇔ l' \preceq l$ and $y \in f(\prof') ⇔ l \preceq l'$.
	
	Finally, by anonymity, $f(\prof) = f(\prof')$, thus, joining our conclusions so far, $x \in f(\prof) ⇔ l \preceq l' ⇔ x \in f(\prof') ⇔ l' \preceq l ⇔ y \in f(\prof)$. At least one of these propositions is true (as $x \in f(\prof) \lor y \in f(\prof)$), thus all are.
\end{proof}

\begin{lemma}
	\label{th:precLVCstr}
	Consider any anonymous, Paretian, MR and LVC rule $f$ and any $\preceq$ such that $f^\preceq = f$. Then, $\forall l, l' \in \intvl{0, \khalf}^N \suchthat [\min l < \min l' \land \max l < \max l']: l \prec l'$.
\end{lemma}
\begin{proof}
	Observe that $(l'(2), l'(1))$ denotes the losses of $l'$ where both individuals are inverted.
	From \cref{th:indiffLVC}, $l' \sim (l'(2), l'(1))$.
	
	Pick any $i \in \argmin l$.
	Define $l'' = \set{(i, \min l'), (\ibar, \max l')}$; observe that $l'' = l' \lor l'' = (l'(2), l'(1))$.
	It follows that $l' \sim l''$.
	
	By hypothesis, $\min l = l(i) < l''(i) = \min l' \land \max l = l(\ibar) < l''(\ibar) = \max l'$. Thus, by \cref{th:precLVC}, $l \prec l''$.
	Together with $l' \sim l''$, this yields $l \prec l'$.
\end{proof}

\begin{theorem}
	With $m ≥ 8$, consider any rule $f$ satisfying RMD and anonymity. The rule is not LVC.
\end{theorem}
\begin{proof}
	Note that $\khalf ≥ 4$. 
	Consider the profile $\prof$:
	\begin{equation}
		\begin{array}{*8{l}}
			x&a_1&a_2&a_3&y&a_4&\ldots&a_{m - 2} \\
			a_{m - 2}&a_{m - 1}&y&x&a_1&a_2&\ldots&a_{m - 3}
		\end{array}.
	\end{equation}
	Note that $\forall 3 ≤ i ≤ m - 3$, $a_2$ Pareto-dominates $a_i$. Thus, $\argmin_{H(\prof) \cap \PE} (d \circ \lprof) = \set{y}$, thus $y \in f(\prof)$.

	It follows that there is no $\preceq$ such that $f^\preceq = f$: if there was such a $\preceq$,
	by \cref{th:precLVCstr} (applicable because RMD implies Pareto and MR), we would have $(0, 3) \prec (4, 2)$, whence $y \notin f^\preceq$.
\end{proof}

Recall that \cref{th:caractRMD} states that a rule $f$ is the resolute RMD rule iff it satisfies anonymity, Strong RMD and LVC.
\commentOC{Check if the proof is clearer without the scoring function.}
\begin{proof}[Proof of \cref{th:caractRMD} - Wrong!]
	First note that $\forall \prof$, $a \in H(\prof) ⇔ s_{\prof}(a) < m^2$.
	
	Consider $f$ being the resolute RMD rule and let’s check that it is Strong RMD.
	Within the first half, $s_{\prof}(a) = m (\max \lprof(a) - \min \lprof(a)) + \min \lprof(a) = m \max \lprof(a) - (m - 1) \min \lprof(a)$. Thus if $a$ PD $b$ within the first half, then $b$ does not win. Also, if $a \notin H(\prof)$, then $a$ does not win, as $\exists b \in H(\prof) \suchthat s_{\prof}(b) < m^2$.
	
	Now consider an LVC rule $f^\preceq = f$ that is anonymous and Strong RMD.
	We have to show that $f$ is the RMD rule, thus, that $f(\prof) = \argmin_{\PE} s_{\prof}$. 
	Observing that $a \in \min_{\PE} s$ is equivalent to $a \in \argmin_{\PE \cap H(\prof)} s$, as $a \in H(\prof) \land a' \notin H(\prof) ⇒ s_{\prof}(a) < s_{\prof}(a')$, an equivalent thesis is that $f(\prof) = \argmin_{\PE \cap H(\prof)} s_{\prof}$.

	Let us write, given $a \in \allalts$, $l_a$ instead of $\lprof(a)$.
	
	For the forward direction, considering any $a \in f(\prof)$, let us show that $a \in \argmin_{\PE \cap H(\prof)} s_{\prof}$.
	
	Strong RMD implies Paretianism and MR, thus, $a \in \PE \cap H(\prof)$.
	
	By definition of LVC, $a \in f^\preceq(\prof)$ is equivalent to $\forall a' \in \allalts: l_{a'} \nprec l_a$, from which follows that $\forall a' \in \PE \cap H(\prof): l_{a'} \nprec l_a$.
	But $[\min l_{a'} < \min l_a \land d(l_a) = d(l_{a'})] ⇒ [\max l_{a'} < \max l_a]$, together implying with $a' \in H(\prof)$, from \cref{th:precLVCstr}, $l_{a'} \prec l_a$. We thus obtain, using the contrapositive, $\forall a' \in \PE \cap H(\prof): [\min l_a ≤ \min l_{a'} \lor d(l_a) ≠ d(l_{a'})]$.
	
	By definition of Strong RMD, $a \in f^\preceq(\prof)$ yields that $\forall a' \in \PE \cap H(\prof): d(l_a) ≤ d(l_{a'})$.
	
	We obtain that $\forall a' \in \PE \cap H(\prof): [d(l_a) ≤ d(l_{a'}) \land \min l_a ≤ \min l_{a'}] \lor d(l_a) < d(l_{a'})$, which is equivalent to $\forall a' \in \PE \cap H(\prof): s_{\prof}(a) ≤ s_{\prof}(a')$ and thus shows that $a \in \argmin_{\PE \cap H(\prof)} s_{\prof}$.
	
	Turning to the backwards direction, we know that $a \in f(\prof) ⇒ \forall a' \in \PE \cap H(\prof): s_{\prof}(a') ≥ s_{\prof}(a) ⇒ \forall a' \in \min_{\PE \cap H(\prof)} s_{\prof}: s_{\prof}(a') = s_{\prof}(a)$.
	
	Given $a, a' \in \PE \cap H(\prof)$, $s_{\prof}(a) = s_{\prof}(a')$ is equivalent to $d(l_a) = d(l_{a'}) \land \min l_a = \min l_{a'}$, itself, using \cref{th:indiffLVC}, implying $a_l \sim a_{l'}$, or equivalently, by LVC, $[a \in f(\prof) ⇔ a' \in f(\prof)]$.
	
	To conclude, pick any $a' \in \min_{\PE \cap H(\prof)} s_{\prof}$ and let us show that $a' \in f(\prof)$. Pick any $a \in f(\prof)$ (which exists as $f(\prof) ≠ \emptyset$). From above, we see that $s_{\prof}(a') = s_{\prof}(a)$, and from above again, we see that $[a \in f(\prof) ⇔ a' \in f(\prof)]$, whence $a' \in f(\prof)$.
\end{proof}

\begin{remark}
	LVC is required because Anonymity and Strong RMD (even with neutrality) does not suffice to obtain the resolute RMD rule: the Strong RMD rule also satisfies those properties.
\end{remark}

\bibliography{bibliototal}
\end{document}

