\RequirePackage[l2tabu, orthodox]{nag}
\documentclass[version=3.21, pagesize, twoside=off, bibliography=totoc, DIV=calc, fontsize=12pt, a4paper]{scrartcl}
%Permits to copy eg x ⪰ y ⇔ v(x) ≥ v(y) from PDF to unicode data, and to search. From pdfTeX users manual. See https://tex.stackexchange.com/posts/comments/1203887.
	\input glyphtounicode
	\pdfgentounicode=1
%Latin Modern has more glyphs than Computer Modern, such as diacritical characters. fntguide commands to load the font before fontenc, to prevent default loading of cmr.
	\usepackage{lmodern}
%Encode resulting accented characters correctly in resulting PDF, permits copy from PDF.
	\usepackage[T1]{fontenc}
%UTF8 seems to be the default in recent TeX installations, but not all, see https://tex.stackexchange.com/a/370280.
	\usepackage[utf8]{inputenc}
%Provides \newunicodechar for easy definition of supplementary UTF8 characters such as → or ≤ for use in source code.
	\usepackage{newunicodechar}
%Text Companion fonts, much used together with CM-like fonts. Provides \texteuro and commands for text mode characters such as \textminus, \textrightarrow, \textlbrackdbl.
	\usepackage{textcomp}
%St Mary’s Road symbol font, used for ⟦ = \llbracket. The \SetSymbolFont command avoids spurious warnings, but also some valid ones, see https://tex.stackexchange.com/a/106719/.
	\usepackage{stmaryrd}\SetSymbolFont{stmry}{bold}{U}{stmry}{m}{n}
%Solves bug in lmodern, https://tex.stackexchange.com/a/261188; probably useful only for unusually big font sizes; and probably better to use exscale instead. Note that the authors of exscale write against this trick.
	%\DeclareFontShape{OMX}{cmex}{m}{n}{
		%<-7.5> cmex7
		%<7.5-8.5> cmex8
		%<8.5-9.5> cmex9
		%<9.5-> cmex10
	%}{}
	%\SetSymbolFont{largesymbols}{normal}{OMX}{cmex}{m}{n}
%More symbols (such as \sum) available in bold version, see https://github.com/latex3/latex2e/issues/71. In article mode (but not in presentation mode), also hides some potentially useful warnings such as when using $\bm{\llbracket}$, see stmaryrd in this document (not sure why).
	\DeclareFontShape{OMX}{cmex}{bx}{n}{%
	   <->sfixed*cmexb10%
	   }{}
	\SetSymbolFont{largesymbols}{bold}{OMX}{cmex}{bx}{n}
%For small caps also in italics, see https://tex.stackexchange.com/questions/32942/italic-shape-needed-in-small-caps-fonts, https://tex.stackexchange.com/questions/284338/italic-small-caps-not-working.
	\usepackage{slantsc}
	\AtBeginDocument{%
		%“Since nearly no font family will contain real italic small caps variants, the best approach is to substitute them by slanted variants.” -- slantsc doc
		%\DeclareFontShape{T1}{lmr}{m}{scit}{<->ssub*lmr/m/scsl}{}%
		%There’s no bold small caps in Latin Modern, we switch to Computer Modern for bold small caps, see https://tex.stackexchange.com/a/22241
		%\DeclareFontShape{T1}{lmr}{bx}{sc}{<->ssub*cmr/bx/sc}{}%
		%\DeclareFontShape{T1}{lmr}{bx}{scit}{<->ssub*cmr/bx/scsl}{}%
	}
%Warn about missing characters.
	\tracinglostchars=2
%Nicer tables: provides \toprule, \midrule, \bottomrule.
	\usepackage{booktabs}
%For new column type X which stretches; can be used together with booktabs, see https://tex.stackexchange.com/a/97137. “tabularx modifies the widths of the columns, whereas tabular* modifies the widths of the inter-column spaces.” Loads array.
	%\usepackage{tabularx}
%math-mode version of "l" column type. Requires \usepackage{array}.
	%\usepackage{array}
	%\newcolumntype{L}{>{$}l<{$}}
%Provides \xpretocmd and loads etoolbox which provides \apptocmd, \patchcmd, \newtoggle… Also loads xparse, which provides \NewDocumentCommand and similar commands intended as replacement of \newcommand in LaTeX3 for defining commands (see https://tex.stackexchange.com/q/98152 and https://github.com/latex3/latex2e/issues/89).
	\usepackage{xpatch}
%for \nexists and because it is a basic package nowadays, see https://tex.stackexchange.com/q/539592/.
	\usepackage{amssymb}
%loads and fixes some bugs in amsmath (a basic, mandatory package nowadays, see Grätzer, More Math Into LaTeX) and provides \DeclarePairedDelimiter. I recommend \begin{equation}, which allows numbering, rather than \[ (and $$ should be avoided), see https://tex.stackexchange.com/questions/503. Relatedly, do not use the displaymath environment: use equation. Do not use the eqnarray environment: use align. This improves spacing. (See l2tabu or amsldoc.)
	\usepackage{mathtools}
%Package frenchb asks to load natbib before babel-french. Package hyperref asks to load natbib before hyperref.
	\usepackage{natbib}

\newtoggle{LCpres}
	\newtoggle{LCart}
	\newtoggle{LCposter}
	\makeatletter
	\@ifclassloaded{beamer}{
		\toggletrue{LCpres}
		\togglefalse{LCart}
		\togglefalse{LCposter}
		\wlog{Presentation mode}
	}{
		\@ifclassloaded{tikzposter}{
			\toggletrue{LCposter}
			\togglefalse{LCpres}
			\togglefalse{LCart}
			\wlog{Poster mode}
		}{
			\toggletrue{LCart}
			\togglefalse{LCpres}
			\togglefalse{LCposter}
			\wlog{Article mode}
		}
	}
	\makeatother%

%Language options ([french, english]) should be on the document level (last is main); except with tikzposter: put [french, english] options next to \usepackage{babel} to avoid warning. beamer uses the \translate command for the appendix: omitting babel results in a warning, see https://github.com/josephwright/beamer/issues/449. Babel also seems required for \refname.
	\iftoggle{LCpres}{
		\usepackage{babel}
	}{
	}
	%\frenchbsetup{AutoSpacePunctuation=false}
%https://ctan.org/pkg/amsmath recommends ntheorem, which supersedes amsthm, which corrects the spacing of proclamations and allows for theoremstyle, but I decided to switch to amsthm with thmtools (mentioned in amsthm doc) because ntheorem “seems essentially unmaintaned and has severe problems”, see https://tex.stackexchange.com/q/535950. Must be loaded after amsmath (from amsthm doc).
		\usepackage{amsthm}
		\usepackage{thmtools}
%listings (1.7) does not allow multi-byte encodings. listingsutf8 works around this only for characters that can be represented in a known one-byte encoding and only for \lstinputlisting. Other workarounds: use literate mechanism; or escape to LaTeX (but breaks alignment).
	%\usepackage{listings}
	%\lstset{tabsize=2, basicstyle=\ttfamily, escapechar=§, literate={é}{{\'e}}1}
%I favor acro over acronym because the former is more recently updated (2018 VS 2015 at time of writing); has a longer user manual (about 40 pages VS 6 pages if not counting the example and implementation parts); has a command for capitalization; and acronym suffers a nasty bug when ac used in section, see https://tex.stackexchange.com/q/103483 (though this might be the fault of the silence package and might be solved in more recent versions, I do not know) and from a bug when used with cleveref, see https://tex.stackexchange.com/q/71364. However, loading it makes compilation time (one pass on this template) go from 0.6 to 1.4 seconds, see https://bitbucket.org/cgnieder/acro/issues/115.
	\usepackage{acro}
	%“All options of acro that have not been mentioned in section 4.1 have to be set up… with… \acsetup{…}” -- acro package doc, cited by the Overleaf support (thanks to them!)
	\acsetup{single}
	\DeclareAcronym{AMCD}{short=AMCD, long={Aide Multicritère à la Décision}}
\DeclareAcronym{AHP}{short=AHP, long={Analytic Hierarchy Process}}
\DeclareAcronym{AR}{short=AR, long={Argumentative Recommender}}
\DeclareAcronym{DA}{short=DA, long={Decision Analysis}}
\DeclareAcronym{DJ}{short=DJ, long={Deliberated Judgment}}
\DeclareAcronym{DM}{short=DM, long={Decision Maker}}
\DeclareAcronym{DP}{short=DP, long={Deliberated Preference}}
\DeclareAcronym{MAVT}{short=MAVT, long={Multiple Attribute Value Theory}}
\DeclareAcronym{MCDA}{short=MCDA, long={Multicriteria Decision Aid}}
\DeclareAcronym{MIP}{short=MIP, long={Mixed Integer Program}}
\DeclareAcronym{SEU}{short=SEU, long={Subjective Expected Utility}}


\iftoggle{LCpres}{
	%I favor fmtcount over nth because it is loaded by datetime anyway; and fmtcount warns about possible conflicts when loaded after nth (“\ordinal already defined use \FCordinal instead”). See also https://english.stackexchange.com/questions/93008.
	\usepackage{fmtcount}
	%For nice input of date of presentation. Must be loaded after the babel package. Has possible problems with srcletter: https://golatex.de/verwendung-von-babel-und-datetime-in-scrlttr2-schlaegt-fehlt-t14779.html.
	\usepackage[nodayofweek]{datetime}
}{
}
%For presentations, Beamer implicitely uses the pdfusetitle option. autonum doc mandates option hypertexnames=false. I want to highlight links only if necessary for the reader to recognize it as a link, to reduce distraction. In presentations, this is already taken care of by beamer (https://tex.stackexchange.com/a/262014). If using colorlinks=true in a presentation, see https://tex.stackexchange.com/q/203056. Crashes the first compilation with tikzposter, just compile again and the problem disappears, see https://tex.stackexchange.com/q/254257.
\makeatletter
\iftoggle{LCpres}{
	\usepackage{hyperref}
}{
	\usepackage[hypertexnames=false, pdfusetitle, linkbordercolor={1 1 1}, citebordercolor={1 1 1}, urlbordercolor={1 1 1}]{hyperref}
	%https://tex.stackexchange.com/a/466235
	\pdfstringdefDisableCommands{%
		\let\thanks\@gobble
	}
}
\makeatother
%urlbordercolor is used both for \url and \doi, which I think shouldn’t be colored, and for \href, thus might want to color manually when required. Requires xcolor.
	\NewDocumentCommand{\hrefblue}{mm}{\textcolor{blue}{\href{#1}{#2}}}
%hyperref doc says: “Package bookmark replaces hyperref’s bookmark organization by a new algorithm (...) Therefore I recommend using this package”.
	\usepackage{bookmark}
%Need to invoke hyperref explicitly to link to line numbers: \hyperlink{lintarget:mylinelabel}{\ref*{lin:mylinelabel}}, with \ref* to disable automatic link. Also see https://tex.stackexchange.com/q/428656 for referencing lines from another document.
	%\usepackage{lineno}
	%\NewDocumentCommand{\llabel}{m}{\hypertarget{lintarget:#1}{}\linelabel{lin:#1}}
	%\setlength\linenumbersep{9mm}
%For complex authors blocks. Seems like authblk wants to be later than hyperref, but sooner than silence. See https://tex.stackexchange.com/q/475513 for the patch to hyperref pdfauthor.
	\ExplSyntaxOn
	\seq_new:N \g_oc_hrauthor_seq
	\NewDocumentCommand{\addhrauthor}{m}{
		\seq_gput_right:Nn \g_oc_hrauthor_seq { #1 }
	}
	\NewExpandableDocumentCommand{\hrauthor}{}{
		\seq_use:Nn \g_oc_hrauthor_seq {,~}
	}
	\ExplSyntaxOff
	{
		\catcode`#=11\relax
		\gdef\fixauthor{\xpretocmd{\author}{\addhrauthor{#2}}{}{}}%
	}
	\iftoggle{LCart}{
		\usepackage{authblk}
		\renewcommand\Affilfont{\small}
		\fixauthor
		\AtBeginDocument{
		    \hypersetup{pdfauthor={\hrauthor}}
		}
	}{
	}
%I do not use floatrow, because it requires an ugly hack for proper functioning with KOMA script (see scrhack doc). Instead, the following command centers all floats (using \centering, as the center environment adds space, http://texblog.net/latex-archive/layout/center-centering/), and I manually place my table captions above and figure captions below their contents (https://tex.stackexchange.com/a/3253).
	\makeatletter
	\g@addto@macro\@floatboxreset\centering
	\makeatother
%Permits to customize enumeration display and references
	%\nottoggle{LCpres}{
		%\usepackage{enumitem} %follow list environments by a string to customize enumeration, example: \begin{description}[itemindent=8em, labelwidth=!] or \begin{enumerate}[label=({\roman*}), ref={\roman*}].
	%}{
	%}
%Provides \Centering, \RaggedLeft, and \RaggedRight and environments Center, FlushLeft, and FlushRight, which allow hyphenation. With tikzposter, seems to cause 1=1 to be printed in the middle of the poster.
	%\usepackage{ragged2e}
%To typeset units by closely following the “official” rules.
	%\usepackage[strict]{siunitx}
%Turns the doi provided by some bibliography styles into URLs.
	\usepackage{doi}
%Makes sure upper case greek letters are italic as well.
	\usepackage{fixmath}
%Provides \mathbb; obsoletes latexsym (see http://tug.ctan.org/macros/latex/base/latexsym.dtx). Relatedly, \usepackage{eucal} to change the mathcal font and \usepackage[mathscr]{eucal} (apparently equivalent to \usepackage[mathscr]{euscript}) to supplement \mathcal with \mathscr. This last option is not very useful as both fonts are similar, and the intent of the authors of eucal was to provide a replacement to mathcal (see doc euscript). Also provides \mathfrak for supplementary letters.
	\usepackage{amsfonts}
%Provides a beautiful (IMHO) \mathscr and really different than \mathcal, for supplementary uppercase letters. But there is no bold version. Alternative: mathrsfs (more slanted), but when used with tikzposter, it warns about size substitution, see https://tex.stackexchange.com/q/495167.
	\usepackage[scr]{rsfso}
%Multiple means to produce bold math: \mathbf, \boldmath (defined to be \mathversion{bold}, see fntguide), \pmb, \boldsymbol (all legacy, from LaTeX base and AMS), \bm (the most recommended one), \mathbold from package fixmath (I don’t see its advantage over \boldsymbol).
%“The \boldsymbol command is obtained preferably by using the bm package, which provides a newer, more powerful version than the one provided by the amsmath package. Generally speaking, it is ill-advised to apply \boldsymbol to more than one symbol at a time.” — AMS Short math guide. “If no bold font appears to be available for a particular symbol, \bm will use ‘poor man’s bold’” — bm. It is “best to load the package after any packages that define new symbol fonts” – bm. bm defines \boldsymbol as synonym to \bm. \boldmath accesses the correct font if it exists; it is used by \bm when appropriate. See https://tex.stackexchange.com/a/10643 and https://github.com/latex3/latex2e/issues/71 for some difficulties with \bm.
	\usepackage{bm}
	\nottoggle{LCpres}{
	%Provides \cref. Unfortunately, cref fails when the language is French and referring to a label whose name contains a colon (https://tex.stackexchange.com/q/83798). Use \cref{sec\string:intro} to work around this. cleveref should go “laster” than hyperref.
		\usepackage[capitalise]{cleveref}
	}{
	}
	\nottoggle{LCposter}{
	%Equations get numbers iff they are referenced. Loading order should be “amsmath → hyperref → cleveref → autonum”, according to autonum doc. Use this in preference to the showonlyrefs option from mathtools, see https://tex.stackexchange.com/q/459918 and autonum doc. See https://tex.stackexchange.com/a/285953 for the etex line. Incompatible with my version of tikzposter (produces “! Improper \prevdepth”). This removes the starred versions, such as equation*. Unfortunately, this prevents using \qedhere in an equation ending a proof, see https://tex.stackexchange.com/q/133358/.
		\expandafter\def\csname ver@etex.sty\endcsname{3000/12/31}\let\globcount\newcount
		\usepackage{autonum}
	}{
	}
%Also loaded by tikz.
	\usepackage{xcolor}
\iftoggle{LCpres}{
	\usepackage{tikz}
	%\usetikzlibrary{babel, matrix, fit, plotmarks, calc, trees, shapes.geometric, positioning, plothandlers, arrows, shapes.multipart}
}{
}
%Vizualization, on top of TikZ
	%\usepackage{pgfplots}
	%\pgfplotsset{compat=1.14}
\usepackage{graphicx}
	\graphicspath{{graphics/}}

%Provides \printlength{length}, useful for debugging.
	%\usepackage{printlen}
	%\uselengthunit{mm}

\iftoggle{LCpres}{
	\usepackage{appendixnumberbeamer}
	%I have yet to see anyone actually use these navigation symbols; let’s disable them
	\setbeamertemplate{navigation symbols}{} 
	\usepackage{preamble/beamerthemeParisFrance}
	\setcounter{tocdepth}{10}
}{
}

%Requires package xcolor.
\definecolor{ao(english)}{rgb}{0.0, 0.5, 0.0}
\NewDocumentCommand{\commentOC}{m}{\textcolor{blue}{\small$\big[$OC: #1$\big]$}}
%Requires package babel and option [french]. According to babel doc, need two braces around \selectlanguage to make the changes really local.
\NewDocumentCommand{\commentOCf}{m}{\textcolor{blue}{{\small\selectlanguage{french}$\big[$OC : #1$\big]$}}}
\NewDocumentCommand{\commentRS}{m}{\textcolor{red}{\small$\big[$RS: #1$\big]$}}
\NewDocumentCommand{\commentMN}{m}{\textcolor{ao(english)}{\small$\big[$MN: #1$\big]$}}

\bibliographystyle{abbrvnat}
\NewDocumentCommand{\possessivecite}{mO{}}{\citeauthor{#1}’s \citeyearpar[#2]{#1}}

%https://tex.stackexchange.com/a/467188, https://tex.stackexchange.com/a/36088 - uncomment if one of those symbols is used.
%\DeclareFontFamily{U} {MnSymbolD}{}
%\DeclareFontShape{U}{MnSymbolD}{m}{n}{
%  <-6> MnSymbolD5
%  <6-7> MnSymbolD6
%  <7-8> MnSymbolD7
%  <8-9> MnSymbolD8
%  <9-10> MnSymbolD9
%  <10-12> MnSymbolD10
%  <12-> MnSymbolD12}{}
%\DeclareFontShape{U}{MnSymbolD}{b}{n}{
%  <-6> MnSymbolD-Bold5
%  <6-7> MnSymbolD-Bold6
%  <7-8> MnSymbolD-Bold7
%  <8-9> MnSymbolD-Bold8
%  <9-10> MnSymbolD-Bold9
%  <10-12> MnSymbolD-Bold10
%  <12-> MnSymbolD-Bold12}{}
%\DeclareSymbolFont{MnSyD} {U} {MnSymbolD}{m}{n}
%\DeclareMathSymbol{\ntriplesim}{\mathrel}{MnSyD}{126}
%\DeclareMathSymbol{\nlessgtr}{\mathrel}{MnSyD}{192}
%\DeclareMathSymbol{\ngtrless}{\mathrel}{MnSyD}{193}
%\DeclareMathSymbol{\nlesseqgtr}{\mathrel}{MnSyD}{194}
%\DeclareMathSymbol{\ngtreqless}{\mathrel}{MnSyD}{195}
%\DeclareMathSymbol{\nlesseqgtrslant}{\mathrel}{MnSyD}{198}
%\DeclareMathSymbol{\ngtreqlessslant}{\mathrel}{MnSyD}{199}
%\DeclareMathSymbol{\npreccurlyeq}{\mathrel}{MnSyD}{228}
%\DeclareMathSymbol{\nsucccurlyeq}{\mathrel}{MnSyD}{229}
%\DeclareFontFamily{U} {MnSymbolA}{}
%\DeclareFontShape{U}{MnSymbolA}{m}{n}{
%  <-6> MnSymbolA5
%  <6-7> MnSymbolA6
%  <7-8> MnSymbolA7
%  <8-9> MnSymbolA8
%  <9-10> MnSymbolA9
%  <10-12> MnSymbolA10
%  <12-> MnSymbolA12}{}
%\DeclareFontShape{U}{MnSymbolA}{b}{n}{
%  <-6> MnSymbolA-Bold5
%  <6-7> MnSymbolA-Bold6
%  <7-8> MnSymbolA-Bold7
%  <8-9> MnSymbolA-Bold8
%  <9-10> MnSymbolA-Bold9
%  <10-12> MnSymbolA-Bold10
%  <12-> MnSymbolA-Bold12}{}
%\DeclareSymbolFont{MnSyA} {U} {MnSymbolA}{m}{n}
%%Rightwards wave arrow: ↝. Alternative: \rightsquigarrow from amssymb, but it’s uglier
%\DeclareMathSymbol{\rightlsquigarrow}{\mathrel}{MnSyA}{160}

%03B3 Greek Small Letter Gamma
\newunicodechar{γ}{\gamma}
%03B4 Greek Small Letter Delta
\newunicodechar{δ}{\delta}
%2115 Double-Struck Capital N
\newunicodechar{ℕ}{\mathbb{N}}
%211D Double-Struck Capital R
\newunicodechar{ℝ}{\mathbb{R}}
%21CF Rightwards Double Arrow with Stroke
\newunicodechar{⇏}{\nRightarrow}
%21D2 Rightwards Double Arrow
\newunicodechar{⇒}{\ensuremath{\Rightarrow}}
%21D4 Left Right Double Arrow
\newunicodechar{⇔}{\Leftrightarrow}
%21DD Rightwards Squiggle Arrow
\newunicodechar{⇝}{\rightsquigarrow}
%2205 Empty Set
\newunicodechar{∅}{\emptyset}
%2212 Minus Sign
\newunicodechar{−}{\ifmmode{-}\else\textminus\fi}
%2227 Logical And
\newunicodechar{∧}{\land}
%2228 Logical Or
\newunicodechar{∨}{\lor}
%2229 Intersection
\newunicodechar{∩}{\cap}
%222A Union
\newunicodechar{∪}{\cup}
%2260 Not Equal To (handy also as text in informal writing)
\newunicodechar{≠}{\ensuremath{\neq}}
%2264 Less-Than or Equal To
\newunicodechar{≤}{\leq}
%2265 Greater-Than or Equal To
\newunicodechar{≥}{\geq}
%2270 Neither Less-Than nor Equal To
\newunicodechar{≰}{\nleq}
%2271 Neither Greater-Than nor Equal To
\newunicodechar{≱}{\ngeq}
%2272 Less-Than or Equivalent To
\newunicodechar{≲}{\lesssim}
%2273 Greater-Than or Equivalent To
\newunicodechar{≳}{\gtrsim}
%2274 Neither Less-Than nor Equivalent To – also, from MnSymbol: \nprecsim, a more exact match to the Unicode symbol; and \npreccurlyeq, too small
\newunicodechar{≴}{\not\preccurlyeq}
%2275 Neither Greater-Than nor Equivalent To
\newunicodechar{≵}{\not\succcurlyeq}
%2279 Neither Greater-Than nor Less-Than – requires MnSymbol; also \nlessgtr from txfonts/pxfonts, \ngtreqless from MnSymbol (but much higher), \ngtrless from MnSymbol (a more exact match to the Unicode symbol); for incomparability (not matching this Unicode symbol), may also consider \ntriplesim from MnSymbol,\nparallelslant from fourier, \between from mathabx, or ⋈
\newunicodechar{≹}{\ngtreqlessslant}
%227A Precedes
\newunicodechar{≺}{\prec}
%227B Succeeds
\newunicodechar{≻}{\succ}
%227C Precedes or Equal To
\newunicodechar{≼}{\preccurlyeq}
%227D Succeeds or Equal To
\newunicodechar{≽}{\succcurlyeq}
%227E Precedes or Equivalent To
\newunicodechar{≾}{\precsim}
%227F Succeeds or Equivalent To
\newunicodechar{≿}{\succsim}
%2280 Does Not Precede
\newunicodechar{⊀}{\nprec}
%2281 Does Not Succeed
\newunicodechar{⊁}{\nsucc}
%2286
\newunicodechar{⊆}{\subseteq}
%22B2 Normal Subgroup Of – using \vartriangleleft from amsfonts, which goes well with \trianglelefteq, \ntriangleright, and so on, also from amsfonts; another possibility is \lhd from latexsym, which seems visually equivalent to \vartriangleleft from amsfonts; latexsym also has ⊴=\unlhd, but doesn’t have a symbol for ⊴. Other related symbols: \triangleleft from latesym package is too small; fdsymbol provides \triangleleft=\medtriangleleft and \vartriangleleft=\smalltriangleleft; MnSymbol provides \medtriangleleft and \vartriangleleft=\lessclosed=\lhd which are smaller than \vartriangleleft from amsfont; \vartriangleleft from mathabx (p. 67), looks different (wider); also \vartriangleleft from boisik (p. 69) looks still different; \vartriangleleft=\lhd from stix are smaller. Oddly enough, \triangleright appears as the LMMathItalic12-Regular font whereas \rhd appears as LASY10 and \vartriangleright appears as MSAM10.
\newunicodechar{⊲}{\vartriangleleft}
%22B3 Contains as Normal Subgroup (also: 25B7 White right-pointing triangle or 25B9 White right-pointing small triangle)
\newunicodechar{⊳}{\vartriangleright}
%22B4 Normal Subgroup of or Equal To
\newunicodechar{⊴}{\trianglelefteq}
%22B5 Contains as Normal Subgroup or Equal To
\newunicodechar{⊵}{\trianglerighteq}
%22C8 Bowtie
\newunicodechar{⋈}{\bowtie}
%22EA Not Normal Subgroup Of
\newunicodechar{⋪}{\ntriangleleft}
%22EB Does Not Contain As Normal Subgroup
\newunicodechar{⋫}{\ntriangleright}
%22EC Not Normal Subgroup of or Equal To
\newunicodechar{⋬}{\ntrianglelefteq}
%22ED Does Not Contain as Normal Subgroup or Equal
\newunicodechar{⋭}{\ntrianglerighteq}
%25A1 White Square
\newunicodechar{□}{\Box}
%27E6 Mathematical Left White Square Bracket – requires stmaryrd (alternative: \text{\textlbrackdbl}, but ugly if used in an italicized text such as a theorem)
\newunicodechar{⟦}{\llbracket}
%27E7 Mathematical Right White Square Bracket
\newunicodechar{⟧}{\rrbracket}
%27FC Long Rightwards Arrow from Bar
\newunicodechar{⟼}{\longmapsto}
%2AB0 Succeeds Above Single-Line Equals Sign
\newunicodechar{⪰}{\succeq}
%301A Left White Square Bracket
\newunicodechar{〚}{\textlbrackdbl}
%301B Right White Square Bracket
\newunicodechar{〛}{\textrbrackdbl}
%→ is defined by default as \textrightarrow, which is invalid in math mode. Same thing for the three other commands. Using \DeclareUnicodeCharacter instead of \newunicodechar because the latter warns about the previous definition.
%→ Rightwards Arrow
\DeclareUnicodeCharacter{2192}{\ifmmode\rightarrow\else\textrightarrow\fi}
%¬ Not Sign
\DeclareUnicodeCharacter{00AC}{\ifmmode\lnot\else\textlnot\fi}
%… Horizontal Ellipsis
\DeclareUnicodeCharacter{2026}{\ifmmode\dots\else\textellipsis\fi}
%× Multiplication Sign
\DeclareUnicodeCharacter{00D7}{\ifmmode\times\else\texttimes\fi}
%Permits to really obtain a straight quote when typing a straight quote; potentially dangerous, see https://tex.stackexchange.com/a/521999
\catcode`\'=\active
\DeclareUnicodeCharacter{0027}{\ifmmode^\prime\else\textquotesingle\fi}


\NewDocumentCommand{\R}{}{ℝ}
\NewDocumentCommand{\N}{}{ℕ}
%\mathscr is rounder than \mathcal.
\NewDocumentCommand{\powerset}{m}{\mathscr{P}(#1)}
%Powerset without zero.
\NewDocumentCommand{\powersetz}{m}{\mathscr{P}^*(#1)}
%https://tex.stackexchange.com/a/45732, works within both \set and \set*, same spacing than \mid (https://tex.stackexchange.com/a/52905).
\NewDocumentCommand{\suchthat}{}{\;\ifnum\currentgrouptype=16 \middle\fi|\;}
%Integer interval.
\NewDocumentCommand{\intvl}{m}{⟦#1⟧}
%Allows for \abs and \abs*, which resizes the delimiters.
\DeclarePairedDelimiter\abs{\lvert}{\rvert}
\DeclarePairedDelimiter\card{\lvert}{\rvert}
\DeclarePairedDelimiter\floor{\lfloor}{\rfloor}
\DeclarePairedDelimiter\ceil{\lceil}{\rceil}
%Perhaps should use U+2016 ‖ DOUBLE VERTICAL LINE here?
\DeclarePairedDelimiter\norm{\lVert}{\rVert}
%From mathtools. Better than using the package braket because braket introduces possibly undesirable space. Then: \begin{equation}\set*{x \in \R^2 \suchthat \norm{x}<5}\end{equation}.
\DeclarePairedDelimiter\set{\{}{\}}
\DeclareMathOperator*{\argmax}{arg\,max}
\DeclareMathOperator*{\argmin}{arg\,min}

%UTR #25: Unicode support for mathematics recommend to use the straight form of phi (by default, given by \phi) rather than the curly one (by default, given by \varphi), and thus use \phi for the mathematical symbol and not \varphi. I however prefer the curly form because the straight form is too easy to mix up with the symbol for empty set.
\let\phi\varphi

%The amssymb solution.
%\NewDocumentCommand{\restr}{mm}{{#1}_{\restriction #2}}
%Another acceptable solution.
%\NewDocumentCommand{\restr}{mm}{{#1|}_{#2}}
%https://tex.stackexchange.com/a/278631; drawback being that sometimes the text collides with the line below.
\NewDocumentCommand\restr{mm}{#1\raisebox{-.5ex}{$|$}_{#2}}


%Decision Theory
\NewDocumentCommand{\allalts}{}{\mathcal{A}}
\NewDocumentCommand{\allcrits}{}{\mathscr{C}}
\NewDocumentCommand{\alts}{}{A}
\NewDocumentCommand{\dm}{}{i}
\NewDocumentCommand{\allF}{}{\mathscr{F}}
\NewDocumentCommand{\allvoters}{}{\mathscr{N}}
\NewDocumentCommand{\voters}{}{N}
\NewDocumentCommand{\allprofs}{}{\linors^N}
\NewDocumentCommand{\prof}{}{\bm{P}}
\NewDocumentCommand{\lprof}{}{\lambda_{\bm{P}}}
\NewDocumentCommand{\linors}{}{\mathscr{L}(\allalts)}
%Thanks to https://tex.stackexchange.com/q/154549
	%\makeatletter
	%\def\@myRgood@#1#2{\mathrel{R^X_{#2}}}
	%\def\myRgood{\@ifnextchar_{\@myRgood@}{\mathrel{R^X}}}
	%\makeatother
\NewDocumentCommand{\pref}{}{\succ}
\NewDocumentCommand{\prefi}{O{i}}{\succ_{#1}}
\NewDocumentCommand{\prefiinv}{O{i}}{\succ_{#1}^{-1}}
\NewDocumentCommand{\ibar}{}{\overline{i}}

\NewDocumentCommand{\lvs}{}{\intvl{0, m - 1}^N}
\NewDocumentCommand{\losses}{}{\intvl{0, m - 1}}
\NewDocumentCommand{\PD}{}{\mathit{PD}(\prof)}
\NewDocumentCommand{\PE}{}{\mathit{PE}(\prof)}

%Rules
\NewDocumentCommand{\rhoP}{}{\rho_{\prof}}
\NewDocumentCommand{\minspread}{O{A}}{\min_{#1}(\sigma \circ \lambda_{\bm{P}})}
\NewDocumentCommand{\mindisp}{O{A}}{\min_{#1}(d \circ \lambda_{\bm{P}})}
\NewDocumentCommand{\FB}{}{\mathit{FB}}
\NewDocumentCommand{\VR}{}{\mathit{VR}}
\NewDocumentCommand{\SL}{}{\mathit{SL}}
\NewDocumentCommand{\PVv}{O{v}}{\mathit{PV}^{#1}}
\NewDocumentCommand{\PVef}{}{\mathit{PV}^{\floor{\frac{m - 1}{2}}}}%f for first
\NewDocumentCommand{\PVes}{}{\mathit{PV}^{\ceil{\frac{m - 1}{2}}}}%s for second
\NewDocumentCommand{\PVe}{}{\mathit{PV^=}}%egalitarian distribution

%Classes
\NewDocumentCommand{\PVcl}{}{\mathcal{PV}}
\NewDocumentCommand{\PVbcl}{}{\mathcal{PV}^b}
\NewDocumentCommand{\PVecl}{}{\mathcal{PV}^=}%egalitarian distribution
\NewDocumentCommand{\PEcl}{}{\mathcal{PE}}
\NewDocumentCommand{\FHcl}{}{\mathcal{FH}}
\NewDocumentCommand{\VCcl}{}{\mathcal{VC}}
\NewDocumentCommand{\VCecl}{}{\mathcal{VC^=}}
\NewDocumentCommand{\ELcl}{}{\mathcal{EL}}


%\NewDocumentCommand{\tikzmark}{m}{%
	\tikz[overlay, remember picture, baseline=(#1.base)] \node (#1) {};%
}

\newlength{\GraphsDNodeSep}
\setlength{\GraphsDNodeSep}{7mm}
\tikzset{/GraphsD/dot/.style={
	shape=circle, fill=black, inner sep=0, minimum size=1mm
}}

% MCDA Drawing Sorting
\newlength{\MCDSCatHeight}
\setlength{\MCDSCatHeight}{6mm}
\newlength{\MCDSAltHeight}
\setlength{\MCDSAltHeight}{4mm}
%separation between two vertical alts
\newlength{\MCDSAltSep}
\setlength{\MCDSAltSep}{2mm}
\newlength{\MCDSCatWidth}
\setlength{\MCDSCatWidth}{3cm}
\newlength{\MCDSAltWidth}
\setlength{\MCDSAltWidth}{2.5cm}
\newlength{\MCDSEvalRowHeight}
\setlength{\MCDSEvalRowHeight}{6mm}
\newlength{\MCDSAltsToCatsSep}
\setlength{\MCDSAltsToCatsSep}{1.5cm}
\newcounter{MCDSNbAlts}
\newcounter{MCDSNbCats}
\newlength{\MCDSArrowDownOffset}
\setlength{\MCDSArrowDownOffset}{0mm}
\tikzset{/MCD/S/alt/.style={
	shape=rectangle, draw=black, inner sep=0, minimum height=\MCDSAltHeight, minimum width=\MCDSAltWidth
}}
\tikzset{/MCD/S/pref/.style={
	shape=ellipse, draw=gray, thick
}}
\tikzset{/MCD/S/cat/.style={
	shape=rectangle, draw=black, inner sep=0, minimum height=\MCDSCatHeight, minimum width=\MCDSCatWidth
}}
\tikzset{/MCD/S/evals matrix/.style={
	matrix, row sep=-\pgflinewidth, column sep=-\pgflinewidth, nodes={shape=rectangle, draw=black, inner sep=0mm, text depth=0.5ex, text height=1em, minimum height=\MCDSEvalRowHeight, minimum width=12mm}, nodes in empty cells, matrix of nodes, inner sep=0mm, outer sep=0mm, row 1/.style={nodes={draw=none, minimum height=0em, text height=, inner ysep=1mm}}
}}

%Git
\newlength{\GitDCommitSep}
\setlength{\GitDCommitSep}{13mm}
\tikzset{/GitD/commit/.style={
	shape=rectangle, draw, minimum width=4em, minimum height=0.6cm
}}
\tikzset{/GitD/branch/.style={
	shape=ellipse, draw, red
}}
\tikzset{/GitD/head/.style={
	shape=ellipse, draw, fill=yellow
}}

%Social Choice
\tikzset{/SCD/profile matrix/.style={
	matrix of math nodes, column sep=3mm, row sep=2mm, nodes={inner sep=0.5mm, anchor=base}
}}
\tikzset{/SCD/rank-profile matrix/.style={
	matrix of math nodes, column sep=3mm, row sep=2mm, nodes={anchor=base}, column 1/.style={nodes={inner sep=0.5mm}}, row 1/.style={nodes={inner sep=0.5mm}}
}}
\tikzset{/SCD/rank-vector/.style={
	draw, rectangle, inner sep=0, outer sep=1mm
}}
\tikzset{/SCD/isolated rank-vector/.style={
	draw, matrix of math nodes, column sep=3mm, inner sep=0, matrix anchor=base, nodes={anchor=base, inner sep=.33em}, ampersand replacement=\&
}}

% GUI
\tikzset{/GUID/button/.style={
	rectangle, very thick, rounded corners, draw=black, fill=black!40%, top color=black!70, bottom color=white
}}

% Logger objects
\tikzset{/loggerD/main/.style={
	shape=rectangle, draw=black, inner sep=1ex, minimum height=7mm
}}
\tikzset{/loggerD/helper/.style={
	shape=rectangle, draw=black, dashed, minimum height=7mm
}}
\tikzset{/loggerD/helper line/.style={
	<->, draw, dotted
}}

% Beliefs
\tikzset{/BeliefsD/attacker/.style={
	shape=rectangle, draw, minimum size=8mm
}}
\tikzset{/BeliefsD/supporter/.style={
	shape=circle, draw
}}



%I find these settings useful in draft mode. Should be removed for final versions.
	%Which line breaks are chosen: accept worse lines, therefore reducing risk of overfull lines. Default = 200.
		\tolerance=2000
	%Accept overfull hbox up to...
		\hfuzz=2cm
	%Reduces verbosity about the bad line breaks.
		\hbadness 5000
	%Reduces verbosity about the underful vboxes.
		\vbadness=1300

\title{Two-person SCRs in a unified framework}
\author{Name}
%\author{Olivier Cailloux}
\affil{Université Paris-Dauphine, PSL Research University, CNRS, LAMSADE, 75016 PARIS, FRANCE\\
%	\href{mailto:olivier.cailloux@dauphine.fr}{olivier.cailloux@dauphine.fr}
}
%\author{Name3}
%\affil{Affil2}
\hypersetup{
	pdfsubject={},
	pdfkeywords={},
}

\begin{document}
\maketitle

\section{Introduction}
\label{sec:intro}
\section{Basic notions and notation}
Let $N = \set{1, 2}$ be the set of individuals and $\allalts$ be the set of alternatives, with $\card{\allalts} = m\geq 2$. 
Given $i \in N$, let $\ibar \in N \setminus \set{i}$ denote the other individual. Let $\powersetz{\allalts}$ denote the set of non-empty subsets of $\allalts$. Let $\linors$ be the set of linear orders over $\mathcal{A}$. We let $\succ \in \linors$ stand for the preference of an individual, written as $\succ_i$ when it belongs to $i\in N$. We write $\prof =(\succ_1,\succ_2) \in \allprofs$ for a (preference) profile. A social choice rule (SCR) is a function $f: \allprofs → \powersetz{\allalts}$.
Given two SCRs $f$, $f'$, we write $f \subset f'$ to indicate that $f$ is a proper subcorrespondence of $f'$.

An SCR $f$ is anonymous iff $f(\prefi[1], \prefi[2]) = f(\prefi[2], \prefi[1])$ for all $(\succ_1,\succ_2) \in \allprofs$.
An SCR $f$ is neutral iff for all permutations $\sigma$ over $\allalts$ and profile $\prof \in \allprofs$, $\sigma \circ f(\prof) = f(\sigma \circ \prof)$.

Given $j, l \in \N$, let $\intvl{j, l} = [j, l] \cap \N $ denote the interval of integer numbers between $j$ and $l$.
The loss vector of $x$ at $\prof$ is $\lprof(x) \in \intvl{0, m - 1}^N$ that associates to each individual her “loss” resulting of selecting $x$ instead of his favorite alternative, that loss being defined as the number of alternatives that the individual prefers to $x$: $\lprof(x)_i = \card{\set{y \in \allalts \suchthat y \prefi x}}$.
%Let $\bigcup_{k \in \losses}(k^N)$ denote the set of constant loss vectors.
Given two loss vectors $l, l' \in \lvs$, we say that $l$ is strictly smaller than $l'$, $l < l'$, iff $\forall i: l_i ≤ l'_i \land l ≠ l'$. We also write $l ≤ l'$ to denote that $l$ is weakly smaller than $l'$, meaning, strictly smaller or equal. Let $\min_N \lprof(x) = \min_{i \in N} \lprof(x)_i \in \N$ and $\sum_N \lprof(x) = \sum_{i \in N} \lprof(x)_i$ denote, respectively, the minimal loss level and the sum of the losses, considering the loss vector $\lprof(x)$.

Let $\PD = \set{x \in \allalts \suchthat \exists y \text{ s.t. } \lprof(y) < \lprof(x)}$ be the Pareto-dominated alternatives in $\prof$. Let $\PE = \allalts \setminus \PD$ be the Pareto-efficient ones.

\section{Four SCRs of the literature}
Given $\prof$ and a loss level $k \in \losses$, define $U(\prof, k) = \set{x \in \allalts \suchthat \lprof(x) ≤ (k, k)}$ as the set of alternatives imposing losses not higher than $k$ for all individuals. We say that such alternatives receive unanimous support at level $k$.
Let $\rhoP = \min \set{k \in \losses \suchthat U(\prof, k) ≠ \emptyset}$ be the least loss level at which some alternative receives unanimous support. By Theorem 1 of Brams and Kilgour (2001) we have $\forall \prof \in \allprofs$, $\rhoP ≤ \ceil{\frac{m - 1}{2}}$.

Given a profile $\prof \in \allprofs$, let $H(\prof) = \set{x \in \allalts \suchthat \lprof(x) ≤ (\ceil{\frac{m - 1}{2}}, \ceil{\frac{m - 1}{2}})}$ denote the set of alternatives reaching the best half of every individual’s preference.
In concordance with the ceiling established by Theorem 1 of Brams and Kilgour (2001), we use the term “half” to mean the smallest integer $k$ that exceeds $m-k$.

\textbf{Fallback Bargaining} is the SCR $\FB$ that picks all alternatives with unanimous support at $\rhoP$: $\FB(\prof) = U(\prof, \rhoP)$. 

The \textbf{Veto-rank} rule $\VR$\footnote{\label{ft:modd} Both $\VR$ and $\SL$ are defined in Clippel et al.XXX for $m$ odd only.} is defined as follows. Each individual vetoes her worst $\floor{\frac{m - 1}{2}}$ alternatives, then the Borda winners among the non vetoed alternatives are picked: $\VR(\prof) = \argmin_{H(\prof)} \sum_N \lprof = \set{x \in H(\prof) \suchthat \forall y \in H(\prof): \sum_N \lprof(x) ≤ \sum_N \lprof(y)}$.

The \textbf{Shortlisting} rule $\SL$\footref{ft:modd} picks the best alternative of individual $1$ that is not among the worst $\floor{\frac{m - 1}{2}}$ alternatives of individual $2$, and the best alternative of $2$ that is not among the worst $\floor{\frac{m - 1}{2}}$ alternatives of $1$. Formally, $\SL(\prof) = \cup_{i \in N} (\argmin_{x \in H^i(\prof)} \lprof(x)_{\ibar})$.

The class of \textbf{Pareto-and-veto} rules, $\PVcl$, contains rules parameterized by $v_1, v_2 \in \intvl{0, m - 1}$ with $v_1 + v_2  ≤ m - 1$ where $v_i$ represents the number of alternatives vetoed by individual $i \in N$ (individuals veto the alternatives at the bottom of their preference).
Given $v_i \in \intvl{0, m - 1}$, define $a_i = m - v_i - 1 \in \intvl{0, m - 1}$ as the highest acceptable loss level for individual $i$. For $v=(v_1,v_2)$, the rule $PV^v = \cap_{i \in N}\set{x \in \allalts \suchthat \lprof(x)_i ≤ a_i} \cap \PE$ picks all alternatives in $\PE$ that no individual vetoes. 
The class $\PVcl = \set{PV^v \suchthat v_1,v_2 \in \intvl{0, m - 1} \: \text{ with } v_1+v_2\leq m-1}$ is the set of those rules.

\begin{remark}
     $\FB, \VR, \SL$ and all SCSs in $\PVcl$ are  neutral. $\FB$, $\VR$ and $\SL$ are also anonymous, while a SCR in $\PVcl$ is anonymous iff $v_1 = v_2$.
\end{remark}

\begin{figure}
        \caption{$\VR(\prof) = \set{b}, \SL(\prof) = \set{a, g}, \FB(\prof) = \set{d}$}
        \label{fig:disjoint}
        $\begin{array}{llll lll | llll ll}
                a&b&c&d&e&f&g&h&i&j&k&l&m\\
                g&h&i&d&b&j&a&c&e&f&k&l&m\\
        \end{array}$
\end{figure}

We start by observing that $\VR$, $\SL$ and $\FB$ can pick disjoint winners.
\begin{proposition}
	$\exists \prof \suchthat \VR(\prof) \cap \SL(\prof) = \emptyset, \VR(\prof) \cap \FB(\prof) = \emptyset, \SL(\prof) \cap \FB(\prof) = \emptyset$.
\end{proposition}
\begin{proof}
	\Cref{fig:disjoint} proves the statement.
\end{proof}

To discuss the relationship of $\FB$, $\SL$ and $\VR$ to members of $\PVcl$, we define a condition that we call the first half property, which mandates that the rule picks its winners among the alternatives placed in the best half of both rankings.
\begin{definition}[First half (FH)]
	$\forall \prof \in \allprofs: f(\prof) \subseteq H(\prof)$.
\end{definition}
Let $\PEcl$ denote the class of rules picking only Pareto efficient alternatives ($\forall \prof: f(\prof) \subseteq \PE$); $\FHcl$ denote the class of rules satisfying the First half property, and similarly for other properties defined in this article.

The next proposition shows that $\VR$, $\SL$ and $\FB$ satisfy FH.
\begin{proposition}
	$\VR, \SL, \FB \in \FHcl$.
\end{proposition}
\begin{proof}
	This follows immediately from the definition of $\VR$ and $\FB$. \commentOC{Do we need to prove anything for $\SL$?}
\end{proof}

\begin{proposition}
	$\PVcl ∩ \FHcl = \{PV^v \in \mathcal{PV} \mid v_i \geq \floor{\frac{m - 1}{2}} \forall i \}$
\end{proposition}
\begin{proof}
	Consider a arbitrary ordering $\prefi$ over $\allalts$, let $\prefiinv$ denote its inverse, and consider a profile $\prof = (\prefi, \prefiinv)$. Observe that $PV^v(\prof) = \set{x \in \allalts \suchthat \lprof(x) = (a_1, a_2) \text{ when } v_1+v_2=m-1}$. 
	This permits to see that $PV^v$ with $v_1+v_2=m-1$ satisfies FH iff $a_1, a_2 ≤ \ceil{\frac{m - 1}{2}}$, equivalently, iff $v_1, v_2 ≥ \floor{\frac{m - 1}{2}}$.
	\commentOC{This is probably a bit too quick.}
\end{proof}

\begin{corollary}
For $m$ odd only, $v_1=v_2=\frac{m-1}{2}$ is FH.\\
For $m$ even, there are SCRs that are FH: $v_1=v_2=\frac{m}{2}-1$; $v_1=\frac{m}{2}$, $v_2=\frac{m}{2}-1$ and $v_1=\frac{m}{2}-1$, $v_2=\frac{m}{2}$.
\end{corollary}
Considering two SCRs $f$ and $f'$, let $f \cup f'$ denote the rule $(f \cup f')(\prof) = f(\prof) \cup f'(\prof)$. 
Given any non empty class of SCRs $F$, let $\bigcup F$ denote the maximal (least resolute) SCR that can be formed by unions of rules of $F$.
Say that a class of rules $F$ is closed under union whenever $\forall f, f' \in F: f \cup f' \in F$. 
Given any non empty class of SCRs $F$ closed under union, the maximal SCR formed by unions of rules of $F$ is a member of $F$, and (evidently) is less resolute than any other member of $F$. Formally: $[f = \bigcup(F)] ⇔ [(\forall f' \in F, f' \subseteq f) \land (f \in F)]$.

\commentRS{Show that there is a unique SCR that is PE, FH and set inclusion maximal which is PVe. Te proof remarks that FH and PE is closed under union.}

\commentOC{Phrase this as: For all $m$, this union is the union of PVecl.}

\begin{proposition}
	For $m$ odd, $\bigcup(\PEcl \cap \FHcl) = \PVe$.
\end{proposition}
\begin{proof}
	%Both classes $\PEcl$ and $\FHcl$ are closed under union. Thus, $\PEcl \cap \FHcl$ is closed under union.
	Define $f$ as the rule that always picks all efficient alternatives ranked in both first halves. This is the least resolute SCR among the class $\PEcl \cap \FHcl$. And it equals $\PVe$.
	\commentOC{To be continued, I suppose. But it seems we do not need the union-closed stuff here.}
\end{proof}

In the next proposition, in fact, conjecture, we use max in the sense of set inclusion (irresoluteness).
\begin{proposition}
	$\max {(\PEcl \cap \FHcl \cap \PVcl)} = \PVecl$.
\end{proposition}

Corollary: A SCR F is FH and Pareto iff F is a subcorrespondence of PV-equalvetoes.”




\commentRS{Show also that FB, SL and VR are subcorrespondences of a PV rule iff that PV rule is PVe. Tje proposition below partly handles that}

\commentOC{Show that this is strict by using the same figure.}
\begin{proposition}
	$\SL, \VR, \FB \subset \PVe$
\end{proposition}

\commentOC{Say that PVe is the FH; SL is max max among the FH; VR maximizes the sum of gains; and FB maximizes the min gain.}


\commentRS{Below are characterizations}
PV-e iff PE, FH, MAskin monotonicity

I recall from a paper of Vincent that FB is characterized by using a Rawlsian maximin axiom plus maximality 

For SL and VR we should look at Clippel. 

\color{green} MN: no characterization of VR to the best of my knowledge \color{black}

Sertel shows (says Danilo Coelho) that the unanimous compromise set (the FB winners?) is the set of Pareto efficient candidates that maximize the worst-of welfare (welfare = nb alternatives less good than the chosen one). \color{green} MN: This is true, right? \color{black}



\section{New SCRs}
\subsection{Based on equal loss}
\begin{definition}[Equal loss (EL)]
\color{green}	$f$ picks some alternative ranked the same by everybody, if any. \color{black}
\end{definition}
\begin{definition}[Paretian equal loss (PEL)]
	$f$ must pick solely the alternative ranked the same by everybody and that is efficient, if there is one.
\end{definition}
\begin{definition}[Paretian compromism compatibility (PCC)]
	Define $\Sigma = \set{\sigma \in (\R^+)^{(\lvs)} \suchthat \sigma^{-1}(0) = \cup_{k \in \losses}(k^N)}$ be the set of all spread measures that attribute the minimal inequality measure to the constant loss vectors and none others.
	\commentOC{Could we perhaps define sigma as a weak order over the loss vectors?}
	$f$ is PCC iff there exists a spread measure $\sigma \in \Sigma$ such that $\forall \prof: f(\prof) \cap \minspread[\PE] ≠ \emptyset$.
\end{definition}

Call the dispersion of a loss vector $l$ at $\prof$ the value $d(l) = \abs{l_1 - l_2}$. 
Note that $d \in \Sigma$ and seems to coincide with multiple commonly used spread measures (to be verified). 
Thus, $(d \circ \lprof)(x) = \max\lprof(x) - \min\lprof(x)$.
\begin{definition}[Dispersion compromism]
	$\forall \prof: f(\prof) \subseteq \mindisp[\PE]$.
\end{definition}

\begin{definition}[Dispersion phobia]
	If $x$ has a strictly higher dispersion than $y$ at $\prof$, then $f$ may not pick $x$.
\end{definition}
\begin{definition}[Loss vector comparing (LVC)]
	Let $\succeq$ be a weak order on $\lvs$. The rule $f^\succeq$ selects all alternatives whose loss vectors are minimal elements of $\restr{{\succeq}}{\lprof(\alts)}$. A rule satisfies LVC iff $\exists {\succeq} \suchthat f = f^\succeq$. (This property is called preorder based in Cailloux and Endriss 2014.)
\end{definition}

\commentOC{I do not understand this.}
A stronger version of PEL is what we used in the paper with Beatrice. Fix a set SIGMA of “acceptable” equality measures and ask from F to always pick an alternative with equal loss according to some member of SIGMA. This is stronger than PEL because being “acceptable” requires that an alternative at the same rank for both players as the most equal loss.

$Paretianism ∩ EL = \emptyset$.



\begin{proposition}
VR, SL, and all rules in PV fail PEL
\end{proposition}
\begin{proposition}
FB satisfies PEL
\end{proposition}


PV, FB, SL and VR, all fail strong PEL. 

In fact, strong PEL and FH are incompatible conditions. 


What are interesting rules that satisfy this stronger version of PEL? 

As Matias suggested, can we use strong PEL to refine VR, SL or FB?

Can we identify an interesting SCR that satisfies strong PEL?

\subsection{Based on refining}
Discuss how discriminative the four rules are. ll four rules are irresolute. Do they have refinements that are anonymous and neutral? The answer is obviously affirmative for PV (as the remaining three rules do the job) but should be addressed for FB, SL and VR. 
The Theorem, stated at the end of page 4 (\hrefblue{https://link.springer.com/article/10.1007/BF00187429}{Sprumont}), is about refinements of FB. It says that FB admits refinements that satisfy the two conditions of the paper, namely CPB and SO. Moreover, any rule that satisfies CPB and SO is a refinement of FB.

We can refer to this when we address the question of refining FB, SL and VR.









\subsection{Rules we invent}
Anonymized PV rules
PV when the sum of vetoes is less than m-1
\section{Miscellaneous comments}
I proposed: [VR selects equal loss ⇒ VR selects all alternatives with "almost equal loss"], but this is contradicted by: (abc; cba).


Danilo Coelho seems to be interested in implementation of FB and shortlisting and related rules (see the \href{https://www.cmss.auckland.ac.nz/2020/06/03/online-social-choice-seminar-series/}{Online Social Choice Seminar})

\color{green}MN: Yes, I agree. By the way, in his paper, he cites a paper by Clippel with a new characterization of Fallback, we might want to cite it.\color{black}

%\bibliography{bibl}

\end{document}

