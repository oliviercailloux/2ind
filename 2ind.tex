\RequirePackage[l2tabu, orthodox]{nag}
\documentclass[version=3.21, pagesize, twoside=off, bibliography=totoc, DIV=calc, fontsize=12pt, a4paper]{scrartcl}
%Permits to copy eg x ⪰ y ⇔ v(x) ≥ v(y) from PDF to unicode data, and to search. From pdfTeX users manual. See https://tex.stackexchange.com/posts/comments/1203887.
	\input glyphtounicode
	\pdfgentounicode=1
%Latin Modern has more glyphs than Computer Modern, such as diacritical characters. fntguide commands to load the font before fontenc, to prevent default loading of cmr.
	\usepackage{lmodern}
%Encode resulting accented characters correctly in resulting PDF, permits copy from PDF.
	\usepackage[T1]{fontenc}
%UTF8 seems to be the default in recent TeX installations, but not all, see https://tex.stackexchange.com/a/370280.
	\usepackage[utf8]{inputenc}
%Provides \newunicodechar for easy definition of supplementary UTF8 characters such as → or ≤ for use in source code.
	\usepackage{newunicodechar}
%Text Companion fonts, much used together with CM-like fonts. Provides \texteuro and commands for text mode characters such as \textminus, \textrightarrow, \textlbrackdbl.
	\usepackage{textcomp}
%St Mary’s Road symbol font, used for ⟦ = \llbracket. The \SetSymbolFont command avoids spurious warnings, but also some valid ones, see https://tex.stackexchange.com/a/106719/.
	\usepackage{stmaryrd}\SetSymbolFont{stmry}{bold}{U}{stmry}{m}{n}
%Solves bug in lmodern, https://tex.stackexchange.com/a/261188; probably useful only for unusually big font sizes; and probably better to use exscale instead. Note that the authors of exscale write against this trick.
	%\DeclareFontShape{OMX}{cmex}{m}{n}{
		%<-7.5> cmex7
		%<7.5-8.5> cmex8
		%<8.5-9.5> cmex9
		%<9.5-> cmex10
	%}{}
	%\SetSymbolFont{largesymbols}{normal}{OMX}{cmex}{m}{n}
%More symbols (such as \sum) available in bold version, see https://github.com/latex3/latex2e/issues/71. In article mode (but not in presentation mode), also hides some potentially useful warnings such as when using $\bm{\llbracket}$, see stmaryrd in this document (not sure why).
	\DeclareFontShape{OMX}{cmex}{bx}{n}{%
	   <->sfixed*cmexb10%
	   }{}
	\SetSymbolFont{largesymbols}{bold}{OMX}{cmex}{bx}{n}
%For small caps also in italics, see https://tex.stackexchange.com/questions/32942/italic-shape-needed-in-small-caps-fonts, https://tex.stackexchange.com/questions/284338/italic-small-caps-not-working.
	\usepackage{slantsc}
	\AtBeginDocument{%
		%“Since nearly no font family will contain real italic small caps variants, the best approach is to substitute them by slanted variants.” -- slantsc doc
		%\DeclareFontShape{T1}{lmr}{m}{scit}{<->ssub*lmr/m/scsl}{}%
		%There’s no bold small caps in Latin Modern, we switch to Computer Modern for bold small caps, see https://tex.stackexchange.com/a/22241
		%\DeclareFontShape{T1}{lmr}{bx}{sc}{<->ssub*cmr/bx/sc}{}%
		%\DeclareFontShape{T1}{lmr}{bx}{scit}{<->ssub*cmr/bx/scsl}{}%
	}
%Warn about missing characters.
	\tracinglostchars=2
%Nicer tables: provides \toprule, \midrule, \bottomrule.
	\usepackage{booktabs}
%For new column type X which stretches; can be used together with booktabs, see https://tex.stackexchange.com/a/97137. “tabularx modifies the widths of the columns, whereas tabular* modifies the widths of the inter-column spaces.” Loads array.
	%\usepackage{tabularx}
%math-mode version of "l" column type. Requires \usepackage{array}.
	%\usepackage{array}
	%\newcolumntype{L}{>{$}l<{$}}
%Provides \xpretocmd and loads etoolbox which provides \apptocmd, \patchcmd, \newtoggle… Also loads xparse, which provides \NewDocumentCommand and similar commands intended as replacement of \newcommand in LaTeX3 for defining commands (see https://tex.stackexchange.com/q/98152 and https://github.com/latex3/latex2e/issues/89).
	\usepackage{xpatch}
%for \nexists and because it is a basic package nowadays, see https://tex.stackexchange.com/q/539592/.
	\usepackage{amssymb}
%loads and fixes some bugs in amsmath (a basic, mandatory package nowadays, see Grätzer, More Math Into LaTeX) and provides \DeclarePairedDelimiter. I recommend \begin{equation}, which allows numbering, rather than \[ (and $$ should be avoided), see https://tex.stackexchange.com/questions/503. Relatedly, do not use the displaymath environment: use equation. Do not use the eqnarray environment: use align. This improves spacing. (See l2tabu or amsldoc.)
	\usepackage{mathtools}
%Package frenchb asks to load natbib before babel-french. Package hyperref asks to load natbib before hyperref.
	\usepackage{natbib}

\newtoggle{LCpres}
	\newtoggle{LCart}
	\newtoggle{LCposter}
	\makeatletter
	\@ifclassloaded{beamer}{
		\toggletrue{LCpres}
		\togglefalse{LCart}
		\togglefalse{LCposter}
		\wlog{Presentation mode}
	}{
		\@ifclassloaded{tikzposter}{
			\toggletrue{LCposter}
			\togglefalse{LCpres}
			\togglefalse{LCart}
			\wlog{Poster mode}
		}{
			\toggletrue{LCart}
			\togglefalse{LCpres}
			\togglefalse{LCposter}
			\wlog{Article mode}
		}
	}
	\makeatother%

%Language options ([french, english]) should be on the document level (last is main); except with tikzposter: put [french, english] options next to \usepackage{babel} to avoid warning. beamer uses the \translate command for the appendix: omitting babel results in a warning, see https://github.com/josephwright/beamer/issues/449. Babel also seems required for \refname.
	\iftoggle{LCpres}{
		\usepackage{babel}
	}{
	}
	%\frenchbsetup{AutoSpacePunctuation=false}
%https://ctan.org/pkg/amsmath recommends ntheorem, which supersedes amsthm, which corrects the spacing of proclamations and allows for theoremstyle, but I decided to switch to amsthm with thmtools (mentioned in amsthm doc) because ntheorem “seems essentially unmaintaned and has severe problems”, see https://tex.stackexchange.com/q/535950. Must be loaded after amsmath (from amsthm doc).
		\usepackage{amsthm}
		\usepackage{thmtools}
%listings (1.7) does not allow multi-byte encodings. listingsutf8 works around this only for characters that can be represented in a known one-byte encoding and only for \lstinputlisting. Other workarounds: use literate mechanism; or escape to LaTeX (but breaks alignment).
	%\usepackage{listings}
	%\lstset{tabsize=2, basicstyle=\ttfamily, escapechar=§, literate={é}{{\'e}}1}
%I favor acro over acronym because the former is more recently updated (2018 VS 2015 at time of writing); has a longer user manual (about 40 pages VS 6 pages if not counting the example and implementation parts); has a command for capitalization; and acronym suffers a nasty bug when ac used in section, see https://tex.stackexchange.com/q/103483 (though this might be the fault of the silence package and might be solved in more recent versions, I do not know) and from a bug when used with cleveref, see https://tex.stackexchange.com/q/71364. However, loading it makes compilation time (one pass on this template) go from 0.6 to 1.4 seconds, see https://bitbucket.org/cgnieder/acro/issues/115.
	\usepackage{acro}
	%“All options of acro that have not been mentioned in section 4.1 have to be set up… with… \acsetup{…}” -- acro package doc, cited by the Overleaf support (thanks to them!)
	\acsetup{single}
	\DeclareAcronym{AMCD}{short=AMCD, long={Aide Multicritère à la Décision}}
\DeclareAcronym{AHP}{short=AHP, long={Analytic Hierarchy Process}}
\DeclareAcronym{AR}{short=AR, long={Argumentative Recommender}}
\DeclareAcronym{DA}{short=DA, long={Decision Analysis}}
\DeclareAcronym{DJ}{short=DJ, long={Deliberated Judgment}}
\DeclareAcronym{DM}{short=DM, long={Decision Maker}}
\DeclareAcronym{DP}{short=DP, long={Deliberated Preference}}
\DeclareAcronym{MAVT}{short=MAVT, long={Multiple Attribute Value Theory}}
\DeclareAcronym{MCDA}{short=MCDA, long={Multicriteria Decision Aid}}
\DeclareAcronym{MIP}{short=MIP, long={Mixed Integer Program}}
\DeclareAcronym{SEU}{short=SEU, long={Subjective Expected Utility}}


\iftoggle{LCpres}{
	%I favor fmtcount over nth because it is loaded by datetime anyway; and fmtcount warns about possible conflicts when loaded after nth (“\ordinal already defined use \FCordinal instead”). See also https://english.stackexchange.com/questions/93008.
	\usepackage{fmtcount}
	%For nice input of date of presentation. Must be loaded after the babel package. Has possible problems with srcletter: https://golatex.de/verwendung-von-babel-und-datetime-in-scrlttr2-schlaegt-fehlt-t14779.html.
	\usepackage[nodayofweek]{datetime}
}{
}
%For presentations, Beamer implicitely uses the pdfusetitle option. autonum doc mandates option hypertexnames=false. I want to highlight links only if necessary for the reader to recognize it as a link, to reduce distraction. In presentations, this is already taken care of by beamer (https://tex.stackexchange.com/a/262014). If using colorlinks=true in a presentation, see https://tex.stackexchange.com/q/203056. Crashes the first compilation with tikzposter, just compile again and the problem disappears, see https://tex.stackexchange.com/q/254257.
\makeatletter
\iftoggle{LCpres}{
	\usepackage{hyperref}
}{
	\usepackage[hypertexnames=false, pdfusetitle, linkbordercolor={1 1 1}, citebordercolor={1 1 1}, urlbordercolor={1 1 1}]{hyperref}
	%https://tex.stackexchange.com/a/466235
	\pdfstringdefDisableCommands{%
		\let\thanks\@gobble
	}
}
\makeatother
%urlbordercolor is used both for \url and \doi, which I think shouldn’t be colored, and for \href, thus might want to color manually when required. Requires xcolor.
	\NewDocumentCommand{\hrefblue}{mm}{\textcolor{blue}{\href{#1}{#2}}}
%hyperref doc says: “Package bookmark replaces hyperref’s bookmark organization by a new algorithm (...) Therefore I recommend using this package”.
	\usepackage{bookmark}
%Need to invoke hyperref explicitly to link to line numbers: \hyperlink{lintarget:mylinelabel}{\ref*{lin:mylinelabel}}, with \ref* to disable automatic link. Also see https://tex.stackexchange.com/q/428656 for referencing lines from another document.
	%\usepackage{lineno}
	%\NewDocumentCommand{\llabel}{m}{\hypertarget{lintarget:#1}{}\linelabel{lin:#1}}
	%\setlength\linenumbersep{9mm}
%For complex authors blocks. Seems like authblk wants to be later than hyperref, but sooner than silence. See https://tex.stackexchange.com/q/475513 for the patch to hyperref pdfauthor.
	\ExplSyntaxOn
	\seq_new:N \g_oc_hrauthor_seq
	\NewDocumentCommand{\addhrauthor}{m}{
		\seq_gput_right:Nn \g_oc_hrauthor_seq { #1 }
	}
	\NewExpandableDocumentCommand{\hrauthor}{}{
		\seq_use:Nn \g_oc_hrauthor_seq {,~}
	}
	\ExplSyntaxOff
	{
		\catcode`#=11\relax
		\gdef\fixauthor{\xpretocmd{\author}{\addhrauthor{#2}}{}{}}%
	}
	\iftoggle{LCart}{
		\usepackage{authblk}
		\renewcommand\Affilfont{\small}
		\fixauthor
		\AtBeginDocument{
		    \hypersetup{pdfauthor={\hrauthor}}
		}
	}{
	}
%I do not use floatrow, because it requires an ugly hack for proper functioning with KOMA script (see scrhack doc). Instead, the following command centers all floats (using \centering, as the center environment adds space, http://texblog.net/latex-archive/layout/center-centering/), and I manually place my table captions above and figure captions below their contents (https://tex.stackexchange.com/a/3253).
	\makeatletter
	\g@addto@macro\@floatboxreset\centering
	\makeatother
%Permits to customize enumeration display and references
	%\nottoggle{LCpres}{
		%\usepackage{enumitem} %follow list environments by a string to customize enumeration, example: \begin{description}[itemindent=8em, labelwidth=!] or \begin{enumerate}[label=({\roman*}), ref={\roman*}].
	%}{
	%}
%Provides \Centering, \RaggedLeft, and \RaggedRight and environments Center, FlushLeft, and FlushRight, which allow hyphenation. With tikzposter, seems to cause 1=1 to be printed in the middle of the poster.
	%\usepackage{ragged2e}
%To typeset units by closely following the “official” rules.
	%\usepackage[strict]{siunitx}
%Turns the doi provided by some bibliography styles into URLs.
	\usepackage{doi}
%Makes sure upper case greek letters are italic as well.
	\usepackage{fixmath}
%Provides \mathbb; obsoletes latexsym (see http://tug.ctan.org/macros/latex/base/latexsym.dtx). Relatedly, \usepackage{eucal} to change the mathcal font and \usepackage[mathscr]{eucal} (apparently equivalent to \usepackage[mathscr]{euscript}) to supplement \mathcal with \mathscr. This last option is not very useful as both fonts are similar, and the intent of the authors of eucal was to provide a replacement to mathcal (see doc euscript). Also provides \mathfrak for supplementary letters.
	\usepackage{amsfonts}
%Provides a beautiful (IMHO) \mathscr and really different than \mathcal, for supplementary uppercase letters. But there is no bold version. Alternative: mathrsfs (more slanted), but when used with tikzposter, it warns about size substitution, see https://tex.stackexchange.com/q/495167.
	\usepackage[scr]{rsfso}
%Multiple means to produce bold math: \mathbf, \boldmath (defined to be \mathversion{bold}, see fntguide), \pmb, \boldsymbol (all legacy, from LaTeX base and AMS), \bm (the most recommended one), \mathbold from package fixmath (I don’t see its advantage over \boldsymbol).
%“The \boldsymbol command is obtained preferably by using the bm package, which provides a newer, more powerful version than the one provided by the amsmath package. Generally speaking, it is ill-advised to apply \boldsymbol to more than one symbol at a time.” — AMS Short math guide. “If no bold font appears to be available for a particular symbol, \bm will use ‘poor man’s bold’” — bm. It is “best to load the package after any packages that define new symbol fonts” – bm. bm defines \boldsymbol as synonym to \bm. \boldmath accesses the correct font if it exists; it is used by \bm when appropriate. See https://tex.stackexchange.com/a/10643 and https://github.com/latex3/latex2e/issues/71 for some difficulties with \bm.
	\usepackage{bm}
	\nottoggle{LCpres}{
	%Provides \cref. Unfortunately, cref fails when the language is French and referring to a label whose name contains a colon (https://tex.stackexchange.com/q/83798). Use \cref{sec\string:intro} to work around this. cleveref should go “laster” than hyperref.
		\usepackage[capitalise]{cleveref}
	}{
	}
	\nottoggle{LCposter}{
	%Equations get numbers iff they are referenced. Loading order should be “amsmath → hyperref → cleveref → autonum”, according to autonum doc. Use this in preference to the showonlyrefs option from mathtools, see https://tex.stackexchange.com/q/459918 and autonum doc. See https://tex.stackexchange.com/a/285953 for the etex line. Incompatible with my version of tikzposter (produces “! Improper \prevdepth”). This removes the starred versions, such as equation*. Unfortunately, this prevents using \qedhere in an equation ending a proof, see https://tex.stackexchange.com/q/133358/.
		\expandafter\def\csname ver@etex.sty\endcsname{3000/12/31}\let\globcount\newcount
		\usepackage{autonum}
	}{
	}
%Also loaded by tikz.
	\usepackage{xcolor}
\iftoggle{LCpres}{
	\usepackage{tikz}
	%\usetikzlibrary{babel, matrix, fit, plotmarks, calc, trees, shapes.geometric, positioning, plothandlers, arrows, shapes.multipart}
}{
}
%Vizualization, on top of TikZ
	%\usepackage{pgfplots}
	%\pgfplotsset{compat=1.14}
\usepackage{graphicx}
	\graphicspath{{graphics/}}

%Provides \printlength{length}, useful for debugging.
	%\usepackage{printlen}
	%\uselengthunit{mm}

\iftoggle{LCpres}{
	\usepackage{appendixnumberbeamer}
	%I have yet to see anyone actually use these navigation symbols; let’s disable them
	\setbeamertemplate{navigation symbols}{} 
	\usepackage{preamble/beamerthemeParisFrance}
	\setcounter{tocdepth}{10}
}{
}

%Requires package xcolor.
\definecolor{ao(english)}{rgb}{0.0, 0.5, 0.0}
\NewDocumentCommand{\commentOC}{m}{\textcolor{blue}{\small$\big[$OC: #1$\big]$}}
%Requires package babel and option [french]. According to babel doc, need two braces around \selectlanguage to make the changes really local.
\NewDocumentCommand{\commentOCf}{m}{\textcolor{blue}{{\small\selectlanguage{french}$\big[$OC : #1$\big]$}}}
\NewDocumentCommand{\commentRS}{m}{\textcolor{red}{\small$\big[$RS: #1$\big]$}}
\NewDocumentCommand{\commentMN}{m}{\textcolor{ao(english)}{\small$\big[$MN: #1$\big]$}}

\bibliographystyle{abbrvnat}
\NewDocumentCommand{\possessivecite}{mO{}}{\citeauthor{#1}’s \citeyearpar[#2]{#1}}

%https://tex.stackexchange.com/a/467188, https://tex.stackexchange.com/a/36088 - uncomment if one of those symbols is used.
%\DeclareFontFamily{U} {MnSymbolD}{}
%\DeclareFontShape{U}{MnSymbolD}{m}{n}{
%  <-6> MnSymbolD5
%  <6-7> MnSymbolD6
%  <7-8> MnSymbolD7
%  <8-9> MnSymbolD8
%  <9-10> MnSymbolD9
%  <10-12> MnSymbolD10
%  <12-> MnSymbolD12}{}
%\DeclareFontShape{U}{MnSymbolD}{b}{n}{
%  <-6> MnSymbolD-Bold5
%  <6-7> MnSymbolD-Bold6
%  <7-8> MnSymbolD-Bold7
%  <8-9> MnSymbolD-Bold8
%  <9-10> MnSymbolD-Bold9
%  <10-12> MnSymbolD-Bold10
%  <12-> MnSymbolD-Bold12}{}
%\DeclareSymbolFont{MnSyD} {U} {MnSymbolD}{m}{n}
%\DeclareMathSymbol{\ntriplesim}{\mathrel}{MnSyD}{126}
%\DeclareMathSymbol{\nlessgtr}{\mathrel}{MnSyD}{192}
%\DeclareMathSymbol{\ngtrless}{\mathrel}{MnSyD}{193}
%\DeclareMathSymbol{\nlesseqgtr}{\mathrel}{MnSyD}{194}
%\DeclareMathSymbol{\ngtreqless}{\mathrel}{MnSyD}{195}
%\DeclareMathSymbol{\nlesseqgtrslant}{\mathrel}{MnSyD}{198}
%\DeclareMathSymbol{\ngtreqlessslant}{\mathrel}{MnSyD}{199}
%\DeclareMathSymbol{\npreccurlyeq}{\mathrel}{MnSyD}{228}
%\DeclareMathSymbol{\nsucccurlyeq}{\mathrel}{MnSyD}{229}
%\DeclareFontFamily{U} {MnSymbolA}{}
%\DeclareFontShape{U}{MnSymbolA}{m}{n}{
%  <-6> MnSymbolA5
%  <6-7> MnSymbolA6
%  <7-8> MnSymbolA7
%  <8-9> MnSymbolA8
%  <9-10> MnSymbolA9
%  <10-12> MnSymbolA10
%  <12-> MnSymbolA12}{}
%\DeclareFontShape{U}{MnSymbolA}{b}{n}{
%  <-6> MnSymbolA-Bold5
%  <6-7> MnSymbolA-Bold6
%  <7-8> MnSymbolA-Bold7
%  <8-9> MnSymbolA-Bold8
%  <9-10> MnSymbolA-Bold9
%  <10-12> MnSymbolA-Bold10
%  <12-> MnSymbolA-Bold12}{}
%\DeclareSymbolFont{MnSyA} {U} {MnSymbolA}{m}{n}
%%Rightwards wave arrow: ↝. Alternative: \rightsquigarrow from amssymb, but it’s uglier
%\DeclareMathSymbol{\rightlsquigarrow}{\mathrel}{MnSyA}{160}

%03B3 Greek Small Letter Gamma
\newunicodechar{γ}{\gamma}
%03B4 Greek Small Letter Delta
\newunicodechar{δ}{\delta}
%2115 Double-Struck Capital N
\newunicodechar{ℕ}{\mathbb{N}}
%211D Double-Struck Capital R
\newunicodechar{ℝ}{\mathbb{R}}
%21CF Rightwards Double Arrow with Stroke
\newunicodechar{⇏}{\nRightarrow}
%21D2 Rightwards Double Arrow
\newunicodechar{⇒}{\ensuremath{\Rightarrow}}
%21D4 Left Right Double Arrow
\newunicodechar{⇔}{\Leftrightarrow}
%21DD Rightwards Squiggle Arrow
\newunicodechar{⇝}{\rightsquigarrow}
%2205 Empty Set
\newunicodechar{∅}{\emptyset}
%2212 Minus Sign
\newunicodechar{−}{\ifmmode{-}\else\textminus\fi}
%2227 Logical And
\newunicodechar{∧}{\land}
%2228 Logical Or
\newunicodechar{∨}{\lor}
%2229 Intersection
\newunicodechar{∩}{\cap}
%222A Union
\newunicodechar{∪}{\cup}
%2260 Not Equal To (handy also as text in informal writing)
\newunicodechar{≠}{\ensuremath{\neq}}
%2264 Less-Than or Equal To
\newunicodechar{≤}{\leq}
%2265 Greater-Than or Equal To
\newunicodechar{≥}{\geq}
%2270 Neither Less-Than nor Equal To
\newunicodechar{≰}{\nleq}
%2271 Neither Greater-Than nor Equal To
\newunicodechar{≱}{\ngeq}
%2272 Less-Than or Equivalent To
\newunicodechar{≲}{\lesssim}
%2273 Greater-Than or Equivalent To
\newunicodechar{≳}{\gtrsim}
%2274 Neither Less-Than nor Equivalent To – also, from MnSymbol: \nprecsim, a more exact match to the Unicode symbol; and \npreccurlyeq, too small
\newunicodechar{≴}{\not\preccurlyeq}
%2275 Neither Greater-Than nor Equivalent To
\newunicodechar{≵}{\not\succcurlyeq}
%2279 Neither Greater-Than nor Less-Than – requires MnSymbol; also \nlessgtr from txfonts/pxfonts, \ngtreqless from MnSymbol (but much higher), \ngtrless from MnSymbol (a more exact match to the Unicode symbol); for incomparability (not matching this Unicode symbol), may also consider \ntriplesim from MnSymbol,\nparallelslant from fourier, \between from mathabx, or ⋈
\newunicodechar{≹}{\ngtreqlessslant}
%227A Precedes
\newunicodechar{≺}{\prec}
%227B Succeeds
\newunicodechar{≻}{\succ}
%227C Precedes or Equal To
\newunicodechar{≼}{\preccurlyeq}
%227D Succeeds or Equal To
\newunicodechar{≽}{\succcurlyeq}
%227E Precedes or Equivalent To
\newunicodechar{≾}{\precsim}
%227F Succeeds or Equivalent To
\newunicodechar{≿}{\succsim}
%2280 Does Not Precede
\newunicodechar{⊀}{\nprec}
%2281 Does Not Succeed
\newunicodechar{⊁}{\nsucc}
%2286
\newunicodechar{⊆}{\subseteq}
%22B2 Normal Subgroup Of – using \vartriangleleft from amsfonts, which goes well with \trianglelefteq, \ntriangleright, and so on, also from amsfonts; another possibility is \lhd from latexsym, which seems visually equivalent to \vartriangleleft from amsfonts; latexsym also has ⊴=\unlhd, but doesn’t have a symbol for ⊴. Other related symbols: \triangleleft from latesym package is too small; fdsymbol provides \triangleleft=\medtriangleleft and \vartriangleleft=\smalltriangleleft; MnSymbol provides \medtriangleleft and \vartriangleleft=\lessclosed=\lhd which are smaller than \vartriangleleft from amsfont; \vartriangleleft from mathabx (p. 67), looks different (wider); also \vartriangleleft from boisik (p. 69) looks still different; \vartriangleleft=\lhd from stix are smaller. Oddly enough, \triangleright appears as the LMMathItalic12-Regular font whereas \rhd appears as LASY10 and \vartriangleright appears as MSAM10.
\newunicodechar{⊲}{\vartriangleleft}
%22B3 Contains as Normal Subgroup (also: 25B7 White right-pointing triangle or 25B9 White right-pointing small triangle)
\newunicodechar{⊳}{\vartriangleright}
%22B4 Normal Subgroup of or Equal To
\newunicodechar{⊴}{\trianglelefteq}
%22B5 Contains as Normal Subgroup or Equal To
\newunicodechar{⊵}{\trianglerighteq}
%22C8 Bowtie
\newunicodechar{⋈}{\bowtie}
%22EA Not Normal Subgroup Of
\newunicodechar{⋪}{\ntriangleleft}
%22EB Does Not Contain As Normal Subgroup
\newunicodechar{⋫}{\ntriangleright}
%22EC Not Normal Subgroup of or Equal To
\newunicodechar{⋬}{\ntrianglelefteq}
%22ED Does Not Contain as Normal Subgroup or Equal
\newunicodechar{⋭}{\ntrianglerighteq}
%25A1 White Square
\newunicodechar{□}{\Box}
%27E6 Mathematical Left White Square Bracket – requires stmaryrd (alternative: \text{\textlbrackdbl}, but ugly if used in an italicized text such as a theorem)
\newunicodechar{⟦}{\llbracket}
%27E7 Mathematical Right White Square Bracket
\newunicodechar{⟧}{\rrbracket}
%27FC Long Rightwards Arrow from Bar
\newunicodechar{⟼}{\longmapsto}
%2AB0 Succeeds Above Single-Line Equals Sign
\newunicodechar{⪰}{\succeq}
%301A Left White Square Bracket
\newunicodechar{〚}{\textlbrackdbl}
%301B Right White Square Bracket
\newunicodechar{〛}{\textrbrackdbl}
%→ is defined by default as \textrightarrow, which is invalid in math mode. Same thing for the three other commands. Using \DeclareUnicodeCharacter instead of \newunicodechar because the latter warns about the previous definition.
%→ Rightwards Arrow
\DeclareUnicodeCharacter{2192}{\ifmmode\rightarrow\else\textrightarrow\fi}
%¬ Not Sign
\DeclareUnicodeCharacter{00AC}{\ifmmode\lnot\else\textlnot\fi}
%… Horizontal Ellipsis
\DeclareUnicodeCharacter{2026}{\ifmmode\dots\else\textellipsis\fi}
%× Multiplication Sign
\DeclareUnicodeCharacter{00D7}{\ifmmode\times\else\texttimes\fi}
%Permits to really obtain a straight quote when typing a straight quote; potentially dangerous, see https://tex.stackexchange.com/a/521999
\catcode`\'=\active
\DeclareUnicodeCharacter{0027}{\ifmmode^\prime\else\textquotesingle\fi}


\NewDocumentCommand{\R}{}{ℝ}
\NewDocumentCommand{\N}{}{ℕ}
%\mathscr is rounder than \mathcal.
\NewDocumentCommand{\powerset}{m}{\mathscr{P}(#1)}
%Powerset without zero.
\NewDocumentCommand{\powersetz}{m}{\mathscr{P}^*(#1)}
%https://tex.stackexchange.com/a/45732, works within both \set and \set*, same spacing than \mid (https://tex.stackexchange.com/a/52905).
\NewDocumentCommand{\suchthat}{}{\;\ifnum\currentgrouptype=16 \middle\fi|\;}
%Integer interval.
\NewDocumentCommand{\intvl}{m}{⟦#1⟧}
%Allows for \abs and \abs*, which resizes the delimiters.
\DeclarePairedDelimiter\abs{\lvert}{\rvert}
\DeclarePairedDelimiter\card{\lvert}{\rvert}
\DeclarePairedDelimiter\floor{\lfloor}{\rfloor}
\DeclarePairedDelimiter\ceil{\lceil}{\rceil}
%Perhaps should use U+2016 ‖ DOUBLE VERTICAL LINE here?
\DeclarePairedDelimiter\norm{\lVert}{\rVert}
%From mathtools. Better than using the package braket because braket introduces possibly undesirable space. Then: \begin{equation}\set*{x \in \R^2 \suchthat \norm{x}<5}\end{equation}.
\DeclarePairedDelimiter\set{\{}{\}}
\DeclareMathOperator*{\argmax}{arg\,max}
\DeclareMathOperator*{\argmin}{arg\,min}

%UTR #25: Unicode support for mathematics recommend to use the straight form of phi (by default, given by \phi) rather than the curly one (by default, given by \varphi), and thus use \phi for the mathematical symbol and not \varphi. I however prefer the curly form because the straight form is too easy to mix up with the symbol for empty set.
\let\phi\varphi

%The amssymb solution.
%\NewDocumentCommand{\restr}{mm}{{#1}_{\restriction #2}}
%Another acceptable solution.
%\NewDocumentCommand{\restr}{mm}{{#1|}_{#2}}
%https://tex.stackexchange.com/a/278631; drawback being that sometimes the text collides with the line below.
\NewDocumentCommand\restr{mm}{#1\raisebox{-.5ex}{$|$}_{#2}}


%Decision Theory
\NewDocumentCommand{\allalts}{}{\mathcal{A}}
\NewDocumentCommand{\allcrits}{}{\mathscr{C}}
\NewDocumentCommand{\alts}{}{A}
\NewDocumentCommand{\dm}{}{i}
\NewDocumentCommand{\allF}{}{\mathscr{F}}
\NewDocumentCommand{\allvoters}{}{\mathscr{N}}
\NewDocumentCommand{\voters}{}{N}
\NewDocumentCommand{\allprofs}{}{\linors^N}
\NewDocumentCommand{\prof}{}{\bm{P}}
\NewDocumentCommand{\lprof}{}{\lambda_{\bm{P}}}
\NewDocumentCommand{\linors}{}{\mathscr{L}(\allalts)}
%Thanks to https://tex.stackexchange.com/q/154549
	%\makeatletter
	%\def\@myRgood@#1#2{\mathrel{R^X_{#2}}}
	%\def\myRgood{\@ifnextchar_{\@myRgood@}{\mathrel{R^X}}}
	%\makeatother
\NewDocumentCommand{\pref}{}{\succ}
\NewDocumentCommand{\prefi}{O{i}}{\succ_{#1}}
\NewDocumentCommand{\prefiinv}{O{i}}{\succ_{#1}^{-1}}
\NewDocumentCommand{\ibar}{}{\overline{i}}

\NewDocumentCommand{\lvs}{}{\intvl{0, m - 1}^N}
\NewDocumentCommand{\losses}{}{\intvl{0, m - 1}}
\NewDocumentCommand{\PD}{}{\mathit{PD}(\prof)}
\NewDocumentCommand{\PE}{}{\mathit{PE}(\prof)}

%Rules
\NewDocumentCommand{\rhoP}{}{\rho_{\prof}}
\NewDocumentCommand{\minspread}{O{A}}{\min_{#1}(\sigma \circ \lambda_{\bm{P}})}
\NewDocumentCommand{\mindisp}{O{A}}{\min_{#1}(d \circ \lambda_{\bm{P}})}
\NewDocumentCommand{\FB}{}{\mathit{FB}}
\NewDocumentCommand{\VR}{}{\mathit{VR}}
\NewDocumentCommand{\SL}{}{\mathit{SL}}
\NewDocumentCommand{\PVv}{O{v}}{\mathit{PV}^{#1}}
\NewDocumentCommand{\PVef}{}{\mathit{PV}^{\floor{\frac{m - 1}{2}}}}%f for first
\NewDocumentCommand{\PVes}{}{\mathit{PV}^{\ceil{\frac{m - 1}{2}}}}%s for second
\NewDocumentCommand{\PVe}{}{\mathit{PV^=}}%egalitarian distribution

%Classes
\NewDocumentCommand{\PVcl}{}{\mathcal{PV}}
\NewDocumentCommand{\PVbcl}{}{\mathcal{PV}^b}
\NewDocumentCommand{\PVecl}{}{\mathcal{PV}^=}%egalitarian distribution
\NewDocumentCommand{\PEcl}{}{\mathcal{PE}}
\NewDocumentCommand{\FHcl}{}{\mathcal{FH}}
\NewDocumentCommand{\VCcl}{}{\mathcal{VC}}
\NewDocumentCommand{\VCecl}{}{\mathcal{VC^=}}
\NewDocumentCommand{\ELcl}{}{\mathcal{EL}}


%\NewDocumentCommand{\tikzmark}{m}{%
	\tikz[overlay, remember picture, baseline=(#1.base)] \node (#1) {};%
}

\newlength{\GraphsDNodeSep}
\setlength{\GraphsDNodeSep}{7mm}
\tikzset{/GraphsD/dot/.style={
	shape=circle, fill=black, inner sep=0, minimum size=1mm
}}

% MCDA Drawing Sorting
\newlength{\MCDSCatHeight}
\setlength{\MCDSCatHeight}{6mm}
\newlength{\MCDSAltHeight}
\setlength{\MCDSAltHeight}{4mm}
%separation between two vertical alts
\newlength{\MCDSAltSep}
\setlength{\MCDSAltSep}{2mm}
\newlength{\MCDSCatWidth}
\setlength{\MCDSCatWidth}{3cm}
\newlength{\MCDSAltWidth}
\setlength{\MCDSAltWidth}{2.5cm}
\newlength{\MCDSEvalRowHeight}
\setlength{\MCDSEvalRowHeight}{6mm}
\newlength{\MCDSAltsToCatsSep}
\setlength{\MCDSAltsToCatsSep}{1.5cm}
\newcounter{MCDSNbAlts}
\newcounter{MCDSNbCats}
\newlength{\MCDSArrowDownOffset}
\setlength{\MCDSArrowDownOffset}{0mm}
\tikzset{/MCD/S/alt/.style={
	shape=rectangle, draw=black, inner sep=0, minimum height=\MCDSAltHeight, minimum width=\MCDSAltWidth
}}
\tikzset{/MCD/S/pref/.style={
	shape=ellipse, draw=gray, thick
}}
\tikzset{/MCD/S/cat/.style={
	shape=rectangle, draw=black, inner sep=0, minimum height=\MCDSCatHeight, minimum width=\MCDSCatWidth
}}
\tikzset{/MCD/S/evals matrix/.style={
	matrix, row sep=-\pgflinewidth, column sep=-\pgflinewidth, nodes={shape=rectangle, draw=black, inner sep=0mm, text depth=0.5ex, text height=1em, minimum height=\MCDSEvalRowHeight, minimum width=12mm}, nodes in empty cells, matrix of nodes, inner sep=0mm, outer sep=0mm, row 1/.style={nodes={draw=none, minimum height=0em, text height=, inner ysep=1mm}}
}}

%Git
\newlength{\GitDCommitSep}
\setlength{\GitDCommitSep}{13mm}
\tikzset{/GitD/commit/.style={
	shape=rectangle, draw, minimum width=4em, minimum height=0.6cm
}}
\tikzset{/GitD/branch/.style={
	shape=ellipse, draw, red
}}
\tikzset{/GitD/head/.style={
	shape=ellipse, draw, fill=yellow
}}

%Social Choice
\tikzset{/SCD/profile matrix/.style={
	matrix of math nodes, column sep=3mm, row sep=2mm, nodes={inner sep=0.5mm, anchor=base}
}}
\tikzset{/SCD/rank-profile matrix/.style={
	matrix of math nodes, column sep=3mm, row sep=2mm, nodes={anchor=base}, column 1/.style={nodes={inner sep=0.5mm}}, row 1/.style={nodes={inner sep=0.5mm}}
}}
\tikzset{/SCD/rank-vector/.style={
	draw, rectangle, inner sep=0, outer sep=1mm
}}
\tikzset{/SCD/isolated rank-vector/.style={
	draw, matrix of math nodes, column sep=3mm, inner sep=0, matrix anchor=base, nodes={anchor=base, inner sep=.33em}, ampersand replacement=\&
}}

% GUI
\tikzset{/GUID/button/.style={
	rectangle, very thick, rounded corners, draw=black, fill=black!40%, top color=black!70, bottom color=white
}}

% Logger objects
\tikzset{/loggerD/main/.style={
	shape=rectangle, draw=black, inner sep=1ex, minimum height=7mm
}}
\tikzset{/loggerD/helper/.style={
	shape=rectangle, draw=black, dashed, minimum height=7mm
}}
\tikzset{/loggerD/helper line/.style={
	<->, draw, dotted
}}

% Beliefs
\tikzset{/BeliefsD/attacker/.style={
	shape=rectangle, draw, minimum size=8mm
}}
\tikzset{/BeliefsD/supporter/.style={
	shape=circle, draw
}}



%I find these settings useful in draft mode. Should be removed for final versions.
	%Which line breaks are chosen: accept worse lines, therefore reducing risk of overfull lines. Default = 200.
		\tolerance=2000
	%Accept overfull hbox up to...
		\hfuzz=2cm
	%Reduces verbosity about the bad line breaks.
		\hbadness 5000
	%Reduces verbosity about the underful vboxes.
		\vbadness=1300

\title{Two principles for two-person social choice}
\author{Name}
%\author{Olivier Cailloux}
\affil{Université Paris-Dauphine, PSL Research University, CNRS, LAMSADE, 75016 PARIS, FRANCE\\
%	\href{mailto:olivier.cailloux@dauphine.fr}{olivier.cailloux@dauphine.fr}
}
%\author{Name3}
%\affil{Affil2}
\hypersetup{
	pdfsubject={},
	pdfkeywords={},
}

\begin{document}
\maketitle

\section{Introduction}
\label{sec:intro}


Two-person discrete social choice models allow a specific interpretation of collective decision making: Bargaining over a finite set of alternatives. Since the seminal model of Nash (1950), for a long time, bargaining problems were formulated over a convex set of alternatives. However, there are many instances where bargaining takes place over a finite set of alternatives. Thus, this simplifying assumption of Nash (1950) excludes several real-life situations. 

Mariotti (1998) is among the first to relax this assumption by characterizing the Nash solution for a finite set of alternatives. His approach is followed by Nagahisa and Tanaka (2002) who, again in a finite setting, characterize the solution of Kalai and Smorodinsky (1975). Both characterizations are built in a cardinal framework. 
 
An ordinal framework of two-person finite bargaining problems is presented by Brams and Kilgour (2001) who introduce and analyze an ordinal solution, namely \textit{fallback bargaining}, that is based on compromising where each of the two bargainers begins by claiming the best outcome with respect to his ranking of alternatives. When the claims of the two bargainers differ, they continue by falling back, in lockstep, to lower ranked alternatives until a mutually (hence unanimously) agreed outcome is found. As this solution is presented in a model that does not admit a disagreement point, fallback bargaining is rather an arbitration rule than being a bargaining solution. An analysis of fallback bargaining in a model with a disagreement point is made by Kıbrıs and Sertel (2007) who rebaptize the solution as \textit{unanimity compromise} and define several variants of it. One of these variants, the \textit{imputational compromise}, is further studied by Conley and Wilkie (2012).
 
As a matter of fact, the compromising approach that underlies fallback bargaining was originally used to design voting rules in settings with more than two individuals with the required support varying from unanimity to simple majority, such as the \textit{Kant-Rawls Social Compromise} by Hurwicz and Sertel (1997) and the \textit{majoritarian compromise} by Sertel and Yilmaz (1999). It also paved the way to new axioms for social choice, such as \textit{efficiency in the degree of compromise} by Ozkal Sanver and Sanver (2004).  Merlin et al. (2019) present a recent comprehensive analysis of voting rules and axioms based on this compromising idea.
 
A closer look at fallback bargaining reveals a principle for two-person social choice. Sprumont (1993) qualifies arbitration rules that maximize the welfare of the least happy individual as being \textit{Rawlsian}. Theorem 3 of Brams and Kilgour (2001) shows the equivalence between the Rawlsian principle and fallback bargaining. Moreover, every individual ranks a fallback bargaining outcome in the upper half of his ranking (Brams and Kilgour 2001). Thus, asking from a social choice rule to always pick an alternative that is ranked among the first half of both players’ rankings presents a well-defined principle, which we qualify as being \textit{minimally Rawlsian}. 

Another principle for two-person social choice is proposed by Cailloux et al. (2022) who propose a different conception of compromising based on the \textit{equal loss principle} that favors outcomes where every individual concedes as equally as possible from his highest ranked alternative. Several two-person social choice rules that are minimally Rawlsian fail to comply with the equal loss principle, suggesting an incompatibility between these two principles.
 
Although the minimal Rawlsian and equal loss principles cover many of the two-person social choice rules, the literature is missing an axiomatic analysis of these rules from this perspective, an observation which forms the subject matter of our paper. We consider the following rules:

Fallback bargaining, as defined by Brams and Kilgour (2001);

The \textit{veto-rank mechanism} where, given an odd number m of alternatives, each individual vetoes $(m−1) / 2$ alternatives and ranks the remaining $(m+1) / 2$. The outcome is the alternative with the minimal sum of ranks among those that have not been vetoed. This mechanism, used for the selection of arbitrators, is comprehensively analyzed by de Clippel et al. (2014).


The \textit{shortlisting procedure} where one individual selects $(m+1) / 2$ alternatives and the other individual decides on the outcome out of that shortlist. This is also a mechanism that is used for the selection of arbitrators and comprehensively analyzed by de Clippel et al. (2014).


The class of \textit{Pareto-and-veto rules} where each individual i vetoes a fixed number $v_i$ of alternatives with $v_1$ + $v_2$ being lower than the total number of alternatives m. The outcome is the set of Pareto optimal alternatives that are not vetoed. This class generalizes the Pareto-and-veto rules analyzed by Laslier et al. (2021) which impose $v_1$ + $v_2$ = $m-1$.
  
This list of rules we consider covers most of the two-person social choice rules in the literature. The literature also admits various interesting real-life procedures expressed as extensive form games, such as those in Anbarci (1993, 2006) and Barberà and Coelho (2021). However, as shown in these papers, the subgame perfect equilibrium outcomes of these games are always among the alternatives that fallback bargaining would choose. 

We now summarize our findings. All rules we consider are Paretian. Fallback bargaining, the veto-rank mechanism and the shortlisting procedure are minimally Rawlsian. The class of Pareto-and-veto rules admits a single minimally Rawlsian member: the one that gives the highest equal veto power to both individuals, i.e., $v_1$ = $v_2$ = $m/2$ when the number $m$ of alternatives is odd and $v_1$ = $v_2$ = $(m-2)/2$ when $m$ is even. Moreover, this Pareto-and-veto rule is a super correspondence of every Paretian and minimally Rawlsian social choice rule. Thus, fallback bargaining, the veto-rank mechanism and the shortlisting procedure are all sub correspondences of the Pareto-and-veto rule with the highest equal veto power.

The equal loss principle we consider favors outcomes that have the same rank for both individuals. Without imposing Pareto optimality separately, this principle may lead to Pareto dominated outcomes. Thus, we consider a Paretian version that favors, among the Pareto optimal outcomes, the one that has the same rank for both individuals. Note that such an alternative, if it exists, will be unique. We define two versions of the Paretian equal loss principle, one being stronger than the other. The stronger version requires that the Pareto optimal alternative that has the same rank for both individuals must be uniquely chosen. Under the weaker version it suffices that this alternative be among the outcomes. The veto-rank mechanism and the shortlisting procedure both fail the weak (hence strong) version of the Paretian equal loss principle.  While some Pareto-and-veto rules, namely xxxxx satisfy the weak Paretian equal loss principle, all of them fail the strong version. On the other hand, fallback bargaining satisfies the strong version of the Paretian equal loss principle, thus showing that this principle is compatible with being minimally Rawlsian. 



Within the spirit of equal loss, we propose the \textit{weak minimal dispersion principle} as another strengthening of the weak equal loss principle. The \textit{dispersion} of an alternative is the difference between the two ranks at which is it placed at the preferences of the two individuals. The weak minimal dispersion principle requires that an alternative whose dispersion is minimal must be among the outcomes. This principle turns out to be logically incompatible with the minimal Rawlsian principle, thus being failed by fallback bargaining.

\commentRS{Say something about characterizations here and cite Congar and Merlin (2012).}

\commentRS{I unwantedly erased the starting part below, sorry}

a set of two individuals and $\allalts$ be a set of alternatives, with $\card{\allalts} = m\geq 2$. 
Given $i \in N$, let $\ibar \in N \setminus \set{i}$ denote the other individual. Let $\powersetz{\allalts}$ denote the set of non-empty subsets of $\allalts$. Let $\linors$ be the set of linear orders over $\mathcal{A}$. We let $\prefi \in \linors$ stand for the preference of individual  $i \in N$ and $\prof = ({\prefi[1]}, {\prefi[2]}) \in \allprofs$ for a preference profile. A social choice rule (SCR) is a function $f: \allprofs → \powersetz{\allalts}$.
Given two SCRs $f$, $f'$, we write $f \subset f'$ to indicate that $f$ is a proper subcorrespondence of $f'$.

An SCR $f$ is anonymous iff $f({\prefi[1]}, {\prefi[2]}) = f({\prefi[2]}, {\prefi[1]})$ for all $({\prefi[1]}, {\prefi[2]}) \in \allprofs$.
An SCR $f$ is neutral iff for all permutations $\sigma$ over $\allalts$ and profile $\prof \in \allprofs$, $\sigma \circ f(\prof) = f(\sigma \circ \prof)$.

Given $p, q \in \N$, let $\intvl{p, q} = [p, q] \cap \N $ denote the interval of integer numbers between $p$ and $q$. The loss of individual $i$ at preference profile $\prof$ for alternative $x$ is  defined as the number of alternatives that the individual prefers to $x$: $\lprof(x)_i = \card{\set{y \in \allalts \suchthat y \prefi x}}$.
The loss vector of $x$ at $\prof$, $\lprof(x)=(\lprof(x)_1, \lprof(x)_2) \in \intvl{0, m - 1}^N$, associates to each individual her loss associated to $x$.

%Let $\bigcup_{k \in \losses}(k^N)$ denote the set of constant loss vectors.
Given two loss vectors $l, l' \in \lvs$, we say that $l$ is weakly smaller than $l'$, $l ≤ l'$, iff $\forall i: l_i ≤ l'_i$. We also write $l < l'$ to denote that $l$ is strictly smaller than $l'$, meaning, weakly smaller and different. Let $\min_N \lprof(x) = \min_{i \in N} \lprof(x)_i \in \N$ 
and $\sum_N \lprof(x) = \sum_{i \in N} \lprof(x)_i$ denote, respectively, the minimal loss and the sum of the losses associated to $\lprof(x)$.

Let $\PE = \set{x \in \allalts \suchthat \nexists y \text{ s.t. } \lprof(y) < \lprof(x)}$ be the set of Pareto-efficient alternatives at $\prof$.
Let $\PEcl$ denote the class of SCRs picking only Pareto efficient alternatives: $\forall \prof: f(\prof) \subseteq \PE$.

We now define several SCRs that we analyze in the paper. 

Formally, given a profile $\prof \in \allprofs$ and an individual $i$, let $H^i(\prof) = \{x \in \allalts \suchthat \lprof(x)_i ≤ \ceil{\frac{m - 1}{2}}\} $ denote the set of alternatives in the best half of player $i$'s preference. 
Given a profile $\prof \in \allprofs$, let $H(\prof) = \set{x \in \allalts \suchthat \lprof(x) ≤ (\ceil{\frac{m - 1}{2}}, \ceil{\frac{m - 1}{2}})}$ denote the set of alternatives reaching the best half of every individual’s preference. 
In concordance with the ceiling established by Theorem 1 of \cite{BramsKilgour2001}, we use the term “half” to mean the smallest integer $k$ that exceeds $m-k$.

Given $\prof$ and a loss level $k \in \losses$, define $U(\prof, k) = \set{x \in \allalts \suchthat \lprof(x) ≤ (k, k)}$ as the set of alternatives imposing losses not higher than $k$ for all individuals. 
We say that such alternatives receive unanimous support at level $k$. Let $\rhoP = \min \set{k \in \losses \suchthat U(\prof, k) ≠ \emptyset}$ be the least loss level at which some alternative receives unanimous support.

\textbf{Fallback Bargaining} is the SCR $\FB$ that picks all alternatives with unanimous support at $\rhoP$: $\FB(\prof) = U(\prof, \rhoP)$. 

The \textbf{Veto-rank} rule $\VR$ is defined as follows. Each individual vetoes her worst $\floor{\frac{m - 1}{2}}$ alternatives, then the Borda winners among the non vetoed alternatives are picked: $\VR(\prof) = \argmin_{H(\prof)} \sum_N \lprof = \set{x \in H(\prof) \suchthat \forall y \in H(\prof): \sum_N \lprof(x) ≤ \sum_N \lprof(y)}$.

The \textbf{Shortlisting} rule $\SL$ picks the best alternative of individual $1$ that is not among the worst $\floor{\frac{m - 1}{2}}$ alternatives of individual $2$, and the best alternative of $2$ that is not among the worst $\floor{\frac{m - 1}{2}}$ alternatives of $1$. The Shortlisting rule is such that
$\SL(\prof) = \cup_{i \in N} (\argmin_{x \in H^i(\prof)} \lprof(x)_{\ibar})$.

Both $\VR$ and $\SL$ are defined in \cite{Clippel} for $m$ odd only.

The class of \textbf{Pareto-and-veto} rules, $\PVcl$, contains rules parametrized by $v_1, v_2 \in \intvl{0, m - 1}$ with $v_1 + v_2  ≤ m - 1$ where $v_i$ represents the number of alternatives vetoed by individual $i \in N$ (individuals veto the alternatives at the bottom of their preference).
Given $v_i \in \intvl{0, m - 1}$, define $a_i = m - v_i - 1 \in \intvl{0, m - 1}$ as the highest acceptable loss level for individual $i$. For $v=(v_1,v_2)$, the rule $\PVv = \cap_{i \in N}\set{x \in \allalts \suchthat \lprof(x)_i ≤ a_i} \cap \PE$ picks all alternatives in $\PE$ that no individual vetoes. 
The class $\PVcl = \set{\PVv \suchthat v_1, v_2 \in \intvl{0, m - 1} \text{ with } v_1 + v_2 \leq m - 1}$ is the set of those rules, and the class $\PVbcl = \set{\PVv \suchthat v_1, v_2 \in \intvl{0, m - 1} \text{ with } v_1 + v_2 = m - 1}$ is the set of rules where the inequality is binding.

\begin{remark}
    $\FB$, $\VR$, $\SL$ and all SCRs in $\PVcl$ are  neutral.
\end{remark}
\begin{remark}
    $\FB$, $\VR$ and $\SL$ are anonymous, while a SCR in $\PVcl$ is anonymous iff $v_1 = v_2$.
\end{remark}

\section{Minimal Rawlsian principle}
\begin{definition}[Minimal Rawlsian (MR)] A SCR satisfies MR if 
	$\forall \prof \in \allprofs,  f(\prof) \subseteq H(\prof)$.
\end{definition}
Let $\mathcal{MR}$ denote the class of rules satisfying the Minimal Rawlsian (MR) property.

\begin{remark}
   \label{th:inFH}
    $\FB, \VR, \SL \in \PEcl \cap \mathcal{MR}$. 
\end{remark}
   

The remark follows from the definition of $\VR$ and $\SL$ since each of these rules selects only alternatives among the top-half alternatives of both individuals; and from Theorem 1 of \cite{BramsKilgour2001} which shows that $\forall \prof \in \allprofs$, $\rhoP ≤ \ceil{\frac{m - 1}{2}}$.

We now discuss the relationship of the class $\PVcl$ to the $\mathcal{MR}$ property. To this end, we define $\PVe = \PVv[\left(\floor{\frac{m - 1}{2}}, \floor{\frac{m - 1}{2}}\right)]$ as the Pareto-and-veto rule that gives the highest equal veto power to both individuals. 
Thus, under $PV^=$, we have $v_1=v_2= \floor{\frac{m - 1}{2}}$, implying $v_1=v_2=\frac{(m-1)}{2}$ when $m$ is odd and $v_1=v_2= \frac{m}{2}-1$ when $m$ is even. 
Note that $PV^=\in\PVbcl$ iff $m$ is odd.

\begin{proposition}\label{propo:diff}
    $\PVcl ∩ \mathcal{MR} = \set{\PVv \in \PVcl \suchthat \forall i: v_i ≥ \floor{\frac{m - 1}{2}}}$.
\end{proposition}
\begin{proof}
	For the “if” part, note that given any profile, the condition $\forall i: v_i ≥ \floor{\frac{m - 1}{2}}$ suffices to guarantee that $\PVv(\prof) \subseteq H(P)$.

	To see the “only if” part, consider an arbitrary ordering $\prefi$ over $\allalts$, let $\prefiinv$ denote its inverse, and consider the profile $\prof = (\prefi, \prefiinv)$.
	Observe that $\PVv(\prof)$ will exclusively pick winners in the first half of voter $i$ only if $v_i ≥ \floor{\frac{m - 1}{2}}$.
\end{proof}
A direct implication of Proposition \ref{propo:diff}   is that the set of $\PVcl$ rules satisfying $\mathcal{MR}$ is quite reduced, as made explicit by the following remark.
\begin{remark}
    When $\forall i \in \set{1, 2}, v_i ≥ \floor{\frac{m - 1}{2}}$, we have $\abs{v_1 - v_2} ≤ 1$.
	Thus, when $m$ is odd, $\PVcl ∩ \mathcal{MR} = \{PV^=\}$, and
	when $m$ is even, $\PVcl ∩ \mathcal{MR} = \set{\PVe, \PVv[(\frac{m}{2}, \frac{m}{2} - 1)], \PVv[(\frac{m}{2} - 1, \frac{m}{2})]}$.
\end{remark}

We now establish the relationship of  $\VR, \SL, \FB$ to the class $\PVcl$. Considering two SCRs $f$ and $f'$, let $f \cup f'$ denote the rule $(f \cup f')(\prof) = f(\prof) \cup f'(\prof)$. 
Given any non empty class of SCRs $F$, let $\bigcup F$ denote the maximal (least resolute) SCR that can be formed by unions of rules of $F$.

\begin{proposition}\label{propo:equal}
	$\bigcup(\PEcl \cap \mathcal{MR}) = \PVe$.
\end{proposition}
\begin{proof}
    Note that $\bigcup(\PEcl \cap \mathcal{MR})$ is, by definition, the SCR that, for each profile, picks all Pareto alternatives that are in the first half of both individuals’ preferences and only those alternatives. 
    We thus have to show that $\forall \prof: \PVe(\prof) = \PE \cap H(\prof)$. By definition of $\PVe$, it suffices to prove that $\cap_i\set{x \in \allalts \suchthat \lprof(x)_i ≤ m - \floor{\frac{m - 1}{2}} - 1} = H(\prof)$. This in turn follows from the definition of $H(\prof)$.
\end{proof}

The observation below follows from \cref{propo:equal}.
\begin{corollary}\label{th:subPVe}
	A SCR $f \in \PEcl \cap \mathcal{MR}$ if and only if $f \subseteq \PVe$.
\end{corollary}

We now establish the relationship between $\FB$, $\VR$ and $\SL$ and show that they can pick disjoint winners.
\begin{proposition}\label{th:different}
	$\exists \prof \suchthat \FB(\prof) \cap \VR(\prof) = \emptyset, \FB(\prof) \cap \SL(\prof) = \emptyset, \VR(\prof) \cap \SL(\prof) = \emptyset$. Furthermore, $\PVe(\prof) ≠ \FB(\prof)$, $\PVe(\prof) ≠ \VR(\prof)$, $\PVe(\prof) ≠ \SL(\prof)$.
\end{proposition}
\begin{proof}
	Consider the following profile $\prof$:
	\begin{equation}
		\label{eq:distinct}
		\begin{array}{llll lll | llll ll}
			a&b&c&d&e&f&g&h&i&j&k&l&m\\
			g&h&i&d&b&j&a&c&e&f&k&l&m\\
		\end{array},
	\end{equation}
	where the first individual prefers $a$ to $b$, $b$ to $c$, etc., and the second individual prefers $g$ to $h$, $h$ to $i$, etc. 
	The bar shows the “half” position.
	The proposition is proven by noting that $\FB(\prof) = \set{d}$, $\VR(\prof) = \set{b}$, $\SL(\prof) = \set{a, g}$ and $\PVe(\prof) = \set{a, b, d, g}$.
\end{proof}

\Cref{th:inFH}, \cref{th:subPVe} and \cref{th:different} lead to the corollary below.
\begin{corollary}
   	$\FB, \VR, \SL \subset \PVe$.
\end{corollary}

\begin{remark}
    The rule $\PVe$ is not merely the union of $\FB$, $\VR$ and $\SL$, as the following profile illustrates.
    \begin{equation}
        \begin{array}{lllll|llll}
                a&b&c&d&e&f&g&h&i\\
                e&c&d&b&a&f&g&h&i\\
        \end{array}.
    \end{equation}
    Here, $\FB(\prof) = \set{c}$, $\VR(\prof) = \set{c}$, $\SL(\prof) = \set{a, e}$ but $\PVe(\prof) = \set{a, b, c, e}$. In fact, no variant of $\VR$ (changing the scoring vector) would elect $b$: its losses are $(1, 3)$ whereas $c$ has losses $(2, 1)$.
\end{remark}

\section{The equal loss principle}
Given $\prof \in \allprofs$, define $S(\prof) = \set{x \in \allalts \suchthat \lprof(x)_1 = \lprof(x)_2}$ as the set of alternatives that are ranked at the same position by both individuals.

\begin{definition}[Weak equal loss (WEL)]
    $\forall \prof \in \allprofs: [S(\prof) ≠ \emptyset] ⇒ f(\prof) \cap S(\prof) ≠ \emptyset$.
\end{definition}

\begin{proposition}
    $\forall m ≥ 3: \PEcl ∩ \WELcl = \emptyset$.
\end{proposition}

Thus, Pareto efficiency is a fortiori incompatible with the following stronger version of the equal loss principle.

\begin{definition}[Equal loss (EL)]
    $\forall \prof \in \allprofs: [S(\prof) ≠ \emptyset] ⇒ f(\prof) \subseteq S(\prof)$.
\end{definition}

We now embed Pareto efficiency into the equal loss requirement by mandating $f$ to pick the (unique) Pareto efficient alternative that is ranked at the same position by both individuals, if there is one.

\begin{definition}[Weak Paretian equal loss (WPEL)]
    $\forall \prof \in \allprofs: [S(\prof) \cap PE(\prof) ≠ \emptyset] ⇒ f(\prof) \cap S(\prof) ≠ \emptyset$.
\end{definition}

\begin{proposition}
	VR and SL fail WPEL.
\end{proposition}
\begin{proof}
	Consider the following profile $\prof$, reused from \cref{eq:distinct}, where $\VR(\prof) = \set{b}$ and $\SL(\prof) = \set{a, g}$:
	\begin{equation}
		\begin{array}{llll lll | llll ll}
			a&b&c&d&e&f&g&h&i&j&k&l&m\\
			g&h&i&d&b&j&a&c&e&f&k&l&m\\
		\end{array}.
	\end{equation}
	WPEL requires to choose at least $d$.
\end{proof}

Some rules from the class $\PVcl$ satisfy WPEL, $\PVe$ being among those.
\begin{proposition}
	For any $m ≥ 3$, a PV rule $\PVv[(v_1, v_2)]$ satisfies WPEL iff its veto levels are both at most $\floor{\frac{m}{2}}$, thus, iff $\max_{i \in \set{1, 2}} v_i ≤ \floor{\frac{m}{2}}$.
\end{proposition}
\begin{proof}
	For all $\prof \in \allprofs$, if $x \in S(\prof) \cap \PE$, then $x$ is not among the last $\floor{\frac{m}{2}}$ ranks, because all alternatives before $x$ for the first voter must be placed after $x$ for the second voter.
	A PV rule with veto parameters at most $\floor{\frac{m}{2}}$ will thus pick all such alternatives $S(\prof) \cap \PE$, as required by WPEL.
	
	For the other direction, observe that there exists $\prof \in \allprofs$ such that for some $x \in \allalts$, $x \in S(\prof) \cap \PE$ and $x$ is positioned just better than the last $\floor{\frac{m}{2}}$ ranks (thus $\exists \prof \in \allprofs, x \in \allalts \suchthat \forall i \in \set{1, 2}: \lprof(x)_i = \floor{\frac{m - 1}{2}}$, leaving $\ceil{\frac{m - 1}{2}} = \floor{\frac{m}{2}}$ positions behind $x$).
	A PV rule such that $\max_{i \in \set{1, 2}} v_i > \floor{\frac{m}{2}}$ will thus not include $x$ in the set of winners, hence, the rule will fail WPEL.
\end{proof}

The rules that fail WPEL will a fortiori fail the following stronger version of the Paretian equal loss property.

\begin{definition}[Paretian equal loss (PEL)]
    $\forall \prof \in \allprofs: [S(\prof) \cap PE(\prof) ≠ \emptyset] ⇒ f(\prof) = S(\prof) \cap PE(\prof)$.
\end{definition}
Thus, VR, SL and those rules in PV that fail WPEL all fail PEL. Furthermore, as we state and show below, even the rules in PV that satisfy WPEL fail PEL.
\begin{proposition}
	When $m ≥ 4$, all rules in PV fail PEL.
\end{proposition}
\begin{proof}
    Let $m = \set{a, b, c, a_4, a_5, …}$.
    Consider the following profile $P$: 
	\begin{equation}
		\begin{array}{llllll}
			a&b&c&a_4&a_5&…\\	c&b&a&a_4&a_5&…\\
		\end{array}.
	\end{equation}
    PEL requires to pick solely $b$. Any PV rule will pick at least either $a$ or $c$ in supplement to $b$. 
    Indeed, we have $a \notin f(\prof)$ iff $v_2 ≥ m - 2$, and $c \notin f(\prof)$ iff $v_1 ≥ m - 2$. Because $v_1 + v_2 ≤ m - 1$, it is impossible to have both $a \notin f(\prof)$ and $c \notin f(\prof)$.
\end{proof}

\begin{remark}
When $m = 3$, $PV^{(1, 1)}$ satisfies PEL.
\end{remark}

\begin{proposition}
	FB satisfies PEL.
\end{proposition}
\begin{proof}
    \commentOC{TODO}
    If an alternative is Pareto and equal-loss, FB chooses it.
    
    Let $x$ be such an alternative and assume that FB does not pick it. Hence, there is some $y$ that Pareto dominates $x$, a contradiction.
\end{proof}

Thus, by satisfying both conditions, FB establishes the compatibility between PEL and MR. However, as discussed below, this compatibility vanishes when another stronger version of WPEL is adopted. 

Call the dispersion of a loss vector $l$ at $\prof$ the value $d(l) = \abs{l_1 - l_2}$. 
Thus, $(d \circ \lprof)(x) = \max\lprof(x) - \min\lprof(x)$.
Note that $d \in \Sigma$ seems to coincide with multiple commonly used spread measures \commentOC{to be verified}. 

Given a profile $\prof \in \allprofs$, define $\min_{\PE} (d \circ \lprof)$ as the minimal dispersion obtained by loss vectors of Paretian alternatives in that profile, and $\argmin_{\PE} (d \circ \lprof)$ as the Paretian alternatives whose loss vectors have minimal dispersion among Paretian alternatives.

Define the weak minimal dispersion ($WMD$) condition as follows.
\begin{definition}[Weak minimal dispersion]
	For any $\prof \in \allprofs$, $\min_{\PE} (d \circ \lprof) ≠ \emptyset ⇒ f(\prof) \cap \min_{\PE} (d \circ \lprof) ≠ \emptyset$.
\end{definition}

\begin{theorem}
 There is no $f\in \mathcal{MR}$ that satisfies WMD.
\end{theorem}
\begin{proof}
    \commentOC{TODO}.
\end{proof}

Thus the minimal Rawlsian principle and WMD are logically incompatible. \commentOC{$\set{FB, VR, SL}$ are MR, thus, they fail WPEL. Also, all rules in PV fail PEL. Not all rules in PV fail WPEL. But do all rules in PV fail WMD?}

\section{Further axiomatizations of the rules}

We define WMD restricted to MR alternatives, namely WMDMR.

WMDMR and WMD are logically independent.

The set of WMDMR alternatives forms a subcorrespondence of FB.

This inclusion can be strict. Here is a description of what this refinement is. Observe that at every profile, FB picks at most two alternatives. Profiles where it picks two alternatives, say x and y, must be of the following form:

…. x ……….y

 ……y………x

where in the dots no alternative appears twice. (Some of the dots may be empty.) Moreover, y for voter 1 and x for voter 2 are at the same level.

Now, x for voter 1 and y for voter 2 may or may not be at the same level. The first WMDMR refines FB when and only when they are not. 

Now consider another axiom, Rawlsian neutrality (RN): Define the social welfare of an alternative x at P, sw(x, P), as the lowest rank at which is is ranked by a voter. We say that F is RN iff given any P and any x, y with sw(x, P) = sw(y, P), we have x is in F(P) iff y is in F(P).

RN and MWDMR are logically independent and their conjunction is in turn equivalent to the maximin principle, thus characterizing FB.


Interestingly, I realize that the veto-rank of Clippel, by its very definition, may have a characterization similar to WMDFH. Isn’t it Borda constrained to the first half.

Short listing should be something around “maximax within the first half.”.


\commentOC{Attempt for describing formally the useful observation of Matías that $\PVe$ is the MR; SL is max max among the MR; VR maximizes the sum of gains; and FB maximizes the min gain. Mistakes are highly possible.}

In this section, in order to clarify how the four rules that we study here compare, we characterize a rule $f$ by stating conditions under which $f$ is the biggest rule satisfying all those conditions. Those conditions are “necessity” principles, indicating a required condition to be a winner, equivalently, a sufficient condition to lose, thus, they have the form: $f(\prof) \subseteq g(\prof)$ for some $g: \allprofs → \powersetz{\allalts}$, indicating that winning requires to be a member of $g(\prof)$, equivalently, that being not in $g(\prof)$ is sufficient to lose.

The least stringent condition is the MR property: $f(\prof) \subseteq H(\prof)$. The rule $\PVe$ is the biggest rule satisfying MR.

Another condition requires that if $\exists y \in H(\prof) \suchthat \min(\lprof(y)) < \min(\lprof(x))$, then $x$ loses. Equivalently, $x \in f(\prof) ⇒ \min(\lprof(x)) ≤ \min_{y \in H(\prof)} \min(\lprof(y))$. The rule $\SL$ is the biggest rule satisfying both MR and that condition. (Note that these two conditions are independent.)

Similarly, $\VR$ is the biggest rule satisfying both MR and the condition: $x \in f(\prof) ⇒ \sum(\lprof(x)) ≤ \min_{y \in H(\prof)} \sum(\lprof(y))$.

These four rules can also be compared by describing them as follows. \commentOC{Here is the description of $\SL$, which can be suitably adapted to other rules.} The $\SL$ rule is the rule that selects any alternative that is MR and that is not dominated by an MR alternative in terms of minimal welfare loses. More precisely, define $\prec$ as the following partial order on the set of loss vectors: $(a, b) \prec (c, d) ⇔ [(a, b) ≤^\text{MR} (c, d) \land \min\set{a, b} < \min\set{c, d}] \lor [(a, b) <^\text{MR} (c, d)]$, where $(a, b) ≤^\text{MR} (c, d)$ iff $(a, b)$ is first half. Then, $\SL$ selects $x$ iff it is an undominated element among the loss vectors that exist in $\prof$; formally, $f(\prof) = \set{x \in \allalts \suchthat \nexists y \in \allalts \suchthat y \prec x}$. 
Defining the uncomparability relation $\sim$ as $l_1 \sim l_2$ iff $l_1 \sim^\text{MR} l_2 \land \min l_1 = \min l_2$, we obtain the weak order ${\preceq} = {\prec} \cup {\sim}$, and $f(\prof) = \argmin^{\preceq}_{x \in \allalts} \lprof(x)$.
\commentOC{We could perhaps also describe $\prec$ as a union of relations, one of which is common to all rules.}

\commentRS{Below are characterizations}
PV-e iff PE, MR, MAskin monotonicity



For SL and VR we should look at Clippel. 

\color{green} MN: no characterization of VR to the best of my knowledge \color{black}


Danilo Coelho  (see the \href{https://www.cmss.auckland.ac.nz/2020/06/03/online-social-choice-seminar-series/}{Online Social Choice Seminar})

cites a paper by Clippel with a new characterization of Fallback, we might want to cite it.

QUESTIONS:


2. Can we identify an interesting SCR that satisfies WMD (or rather MD)?

3. Can we use WMD to refine PV, VR, SL or FB? Discuss refinements of these rules. The Theorem, stated at the end of page 4 (\hrefblue{https://link.springer.com/article/10.1007/BF00187429}{Sprumont}), is about refinements of FB. It says that FB admits refinements that satisfy the two conditions of the paper, namely CPB and SO. Moreover, any rule that satisfies CPB and SO is a refinement of FB. Here are a few examples that illustrate the point of the question. At the profile 
\begin{center}
	$
	\begin{array}{cccccc}
		\mathbf{i_1} \quad &a&b&c&d&e\\
		\mathbf{i_2} \quad &d&c&a&b&e\\
	\end{array}
	$
\end{center}

PV would pick a-c and through SPEL we could refine this to the singleton c (which is also a refinement of FB).What is this new rule? It refines PV. It is not FB. Similarly, at
\begin{center}
	$
	\begin{array}{cccccc}
		\mathbf{i_1} \quad &a&b&c&d&e\\
		\mathbf{i_2} \quad &c&b&a&d&e\\
	\end{array}
	$
\end{center}

PV picks a-b-c, SPEL refines it to b (which is the FB outcome) while SL picks a-c.

\section{Miscellaneous material}
NOW: introduce variable sized rules from the start, in supplement to (even, using) fixed sized rules. Introduce something like PCC below, observe that it is equivalent to WPEL but that it can be used to straighten it, restrict the class $\Sigma$ of acceptable spread measures and observe that it is equivalent to MD (defined over variable sized electorates). In other words, determine a version of Strong PEL that is sufficiently strong for incompatibility with MR using an appropriately restricted class of measures sigma.
Variants of the definitions below could be useful.

\begin{definition}[Paretian compromism compatibility (PCC)]
	Define $\Sigma = \set{\sigma \in (\R^+)^{(\lvs)} \suchthat \sigma^{-1}(0) = \cup_{k \in \losses}(k^N)}$ be the set of all spread measures that attribute the minimal inequality measure to the constant loss vectors and none others.
	The rule $f$ is PCC iff there exists a spread measure $\sigma \in \Sigma$ such that $\forall \prof: f(\prof) \cap \minspread[\PE] ≠ \emptyset$.
\end{definition}

%\begin{definition}[Dispersion phobia]
%	If $x$ has a strictly higher dispersion than $y$ at $\prof$, then $f$ may not pick $x$.
%\end{definition}
\begin{definition}[Loss vector comparing (LVC)]
	Let $\succeq$ be a weak order on $\lvs$. The rule $f^\succeq$ selects all alternatives whose loss vectors are minimal elements of $\restr{{\succeq}}{\lprof(\alts)}$. A rule satisfies LVC iff $\exists {\succeq} \suchthat f = f^\succeq$. (This property is called preorder based in Cailloux and Endriss 2014.)
\end{definition}

\begin{definition}[Dispersion compromism]
	$\forall \prof: f(\prof) \subseteq \mindisp[\PE]$.
\end{definition}

\bibliography{bibliototal}

\appendix
\section{Previously in Section 3}



\subsection{Veto core}
\commentMN{I have been reading Moulin (1983)'s book on veto core correspondence (that is Pareto and veto rule). He mentions that for each outcome in the Pareto and veto rule, there is an order of sincere vetoes that isolates it.
Namely,
Consider an order of players $\{1,2,1,2,1,2,1,2\}$ and the following naive algorithm as follows:\\
\noindent Step 1: player 1 removes his worst alternative in $A$ (denoted $B_1$)\\
\noindent Step 2: player 2 removes his worst alternative in $A\setminus B_1$ (denoted $B_2$)\\
\noindent Step 3: player 1 removes his worst alternative in $A\setminus B_1\cup B_2$ (denoted $B_3$)\\
\noindent Step 4: player 2 removes his worst alternative in $A\setminus B_1\cup B_2 \cup B_3$\\
In the example of Remark 5, this sequence leads to $c$, the $FB$ outcome.
\textbf{Question 1:} Is it always the case that this alternating sequence leads to $FB$?
To obtain $e$ again in the example of Remark 5, the sequence $\{1,1,1,1,2,2,2,2\}$ seems to work. Similarly, to obtain $a$, the sequence $\{2,2,2,2,1,1,1,1\}$ seems to work.
Finally, to isolate $b$, the sequence $\{1,1,1,1,1,1,1,2\}$ seems to work.
This leads me to think that
\textbf{Question 2:} $PV^=$ could be the union of any sequence of length $n-1$. 
\textbf{Question 3:} Similarly, $FB$ consists only of the extreme sequences $\{1,1,1,...,1,2,\ldots,2\}$ and $\{2,2,2,...,2,1,\ldots,1\}$.
\textbf{Question 4:} Which are the sequences that isolate $FB$?
}

Let $s \in \set{1, 2}^{m - 1}$ be a sequence of turns. 
Define a corresponding sequence of subsets of alternatives, $A^s \in \powersetz{\allalts}^m$, as follows: $A^s_1 = \allalts$ and $A^s_{i + 1} = A^s_i \setminus \argmax_{x \in A^s_i} \lprof(x)_{s_i}$. 
Note that $\card{A^s_m} = 1$. 
The \emph{simple veto core} rule $f^s$ is the SCR $f^s(\prof) = A^s_m$ (TODO cite Moulin).
Let $\emptyset ≠ S \subseteq \set{1, 2}^{m - 1}$. Define $f^S = \cup_{s \in S} f^s$.
The set of \emph{veto core} rules is the set $\set{f^S \suchthat \emptyset ≠ S \subseteq \set{1, 2}^{m - 1}}$.

Define $S^= = \set{s \in \set{1, 2}^{m - 1} \suchthat \card{s^{-1}(1)} = \card{s^{-1}(2)}}$.
%%%%%The set of \emph{veto core egalitarian} rules is $\VCecl = \set{f^S \suchthat \emptyset ≠ S \subseteq \set{1, 2}^{m - 1} \land \card{S^{-1}(1)} = \card{S^{-1}(2)}}$.
For $m$ odd, the \emph{veto core egalitarian} rule is $f^{S^=}$.

\begin{proposition}[Might be in Moulin]
    For any sequence of turn $s \in \set{1, 2}^{m - 1}$, $f^s$ is Pareto. 
    Therefore, for any set of sequences $\emptyset ≠ S \subseteq \set{1, 2}^{m - 1}$, $f^S$ is Pareto.
\end{proposition}

\begin{conjecture}
	\label{th:sUnique}
	$\forall m, s, s' \in \set{1, 2}^{m - 1}: s ≠ s' ⇒ f^s ≠ f^{s'}$.
\end{conjecture}
\begin{proof}
	Given a sequence $s$, let $s \circ (+k)$ denote the sequence in $\set{1, 2}^{m - 1 - k}$ that skips the first $k$ elements, thus such that $(s \circ (+k))_i = s_{i + k}$.
	
	Observe first that if $s ≠ s'$, there must be some suffixes of these sequences, of a common length, that give different number of turns to the players (equivalently, to the player $1$).
	Formally: $\exists k \in \intvl{0, m - 2} \suchthat \card{(s \circ (+k))}^{-1}(1) ≠ \card{(s' \circ (+k))}^{-1}(1)$.
	Pick such a $k$ and define $r = (s \circ +k)$ and $r' = (s' \circ +k)$, thus with $r, r' \in \set{1, 2}^{m - 1 - k}$.

	Define $\prof$ by assigning to the voters opposite preferences to the $m - k$ first alternatives and equal preferences to the last $k$ alternatives; more explicitly, ${\prefi[1]} = (a_1, …, a_{m - k}, a_{m - k + 1}, …, a_m)$ and ${\prefi[2]} = (a_{m - k}, …, a_1, a_{m - k + 1}, …, a_m)$. 
	
	Considering $f^s(\prof)$ and $f^{s'}(\prof)$, the first $k$ alternatives “vetoed“ by both $s$ and $s'$ are the $k$ alternatives that the voters agree are the worst, namely, $\set{a_{m - k + 1}, … a_m}$. Considering now any of the rest of the sequences, say (wlog) $r$, we see that the remaining “vetoed” alternatives depends only on the number of turns that each player plays in $r$, namely, $\card{r^{-1}(1)}$ and $\card{r^{-1}(2)}$, as the voters have opposite preferences on the alternatives that have not yet been vetoed after $k$ turns: from the remaining alternatives $\set{a_1, …, a_{m - k}}$, the worst $\card{r^{-1}(i)}$ alternatives from voter $i$ will be vetoed (for $i \in \set{1, 2}$). Thus, $f^s(\prof) = \set{a_{m - k - \card{r^{-1}(1)}}}$ and $f^{s'}(\prof) = \set{a_{m - k - \card{{r'}^{-1}(1)}}}$. These are distinct, as $\card{r^{-1}(1)} ≠ \card{{r'}^{-1}(1)}$.
\end{proof}

\commentOC{What about $S ≠ S'$ but $f^S = f^{S'}$?}

A bijective sequence of $m$ alternatives is a sequence $e \in \allalts^m$ that covers all the alternatives, thus, such that $e_{\intvl{1, m}} = \allalts$.
We can interpret any such bijective sequence $e \in \allalts^m$ as a “veto sequence” and associate to it the sequence of subsets of alternatives, defined, for $0 ≤ j ≤ m - 1$, as $e_{\intvl{1, j}} \subset \allalts$, representing all the alternatives that have been vetoed after turn $j$ (thus with $e_{\intvl{1, 0}} = \emptyset$); and the sequence of complements, defined, for $1 ≤ j ≤ m$, as $\allalts \setminus e_{\intvl{1, j - 1}}$, or equivalently, as $e_{\intvl{j, m}}$, representing the alternatives that remain before turn $j$.

We say that a bijective sequence $e \in \allalts^m$ is a possible elimination sequence for $\prof = \set{(1, {\prefi[1]}), (2, {\prefi[2]})}$ iff $\forall j \in \intvl{1, m - 1}: \exists i \in \set{1, 2} \suchthat e_j = \min_{e_{\intvl{j, m}}} {\prefi}$.

Given a bijective sequence $e \in \allalts^m$ and a preference ${\pref} \in \linors$, we say that the possible vetoes associated to $e$ and $\pref$ are $A^{e, {\pref}} = \set{\min_{e_{\intvl{j, m}}} {\prefi} \suchthat j \in \intvl{1, m - 1}} \subseteq \allalts$.

The following lemma shows, losely speaking, that any profile and possible elimination sequence corresponds to sequences of turns that give to each voter the “responsability of vetoing” some of the alternatives among its possible vetoes.
\begin{lemma}[To be verified]
	\label{th:seqATos}
	Consider a profile $\prof = \set{(1, {\prefi[1]}), (2, {\prefi[2]})}$ and $e \in \allalts^m$, a possible elimination sequence for $\prof$.
	Let $A^{e, {\prefi}}$ be the corresponding possible vetoes for each voter $i \in \set{1, 2}$.
	Pick any two numbers $(k_1, k_2)$ such that $k_1 + k_2 = m - 1$ and $k_i ≥ \card{(A^{e, {\prefi}} \setminus A^{e, {\prefi[\ibar]}})}$.
	
	Then, $\exists s \in \set{1, 2}^{m - 1} \suchthat f^s(\prof) = \set{e_m}$ satisfying $\forall i \in \set{1, 2}: \card{s^{-1}(i)} = k_i$.
\end{lemma}
\begin{proof}
	Observe that $\cup_{i \in \set{1, 2}} A^{e, {\prefi}} = \set{\min_{e_{\intvl{j, m}}} {\prefi} \suchthat i \in \set{1, 2}, j \in \intvl{1, m - 1}}$ (by definition of $A^{e, {\prefi}}$), the latter set being also equal to $e_{\intvl{1, m - 1}} = \allalts \setminus \set{e_m}$, as $e$ is a possible elimination sequence for $\prof$. Also, $\cup_{i \in \set{1, 2}} A^{e, {\prefi}}$ can be written as $(A^{e, {\prefone}} \setminus A^{e, {\preftwo}}) \cup (A^{e, {\preftwo}} \setminus A^{e, {\prefone}}) \cup (A^{e, {\prefone}} \cap A^{e, {\preftwo}})$.
	
	Thus, we can partition $A^{e, {\prefone}} \cup A^{e, {\preftwo}}$ into disjoint sets $K_1, K_2$ such that $(A^{e, {\prefi}} \setminus A^{e, {\prefi[\ibar]}} \subseteq K_i \subseteq A^{e, {\prefi}}$, $\card{K_i} = k_i$ and $K_1 \cup K_2 = e_{\intvl{1, m - 1}}$.
	
	Define, $\forall j \in \intvl{1, m - 1}$: $s_j = i$ iff $e_j \in K_i$.
\end{proof}

For $x \in \allalts$, let ${\prefi}(x) \subseteq \allalts$ designate the strict lower contour set of $x$ and ${\prefeqi}(x) \subseteq \allalts$ designate the lower contour set of $x$ including $x$.
\begin{lemma}
	\label{th:coveringToSeqs}
	Consider $x \in \allalts$, $\prof = \set{(1, {\prefi[1]}), (2, {\prefi[2]})}$.
	Assume that $\cup_i {\prefeqi}(x) = \allalts$ (equivalently, $x \in \PE$).
	
	Define $K_1^u = \allalts \setminus {\prefeqtwo}(x)$.
	Consider any $k_1 \in \intvl{\card{K_1^u}, \card{{\prefone}(x)}}$. Define $k_2 = m - 1 - k_1$.

	Then, $\exists s \in \set{1, 2}^{m - 1} \suchthat f^s(\prof) = \set{x}$ satisfying $\forall i \in \set{1, 2}: \card{s^{-1}(i)} = k_i$.
\end{lemma}
\begin{proof}
	Define $B$ as the bottom $k_1 - \card{K_1^u}$ alternatives from ${\prefone}(x) \setminus K_1^u$.
	Define $K_1 = K_1^u \cup B$.
	Note that $K_1^u \subseteq K_1 \subseteq {\prefone}(x)$ and $\card{K_1} = k_1$.
	Define $K_2 = \allalts \setminus \set{x} \setminus K_1$.
	
	Define $e$ as any sequence that satisfies $\forall i: [y \in K_i ⇒ {\prefi}(y) \text{ is before } y \text{ in } e]$. Apply \cref{th:seqATos}.
\end{proof}

\begin{conjecture}
	\label{th:vce}
	For $m$ odd, $f^{S^=} = \bigcup (\PEcl \cap \mathcal{MR}) = \PVe$.
\end{conjecture}
\begin{proof}
	To see that $f^{S^=} \subseteq \bigcup (\PEcl \cap \mathcal{MR})$. Any sequence leads to Pareto efficient (to be detailed). Let $2k+1$ be the number of alternatives. Since any sequence in $S^=$ gives each voter the same number of turns (hence $k$), the outcome is never among the worst $k$ alternatives of each voter. It follows that the outcome is Pareto-efficient and in the first half of each voter's preferences, concluding the proof.
	
	About the other direction. 
	\footnote{Example (Preliminary): abcdefg, gbacdef, b $\Longrightarrow$ 112212.}
	From first half: at least k alternatives in the lower contour set of $x$ for each voter. Apply \cref{th:coveringToSeqs}.
\end{proof}

\begin{conjecture}
	\label{th:vrNotVce}
	$\forall \emptyset ≠ S \subseteq S^=: \VR ≠ f^{S}$.
\end{conjecture}
\begin{proof}
    ?
\end{proof}

\begin{conjecture}
	\label{th:slVce}
	$\SL = f^{\set{(1, 1, …), (2, 2, …)}}$.
\end{conjecture}

\begin{remark}
	\label{th:fbVce}
	It is false that
	$\FB = f^{\set{(1, 2,1,2, …), (2, 1,2,1 …)}}$.
    \begin{equation}
        \begin{array}{lllll}
                a&b&c&d&e\\
                d&c&b&e&a\\
        \end{array}.
    \end{equation}
    $\FB(\prof) = \set{b, c}$ and $f^{\set{(1, 2,1,2, …), (2, 1,2,1 …)}}=\{c\}$.
\end{remark}

\commentOC{Is $\VCecl$ the set of anonymous veto core rules, by chance?}

\commentRS{Below are characterizations - TODO move to introduction}
PV-e iff PE, FH, MAskin monotonicity

I recall from a paper of Vincent that FB is characterized by using a Rawlsian maximin axiom plus maximality 

For SL and VR we should look at Clippel. 

\color{green} MN: no characterization of VR to the best of my knowledge \color{black}

Sertel shows (says Danilo Coelho) that the unanimous compromise set (the FB winners?) is the set of Pareto efficient candidates that maximize the worst-of welfare (welfare = nb alternatives less good than the chosen one). \color{green} MN: This is true, right? \color{black}

\commentOC{Would it be interesting to attempt to characterize scoring rules that attribute an infinitely negative weight to the ranks in the bottom half?}

