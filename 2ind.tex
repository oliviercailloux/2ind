\RequirePackage[l2tabu, orthodox]{nag}
\documentclass[version=3.21, pagesize, twoside=off, bibliography=totoc, DIV=calc, fontsize=12pt, a4paper]{scrartcl}
\input{preamble/packages}
\input{preamble/redac}
\input{preamble/math_basics}
%Decision Theory
\NewDocumentCommand{\allalts}{}{\mathcal{A}}
\NewDocumentCommand{\allcrits}{}{\mathscr{C}}
\NewDocumentCommand{\alts}{}{A}
\NewDocumentCommand{\dm}{}{i}
\NewDocumentCommand{\allF}{}{\mathscr{F}}
\NewDocumentCommand{\allvoters}{}{\mathscr{N}}
\NewDocumentCommand{\voters}{}{N}
\NewDocumentCommand{\allprofs}{}{\linors^N}
\NewDocumentCommand{\prof}{}{\bm{P}}
\NewDocumentCommand{\lprof}{}{\lambda_{\bm{P}}}
\NewDocumentCommand{\linors}{}{\mathscr{L}(\allalts)}
%Thanks to https://tex.stackexchange.com/q/154549
	%\makeatletter
	%\def\@myRgood@#1#2{\mathrel{R^X_{#2}}}
	%\def\myRgood{\@ifnextchar_{\@myRgood@}{\mathrel{R^X}}}
	%\makeatother
\NewDocumentCommand{\pref}{}{\succ}
\NewDocumentCommand{\prefi}{O{i}}{\succ_{#1}}
\NewDocumentCommand{\prefeqi}{O{i}}{\succeq_{#1}}
\NewDocumentCommand{\prefone}{}{\succ_1}
\NewDocumentCommand{\preftwo}{}{\succ_2}
\NewDocumentCommand{\prefeqone}{}{\succeq_1}
\NewDocumentCommand{\prefeqtwo}{}{\succeq_2}
\NewDocumentCommand{\prefiinv}{O{i}}{\succ_{#1}^{-1}}
\NewDocumentCommand{\ibar}{}{\overline{i}}

\NewDocumentCommand{\lvs}{}{\intvl{0, m - 1}^N}
\NewDocumentCommand{\losses}{}{\intvl{0, m - 1}}
\NewDocumentCommand{\PD}{}{\mathit{PD}(\prof)}
\NewDocumentCommand{\PE}{}{\mathit{PE}(\prof)}

%Rules
\NewDocumentCommand{\rhoP}{}{\rho_{\prof}}
\NewDocumentCommand{\minspread}{O{A}}{\min_{#1}(\sigma \circ \lambda_{\bm{P}})}
\NewDocumentCommand{\mindisp}{O{A}}{\min_{#1}(d \circ \lambda_{\bm{P}})}
\NewDocumentCommand{\FB}{}{\mathit{FB}}
\NewDocumentCommand{\VR}{}{\mathit{VR}}
\NewDocumentCommand{\SL}{}{\mathit{SL}}
\NewDocumentCommand{\PVv}{O{v}}{\mathit{PV}^{#1}}
\NewDocumentCommand{\PVef}{}{\mathit{PV}^{\floor{\frac{m - 1}{2}}}}%f for first
\NewDocumentCommand{\PVes}{}{\mathit{PV}^{\ceil{\frac{m - 1}{2}}}}%s for second
\NewDocumentCommand{\PVe}{}{\mathit{PV^=}}%egalitarian distribution

%Classes
\NewDocumentCommand{\PVcl}{}{\mathcal{PV}}
\NewDocumentCommand{\PVbcl}{}{\mathcal{PV}^b}
\NewDocumentCommand{\PVecl}{}{\mathcal{PV}^=}%egalitarian distribution
\NewDocumentCommand{\PEcl}{}{\mathcal{PE}}
\NewDocumentCommand{\FHcl}{}{\mathcal{FH}}
\NewDocumentCommand{\VCcl}{}{\mathcal{VC}}
\NewDocumentCommand{\VCecl}{}{\mathcal{VC^=}}
\NewDocumentCommand{\ELcl}{}{\mathcal{EL}}
\NewDocumentCommand{\WELcl}{}{\mathcal{WEL}}
\NewDocumentCommand{\PELcl}{}{\mathcal{PEL}}
\NewDocumentCommand{\WPELcl}{}{\mathcal{WPEL}}


%\input{preamble/draw}

%I find these settings useful in draft mode. Should be removed for final versions.
	%Which line breaks are chosen: accept worse lines, therefore reducing risk of overfull lines. Default = 200.
		\tolerance=2000
	%Accept overfull hbox up to...
		\hfuzz=2cm
	%Reduces verbosity about the bad line breaks.
		\hbadness 5000
	%Reduces verbosity about the underful vboxes.
		\vbadness=1300

\title{Two-person SCRs in a unified framework}
\author{Name}
%\author{Olivier Cailloux}
\affil{Université Paris-Dauphine, PSL Research University, CNRS, LAMSADE, 75016 PARIS, FRANCE\\
%	\href{mailto:olivier.cailloux@dauphine.fr}{olivier.cailloux@dauphine.fr}
}
%\author{Name3}
%\affil{Affil2}
\hypersetup{
	pdfsubject={},
	pdfkeywords={},
}

\begin{document}
\maketitle

\section{Introduction}
\label{sec:intro}
\section{Basic notions and notation}
Let $N = \set{1, 2}$ be the set of individuals and $\allalts$ be the set of alternatives, with $\card{\allalts} = m$. 
Let $\powersetz{\allalts}$ denote the non empty subsets of $\allalts$.
Let $\linors$ be the set of linear orders over $\alts$.
The set of possible profiles is $\allprofs$. 
A social choice rule (SCR) is a function $f: \allprofs → \powersetz{\allalts}$.
Given two SCRs $f$, $f'$, we write $f \subset f'$ to indicate that $f$ is a proper subcorrespondence of $f'$.
Let $\PO$ be the Pareto-optimal alternatives in $\prof$.
\commentRS{define PO formally.}
\commentRS{define loss formally without referring to rank.}

Given $j, l \in \N$, let $\intvl{j, l} = [j, l] \cap \N = \set{k \in \N \suchthat j ≤ k ≤ l}$ denote the interval of integer numbers between $j$ and $l$.



The loss vector of $x$ at $\prof$ is $\lprof(x) \in \intvl{0, m - 1}^N$ that associates to each individual the number of ranks the individual lost by having $x$ picked instead of his favorite alternative (that is, the number of alternatives that the individual prefers to $x$). 
Let $\bigcup_{k \in \losses}(k^N)$ denote the set of constant loss vectors.
\commentRS{define anonymity and neutrality}

\section{Four SCRs of the literature}

Given $\prof$, we say that an alternative has $k$ supports at rank $r$ iff exactly $k$ individuals place the alternative at rank $r$ or better. Let $\rho$ (depending on $\prof$, omitted in the notation) be the best rank at which some alternative receives unanimous support.
The rule $\FB$ picks all alternatives with two supports at $\rho$. Note that this set contains either one or two alternatives.

The class $\PV$ contains rules parametrized by $v_1 \in \intvl{0, m - 1}$. Given a parameter $v_1$, define $v_2 = (m - 1) - v_1$. The value $v_i$ represents the number of alternatives vetoed by individual $i$, with $i \in N$ (individuals veto the alternatives at the bottom of their preference). The rule $\PVv$ picks all alternatives in $\PO$ that nobody vetoed. Let $\set{\PVef, \PVes} = \PVe \subset \PV$ be the class of Pareto and Veto rules that distribute veto power in the most egalitarian manner possible. Equivalently, $\PVe = \set{\PVv \suchthat \abs{v_1 - v_2} ≤ 1}$. When $m$ is odd, we use $\PVe$ also to denote the rule $\PVef = \PVes$.

The rule $\VR$ is defined only for $m$ odd. Each individual veto her worst $\frac{m - 1}{2}$ alternatives, then the Borda winners among the non vetoed alternatives are picked.

The rule $\SL$ is defined only for $m$ odd. It picks the best alternative of $1$ that is not among the worst $\frac{m - 1}{2}$ alternatives of $2$, and the best alternative of $2$ that is not among the worst $\frac{m - 1}{2}$ alternatives of $1$.

\commentRS{define FB, VR, SL and the class PV specifying PVe a bit more formally. Distingush between script PV as a class and non script PV as a rule.}

\commentRS{Make formal remarks about their anonymity and neutrality properties.}
\Cref{fig:disjoint} illustrate that $\VR$, $\SL$ and $\FB$ can pick disjoint winners.
\commentRS{Show through a formal proposition that FB, SL and VR can be disjoint.}
To discuss the relationship of FB, SL and VR to members of PVscript, we define condition we call the first-half property.
\begin{definition}[First half (FH)]
	$\forall \prof$, $f(\prof)$ must pick among the alternatives that both individuals rank within the first $\ceil{(m + 1) / 2}$ ranks.
\end{definition}

\commentRS{Show that FB, SL and VR satisfy FH.}

\begin{proposition}
	$\PV ∩ FH = \PVe$
\end{proposition}
\begin{proof}
	$\PVv$ satisfies FH iff $v_1, v_2 ≥ \floor{\frac{m - 1}{2}}$.
\end{proof}

Considering two SCRs $f$ and $f'$, let $f \cup f'$ denote the rule $(f \cup f')(\prof) = f(\prof) \cup f'(\prof)$. Say that a class of rules $F$ is closed under union whenever $\forall f, f' \in F: f \cup f' \in F$. For such a class, if it is non empty, $[(\forall f \in F, f \subseteq f') \land (f' \in F)] ⇔ [f' = \bigcup(F)]$.
\commentRS{Show that there is a unique SCR that is PO, FH and set inclusion maximal which is PVe. Te proof remarks that FH and PO is closed under union.}

\begin{proposition}
	$\bigcup(Paretianism \cap FH) = \PVe$
\end{proposition}
Corollary: A SCR F is FH and Pareto iff F is a subcorrespondence of PV-equalvetoes.”




\commentRS{Show also that FB, SL and VR are subcorrespondences of a PV rule iff that PV rule is PVe. Tje proposition below partly handles that}

\begin{proposition}
	$\SL, \VR, \FB \subset \PVe$
\end{proposition}



\begin{figure}
	\caption{$[\FB(\prof) = \set{b}, \SL(\prof) = \set{a, c}] \subset [\VR(\prof) = \PVe(\prof) = \set{a, b, c}]$}
	\label{fig:cbade}
	$\begin{array}{l}
		abcde\\
		cbade
	\end{array}$
\end{figure}
\begin{figure}
	\caption{$[\VR(\prof) = \set{a}] \subset [\FB(\prof) = \SL(\prof) = \PVe(\prof) = \set{a, c}]$}
	\label{fig:dcabe}
	$\begin{array}{l}
		abcde\\
		dcabe
	\end{array}$
\end{figure}
\begin{figure}
	\caption{$\VR(\prof) = \set{b}, \SL(\prof) = \set{a, d}, \VR(\prof) \cap \SL(\prof) = \emptyset$}
	\label{fig:dbcaefg}
	$\begin{array}{l}
		abcdefg\\
		dbcaefg
	\end{array}$
\end{figure}
\begin{figure}
	\caption{$\VR(\prof) = \set{a}, \SL(\prof) = \set{a, c}, \FB(\prof) = \set{c}, \VR(\prof) \cap \FB(\prof) = \emptyset$}
	\label{fig:fgcabde}
	$\begin{array}{l}
		abcdefg\\
		fgcabde
	\end{array}$
\end{figure}
\begin{figure}
	\caption{$\VR(\prof) = \set{b}, \SL(\prof) = \set{a, g}, \FB(\prof) = \set{d}$}
	\label{fig:disjoint}
	$\begin{array}{llll lll | llll ll}
		a&b&c&d&e&f&g&h&i&j&k&l&m\\
		g&h&i&d&b&j&a&c&e&f&k&l&m\\
	\end{array}$
\end{figure}


\commentRS{Below are characterizations}
PV-e iff PO, FH, MAskin monotonicity

I recall from a paper of Vincent that FB is characterized by using a Rawlsian maximin axiom plus maximality 

For SL and VR we should look at Clippel. 

Sertel shows (says Danilo Coelho) that the unanimous compromise set (the FB winners?) is the set of Pareto efficient candidates that maximize the worst-of welfare (welfare = nb alternatives less good than the chosen one)



\section{New SCRs}
\subsection{Based on equal loss}
\begin{definition}[Equal loss (EL)]
	$f$ must pick at least some alternative ranked the same by everybody, if there is one.
\end{definition}
\begin{definition}[Paretian equal loss (PEL)]
	$f$ must pick solely the alternative ranked the same by everybody and that is efficient, if there is one.
\end{definition}
\begin{definition}[Paretian compromism compatibility (PCC)]
	Define $\Sigma = \set{\sigma \in (\R^+)^{(\lvs)} \suchthat \sigma^{-1}(0) = \cup_{k \in \losses}(k^N)}$ be the set of all spread measures that attribute the minimal inequality measure to the constant loss vectors and none others.
	\commentOC{Could we perhaps define sigma as a weak order over the loss vectors?}
	$f$ is PCC iff there exists a spread measure $\sigma \in \Sigma$ such that $\forall \prof: f(\prof) \cap \minspread[\PO] ≠ \emptyset$.
\end{definition}

Call the dispersion of a loss vector $l$ at $\prof$ the value $d(l) = \abs{l_1 - l_2}$. 
Note that $d \in \Sigma$ and seems to coincide with multiple commonly used spread measures (to be verified). 
Thus, $(d \circ \lprof)(x) = \max\lprof(x) - \min\lprof(x)$.
\begin{definition}[Dispersion compromism]
	$\forall \prof: f(\prof) \subseteq \mindisp[\PO]$.
\end{definition}

\begin{definition}[Dispersion phobia]
	If $x$ has a strictly higher dispersion than $y$ at $\prof$, then $f$ may not pick $x$.
\end{definition}
\begin{definition}[Loss vector comparing (LVC)]
	Let $\succeq$ be a weak order on $\lvs$. The rule $f^\succeq$ selects all alternatives whose loss vectors are minimal elements of $\restr{{\succeq}}{\lprof(\alts)}$. A rule satisfies LVC iff $\exists {\succeq} \suchthat f = f^\succeq$. (This property is called preorder based in Cailloux and Endriss 2014.)
\end{definition}

\commentOC{I do not understand this.}
A stronger version of PEL is what we used in the paper with Beatrice. Fix a set SIGMA of “acceptable” equality measures and ask from F to always pick an alternative with equal loss according to some member of SIGMA. This is stronger than PEL because being “acceptable” requires that an alternative at the same rank for both players as the most equal loss.

$Paretianism ∩ EL = \emptyset$.



\begin{proposition}
VR, SL, and all rules in PV fail PEL
\end{proposition}
\begin{proposition}
FB satisfies PEL
\end{proposition}


PV, FB, SL and VR, all fail strong PEL. 

In fact, strong PEL and FH are incompatible conditions. 


What are interesting rules that satisfy this stronger version of PEL? 

As Matias suggested, can we use strong PEL to refine VR, SL or FB?

Can we identify an interesting SCR that satisfies strong PEL?

\subsection{Based on refining}
Discuss how discriminative the four rules are. ll four rules are irresolute. Do they have refinements that are anonymous and neutral? The answer is obviously affirmative for PV (as the remaining three rules do the job) but should be addressed for FB, SL and VR. 
The Theorem, stated at the end of page 4 (\hrefblue{https://link.springer.com/article/10.1007/BF00187429}{Sprumont}), is about refinements of FB. It says that FB admits refinements that satisfy the two conditions of the paper, namely CPB and SO. Moreover, any rule that satisfies CPB and SO is a refinement of FB.

We can refer to this when we address the question of refining FB, SL and VR.









\subsection{Rules we invent}
Anonymized PV rules
\section{Miscellaneous comments}
I proposed: [VR selects equal loss ⇒ VR selects all alternatives with "almost equal loss"], but this is contradicted by: (abc; cba).


Danilo Coelho seems to be interested in implementation of FB and shortlisting and related rules (see the \href{https://www.cmss.auckland.ac.nz/2020/06/03/online-social-choice-seminar-series/}{Online Social Choice Seminar}

%\bibliography{bibl}

\end{document}

