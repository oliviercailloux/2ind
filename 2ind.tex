\RequirePackage[l2tabu]{nag}
\documentclass[version=3.21, pagesize, twoside=off, bibliography=totoc, DIV=calc, fontsize=12pt, a4paper]{scrartcl}
%Permits to copy eg x ⪰ y ⇔ v(x) ≥ v(y) from PDF to unicode data, and to search. From pdfTeX users manual. See https://tex.stackexchange.com/posts/comments/1203887.
	\input glyphtounicode
	\pdfgentounicode=1
%Latin Modern has more glyphs than Computer Modern, such as diacritical characters. fntguide commands to load the font before fontenc, to prevent default loading of cmr.
	\usepackage{lmodern}
%Encode resulting accented characters correctly in resulting PDF, permits copy from PDF.
	\usepackage[T1]{fontenc}
%UTF8 seems to be the default in recent TeX installations, but not all, see https://tex.stackexchange.com/a/370280.
	\usepackage[utf8]{inputenc}
%Provides \newunicodechar for easy definition of supplementary UTF8 characters such as → or ≤ for use in source code.
	\usepackage{newunicodechar}
%Text Companion fonts, much used together with CM-like fonts. Provides \texteuro and commands for text mode characters such as \textminus, \textrightarrow, \textlbrackdbl.
	\usepackage{textcomp}
%St Mary’s Road symbol font, used for ⟦ = \llbracket. The \SetSymbolFont command avoids spurious warnings, but also some valid ones, see https://tex.stackexchange.com/a/106719/.
	\usepackage{stmaryrd}\SetSymbolFont{stmry}{bold}{U}{stmry}{m}{n}
%Solves bug in lmodern, https://tex.stackexchange.com/a/261188; probably useful only for unusually big font sizes; and probably better to use exscale instead. Note that the authors of exscale write against this trick.
	%\DeclareFontShape{OMX}{cmex}{m}{n}{
		%<-7.5> cmex7
		%<7.5-8.5> cmex8
		%<8.5-9.5> cmex9
		%<9.5-> cmex10
	%}{}
	%\SetSymbolFont{largesymbols}{normal}{OMX}{cmex}{m}{n}
%More symbols (such as \sum) available in bold version, see https://github.com/latex3/latex2e/issues/71. In article mode (but not in presentation mode), also hides some potentially useful warnings such as when using $\bm{\llbracket}$, see stmaryrd in this document (not sure why).
	\DeclareFontShape{OMX}{cmex}{bx}{n}{%
	   <->sfixed*cmexb10%
	   }{}
	\SetSymbolFont{largesymbols}{bold}{OMX}{cmex}{bx}{n}
%For small caps also in italics, see https://tex.stackexchange.com/questions/32942/italic-shape-needed-in-small-caps-fonts, https://tex.stackexchange.com/questions/284338/italic-small-caps-not-working.
	\usepackage{slantsc}
	\AtBeginDocument{%
		%“Since nearly no font family will contain real italic small caps variants, the best approach is to substitute them by slanted variants.” -- slantsc doc
		%\DeclareFontShape{T1}{lmr}{m}{scit}{<->ssub*lmr/m/scsl}{}%
		%There’s no bold small caps in Latin Modern, we switch to Computer Modern for bold small caps, see https://tex.stackexchange.com/a/22241
		%\DeclareFontShape{T1}{lmr}{bx}{sc}{<->ssub*cmr/bx/sc}{}%
		%\DeclareFontShape{T1}{lmr}{bx}{scit}{<->ssub*cmr/bx/scsl}{}%
	}
%Warn about missing characters.
	\tracinglostchars=2
%Nicer tables: provides \toprule, \midrule, \bottomrule.
	\usepackage{booktabs}
%For new column type X which stretches; can be used together with booktabs, see https://tex.stackexchange.com/a/97137. “tabularx modifies the widths of the columns, whereas tabular* modifies the widths of the inter-column spaces.” Loads array.
	%\usepackage{tabularx}
%math-mode version of "l" column type. Requires \usepackage{array}.
	%\usepackage{array}
	%\newcolumntype{L}{>{$}l<{$}}
%Provides \xpretocmd and loads etoolbox which provides \apptocmd, \patchcmd, \newtoggle… Also loads xparse, which provides \NewDocumentCommand and similar commands intended as replacement of \newcommand in LaTeX3 for defining commands (see https://tex.stackexchange.com/q/98152 and https://github.com/latex3/latex2e/issues/89).
	\usepackage{xpatch}
%for \nexists and because it is a basic package nowadays, see https://tex.stackexchange.com/q/539592/.
	\usepackage{amssymb}
%loads and fixes some bugs in amsmath (a basic, mandatory package nowadays, see Grätzer, More Math Into LaTeX) and provides \DeclarePairedDelimiter. I recommend \begin{equation}, which allows numbering, rather than \[ (and $$ should be avoided), see https://tex.stackexchange.com/questions/503. Relatedly, do not use the displaymath environment: use equation. Do not use the eqnarray environment: use align. This improves spacing. (See l2tabu or amsldoc.)
	\usepackage{mathtools}
%Package frenchb asks to load natbib before babel-french. Package hyperref asks to load natbib before hyperref.
	\usepackage{natbib}

\newtoggle{LCpres}
	\newtoggle{LCart}
	\newtoggle{LCposter}
	\makeatletter
	\@ifclassloaded{beamer}{
		\toggletrue{LCpres}
		\togglefalse{LCart}
		\togglefalse{LCposter}
		\wlog{Presentation mode}
	}{
		\@ifclassloaded{tikzposter}{
			\toggletrue{LCposter}
			\togglefalse{LCpres}
			\togglefalse{LCart}
			\wlog{Poster mode}
		}{
			\toggletrue{LCart}
			\togglefalse{LCpres}
			\togglefalse{LCposter}
			\wlog{Article mode}
		}
	}
	\makeatother%

%Language options ([french, english]) should be on the document level (last is main); except with tikzposter: put [french, english] options next to \usepackage{babel} to avoid warning. beamer uses the \translate command for the appendix: omitting babel results in a warning, see https://github.com/josephwright/beamer/issues/449. Babel also seems required for \refname.
	\iftoggle{LCpres}{
		\usepackage{babel}
	}{
	}
	%\frenchbsetup{AutoSpacePunctuation=false}
%https://ctan.org/pkg/amsmath recommends ntheorem, which supersedes amsthm, which corrects the spacing of proclamations and allows for theoremstyle, but I decided to switch to amsthm with thmtools (mentioned in amsthm doc) because ntheorem “seems essentially unmaintaned and has severe problems”, see https://tex.stackexchange.com/q/535950. Must be loaded after amsmath (from amsthm doc).
		\usepackage{amsthm}
		\usepackage{thmtools}
%listings (1.7) does not allow multi-byte encodings. listingsutf8 works around this only for characters that can be represented in a known one-byte encoding and only for \lstinputlisting. Other workarounds: use literate mechanism; or escape to LaTeX (but breaks alignment).
	%\usepackage{listings}
	%\lstset{tabsize=2, basicstyle=\ttfamily, escapechar=§, literate={é}{{\'e}}1}
%I favor acro over acronym because the former is more recently updated (2018 VS 2015 at time of writing); has a longer user manual (about 40 pages VS 6 pages if not counting the example and implementation parts); has a command for capitalization; and acronym suffers a nasty bug when ac used in section, see https://tex.stackexchange.com/q/103483 (though this might be the fault of the silence package and might be solved in more recent versions, I do not know) and from a bug when used with cleveref, see https://tex.stackexchange.com/q/71364. However, loading it makes compilation time (one pass on this template) go from 0.6 to 1.4 seconds, see https://bitbucket.org/cgnieder/acro/issues/115.
	\usepackage{acro}
	%“All options of acro that have not been mentioned in section 4.1 have to be set up… with… \acsetup{…}” -- acro package doc, cited by the Overleaf support (thanks to them!)
	\acsetup{single}
	\DeclareAcronym{AMCD}{short=AMCD, long={Aide Multicritère à la Décision}}
\DeclareAcronym{AHP}{short=AHP, long={Analytic Hierarchy Process}}
\DeclareAcronym{AR}{short=AR, long={Argumentative Recommender}}
\DeclareAcronym{DA}{short=DA, long={Decision Analysis}}
\DeclareAcronym{DJ}{short=DJ, long={Deliberated Judgment}}
\DeclareAcronym{DM}{short=DM, long={Decision Maker}}
\DeclareAcronym{DP}{short=DP, long={Deliberated Preference}}
\DeclareAcronym{MAVT}{short=MAVT, long={Multiple Attribute Value Theory}}
\DeclareAcronym{MCDA}{short=MCDA, long={Multicriteria Decision Aid}}
\DeclareAcronym{MIP}{short=MIP, long={Mixed Integer Program}}
\DeclareAcronym{SEU}{short=SEU, long={Subjective Expected Utility}}


\iftoggle{LCpres}{
	%I favor fmtcount over nth because it is loaded by datetime anyway; and fmtcount warns about possible conflicts when loaded after nth (“\ordinal already defined use \FCordinal instead”). See also https://english.stackexchange.com/questions/93008.
	\usepackage{fmtcount}
	%For nice input of date of presentation. Must be loaded after the babel package. Has possible problems with srcletter: https://golatex.de/verwendung-von-babel-und-datetime-in-scrlttr2-schlaegt-fehlt-t14779.html.
	\usepackage[nodayofweek]{datetime}
}{
}
%For presentations, Beamer implicitely uses the pdfusetitle option. autonum doc mandates option hypertexnames=false. I want to highlight links only if necessary for the reader to recognize it as a link, to reduce distraction. In presentations, this is already taken care of by beamer (https://tex.stackexchange.com/a/262014). If using colorlinks=true in a presentation, see https://tex.stackexchange.com/q/203056. Crashes the first compilation with tikzposter, just compile again and the problem disappears, see https://tex.stackexchange.com/q/254257.
\makeatletter
\iftoggle{LCpres}{
	\usepackage{hyperref}
}{
	\usepackage[hypertexnames=false, pdfusetitle, linkbordercolor={1 1 1}, citebordercolor={1 1 1}, urlbordercolor={1 1 1}]{hyperref}
	%https://tex.stackexchange.com/a/466235
	\pdfstringdefDisableCommands{%
		\let\thanks\@gobble
	}
}
\makeatother
%urlbordercolor is used both for \url and \doi, which I think shouldn’t be colored, and for \href, thus might want to color manually when required. Requires xcolor.
	\NewDocumentCommand{\hrefblue}{mm}{\textcolor{blue}{\href{#1}{#2}}}
%hyperref doc says: “Package bookmark replaces hyperref’s bookmark organization by a new algorithm (...) Therefore I recommend using this package”.
	\usepackage{bookmark}
%Need to invoke hyperref explicitly to link to line numbers: \hyperlink{lintarget:mylinelabel}{\ref*{lin:mylinelabel}}, with \ref* to disable automatic link. Also see https://tex.stackexchange.com/q/428656 for referencing lines from another document.
	%\usepackage{lineno}
	%\NewDocumentCommand{\llabel}{m}{\hypertarget{lintarget:#1}{}\linelabel{lin:#1}}
	%\setlength\linenumbersep{9mm}
%For complex authors blocks. Seems like authblk wants to be later than hyperref, but sooner than silence. See https://tex.stackexchange.com/q/475513 for the patch to hyperref pdfauthor.
	\ExplSyntaxOn
	\seq_new:N \g_oc_hrauthor_seq
	\NewDocumentCommand{\addhrauthor}{m}{
		\seq_gput_right:Nn \g_oc_hrauthor_seq { #1 }
	}
	\NewExpandableDocumentCommand{\hrauthor}{}{
		\seq_use:Nn \g_oc_hrauthor_seq {,~}
	}
	\ExplSyntaxOff
	{
		\catcode`#=11\relax
		\gdef\fixauthor{\xpretocmd{\author}{\addhrauthor{#2}}{}{}}%
	}
	\iftoggle{LCart}{
		\usepackage{authblk}
		\renewcommand\Affilfont{\small}
		\fixauthor
		\AtBeginDocument{
		    \hypersetup{pdfauthor={\hrauthor}}
		}
	}{
	}
%I do not use floatrow, because it requires an ugly hack for proper functioning with KOMA script (see scrhack doc). Instead, the following command centers all floats (using \centering, as the center environment adds space, http://texblog.net/latex-archive/layout/center-centering/), and I manually place my table captions above and figure captions below their contents (https://tex.stackexchange.com/a/3253).
	\makeatletter
	\g@addto@macro\@floatboxreset\centering
	\makeatother
%Permits to customize enumeration display and references
	%\nottoggle{LCpres}{
		%\usepackage{enumitem} %follow list environments by a string to customize enumeration, example: \begin{description}[itemindent=8em, labelwidth=!] or \begin{enumerate}[label=({\roman*}), ref={\roman*}].
	%}{
	%}
%Provides \Centering, \RaggedLeft, and \RaggedRight and environments Center, FlushLeft, and FlushRight, which allow hyphenation. With tikzposter, seems to cause 1=1 to be printed in the middle of the poster.
	%\usepackage{ragged2e}
%To typeset units by closely following the “official” rules.
	%\usepackage[strict]{siunitx}
%Turns the doi provided by some bibliography styles into URLs.
	\usepackage{doi}
%Makes sure upper case greek letters are italic as well.
	\usepackage{fixmath}
%Provides \mathbb; obsoletes latexsym (see http://tug.ctan.org/macros/latex/base/latexsym.dtx). Relatedly, \usepackage{eucal} to change the mathcal font and \usepackage[mathscr]{eucal} (apparently equivalent to \usepackage[mathscr]{euscript}) to supplement \mathcal with \mathscr. This last option is not very useful as both fonts are similar, and the intent of the authors of eucal was to provide a replacement to mathcal (see doc euscript). Also provides \mathfrak for supplementary letters.
	\usepackage{amsfonts}
%Provides a beautiful (IMHO) \mathscr and really different than \mathcal, for supplementary uppercase letters. But there is no bold version. Alternative: mathrsfs (more slanted), but when used with tikzposter, it warns about size substitution, see https://tex.stackexchange.com/q/495167.
	\usepackage[scr]{rsfso}
%Multiple means to produce bold math: \mathbf, \boldmath (defined to be \mathversion{bold}, see fntguide), \pmb, \boldsymbol (all legacy, from LaTeX base and AMS), \bm (the most recommended one), \mathbold from package fixmath (I don’t see its advantage over \boldsymbol).
%“The \boldsymbol command is obtained preferably by using the bm package, which provides a newer, more powerful version than the one provided by the amsmath package. Generally speaking, it is ill-advised to apply \boldsymbol to more than one symbol at a time.” — AMS Short math guide. “If no bold font appears to be available for a particular symbol, \bm will use ‘poor man’s bold’” — bm. It is “best to load the package after any packages that define new symbol fonts” – bm. bm defines \boldsymbol as synonym to \bm. \boldmath accesses the correct font if it exists; it is used by \bm when appropriate. See https://tex.stackexchange.com/a/10643 and https://github.com/latex3/latex2e/issues/71 for some difficulties with \bm.
	\usepackage{bm}
	\nottoggle{LCpres}{
	%Provides \cref. Unfortunately, cref fails when the language is French and referring to a label whose name contains a colon (https://tex.stackexchange.com/q/83798). Use \cref{sec\string:intro} to work around this. cleveref should go “laster” than hyperref.
		\usepackage[capitalise]{cleveref}
	}{
	}
	\nottoggle{LCposter}{
	%Equations get numbers iff they are referenced. Loading order should be “amsmath → hyperref → cleveref → autonum”, according to autonum doc. Use this in preference to the showonlyrefs option from mathtools, see https://tex.stackexchange.com/q/459918 and autonum doc. See https://tex.stackexchange.com/a/285953 for the etex line. Incompatible with my version of tikzposter (produces “! Improper \prevdepth”). This removes the starred versions, such as equation*. Unfortunately, this prevents using \qedhere in an equation ending a proof, see https://tex.stackexchange.com/q/133358/.
		\expandafter\def\csname ver@etex.sty\endcsname{3000/12/31}\let\globcount\newcount
		\usepackage{autonum}
	}{
	}
%Also loaded by tikz.
	\usepackage{xcolor}
\iftoggle{LCpres}{
	\usepackage{tikz}
	%\usetikzlibrary{babel, matrix, fit, plotmarks, calc, trees, shapes.geometric, positioning, plothandlers, arrows, shapes.multipart}
}{
}
%Vizualization, on top of TikZ
	%\usepackage{pgfplots}
	%\pgfplotsset{compat=1.14}
\usepackage{graphicx}
	\graphicspath{{graphics/}}

%Provides \printlength{length}, useful for debugging.
	%\usepackage{printlen}
	%\uselengthunit{mm}

\iftoggle{LCpres}{
	\usepackage{appendixnumberbeamer}
	%I have yet to see anyone actually use these navigation symbols; let’s disable them
	\setbeamertemplate{navigation symbols}{} 
	\usepackage{preamble/beamerthemeParisFrance}
	\setcounter{tocdepth}{10}
}{
}

%Requires package xcolor.
\definecolor{ao(english)}{rgb}{0.0, 0.5, 0.0}
\NewDocumentCommand{\commentOC}{m}{\textcolor{blue}{\small$\big[$OC: #1$\big]$}}
%Requires package babel and option [french]. According to babel doc, need two braces around \selectlanguage to make the changes really local.
\NewDocumentCommand{\commentOCf}{m}{\textcolor{blue}{{\small\selectlanguage{french}$\big[$OC : #1$\big]$}}}
\NewDocumentCommand{\commentRS}{m}{\textcolor{red}{\small$\big[$RS: #1$\big]$}}
\NewDocumentCommand{\commentMN}{m}{\textcolor{ao(english)}{\small$\big[$MN: #1$\big]$}}

\bibliographystyle{abbrvnat}
\NewDocumentCommand{\possessivecite}{mO{}}{\citeauthor{#1}’s \citeyearpar[#2]{#1}}

%https://tex.stackexchange.com/a/467188, https://tex.stackexchange.com/a/36088 - uncomment if one of those symbols is used.
%\DeclareFontFamily{U} {MnSymbolD}{}
%\DeclareFontShape{U}{MnSymbolD}{m}{n}{
%  <-6> MnSymbolD5
%  <6-7> MnSymbolD6
%  <7-8> MnSymbolD7
%  <8-9> MnSymbolD8
%  <9-10> MnSymbolD9
%  <10-12> MnSymbolD10
%  <12-> MnSymbolD12}{}
%\DeclareFontShape{U}{MnSymbolD}{b}{n}{
%  <-6> MnSymbolD-Bold5
%  <6-7> MnSymbolD-Bold6
%  <7-8> MnSymbolD-Bold7
%  <8-9> MnSymbolD-Bold8
%  <9-10> MnSymbolD-Bold9
%  <10-12> MnSymbolD-Bold10
%  <12-> MnSymbolD-Bold12}{}
%\DeclareSymbolFont{MnSyD} {U} {MnSymbolD}{m}{n}
%\DeclareMathSymbol{\ntriplesim}{\mathrel}{MnSyD}{126}
%\DeclareMathSymbol{\nlessgtr}{\mathrel}{MnSyD}{192}
%\DeclareMathSymbol{\ngtrless}{\mathrel}{MnSyD}{193}
%\DeclareMathSymbol{\nlesseqgtr}{\mathrel}{MnSyD}{194}
%\DeclareMathSymbol{\ngtreqless}{\mathrel}{MnSyD}{195}
%\DeclareMathSymbol{\nlesseqgtrslant}{\mathrel}{MnSyD}{198}
%\DeclareMathSymbol{\ngtreqlessslant}{\mathrel}{MnSyD}{199}
%\DeclareMathSymbol{\npreccurlyeq}{\mathrel}{MnSyD}{228}
%\DeclareMathSymbol{\nsucccurlyeq}{\mathrel}{MnSyD}{229}
%\DeclareFontFamily{U} {MnSymbolA}{}
%\DeclareFontShape{U}{MnSymbolA}{m}{n}{
%  <-6> MnSymbolA5
%  <6-7> MnSymbolA6
%  <7-8> MnSymbolA7
%  <8-9> MnSymbolA8
%  <9-10> MnSymbolA9
%  <10-12> MnSymbolA10
%  <12-> MnSymbolA12}{}
%\DeclareFontShape{U}{MnSymbolA}{b}{n}{
%  <-6> MnSymbolA-Bold5
%  <6-7> MnSymbolA-Bold6
%  <7-8> MnSymbolA-Bold7
%  <8-9> MnSymbolA-Bold8
%  <9-10> MnSymbolA-Bold9
%  <10-12> MnSymbolA-Bold10
%  <12-> MnSymbolA-Bold12}{}
%\DeclareSymbolFont{MnSyA} {U} {MnSymbolA}{m}{n}
%%Rightwards wave arrow: ↝. Alternative: \rightsquigarrow from amssymb, but it’s uglier
%\DeclareMathSymbol{\rightlsquigarrow}{\mathrel}{MnSyA}{160}

%03B3 Greek Small Letter Gamma
\newunicodechar{γ}{\gamma}
%03B4 Greek Small Letter Delta
\newunicodechar{δ}{\delta}
%2115 Double-Struck Capital N
\newunicodechar{ℕ}{\mathbb{N}}
%211D Double-Struck Capital R
\newunicodechar{ℝ}{\mathbb{R}}
%21CF Rightwards Double Arrow with Stroke
\newunicodechar{⇏}{\nRightarrow}
%21D2 Rightwards Double Arrow
\newunicodechar{⇒}{\ensuremath{\Rightarrow}}
%21D4 Left Right Double Arrow
\newunicodechar{⇔}{\Leftrightarrow}
%21DD Rightwards Squiggle Arrow
\newunicodechar{⇝}{\rightsquigarrow}
%2205 Empty Set
\newunicodechar{∅}{\emptyset}
%2212 Minus Sign
\newunicodechar{−}{\ifmmode{-}\else\textminus\fi}
%2227 Logical And
\newunicodechar{∧}{\land}
%2228 Logical Or
\newunicodechar{∨}{\lor}
%2229 Intersection
\newunicodechar{∩}{\cap}
%222A Union
\newunicodechar{∪}{\cup}
%2260 Not Equal To (handy also as text in informal writing)
\newunicodechar{≠}{\ensuremath{\neq}}
%2264 Less-Than or Equal To
\newunicodechar{≤}{\leq}
%2265 Greater-Than or Equal To
\newunicodechar{≥}{\geq}
%2270 Neither Less-Than nor Equal To
\newunicodechar{≰}{\nleq}
%2271 Neither Greater-Than nor Equal To
\newunicodechar{≱}{\ngeq}
%2272 Less-Than or Equivalent To
\newunicodechar{≲}{\lesssim}
%2273 Greater-Than or Equivalent To
\newunicodechar{≳}{\gtrsim}
%2274 Neither Less-Than nor Equivalent To – also, from MnSymbol: \nprecsim, a more exact match to the Unicode symbol; and \npreccurlyeq, too small
\newunicodechar{≴}{\not\preccurlyeq}
%2275 Neither Greater-Than nor Equivalent To
\newunicodechar{≵}{\not\succcurlyeq}
%2279 Neither Greater-Than nor Less-Than – requires MnSymbol; also \nlessgtr from txfonts/pxfonts, \ngtreqless from MnSymbol (but much higher), \ngtrless from MnSymbol (a more exact match to the Unicode symbol); for incomparability (not matching this Unicode symbol), may also consider \ntriplesim from MnSymbol,\nparallelslant from fourier, \between from mathabx, or ⋈
\newunicodechar{≹}{\ngtreqlessslant}
%227A Precedes
\newunicodechar{≺}{\prec}
%227B Succeeds
\newunicodechar{≻}{\succ}
%227C Precedes or Equal To
\newunicodechar{≼}{\preccurlyeq}
%227D Succeeds or Equal To
\newunicodechar{≽}{\succcurlyeq}
%227E Precedes or Equivalent To
\newunicodechar{≾}{\precsim}
%227F Succeeds or Equivalent To
\newunicodechar{≿}{\succsim}
%2280 Does Not Precede
\newunicodechar{⊀}{\nprec}
%2281 Does Not Succeed
\newunicodechar{⊁}{\nsucc}
%2286
\newunicodechar{⊆}{\subseteq}
%22B2 Normal Subgroup Of – using \vartriangleleft from amsfonts, which goes well with \trianglelefteq, \ntriangleright, and so on, also from amsfonts; another possibility is \lhd from latexsym, which seems visually equivalent to \vartriangleleft from amsfonts; latexsym also has ⊴=\unlhd, but doesn’t have a symbol for ⊴. Other related symbols: \triangleleft from latesym package is too small; fdsymbol provides \triangleleft=\medtriangleleft and \vartriangleleft=\smalltriangleleft; MnSymbol provides \medtriangleleft and \vartriangleleft=\lessclosed=\lhd which are smaller than \vartriangleleft from amsfont; \vartriangleleft from mathabx (p. 67), looks different (wider); also \vartriangleleft from boisik (p. 69) looks still different; \vartriangleleft=\lhd from stix are smaller. Oddly enough, \triangleright appears as the LMMathItalic12-Regular font whereas \rhd appears as LASY10 and \vartriangleright appears as MSAM10.
\newunicodechar{⊲}{\vartriangleleft}
%22B3 Contains as Normal Subgroup (also: 25B7 White right-pointing triangle or 25B9 White right-pointing small triangle)
\newunicodechar{⊳}{\vartriangleright}
%22B4 Normal Subgroup of or Equal To
\newunicodechar{⊴}{\trianglelefteq}
%22B5 Contains as Normal Subgroup or Equal To
\newunicodechar{⊵}{\trianglerighteq}
%22C8 Bowtie
\newunicodechar{⋈}{\bowtie}
%22EA Not Normal Subgroup Of
\newunicodechar{⋪}{\ntriangleleft}
%22EB Does Not Contain As Normal Subgroup
\newunicodechar{⋫}{\ntriangleright}
%22EC Not Normal Subgroup of or Equal To
\newunicodechar{⋬}{\ntrianglelefteq}
%22ED Does Not Contain as Normal Subgroup or Equal
\newunicodechar{⋭}{\ntrianglerighteq}
%25A1 White Square
\newunicodechar{□}{\Box}
%27E6 Mathematical Left White Square Bracket – requires stmaryrd (alternative: \text{\textlbrackdbl}, but ugly if used in an italicized text such as a theorem)
\newunicodechar{⟦}{\llbracket}
%27E7 Mathematical Right White Square Bracket
\newunicodechar{⟧}{\rrbracket}
%27FC Long Rightwards Arrow from Bar
\newunicodechar{⟼}{\longmapsto}
%2AB0 Succeeds Above Single-Line Equals Sign
\newunicodechar{⪰}{\succeq}
%301A Left White Square Bracket
\newunicodechar{〚}{\textlbrackdbl}
%301B Right White Square Bracket
\newunicodechar{〛}{\textrbrackdbl}
%→ is defined by default as \textrightarrow, which is invalid in math mode. Same thing for the three other commands. Using \DeclareUnicodeCharacter instead of \newunicodechar because the latter warns about the previous definition.
%→ Rightwards Arrow
\DeclareUnicodeCharacter{2192}{\ifmmode\rightarrow\else\textrightarrow\fi}
%¬ Not Sign
\DeclareUnicodeCharacter{00AC}{\ifmmode\lnot\else\textlnot\fi}
%… Horizontal Ellipsis
\DeclareUnicodeCharacter{2026}{\ifmmode\dots\else\textellipsis\fi}
%× Multiplication Sign
\DeclareUnicodeCharacter{00D7}{\ifmmode\times\else\texttimes\fi}
%Permits to really obtain a straight quote when typing a straight quote; potentially dangerous, see https://tex.stackexchange.com/a/521999
\catcode`\'=\active
\DeclareUnicodeCharacter{0027}{\ifmmode^\prime\else\textquotesingle\fi}


\NewDocumentCommand{\R}{}{ℝ}
\NewDocumentCommand{\N}{}{ℕ}
%\mathscr is rounder than \mathcal.
\NewDocumentCommand{\powerset}{m}{\mathscr{P}(#1)}
%Powerset without zero.
\NewDocumentCommand{\powersetz}{m}{\mathscr{P}^*(#1)}
%https://tex.stackexchange.com/a/45732, works within both \set and \set*, same spacing than \mid (https://tex.stackexchange.com/a/52905).
\NewDocumentCommand{\suchthat}{}{\;\ifnum\currentgrouptype=16 \middle\fi|\;}
%Integer interval.
\NewDocumentCommand{\intvl}{m}{⟦#1⟧}
%Allows for \abs and \abs*, which resizes the delimiters.
\DeclarePairedDelimiter\abs{\lvert}{\rvert}
\DeclarePairedDelimiter\card{\lvert}{\rvert}
\DeclarePairedDelimiter\floor{\lfloor}{\rfloor}
\DeclarePairedDelimiter\ceil{\lceil}{\rceil}
%Perhaps should use U+2016 ‖ DOUBLE VERTICAL LINE here?
\DeclarePairedDelimiter\norm{\lVert}{\rVert}
%From mathtools. Better than using the package braket because braket introduces possibly undesirable space. Then: \begin{equation}\set*{x \in \R^2 \suchthat \norm{x}<5}\end{equation}.
\DeclarePairedDelimiter\set{\{}{\}}
\DeclareMathOperator*{\argmax}{arg\,max}
\DeclareMathOperator*{\argmin}{arg\,min}

%UTR #25: Unicode support for mathematics recommend to use the straight form of phi (by default, given by \phi) rather than the curly one (by default, given by \varphi), and thus use \phi for the mathematical symbol and not \varphi. I however prefer the curly form because the straight form is too easy to mix up with the symbol for empty set.
\let\phi\varphi

%The amssymb solution.
%\NewDocumentCommand{\restr}{mm}{{#1}_{\restriction #2}}
%Another acceptable solution.
%\NewDocumentCommand{\restr}{mm}{{#1|}_{#2}}
%https://tex.stackexchange.com/a/278631; drawback being that sometimes the text collides with the line below.
\NewDocumentCommand\restr{mm}{#1\raisebox{-.5ex}{$|$}_{#2}}


%Decision Theory
\NewDocumentCommand{\allalts}{}{\mathcal{A}}
\NewDocumentCommand{\allcrits}{}{\mathscr{C}}
\NewDocumentCommand{\alts}{}{A}
\NewDocumentCommand{\dm}{}{i}
\NewDocumentCommand{\allF}{}{\mathscr{F}}
\NewDocumentCommand{\allvoters}{}{\mathscr{N}}
\NewDocumentCommand{\voters}{}{N}
\NewDocumentCommand{\allprofs}{}{\linors^N}
\NewDocumentCommand{\prof}{}{\bm{P}}
\NewDocumentCommand{\lprof}{}{\lambda_{\bm{P}}}
\NewDocumentCommand{\linors}{}{\mathscr{L}(\allalts)}
%Thanks to https://tex.stackexchange.com/q/154549
	%\makeatletter
	%\def\@myRgood@#1#2{\mathrel{R^X_{#2}}}
	%\def\myRgood{\@ifnextchar_{\@myRgood@}{\mathrel{R^X}}}
	%\makeatother
\NewDocumentCommand{\pref}{}{\succ}
\NewDocumentCommand{\prefi}{O{i}}{\succ_{#1}}
\NewDocumentCommand{\prefiinv}{O{i}}{\succ_{#1}^{-1}}
\NewDocumentCommand{\ibar}{}{\overline{i}}

\NewDocumentCommand{\lvs}{}{\intvl{0, m - 1}^N}
\NewDocumentCommand{\losses}{}{\intvl{0, m - 1}}
\NewDocumentCommand{\PD}{}{\mathit{PD}(\prof)}
\NewDocumentCommand{\PE}{}{\mathit{PE}(\prof)}

%Rules
\NewDocumentCommand{\rhoP}{}{\rho_{\prof}}
\NewDocumentCommand{\minspread}{O{A}}{\min_{#1}(\sigma \circ \lambda_{\bm{P}})}
\NewDocumentCommand{\mindisp}{O{A}}{\min_{#1}(d \circ \lambda_{\bm{P}})}
\NewDocumentCommand{\FB}{}{\mathit{FB}}
\NewDocumentCommand{\VR}{}{\mathit{VR}}
\NewDocumentCommand{\SL}{}{\mathit{SL}}
\NewDocumentCommand{\PVv}{O{v}}{\mathit{PV}^{#1}}
\NewDocumentCommand{\PVef}{}{\mathit{PV}^{\floor{\frac{m - 1}{2}}}}%f for first
\NewDocumentCommand{\PVes}{}{\mathit{PV}^{\ceil{\frac{m - 1}{2}}}}%s for second
\NewDocumentCommand{\PVe}{}{\mathit{PV^=}}%egalitarian distribution

%Classes
\NewDocumentCommand{\PVcl}{}{\mathcal{PV}}
\NewDocumentCommand{\PVbcl}{}{\mathcal{PV}^b}
\NewDocumentCommand{\PVecl}{}{\mathcal{PV}^=}%egalitarian distribution
\NewDocumentCommand{\PEcl}{}{\mathcal{PE}}
\NewDocumentCommand{\FHcl}{}{\mathcal{FH}}
\NewDocumentCommand{\VCcl}{}{\mathcal{VC}}
\NewDocumentCommand{\VCecl}{}{\mathcal{VC^=}}
\NewDocumentCommand{\ELcl}{}{\mathcal{EL}}



%I find these settings useful in draft mode. Should be removed for final versions.
	%Which line breaks are chosen: accept worse lines, therefore reducing risk of overfull lines. Default = 200.
		\tolerance=2000
	%Accept overfull hbox up to...
		\hfuzz=2cm
	%Reduces verbosity about the bad line breaks.
		\hbadness 5000
	%Reduces verbosity about the underful vboxes.
		\vbadness=1300

\title{Two principles for two-person social choice}
\author{Name}
%\author{Olivier Cailloux}
\affil{Université Paris-Dauphine, PSL Research University, CNRS, LAMSADE, 75016 PARIS, FRANCE\\
%	\href{mailto:olivier.cailloux@dauphine.fr}{olivier.cailloux@dauphine.fr}
}
%\author{Name3}
%\affil{Affil2}
\hypersetup{
	pdfsubject={},
	pdfkeywords={},
}

\begin{document}
\maketitle

\section{Introduction}
\label{sec:intro}

\commentRS{As a general concern about the presentation of our results, we must decide what is a proposition what is a theorem. I prefer calling all propositions (except 9) as "theorem" and Proposition 9 is a lemma.}
Two-person discrete social choice models allow a specific interpretation of collective decision making: Bargaining over a finite set of alternatives. Since the seminal model of \cite{Nash1950}, for a long time, bargaining problems were formulated over a convex set of alternatives. However, there are many instances where bargaining takes place over a finite set of alternatives. Thus, this simplifying assumption of \cite{Nash1950} excludes several real-life situations. 

\cite{Mariotti1998} is among the first to relax this assumption by characterizing the Nash solution for a finite set of alternatives. His approach is followed by \cite{nagahisa2002axiomatization} who, again in a finite setting, characterize the solution of \cite{kalai1975other}. Both characterizations are built in a cardinal framework. 
 
An ordinal framework of two-person finite bargaining problems is presented by \cite{BramsKilgour2001} who introduce and analyze an ordinal solution, namely \textit{fallback bargaining}, that is based on compromising where each of the two bargainers begins by claiming the best outcome with respect to his ranking of alternatives. When the claims of the two bargainers differ, they continue by falling back, in lockstep, to lower ranked alternatives until a mutually (hence unanimously) agreed outcome is found. A characterization of fallback bargaining is provided by \cite{de2012reason}. As this solution is presented in a model that does not admit a disagreement point, fallback bargaining is rather an arbitration rule than being a bargaining solution. An analysis of fallback bargaining in a model with a disagreement point is made by \cite{KibrisSertel2007} who rebaptize the solution as \textit{unanimity compromise} and define several variants of it. One of these variants, the \textit{imputational compromise}, is further studied by \cite{ConleyWilkie2012}.
 
As a matter of fact, the compromising approach that underlies fallback bargaining was originally used to design voting rules in settings with more than two individuals with the required support varying from unanimity to simple majority, such as the \textit{Kant-Rawls Social Compromise} by \cite{HurwiczSertel1997}  and the \textit{majoritarian compromise} by \cite{sertel1999majoritarian}. It also paved the way to new axioms for social choice, such as \textit{majoritarian approval} and \textit{majoritarian optimality} as well as \textit{efficiency in the degree of compromise} by \cite{ozkal2004efficiency}.
\cite{merlin2019compromise} present a recent comprehensive analysis of voting rules and axioms based on this compromising idea.

A closer look at fallback bargaining reveals a principle for two-person social choice.

\cite{Sprumont1993} qualifies arbitration rules that maximize the welfare of the least happy individual as being \textit{Rawlsian}. \cite{congar2012characterization} characterize the Rawlsian principle within the framework of social welfare functions. For social choice rules, \cite{BramsKilgour2001} establish the equivalence between the Rawlsian principle and fallback bargaining. Moreover, they show that every individual ranks a fallback bargaining outcome in the upper half of his ranking. Thus, at every preference profile, there is an alternative that both individuals rank in the upper half of their preference. In other words, the least happy individual of the society can always be granted a welfare within the first half of his preference ranking. We qualify a social choice rule that complies with this possibility as \textit{minimally Rawlsian}.
\footnote{In a two-person collective choice framework with an interpretation that is more specific than ours, \cite{Clippel} call this condition the \textit{minimal satisfaction test}.}

Another principle for two-person social choice is proposed by \cite{cailloux2022compromising} who propose a different conception of compromising based on the \textit{equal loss principle} that favors outcomes where every individual concedes as equally as possible from his highest ranked alternative. Several two-person social choice rules that are minimally Rawlsian fail to comply with the equal loss principle, suggesting an incompatibility between these two principles.
 
Although the minimal Rawlsian and equal loss principles cover many of the two-person social choice rules, the literature is missing an axiomatic analysis of these rules from this perspective, an observation which forms the subject matter of our paper. We consider the following rules:

Fallback bargaining, as defined by \cite{BramsKilgour2001};

The \textit{veto-rank mechanism} where, given an odd number m of alternatives, each individual vetoes $(m−1) / 2$ alternatives and ranks the remaining $(m+1) / 2$. The outcome is the alternative with the minimal sum of ranks among those that have not been vetoed. The \textit{shortlisting procedure} where one individual selects $(m+1) / 2$ alternatives and the other individual decides on the outcome out of that shortlist. Both mechanisms are used for the selection of arbitrators and their strategic aspects are comprehensively analyzed by \cite{Clippel}.


The class of \textit{Pareto-and-veto rules} where each individual i vetoes a fixed number $v_i$ of alternatives with $v_1$ + $v_2$ being lower than the total number of alternatives m. The outcome is the set of Pareto optimal alternatives that are not vetoed. This class generalizes the Pareto-and-veto rules analyzed by \cite{laslier2021solution} which impose $v_1$ + $v_2$ = $m-1$.
  
We consider most of the two-person social choice rules in the literature. The literature also admits various interesting real-life procedures expressed as extensive form games, such as those in \cite{anbarci1993noncooperative, anbarci2006finite} and \cite{barbera2022compromising}. However, as shown in these papers, the subgame perfect equilibrium outcomes of these games are always among the alternatives that fallback bargaining would choose. 

We now summarize our findings. All rules we consider are Paretian. Fallback bargaining, the veto-rank mechanism and the shortlisting procedure are minimally Rawlsian. The class of minimally Rawlsian Pareto-and-veto rules is restricted to those that gives the highest (almost) equal veto power to both individuals. Moreover, when the veto power is equal, the corresponding Pareto-and-veto rule is a super correspondence of every Paretian and minimally Rawlsian social choice rule. Thus, fallback bargaining, the veto-rank mechanism and the shortlisting procedure are all sub correspondences of the Pareto-and-veto rule with the highest equal veto power.

The equal loss principle we consider favors outcomes that have the same rank for both individuals. Without imposing Pareto optimality separately, this principle may lead to Pareto dominated outcomes. Thus, we consider a Paretian version that favors, among the Pareto optimal outcomes, the one that has the same rank for both individuals. Note that such an alternative, if it exists, will be unique. We define two versions of the Paretian equal loss principle, one being stronger than the other. The stronger version requires that the Pareto optimal alternative that has the same rank for both individuals must be uniquely chosen. Under the weaker version it suffices that this alternative be among the outcomes. The veto-rank mechanism and the shortlisting procedure both fail the weak (hence strong) version of the Paretian equal loss principle. While Pareto-and-veto rules that endow individuals with a veto power that doesn't exceed $\floor{\frac{m}{2}}$ satisfy the weak Paretian equal loss principle, all of them fail the strong version. On the other hand, fallback bargaining satisfies the strong Paretian equal loss principle, thus showing that this principle is compatible with being minimally Rawlsian.
 
Within the spirit of equal loss, we propose the \textit{minimal dispersion principle} as another strengthening of the (weak) Paretian equal loss principle. The \textit{dispersion} of an alternative is the difference between the two ranks at which is it placed at the preferences of the two individuals. The minimal dispersion principle requires that an alternative whose dispersion is minimal must be among the outcomes. \footnote{
In a framework with a disagreement outcome, \cite{KibrisSertel2007} argue that the finite version of the \textit{equal area rule} (see \cite{thomson1994cooperative}) minimizes the difference between losses with respect to individually rational alternatives. They also show that this rule differs from unanimity compromise which is fallback bargaining defined in their more general framework.} Not only fallback bargaining fails this principle, but the minimal dispersion principle turns out to be logically incompatible with the minimal Rawlsian principle. As a result, among the social choice rules we consider, the only candidates to satisfy the minimal dispersion principle are the Pareto-and-veto rules that fail the minimal Rawlsian principle. As a matter of fact, those that endow individuals with a veto power that doesn't exceed a third of the available alternatives turn out to satisfy the minimal dispersion principle. Recall that this upper bound is a half for the satisfaction of the weaker Paretian equal loss principle.

Given the incompatibility between the two principles, we introduce the Rawlsian minimal dispersion principle that requires the outcome to contain the Rawlsian alternatives whose loss vectors have minimal dispersion. It turns out that, except the Pareto-and-veto rule that gives the highest equal veto power to both individuals, it is failed by all social choice rules we consider,

Section 2 introduces the basic notions and notation. Sections 3 and 4 are devoted to the minimal Rawlsian and equal loss principles, respectively. Section 5 introduces the Rawlsian minimal dispersion principle. Section 6 makes some concluding remarks.



\section{Basic notions and notation}

Let $N$ be a set of two individuals and $\allalts$ be a set of alternatives, with $\card{\allalts} = m\geq 2$. 
Given $i \in N$, let $\ibar \in N \setminus \set{i}$ denote the other individual. Let $\powersetz{\allalts}$ denote the set of non-empty subsets of $\allalts$. Let $\linors$ be the set of linear orders over $\allalts$. We let ${\prefi} \in \linors$ stand for the preference of individual  $i \in N$ and $\prof = ({\prefi[1]}, {\prefi[2]}) \in \allprofs$ for a preference profile. 
A social choice rule (SCR) is a function $f: \allprofs → \powersetz{\allalts}$.
Given two SCRs $f$, $f'$, we write $f \subset f'$ to indicate that $f$ is a proper subcorrespondence of $f'$.

An SCR $f$ is anonymous iff $f({\prefi[1]}, {\prefi[2]}) = f({\prefi[2]}, {\prefi[1]})$ for all $({\prefi[1]}, {\prefi[2]}) \in \allprofs$.
An SCR $f$ is neutral iff for all permutations $\sigma$ over $\allalts$ and profile $\prof \in \allprofs$, $\sigma \circ f(\prof) = f(\sigma \circ \prof)$.

Given $p, q \in \R$, let $\intvl{p, q} = [p, q] \cap \N $ denote the interval of integer numbers between $p$ and $q$. The loss of individual $i$ at preference profile $\prof$ for alternative $x$ is  defined as the number of alternatives that the individual prefers to $x$: $\lprof(x)_i = \card{\set{y \in \allalts \suchthat y \prefi x}}$.
The loss vector of $x$ at $\prof$, $\lprof(x)=(\lprof(x)_1, \lprof(x)_2) \in \intvl{0, m - 1}^N$, associates to each individual her loss associated to $x$.

Given two loss vectors $l, l' \in \lvs$, we say that $l$ is weakly smaller than $l'$, $l ≤ l'$, iff $\forall i: l_i ≤ l'_i$. We also write $l < l'$ to denote that $l$ is strictly smaller than $l'$, meaning, weakly smaller and different. Let $\min_N \lprof(x) = \min_{i \in N} \lprof(x)_i \in \N$ 
and $\sum_N \lprof(x) = \sum_{i \in N} \lprof(x)_i$ denote, respectively, the minimal loss and the sum of the losses associated to $\lprof(x)$.

Let $\PE = \set{x \in \allalts \suchthat \nexists y \text{ s.t. } \lprof(y) < \lprof(x)}$ be the set of Pareto-efficient alternatives at $\prof$.
Let $\PEcl = \set{f: \allprofs → \powersetz{\allalts} \suchthat \forall \prof: f(\prof) \subseteq \PE}$ denote the class of SCRs picking only Pareto efficient alternatives. Similarly, we systematically use calligraphic letters to denote the class of rules satisfying a given property.

In concordance with the ceiling established by Theorem 1 of \cite{BramsKilgour2001}, we use the term “half” to mean the smallest integer $k$ that exceeds $m-k$ and for every profile $\prof \in \allprofs$, we let $H(\prof) = \set{x \in \allalts \suchthat \lprof(x) ≤ (\ceil{\frac{m - 1}{2}}, \ceil{\frac{m - 1}{2}})}$ denote the set of alternatives reaching the best half of every individual’s preference. Given $\prof$ and a loss level $k \in \losses$, define $U(\prof, k) = \set{x \in \allalts \suchthat \lprof(x) ≤ (k, k)}$ as the set of alternatives imposing losses not higher than $k$ for all individuals. 
We say that such alternatives receive unanimous support at level $k$. Let $\rhoP = \min \set{k \in \losses \suchthat U(\prof, k) ≠ \emptyset}$ be the least loss level at which some alternative receives unanimous support.

We now define several SCRs that we analyze in the paper. 

\textbf{Fallback Bargaining} is the SCR $\FB$ that picks all alternatives with unanimous support at $\rhoP$: $\FB(\prof) = U(\prof, \rhoP)$. 

The \textbf{Veto-rank} rule $\VR$ is defined as follows. Each individual vetoes her worst $\floor{\frac{m - 1}{2}}$ alternatives, then the Borda winners among the non vetoed alternatives are picked: $\VR(\prof) = \argmin_{H(\prof)} \sum_N \lprof = \set{x \in H(\prof) \suchthat \forall y \in H(\prof): \sum_N \lprof(x) ≤ \sum_N \lprof(y)}$.

The \textbf{Shortlisting} rule $\SL$ picks the best alternative of individual $1$ that is not among the worst $\floor{\frac{m - 1}{2}}$ alternatives of individual $2$, and the best alternative of $2$ that is not among the worst $\floor{\frac{m - 1}{2}}$ alternatives of $1$. The Shortlisting rule is such that
$\SL(\prof) = \cup_{i \in N} (\argmin_{x \in H^i(\prof)} \lprof(x)_{\ibar})$.

Both $\VR$ and $\SL$ are defined in \cite{Clippel} for $m$ odd only.

The class of \textbf{Pareto-and-veto} rules, $\PVcl$, contains rules parametrized by $v_1, v_2 \in \intvl{0, m - 1}$ with $v_1 + v_2  ≤ m - 1$ where $v_i$ represents the number of alternatives vetoed by individual $i \in N$ (individuals veto the alternatives at the bottom of their preference).
Given $v_i \in \intvl{0, m - 1}$, define $a_i = m - v_i - 1 \in \intvl{0, m - 1}$ as the highest acceptable loss level for individual $i$. For $v=(v_1,v_2)$, the rule $\PVv = \cap_{i \in N}\set{x \in \allalts \suchthat \lprof(x)_i ≤ a_i} \cap \PE$ picks all alternatives in $\PE$ that no individual vetoes. 
The class $\PVcl = \set{\PVv \suchthat v_1, v_2 \in \intvl{0, m - 1} \text{ with } v_1 + v_2 \leq m - 1}$ is the set of those rules, and the class $\PVbcl = \set{\PVv \suchthat v_1, v_2 \in \intvl{0, m - 1} \text{ with } v_1 + v_2 = m - 1}$ is the set of rules where the inequality is binding.

\begin{remark}
    $\FB$, $\VR$, $\SL$ and all SCRs in $\PVcl$ are  neutral.
\end{remark}
\begin{remark}
    $\FB$, $\VR$ and $\SL$ are anonymous, while a SCR $\PVv \in \PVcl$ is anonymous iff $v_1 = v_2$.
\end{remark}

\section{The minimal Rawlsian principle}

\begin{definition}[$k$-Rawlsianism] Given $k \in \intvl{0, m - 1}$, an SCR is $k$-Rawlsian iff it selects its winners among those alternatives whose losses are within $\intvl{0, k}$ for both individuals. Formally:
	$\forall \prof \in \allprofs,  f(\prof) \subseteq \lprofinv(\intvl{0, k} × \intvl{0, k})$.
\end{definition}
It follows from \citet[Theorem 1]{BramsKilgour2001} that $k$-Rawlsianism can be satisfied if, and only if, $k$ is at least “half”.
\begin{proposition}
    There exist $k$-Rawlsian SCRs iff $k  ≥ \khalf$.
\end{proposition}
It follows that the strongest satisfiable version of $k$-Rawlsianism is when $k$ equals $\khalf$. On the other hand, this choice of $k$ reflects a general bound that renders $k$-Rawlsianism satisfiable at every preference profile while there are several profiles where the minimal loss is lower than $k$. Thus, we qualify $\khalf$-Rawlsianism as “minimal Rawlsianism”, which we formally define as follows.
\begin{definition}[Minimal Rawlsianism (\MRprop)] An SCR $f$ satisfies \MRprop{} if 
	$\forall \prof \in \allprofs,  f(\prof) \subseteq H(\prof)$.
\end{definition}

\commentOC{Drop this?}
Let $\MRcl$ denote the class of minimally Rawlsian (\MRprop) SCRs.

\begin{theorem}
	\label{th:inFH}
	$\FB, \VR, \SL \in \PEcl \cap \MRcl$. 
\end{theorem}
 \begin{proof}
	For $\VR$ and $\SL$, the proof follows from the definition of the SCRs since each of them selects only alternatives among the top-half alternatives of both individuals. For $\FB$, Theorem 1 of \cite{BramsKilgour2001} shows that $\forall \prof \in \allprofs$, $\rhoP ≤ \ceil{\frac{m - 1}{2}}$.\end{proof} 
   
We now discuss the relationship of the class $\PVcl$ to the \MRprop{} property. To this end, we define $\PVe = \PVv[\left(\floor{\frac{m - 1}{2}}, \floor{\frac{m - 1}{2}}\right)]$ as the Pareto-and-veto rule that gives the highest equal veto power to both individuals. 
Thus, under $PV^=$, we have $v_1=v_2= \floor{\frac{m - 1}{2}}$, implying $v_1=v_2=\frac{(m-1)}{2}$ when $m$ is odd and $v_1=v_2= \frac{m}{2}-1$ when $m$ is even. 
Note that $PV^=\in\PVbcl$ iff $m$ is odd.

\begin{theorem}
 	When $m$ is odd, $\PVcl ∩ \MRcl = \{PV^=\}$, and
	when $m$ is even, $\PVcl ∩ \MRcl = \set{\PVe, \PVv[(\frac{m}{2}, \frac{m}{2} - 1)], \PVv[(\frac{m}{2} - 1, \frac{m}{2})]}$.
\end{theorem}
 \begin{proof}
	We start by claiming that $\PVcl ∩ \MRcl = \set{\PVv \in \PVcl \suchthat \forall i: v_i ≥ \floor{\frac{m - 1}{2}}}$. To show the “if” part of the claim, note that given any profile, the condition $\forall i: v_i ≥ \floor{\frac{m - 1}{2}}$ suffices to guarantee that $\PVv(\prof) \subseteq H(P)$. To see the “only if” part, consider an arbitrary ordering $\prefi$ over $\allalts$, let $\prefiinv$ denote its inverse, and consider the profile $\prof = (\prefi, \prefiinv)$.
	Observe that $\PVv(\prof)$ will exclusively pick winners in the first half of individual $i$ only if $v_i ≥ \floor{\frac{m - 1}{2}}$.
	
	Having shown the claim, note that When $\forall i \in \set{1, 2}, v_i ≥ \floor{\frac{m - 1}{2}}$, we have $\abs{v_1 - v_2} ≤ 1$.
	Thus, when $m$ is odd, $\PVcl ∩ \MRcl = \{PV^=\}$, and
	when $m$ is even, $\PVcl ∩ \MRcl = \set{\PVe, \PVv[(\frac{m}{2}, \frac{m}{2} - 1)], \PVv[(\frac{m}{2} - 1, \frac{m}{2})]}$.\end{proof} 


We now establish the relationship of  $\VR, \SL, \FB$ to the class $\PVcl$. Considering two SCRs $f$ and $f'$, let $f \cup f'$ denote the rule $(f \cup f')(\prof) = f(\prof) \cup f'(\prof)$. 
Given any non empty class of SCRs $F$, let $\bigcup F$ denote the maximal (least resolute) SCR that can be formed by unions of rules of $F$.

\begin{proposition}\label{propo:equal}
	$\bigcup(\PEcl \cap \MRcl) = \PVe$.
\end{proposition}
\begin{proof}
    Note that $\bigcup(\PEcl \cap \MRcl)$ is, by definition, the SCR that, for each profile, picks all Pareto alternatives that are in the first half of both individuals’ preferences and only those alternatives. 
    We thus have to show that $\forall \prof: \PVe(\prof) = \PE \cap H(\prof)$. By definition of $\PVe$, it suffices to prove that $\cap_i\set{x \in \allalts \suchthat \lprof(x)_i ≤ m - \floor{\frac{m - 1}{2}} - 1} = H(\prof)$. This in turn follows from the definition of $H(\prof)$.
\end{proof}

The observation below follows from \cref{propo:equal}.
\begin{corollary}\label{th:subPVe}
	A SCR $f \in \PEcl \cap \MRcl$ if and only if $f \subseteq \PVe$.
\end{corollary}

We now establish the relationship between $\FB$, $\VR$ and $\SL$ and show that they can pick disjoint winners.
\begin{proposition}\label{th:different}
	$\exists \prof \suchthat \FB(\prof) \cap \VR(\prof) = \emptyset, \FB(\prof) \cap \SL(\prof) = \emptyset, \VR(\prof) \cap \SL(\prof) = \emptyset$. Furthermore, $\PVe(\prof) ≠ \FB(\prof)$, $\PVe(\prof) ≠ \VR(\prof)$, $\PVe(\prof) ≠ \SL(\prof)$.
\end{proposition}
\begin{proof}
	Consider the following profile $\prof$:
	\begin{equation}
		\label{eq:distinct}
		\begin{array}{llll lll | llll ll}
			a&b&c&d&e&f&g&h&i&j&k&l&m\\
			g&h&i&d&b&j&a&c&e&f&k&l&m\\
		\end{array},
	\end{equation}
	where the first individual prefers $a$ to $b$, $b$ to $c$, etc., and the second individual prefers $g$ to $h$, $h$ to $i$, etc. 
	The bar shows the “half” position.
	The proposition is proven by noting that $\FB(\prof) = \set{d}$, $\VR(\prof) = \set{b}$, $\SL(\prof) = \set{a, g}$ and $\PVe(\prof) = \set{a, b, d, g}$.
\end{proof}

\Cref{th:inFH}, \cref{th:subPVe} and \cref{th:different} lead to the corollary below.
\begin{corollary}
   	$\FB, \VR, \SL \subset \PVe$.
\end{corollary}

\begin{remark}
    The rule $\PVe$ is not merely the union of $\FB$, $\VR$ and $\SL$, as the following profile illustrates.
    \begin{equation}
        \begin{array}{lllll|llll}
                a&b&c&d&e&f&g&h&i\\
                e&c&d&b&a&f&g&h&i\\
        \end{array}.
    \end{equation}
    Here, $\FB(\prof) = \set{c}$, $\VR(\prof) = \set{c}$, $\SL(\prof) = \set{a, e}$ but $\PVe(\prof) = \set{a, b, c, e}$. In fact, no variant of $\VR$ (changing the scoring vector) would elect $b$: its losses are $(1, 3)$ whereas $c$ has losses $(2, 1)$.
\end{remark}

\section{The equal loss principle}
Given $\prof \in \allprofs$, define $S(\prof) = \set{x \in \allalts \suchthat \lprof(x)_1 = \lprof(x)_2}$ as the set of alternatives that are ranked at the same position by both individuals.

\begin{definition}[Equal loss (EL)]
    $\forall \prof \in \allprofs: [S(\prof) ≠ \emptyset] ⇒ f(\prof) \cap S(\prof) ≠ \emptyset$.
\end{definition}

\begin{proposition}
    $\forall m ≥ 3: \PEcl ∩ \ELcl = \emptyset$.
\end{proposition}
\commentRS{Proposition 4 is missing a proof (that is rather straightforward).}
Thus, Pareto efficiency is a fortiori incompatible with the following stronger version of the equal loss principle.

\begin{definition}[Strong equal loss]
    $\forall \prof \in \allprofs: [S(\prof) ≠ \emptyset] ⇒ f(\prof) \subseteq S(\prof)$.
\end{definition}

We now embed Pareto efficiency into the equal loss requirement by mandating $f$ to pick the (unique) Pareto efficient alternative that is ranked at the same position by both individuals, if there is one.

\begin{definition}[Paretian equal loss (PEL)]
    $\forall \prof \in \allprofs: [S(\prof) \cap \PE ≠ \emptyset] ⇒ f(\prof) \cap S(\prof) ≠ \emptyset$.
\end{definition}

\begin{proposition}
	$\VR$ and $\SL$ fail PEL.
\end{proposition}
\begin{proof}
	Consider the following profile $\prof$, reused from \cref{eq:distinct}, where $\VR(\prof) = \set{b}$ and $\SL(\prof) = \set{a, g}$:
	\begin{equation}
		\begin{array}{llll lll | llll ll}
			a&b&c&d&e&f&g&h&i&j&k&l&m\\
			g&h&i&d&b&j&a&c&e&f&k&l&m\\
		\end{array}.
	\end{equation}
	PEL requires to choose at least $d$.
\end{proof}

Some rules from the class $\PVcl$ satisfy PEL, $\PVe$ being among those.
\begin{proposition}
\label{propo:pel}	For any $m ≥ 3$, a rule $\PVv[(v_1, v_2)] \in \PVcl$ satisfies PEL iff its veto levels are both at most $\floor{\frac{m}{2}}$, thus, iff $\max_{i \in \set{1, 2}} v_i ≤ \floor{\frac{m}{2}}$.
\end{proposition}
\begin{proof}
	For all $\prof \in \allprofs$, if $x \in S(\prof) \cap \PE$, then $x$ is not among the last $\floor{\frac{m}{2}}$ ranks, because all alternatives before $x$ for the first individual must be placed after $x$ for the second individual.
	A PV rule with veto parameters at most $\floor{\frac{m}{2}}$ will thus pick all such alternatives $S(\prof) \cap \PE$, as required by PEL.
	
	For the other direction, observe that there exists $\prof \in \allprofs$ such that for some $x \in \allalts$, $x \in S(\prof) \cap \PE$ and $x$ is positioned just better than the last $\floor{\frac{m}{2}}$ ranks (thus $\exists \prof \in \allprofs, x \in \allalts \suchthat \forall i \in \set{1, 2}: \lprof(x)_i = \floor{\frac{m - 1}{2}}$, leaving $\ceil{\frac{m - 1}{2}} = \floor{\frac{m}{2}}$ positions behind $x$).
	A PV rule such that $\max_{i \in \set{1, 2}} v_i > \floor{\frac{m}{2}}$ will thus not include $x$ in the set of winners, hence, the rule will fail PEL.
\end{proof}
\commentRS{Should we add a corollary about binding PV rules that satisfy PEL?}
The rules that fail PEL will a fortiori fail the following stronger version of the Paretian equal loss property which requires that the Pareto efficient alternative ranked at the same position by both individuals, if it exists, is the unique outcome.

\begin{definition}[Strong Paretian equal loss (SPEL)]
    $\forall \prof \in \allprofs: [S(\prof) \cap \PE ≠ \emptyset] ⇒ f(\prof) = S(\prof) \cap \PE$.
\end{definition}
Thus, $\VR$, $\SL$ and those rules in $\PVcl$ that fail PEL all fail SPEL. Furthermore, as we state and show below, even the rules in $\PVcl$ that satisfy PEL fail SPEL.
\begin{proposition}
	When $m ≥ 4$, all rules in $\PVcl$ fail SPEL.
\end{proposition}
\begin{proof}
    Let $\allalts = \set{a, b, c, a_4, a_5, …}$.
    Consider the following profile $P$: 
	\begin{equation}
		\begin{array}{llllll}
			a&b&c&a_4&a_5&…\\	c&b&a&a_4&a_5&…\\
		\end{array}.
	\end{equation}
    SPEL requires to pick solely $b$. Any rule $f \in \PVcl$ will pick at least either $a$ or $c$ in supplement to $b$. 
    Indeed, we have $a \notin f(\prof)$ iff $v_2 ≥ m - 2$, and $c \notin f(\prof)$ iff $v_1 ≥ m - 2$. Because $v_1 + v_2 ≤ m - 1$ and $m ≥ 4$, we know that $v_1 + v_2 ≤ m - 1 + (m - 4) = m - 2 + m - 3$. Now if $c \notin f(\prof)$ (equivalently, $m - 2 ≤ v_1$), then $v_1 + v_2≤ v_1 + m - 3$, then $a \in f(\prof)$.
\end{proof}

\begin{remark}
When $m = 3$, $\PVv[(1, 1)]$ satisfies SPEL.
\end{remark}

\begin{proposition}
	$\FB$ satisfies SPEL.
\end{proposition}
\begin{proof}
	Pick any profile $\prof$ and any $x \in PE(\prof)$ with $\lprof(x) = (k, k)$ for some $k \in \intvl{0, m - 1}$. As $\FB$ minimizes the maximal loss, we have $\max \lprof(y) ≥ \max \lprof(x) = \min \lprof(x) = k$ for all $y \in \FB(\prof)$. Since $x \in PE(\prof)$, we have $x = y$, hence $\FB(\prof) = \set{x}$, showing that $\FB$ satisfies SPEL.
\end{proof}

SPEL, failed by all rules we consider but one, allows to distinguish $\FB$ from the rest. Moreover, by satisfying both conditions, $\FB$ establishes the compatibility between SPEL and MR. However, as discussed below, this compatibility vanishes when another stronger version of PEL is adopted. 

Call the dispersion of a loss vector $l$ at $\prof$ the value $d(l) = \abs{l_1 - l_2}$. 
Thus, $(d \circ \lprof)(x) = \max\lprof(x) - \min\lprof(x)$.
Note that $d \in \Sigma$ coincides with multiple commonly used spread measures, as shown in \cref{sec:spreads}.

Given a profile $\prof \in \allprofs$, define $\min_{\PE} (d \circ \lprof)$ as the minimal dispersion obtained by loss vectors of Paretian alternatives in that profile, and $\argmin_{\PE} (d \circ \lprof)$ as the Paretian alternatives whose loss vectors have minimal dispersion among Paretian alternatives.

Define the minimal dispersion ($MD$) condition as follows.
\commentRS{To prove Proposition 11, we need to make MD stronger so that it requires f(P) to contain all Pareto alternatives with a minimal dispersion. When MD gets stronger, the impossibility expressed by Theorem 3 prevails. Proposition 9 is independent of the definition of MD. Proposition 10 exploits that fact that the intersection is non-empty, which is independent of the definition of MD and once the non-emptiness is ensured, that well-defined rule also satisfies the stronger version of MD. Thus, making MD stronger has no impact on Theorem 3, Proposition 9 and Proposition 10. Am I right?}

\commentMN{MD will then read $f(\prof)\supseteq \arg min_{\PE} (d \circ \lprof)$, $\forall \prof \in \allprofs$, right?} 

\commentRS{Yes, it is this set inclusion relation. However, I wouldn't put "not being equal to the empty-set" to the definition of the condition because this would appear as if this is a part of the condition. However, this is a fact independent of the condition. Perhaps we can remark this before stating the condition.}

\commentMN{Now, regarding its implications. I think that it's ok to do this replacement in Theorem 3 and Proposition 10 and 11 (no role at all in Proposition 9, right?).}

\commentRS{Yes, this is what I meant. Perhaps, some revision in certain expressions in the proofs of Theroem 3 and Proposition 10 may be needed. Proposition 11 is already proven according to the stronger version.}
\begin{definition}[Minimal dispersion]
	$f(\prof) \cap \argmin_{\PE} (d \circ \lprof) ≠ \emptyset$, $\forall \prof \in \allprofs$.
\end{definition}
MD requires the outcome to contain the Paretian alternatives whose loss vectors have minimal dispersion. As such, MD is another strengthening of PEL while it is logically independent of SPEL. Nevertheless, although there exists rules that satisfy both SPEL and MR, MD turns out to be logically incompatible with MR.

The following result will be useful to explore the precise relationship between MD and MR.
\begin{remark}
	\label{remark1} 
	For any $\prof$, if $x \in \PE$ then
$\lprof(x)_1+\lprof(x)_2\leq m-1$. 
To see this, suppose, for a contradiction, that $x \in \PE$ while
$\lprof(x)_1+\lprof(x)_2\geq m$. Thus, $\#\{y\in \mathcal{A}\mid y \succ_1 x\}+\#\{y\in \mathcal{A}\mid y \succ_2 x\}\geq m$, which implies $\{y\in \mathcal{A}\mid y \succ_1 x\}\cap \{y\in \mathcal{A}\mid y \succ_2 x\}\neq \emptyset$, contradicting $x \in \PE$.
	\commentOC{I leave the remark here as this fact is used in the proof below, but as we use it only once now (I think), we might want to drop the remark. I chose to not refer to this remark for now in the proof below and inlined its reasoning.}
	\commentRS{We use and explicitly refer to it in the proof of Proposition 9.}
\end{remark}
In the following, given $S \subseteq \intvl{0, m - 1} × \intvl{0, m - 1}$, we write $\lprofinv(S) = \set{x \in \allalts \suchthat \lprof(x) \in S}$ to denote the set of alternatives whoses losses are within $S$. So, for example, $H(\prof) = \lprofinv(\intvl{0, \khalf} × \intvl{0, \khalf})$.
\commentOC{Consider externalizing the implication: $m ≥ 3$ and $t ≥ \frac{2m - 4}{3}$ ⇒ $t ≥ \khalf$.}
\begin{theorem}
	\label{th:caractEmpty}
	Given $m ≥ 3$, $\forall t \in \intvl{0, m - 1}$, the following statements are logically equivalent: 
	\begin{enumerate}
		\item \label{it:tbound} $t ≥ \frac{2m - 4}{3}$
		\item \label{it:Pt} $\forall \prof: \lprofinv(\intvl{0, t} × \intvl{0, t}) \cap \argmin_{\PE} (d \circ \lprof) ≠ \emptyset$
		\item \label{it:Ptbig2} $\forall \prof: \lprofinv(\intvl{0, t} × \intvl{0, m - 1}) \cap \argmin_{\PE} (d \circ \lprof) ≠ \emptyset$
		\item \label{it:Ptbig1} $\forall \prof: \lprofinv(\intvl{0, m - 1} × \intvl{0, t}) \cap \argmin_{\PE} (d \circ \lprof) ≠ \emptyset$.
	\end{enumerate}
\end{theorem}
\commentRS{Regarding Theorem 3, statement 2 implies 3 and 4 just by set inclusion, as we state in the proof. On the other hand, I think either of 3 or 4 implies 2 by the anonymity of minimal dispersion. If so, then 2, 3 and 4 are equivalent independent of the value of t, in which case we should state the theorem as the equivalence of 1 and 2.}
\begin{proof}
	For the claim $\ref{it:tbound} ⇒ 2$, let us consider any $t ≥ \frac{2m - 4}{3}$ and any profile $\prof$, and let us show that some alternative lies in $\lprofinv(\intvl{0, t} × \intvl{0, t}) \cap \argmin_{\PE} (d \circ \lprof)$.
	
	Consider any $x \in \argmin_{\PE} (d \circ \lprof)$, and let $i$ denote any individual such that $\max \lprof(x) = \lprof(x)_i$. 
	Observe that if $\lprof(x)_i ≤ t$, then also $\lprof(x)_{\ibar} ≤ t$ and thus $x \in \lprofinv(\intvl{0, t} × \intvl{0, t}) \cap \argmin_{\PE} (d \circ \lprof)$, and the proof is done. Thus, assume that $\lprof(x)_i > t$.
%	The alternative $x$ thus has $k$ alternatives better than it for the individual $i$, and therefore must have at least those $k$ alternatives worst than it for the other individual (otherwise it is Pareto-dominated by at least one of those $k$ alternatives). Thus $\lprof(x)_{\ibar} ≤ m - 1 - k$. In other words, the losses of $x$ are $k$ and at most $m - 1 - k$, with $k$ the highest of these two losses. 
	
%	If [equivalently, ], then 
	Let $A = \set{a \suchthat \lprof(a)_i ≤ t}$ designate the $t + 1$ top alternatives for $i$. 
	Define $y = \argmin_A \lprof(.)_{\ibar}$ as the best alternative for individual $\ibar$ among $A$. 
%	By construction thus, $\forall a ≠ y \in A: \lprof(y)_{\ibar} < \lprof(a)_{\ibar}$, hence, $\lprof(y)_{\ibar} ≤ m - 1 - \card{(A \setminus \set{y})} = m - 1 - t$.
	
	Towards showing that $y \in \lprofinv(\intvl{0, t} × \intvl{0, t}) \cap \argmin_{\PE} (d \circ \lprof)$, observe that $y \in \PE$: for $i$, only the alternatives in $A$ may be better than $y$, and those are worst than $y$ for $\ibar$ by definition of $y$.
	
	Because $\lprof(x)_i > t$ and $y \in A$, we see that $\lprof(y)_i < \lprof(x)_i$.
	It follows that $\lprof(x)_{\ibar} < \lprof(y)_{\ibar}$, otherwise $y$ Pareto-dominates $x$; whence $\lprof(y)_i < \lprof(y)_{\ibar}$ otherwise $\lprof(x)_{\ibar} < \lprof(y)_{\ibar} ≤ \lprof(y)_i < \lprof(x)_i$ and $d(\lprof(y)) < d(\lprof(x))$, contradicting $x \in \argmin_{\PE} (d \circ \lprof)$.
	
	 Observe that as $t \in \N$, $t ≥ \frac{2m - 4}{3} ⇒ t ≥ \ceil{\frac{2m - 4}{3}}$, and when $m ≥ 3$, $\ceil{\frac{2m - 4}{3}} ≥ \khalf$ \commentOC{I ignore how to prove that shortly, here are the gory details. When $m$ odd, $m ≥ 5 ⇒ \frac{2m - 4}{3} ≥ \frac{m - 1}{2} = \khalf$ and when $m$ is even, $m ≥ 8 ⇒ \frac{2m - 4}{3} ≥ \frac{m}{2} = \khalf$; when $m = 3$, $\ceil{\frac{2m - 4}{3}} = \ceil{\frac{2}{3}} = 1 = \khalf$; when $m = 4$, $\ceil{\frac{2m - 4}{3}} = \ceil{\frac{4}{3}} = 2 = \khalf$; when $m = 6$, $\ceil{\frac{2m - 4}{3}} = \ceil{\frac{8}{3}} = 3 = \khalf$}; whence $t ≥ \khalf$.
	
	By construction, $\forall a ≠ y \in A: \lprof(y)_{\ibar} < \lprof(a)_{\ibar}$, hence, $\lprof(y)_{\ibar} ≤ m - 1 - \card{(A \setminus \set{y})} = m - 1 - t$.
	Also, $t ≥ \khalf$ thus $m - 1 - t ≤ m - 1 - \khalf = \floor{\frac{m - 1}{2}} ≤ t$.
	These two facts, together with $\lprof(y)_i < \lprof(y)_{\ibar}$, establish that $\lprof(y)_i < \lprof(y)_{\ibar}≤ m - 1 - t ≤ t$, which yields that $y \in \lprofinv(\intvl{0, t} × \intvl{0, t})$ and that $d(\lprof(y)) ≤ m - 1 - t$. 
	
	Observe now that $\min \lprof(x) ≤ m - 1 - \max \lprof(x)$: the alternative $x$ has $\max \lprof(x) = \lprof(x)_i$ alternatives better than it for the individual $i$, and therefore must have at least those $\max \lprof(x)$ alternatives worst than it for the individual $\ibar$, otherwise it is Pareto-dominated by at least one of those $\max \lprof(x)$ alternatives. This yields $d(\lprof(x)) = \max \lprof(x) - \min\lprof(x) ≥ \max \lprof(x) - (m - 1 - \max \lprof(x)) = 2 \lprof(x)_i - m + 1 ≥ 2 (t + 1) - m + 1$.
	
	To conclude, note that the condition $t ≥ \frac{2m - 4}{3}$ is equivalent to $t + 1 ≥ \frac{2m - 1}{3}$, itself equivalent to $2 (t + 1) - m + 1 ≥ m - t - 1$, hence $d(\lprof(x)) ≥ m - t - 1 ≥ d(\lprof(y))$, which implies that $y \in \argmin_{\PE} (d \circ \lprof)$.
	
	\commentOC{It would be more elegant to show that any other alternative than $y$ has either $\max\lprof(x) ≤ t$ or $d(x) ≥ d(y)$.}
	
	The claims $\ref{it:Pt} ⇒ \ref{it:Ptbig2}$ and $\ref{it:Pt} ⇒ \ref{it:Ptbig1}$ and are seen to hold simply thanks to set inclusion.
	
	Turning now to the claim $\ref{it:Ptbig2} ⇒ \ref{it:tbound}$, assume, considering the contrapositive, that $t < \frac{2m - 4}{3}$, and let us define a profile $\prof$ such that $\lprofinv(\intvl{0, t} × \intvl{0, m - 1}) \cap \argmin_{\PE} (d \circ \lprof) = \emptyset$. (That proof applies equally to the claim $\ref{it:Ptbig1} ⇒ \ref{it:tbound}$ by inverting the role of individuals 1 and 2.)
	
	Observing that $m = t + 1 + (m - t - 2) + 1$, let us name the alternatives $\set{a_1, …, a_{t + 1}, c_1, …, c_{m - t - 2}, x}$.
	Define the sequences of alternatives $A = (a_1, …, a_{t + 1})$ and $C = (c_1, …, c_{m - t - 2})$. 
	Define the preference of individual $1$ as $(A, x, C)$ and the preference of individual $2$ as $(C, x, A)$.

	Observe that as $\lprof(x)_1 = t + 1$, $x \notin \lprofinv(\intvl{0, t} × \intvl{0, m - 1})$, and $x \in \PE$, so the claim will be proven by showing that $\forall y \in A \cup C: d(\lprof(y)) > d(\lprof(x))$, so that $\forall y \in A \cup C: y \notin \argmin_{\PE} (d \circ \lprof)$.
	
	Note that $\card{A} - \card{C} ≥ 0$ as $\card{A} - \card{C} = t + 1 - (m - t - 2) = 2t + 3 - m ≥ 2\khalf + 3 - m ≥ 2 \frac{m - 1}{2} + 3 - m = 2$.
	Thus, $d(\lprof(x)) = \card{A} - \card{C} = 2t + 3 - m$.
	
	Considering $a_i \in A$, $\lprof(a_i) = (i - 1, \card{C} + 1 + i - 1)$ thus $d(\lprof(a_i)) = m - t - 1$.
	It follows that $d(\lprof(a_i)) > d(\lprof(x))$, equivalently, $m - t - 1 > 2t + 3 - m$, as $3t < 2m - 4$ by hypothesis.
	
	And considering $c_i \in C$, $\lprof(c_i) = (\card{A} + 1 + (i - 1), i - 1)$ thus $d(\lprof(c_i)) = t + 2$.
	Using $\card{A} ≥ \card{C}$, it follows that $d(\lprof(c_i)) = \card{A} + 1 ≥ \card{C} + 1 = d(\lprof(a_i)) > d(\lprof(x))$, thus $d(\lprof(c_i)) > d(\lprof(x))$.
\end{proof}

\begin{corollary}
	\label{th:noMRMD}
	Some $f \in \MRcl$ satisfies MD iff $m ≤ 6 \lor m = 8$.
\end{corollary}
\begin{proof}
	Fix $t = \khalf$ and use \cref{th:caractEmpty} to obtain:
	\begin{equation}
		m ≤ \frac{3 \khalf + 4}{2} ⇔ \forall \prof: H(P) \cap \argmin_{\PE} (d \circ \lprof) ≠ \emptyset.
	\end{equation}
	The left hand side is equivalent, when $m$ is odd, to $m ≤ \frac{3m + 5}{4}$ thus $m ≤ 5$, and when $m$ is even, to $m ≤ \frac{3m + 8}{4}$ thus $m ≤ 8$.
	
	It follows that when $m ≤ 6$ or $m = 8$, the SCR $H(\prof) \cap \argmin_{\PE} (d \circ \lprof)$ is well-defined and satisfies MR and MD by construction (\cref{th:caractEmpty} does not apply for $m = 2$; in that case,  $H(\prof) = \lprofinv(\intvl{0, 1} × \intvl{0, 1} = \prof$ thus $H(\prof) \cap \argmin_{\PE} (d \circ \lprof) ≠ \emptyset$ also holds); and when $m = 7$ or $m ≥ 9$, there is no $f \in \MRcl$ that satisfies MD.
\end{proof}
 
The minimal Rawlsian and minimal dispersion principles being logically incompatible, the SCRs that satisfy MR (namely $\FB$, $\VR$, $\SL$ and the rules in $\PVcl$ identified by Theorem 2) all fail MD. As Theorem 2 shows that most rules in $\PVcl$ fail MR, the following result which determines the subclass of $\PVcl$ satisfying MD is of interest. 

\begin{corollary}
	For $m ≥ 3$, $\PVv[(v_1, v_2)]$ satisfies MD iff $\max_{i \in \set{1, 2}} v_i ≤ \frac{m + 1}{3}$.
\end{corollary}
\begin{proof}
	Define $t = \min_{i \in \set{1, 2}} (m - 1 - v_i) \in \N$.
	Observe that $\max_{i \in \set{1, 2}} v_i ≤ \frac{m + 1}{3} ⇔ \forall i \in \set{1, 2}: v_i ≤ \frac{m + 1}{3} ⇔ \forall i \in \set{1, 2}: m - 1 - v_i ≥ \frac{2m - 4}{3} ⇔ t ≥ \frac{2m - 4}{3}$.
	
	If $t = m - 1 - v_1$, define $S = \intvl{0, t} × \intvl{0, m - 1}$, otherwise (implying that $t = m - 1 - v_2$), define $S = \intvl{0, m - 1} × \intvl{0, t}$.
	By definition, $\forall \prof: \PVv[(v_1, v_2)](\prof) = \lprofinv(\intvl{0, m - 1 - v_1} × \intvl{0, m - 1 - v_2}) \cap \PE$.
	It follows that $\forall \prof$:
	\begin{equation}
		\lprofinv(\intvl{0, t} × \intvl{0, t}) \cap \PE \subseteq \PVv[(v_1, v_2)](\prof)
	\end{equation} 
	and
	\begin{equation}
		\PVv[(v_1, v_2)](\prof) \subseteq \lprofinv(S) \cap \PE,
	\end{equation}
	therefore (intersecting all sets with $\argmin_{\PE} (d \circ \lprof)$ and using $\PE \cap \argmin_{\PE} (d \circ \lprof) = \argmin_{\PE} (d \circ \lprof)$):
	\begin{equation}
		\lprofinv(\intvl{0, t} × \intvl{0, t}) \cap \argmin_{\PE} (d \circ \lprof) \subseteq \PVv[(v_1, v_2)](\prof) \cap \argmin_{\PE} (d \circ \lprof)
	\end{equation} 
	and
	\begin{equation}
		\PVv[(v_1, v_2)](\prof) \cap \argmin_{\PE} (d \circ \lprof) \subseteq \lprofinv(S) \cap \argmin_{\PE} (d \circ \lprof).
	\end{equation}
	
	It follows from \cref{th:caractEmpty} that $t ≥ \frac{2m - 4}{3} ⇔ \forall \prof: \lprofinv(\intvl{0, t} × \intvl{0, t}) \cap \argmin_{\PE} (d \circ \lprof) ≠ \emptyset ⇔ \forall \prof: \lprofinv(S) \cap \argmin_{\PE} (d \circ \lprof) ≠ \emptyset$, and thus $\max_{i \in \set{1, 2}} v_i ≤ \frac{m + 1}{3} ⇔ t ≥ \frac{2m - 4}{3} ⇔ \forall \prof: \PVv[(v_1, v_2)](\prof) \cap \argmin_{\PE} (d \circ \lprof) ≠ \emptyset$.
%	Assume that $\max_{i \in \set{1, 2}} v_i ≤ \frac{m + 1}{3}$ (thus $t ≥ \frac{2m - 4}{3}$).
%	Note that 
%	We can apply \cref{th:caractEmpty} to obtain $\lprofinv(\intvl{0, t} × \intvl{0, t}) \cap \argmin_{\PE} (d \circ \lprof) ≠ \emptyset$, whence $\lprofinv(\intvl{0, t} × \intvl{0, t}) \cap \PE \cap \argmin_{\PE} (d \circ \lprof) ≠ \emptyset$ and thus $\PVv[(v_1, v_2)](\prof) \cap \argmin_{\PE} (d \circ \lprof) ≠ \emptyset$.
%
%	Assume that $\max_{i \in \set{1, 2}} v_i > \frac{m + 1}{3}$ (thus $t < \frac{2m - 4}{3}$).
%	By \cref{th:caractEmpty}, $\exists \prof \suchthat \lprofinv(\intvl{0, t} × \intvl{0, m - 1}) \cap \argmin_{\PE} (d \circ \lprof) = \emptyset$.
%	It follows from $\PVv[(v_1, v_2)](\prof) \subseteq \lprofinv(\intvl{0, t} × \intvl{0, m - 1})$ that $\PVv[(v_1, v_2)](\prof) \cap \argmin_{\PE} (d \circ \lprof) = \emptyset$.
\end{proof}

\commentOC{I suppose that we have to choose whether we keep the above approach or the one below.}
\commentOC{(I suppose that this is clear for everybody but just to be sure.) \Cref{propo:pv} says that for $k = 1$, $r = 2$ (hence $m = 5$) and $v = (2, 2)$, the rule fails MD, which is, I suppose, true for the strong definition of MD but false for the weak one. Indeed, the proof of \cref{propo:pv} does not apply to the weak one for $r = 2$ as $d(a_1) = k + 1 = d(x)$ so picking either $a_1$ or $x$ is fine.}
\begin{proposition}\label{propo:pv}
	Take any $m ≥ 3$ and write $m=3k+r$ for $k\in \mathbb{N}$, $r\in\{0,1,2\}$. $\PVv[(v_1, v_2)]$ satisfies MD iff $\max_{i \in \set{1, 2}} v_i ≤ k$.
\end{proposition}
\begin{proof}
To see the "only if" part, suppose that $v_{i}>k$ for some $i$. Let $i=1$, without loss of generality. We consider three cases, one for each possible value of $r$.

\medskip

For $r=0$, consider the profile $\prof$


	\begin{equation}
		\begin{array}{lllllll}
			a_1&\ldots&a_{2k}&a_{2k+1}&\ldots&\ldots&a_{3k}\\	a_{2k+1}&\ldots&a_{3k}&a_{2k}&a_1 &\ldots&a_{2k-1}\\
		\end{array}.
	\end{equation}
	
	


$\bigskip $

where $PE(\prof )=\{a_{1}, a_{2k+1},a_{2k}\}$ while
 $d(\lprof(a_{1})) = 2k$, $d(\lprof(a_{2k+1})) = k+1$ and $d(\lprof(a_{2k})) = k-1$. Thus, any SCR that satisfies MD must pick $a_{2k}$, which is, however, vetoed by individual 1.

\medskip

For $r=1$, consider the profile $\prof'$

	\begin{equation}
		\begin{array}{lllllll}
			a_1&\ldots&a_{2k}&x&a_{2k+1}&\ldots&a_{3k}\\	a_{2k+1}&\ldots&a_{3k}&x&a_1 &\ldots&a_{2k}\\
		\end{array}.
	\end{equation}


where $PE(\prof' )=\{a_{1}, a_{2k+1}, x\}$ while
 $d(\lambda_{\prof'}(a_{1})) =2k+1$, $d(\lambda_{\prof'}(a_{2k+1})) = k+1$ and $d(\lambda_{\prof'}(x)) = k$. Thus, any SCR that satisfies
MD must pick $x$, which is, however, vetoed by individual 1. 
\bigskip 

For $r=2$, consider the profile $\prof^{\prime\prime}$

	\begin{equation}
		\begin{array}{llllllll}
			a_1&\ldots&a_{2k}&y&x&a_{2k+1}&\ldots&a_{3k}\\	a_{2k+1}&\ldots&a_{3k}&x&a_1 &\ldots&a_{2k}&y\\
		\end{array}.
	\end{equation}


where $PE(\prof^{\prime\prime} )=\{a_{1}, a_{2k+1}, x\}$ while
 $d(\lambda_{\prof^{\prime\prime}}(a_{1})) =k+1$, $d(\lambda_{\prof^{\prime\prime}}(a_{2k+1})) = 2k+2$ and $d(\lambda_{\prof^{\prime\prime}}(x)) = k+1$. Thus, any SCR that satisfies
MD must pick $x$, which is, however, vetoed by individual 1. 


To see the "if" part, let max$_{i\in \{1,2\}}$ $v_{i}\leq $ $k.$ Take any $%
\prof$ and any $x\in \boldsymbol{argmin}_{\mathcal{PE}(\prof)}(d\cdot
\lambda _{P})$. Suppose, towards contradiction, $x\notin PV^{(v_{1},\text{ }%
v_{2})}(P)$, hence $\lambda _{P}(x)_{i}\geq $ $m-v_{i}$ for some $i\in
\left\{ 1,\text{ }2\right\} $. Let, without loss of generality, $i=1.$ So, $%
\lambda _{P}(x)_{1}\geq $ $m-v_{1}$. By Remark \ref{remark1}, $\lambda _{P}(x)_{2}\leq
m-1-\lambda _{P}(x)_{1}.$ Thus, $\lambda _{P}(x)_{2}\leq v_{1}-1$. As $%
v_{1}\leq $ $k$, we have $v_{1}-1<$ $m-v_{1}$, hence $\lambda _{P}(x)_{2}<$ $%
\lambda _{P}(x)_{1}.$ Thus, $d\cdot \lambda _{P}(x)=\lambda
_{P}(x)_{1}-\lambda _{P}(x)_{2}\geq m-2v_{1}+1.$

Now take $y\in PE(\prof)$ with $y\succ _{1}x$ such that
there is no $z\in PE(\prof)$ with $y\succ _{1}z$ $\succ
_{1}x$. Write $I(y,$ $x;$ $\succ _{1})=\left\{ z\in A:y\succ _{1}z\succ
_{1}x\right\} $ for the set of alternatives between $y$ and $x$ at $%
\succ _{1}.$ By construction, $I(y,$ $x;$ $\succ _{1})$ $\subseteq
A\backslash PE(\prof).$

Consider first the case $I(y,$ $x;$ $\succ _{1})$ $=\varnothing $. Thus, $%
\lambda _{P}(y)_{1}=\lambda _{P}(x)_{1}-1$. As $x\in PE(\prof)$, we have $x\succ
_{2}y$, thus $\lambda _{P}(y)_{2}>\lambda _{P}(x)_{2}$. Observe that $d\cdot
\lambda _{P}(y)<d\cdot \lambda _{P}(x)$ when $\lambda _{P}(y)_{2}\leq
\lambda _{P}(y)_{1}$, which contradicts $x\in \boldsymbol{argmin}_{\mathcal{%
PE}(P)}(d\cdot \lambda _{P})$. Thus $\lambda _{P}(y)_{2}>\lambda _{P}(y)_{1}$%
, in which case the highest value of $d\cdot \lambda _{P}(y)$ is $m-\lambda
_{P}(x)_{1}$, which is attained when $\lambda _{P}(y)_{2}=m-1$. As $x\in 
\boldsymbol{argmin}_{\mathcal{PE}(P)}(d\cdot \lambda _{P}),$ we have $d\cdot
\lambda _{P}(y)\geq d\cdot \lambda _{P}(x)$, thus $m-\lambda _{P}(x)_{1}\geq
m-2v_{1}+1$, equivalently $\lambda _{P}(x)_{1}\leq 2v_{1}-1$. However, $m-v_{1}>2v_{1}-1$ when $%
v_{1}\leq $ $k$, leading to a contradiction since $%
\lambda _{P}(x)_{1}\geq $ $m-v_{1}$ by assumption.

Now consider the case $I(y,$ $x;$ $\succ _{1})$ $\neq \varnothing $. For any 
$a \in I(y,$ $x;$ $\succ _{1}),$ write $\Delta (a )\in A$ for some
alternative that Pareto dominates $a $, i.e., $\Delta (a )\succ
_{i}a $ $\forall i\in \{1,$ $2\}$. In general, $\Delta (a )$ may
or may not be in $I(y,$ $x;$ $\succ _{1})$. However, w.l.o.g. we can pick $\Delta (a )$ so that $\Delta (a )$ $\notin $
$I(y,$ $x;$ $\succ _{1}),$\footnote{Observe that $I(y,$ $x;$ $\succ _{1})$ does not contain Pareto efficient alternatives. Hence, for any alternative in this set, there is some other alternative that Pareto dominates it. Since $I(y, x; \succ _{1})$ is finite and the Pareto dominance
relation is transitive, the claim holds.}, which implies either $\Delta (a )=y$ or $%
\Delta (a )\succ _{1}y$. If $\Delta (a )=y$ then $y\succ
_{2}a $ by Pareto dominance. If $\Delta (a
)\succ _{1}y$ then $y\succ _{2}\Delta (a )$ as $y\in PE(\prof)$. Thus for
any $a \in I(y,$ $x;$ $\succ _{1}),$ we have either  $y\succ _{2}a 
$ or $y\succ _{2}\Delta (a ).$ In case $\Delta (a)=$ $\Delta (a^{\prime
})$ for some distinct $a ,$ $a ^{\prime }\in I(y,$ $x;$ $\succ
_{1})$, we have $\Delta (a)\succ _{2}a^{\prime }$, implying $y\succ
_{2}a^{\prime }$. As a result, $\lambda _{P}(y)_{2}\leq $ $m-\#I(y,x;\succ
_{1})-1$. Note also $\lambda _{P}(y)_{1}=\lambda _{P}(x)_{1}-\#I(y,x;\succ
_{1})-1$. As $x\in PE(\prof)$, we have $x\succ _{2}y$, thus $\lambda
_{P}(y)_{2}>\lambda _{P}(x)_{2}$. Observe that $d\cdot \lambda
_{P}(y)<d\cdot \lambda _{P}(x)$ when $\lambda _{P}(y)_{2}\leq \lambda
_{P}(y)_{1}$, which contradicts $x\in \boldsymbol{argmin}_{\mathcal{PE}%
(P)}(d\cdot \lambda _{P})$. Thus $\lambda _{P}(y)_{2}>\lambda _{P}(y)_{1}$,
in which case the highest value of $d\cdot \lambda _{P}(y)$ is $m-\lambda
_{P}(x)_{1}$, which is attained at $\lambda _{P}(y)_{2}=m-\#I(y,x;\succ
_{1})-1$. As $x\in \boldsymbol{argmin}_{\mathcal{PE}(P)}(d\cdot \lambda
_{P}),$ we have $d\cdot \lambda _{P}(y)\geq d\cdot \lambda _{P}(x)$, thus $%
m-\lambda _{P}(x)_{1}\geq m-2v_{1}+1$, equivalently $\lambda _{P}(x)_{1}\leq
2v_{1}-1$. However, $%
m-v_{1}>2v_{1}-1$ when $v_{1}\leq $ $k$, leading to a contradiction, since $\lambda _{P}(x)_{1}\geq $ $m-v_{1}$ by assumption.
\end{proof}

\begin{remark}
   As MD implies PEL, the class of PV rules that satisfy MD is a subclass of those that satisfy PEL. This relationship can be more precisely observed by comparing Propositions \ref{propo:pel} and \ref{propo:pv}. 
   
\end{remark}

\section{Reconciling the two principles}

Given the incompatibility between MR and MD, one can attempt to reconcile the two principles by imposing minimal dispersion among the alternatives that are minimally Rawlsian. The Rawlsian minimal dispersion ($RMD$) principle requires the outcome to contain the Rawlsian alternatives whose loss vectors have minimal dispersion.


\commentRS{The definition and the proposition below must be adapted to the new definition of MD. Moreover, we should add Pareto efficiency.}


\commentMN{RMD will then read $f(\prof)\supseteq \arg min_{\PE\cap H(P)} (d \circ \lprof)$, $\forall \prof \in \allprofs$, right?} 
\commentMN{The Rawlsian minimal dispersion ($RMD$) principle requires the outcome to contain the alternatives whose loss vectors have minimal dispersion among the Pareto efficient ones in the first-half, right?.}
\commentRS{Yes}
\begin{definition}[Rawlsian minimal dispersion]
	$f(\prof) \cap \arg min_{H(\mathbf{P})} (d \circ \lprof) ≠ \emptyset$, $\forall \prof \in \allprofs$.
\end{definition}


\begin{proposition}
     $\FB$, $\VR$, $\SL$ and all rules in $\PVcl$ except $\PVe$ fail RMD.
\end{proposition}

\begin{proof}
Consider the following profile $\mathbf{P}$ with 11 alternatives.

	\begin{equation}
		\begin{array}{llllllll}
			x&a&b&c&d&y&w&\ldots \\		e&f&g&y&x&h&w&\ldots
		\end{array}.
	\end{equation}


$H(\prof)=\{x, y\}$ and for an SCR $f$ to satisfy RMD, we must have $y\in   f(\prof)$. As $\FB(\prof)$ = $\VR(\prof)=\{x\}$, both rules fail RMD. 

To see that $\SL$ fails RMD, consider the following profile $\prof$ with $\SL(\prof)=\set{a, c}$ while RMD requires $b \in \SL(\prof)$. 


	\begin{equation}
		\begin{array}{lllll}
			a&b&c&d&e \\		c&b&a&d&e
		\end{array}.
	\end{equation}


Regarding rules in $\PVcl$, we first show that for $m$ even, $\PVv[(\frac{m}{2}, \frac{m}{2} - 1)]$ and $\PVv[(\frac{m}{2} - 1, \frac{m}{2})]$ both fail RMD. To see this, consider the profile $\prof$

	\begin{equation}
		\begin{array}{llll}
			a&b&c&d \\		d&c&a&b
		\end{array}
	\end{equation}
where $\PVv[(\frac{m}{2}, \frac{m}{2} - 1)](\prof)=\{a\}$ while RMD requires $c \in \PVv[(\frac{m}{2}, \frac{m}{2} - 1)](\prof)$.  selects $a$ whereas $c$ should be included in the selection if the rule satisfied $RMD$. A similar argument shows that $\PVv[(\frac{m}{2} - 1, \frac{m}{2})]$ fails RMD.

Thus, by Theorem 2, among the rules in $\PVcl$, $\PVe$ is the only candidate to satisfy RMD. As $\PVe$ chooses all Pareto efficient alternatives in $H(\prof)$, it always picks the alternative that minimizes the loss dispersion.
\end{proof}

\section{Concluding remarks}

Axiomatic analysis of social choice rules with or without strategic concerns present two strands of the literature that complement each other. This complementarity appears less balanced in two-person collective choice problems where there is a clear focus on a strategic analysis that usually adopts subgame perfect equilibrium as the solution concept. \footnote{While we will not restate here the relevant papers already cited in the introduction, we wish to remark that the focus on subgame perfect equilibrium can be explained by the classical result of Hurwicz and Schmeidler (1978) and Maskin (1999) on the impossibility Nash implementing Pareto efficient two-person social choice rules.} 

The richness of the non-strategic axiomatic analysis of collective choice with three or more individuals is accompanied by a wealthy list of conceived social choice rules. On the other hand, for two individuals, it is hard to name a prominent social choice rule beyond fallback bargaining, the shortlisting procedure, the veto-rank mechanism and the class of Pareto-and-veto rules. Moreover, these social choice rules are conceived under different interpretations of the two-person collective choice model, thus being analyzed from rather different perspectives.

We bring a consideration based on a common interpretation and perspective. The axiomatic analysis we propose is free of strategic concerns and relies on two basic principles that we identify, namely the minimally Rawlsian principle and the equal loss principle. These two principles that emerge from the existing literature exhibit an incompatibility. More precisely, no minimally Rawlsian social choice rule can minimize the dispersion of the loss vector.

In front of this incompatibility, the literature seems to favor the minimally Rawlsian principle, as fallback bargaining, the shortlisting procedure and the veto-rank mechanism are minimally Rawlsian. Moreover, within the class of Pareto-and-veto rules, perhaps the most prominent ones, namely those which give both individuals the highest equal or almost equal veto power are minimally Rawlsian. By the established incompatibility, these rules cannot minimize loss dispersion but, except the Pareto-and-veto rule that gives the highest equal veto power, they don’t even minimize loss dispersion among the Paretian alternatives that are ranked within the first half of both individuals.

Among the rules we consider, those that minimize loss dispersion are the Pareto-and-veto rules that give each individual a veto power that doesn't exceed a third of the total number of alternatives. These rules will typically make coarse choices with several tied alternatives. Aiming at more refined outcomes that minimize loss dispersion, one can directly pick at every preference profile the Pareto efficient alternatives that minimize loss dispersion or the Pareto efficient alternatives within the first half of both individuals that minimize loss dispersion. Seemingly none of these two rules are considered in the literature and their properties are not known, which suggest that there is room to conceive and analyze new social choice rules.


\commentMN{I like it}

\appendix
\section{More results}
\begin{proposition}
	\label{th:maxDisp}
	Given any profile $\prof$, $\exists x \in H(\prof) \cap \PE \suchthat d(\lprof(x)) ≤ \khalf$.
	That bound is tight, in other words, $\exists \prof \suchthat \forall x \in H(\prof): d(\lprof(x)) ≥ \khalf$.
\end{proposition}
\begin{proof}
	About the first claim, consider the $\khalf + 1$ alternatives having losses for the individual 1 $\lprof(x)_1 ≤ \khalf$.
	At least one of those alternatives will have a loss for the individual 2 within $\intvl{0, \khalf}$ (as the complementary interval of possible losses is $\intvl{\khalf + 1, m - 1}$ which has cardinality $\floor{\frac{m - 1}{2}} < \khalf + 1$). Such alternatives $\set{x \in \allalts \suchthat \lprof(x)_1 ≤ \khalf \land \lprof(x)_2 ≤ \khalf}$ have dispersion $d(\lprof(x)) ≤ \khalf$.
	The best of those for individual $2$, that is, $\argmin_{x \in \allalts \suchthat \lprof(x)_1 ≤ \khalf \land \lprof(x)_2 ≤ \khalf} \lprof(x)_2$, is Pareto-efficient.
	
	About the second claim, write $\allalts = \set{a_1, …, a_m}$, define $S_1$ as the sequence $(a_1, …, a_{\floor{\frac{m + 1}{2}}})$ and $S_2$ as the sequence $(a_{\floor{\frac{m + 1}{2}} + 1}, …, a_m)$, and let $\prof$ contain the preference ordering $(S_1, S_2)$ for individual $1$ and $(S_2, S_1)$ for individual $2$. 
	Alternatives in $S_2$ have dispersion $\floor{\frac{m + 1}{2}}$ (the length of $S_1$) and 
	alternatives in $S_1$ have dispersion $m - (\floor{\frac{m + 1}{2}} + 1) + 1 = (m + 1) - \floor{\frac{m + 1}{2}} - 1 = \ceil{\frac{m + 1}{2}} - 1 = \ceil{\frac{m - 1}{2}}$ (the length of $S_2$).
\end{proof}

\section{Spread measures}
\label{sec:spreads}
We reuse the following definitions from \cite{cailloux2022compromising}, letting $n$ denote the number of individuals, $l \in \R^n$ denote a generalized loss tuple and $\bar{l} = \sum_{i = 1}^n l_i / n$ denote the arithmetic mean of the losses:
\begin{itemize}
	\item the mean absolute difference $\sigma_{mad}(l)= \frac{1}{n^2} \sum_{i = 1}^n\sum_{j = 1}^n \abs{l_i - l_j}$;
	\item the average absolute deviation $\sigma_{ad}(l)= \frac{\sum_{i = 1}^n \abs{l_i - \bar{l}}}{n}$;
	\item the standard deviation $\sigma_{sd}(l)= \sqrt{\frac{\sum_{i = 1}^n(l_i - \bar{l})^2}{n}}$;
	\item the Gini coefficient $\sigma_{G}(l)= \frac{\sum_{i = 1}^n\sum_{j = 1}^n \abs{l_i - l_j}}{2 n \sum_{i = 1}^n l_i}$.
\end{itemize} 


\commentMN{TO BE CHECKED: 
When $n = 2$ as in our case, it is plain that $\forall i \in \set{1, 2}: \abs{l_i - \bar{l}} = \frac{\abs{l_1 - l_2}}{2}$;  $\sum_{i = 1}^n\sum_{j = 1}^n \abs{l_i - l_j} = 2 \abs{l_1 - l_2}$; $\sigma_{mad} = \sigma_{ad} = \sigma_{sd} = \frac{d}{2}$ and $\sigma_G = \frac{\abs{l_1 - l_2}}{2 (l_1 + l_2)}$. 
Thus, $\sigma_{mad}$, $\sigma_{ad}$ and $\sigma_{sd}$ coincide with $d$, but $\sigma_G$ does not. 
For example, the Gini coefficient considers $(49, 51)$ as less unequal than $(0, 1)$ whereas $d$ orders these tuples reversely.}\commentMN{I agree}

\section{Further axiomatizations of the rules}
\subsection{A characterisation of FB based on weight comparison}
For any $w_m, w_i \in [0, 1]$ with $w_m + w_i = 1$, let $f^{w_m, w_i}$ denote the rule that selects the maximal elements according to the disutility of the loss vectors weighted by $w_m$ and $w_i$, as follows: $f^{w_m, w_i}(\prof) = \argmax_{x \in \allalts} -w_m(\min\lprof(x)) - w_i(\max \lprof(x) - \min \lprof(x))$. \commentMN{Why do we have $\max \lprof(x) - \min \lprof(x)$?? What I mean is that the previous expression calls for an explanation. I can add that but it maybe safer to discuss a bit before. Actually, the class of these rules is not immediate to me.}

\begin{remark}
	The currently, and temporarily, adopted notations for $w_m$ and $w_i$ use $m$ and $i$ to stand for \emph{minimal loss aversion} and \emph{inequality aversion}, and these letters in that context have nothing to do with the $m$ and $i$ previously defined for the number of candidates and a generic individual. I leave this is as-is for now and hope this introduces no confusion.\commentMN{What about $w_{ineq}$ rather than $w_i$ and $w_{min}$ for $w_m$?}
\end{remark}

\begin{definition}[Weight-comparing]
	$f$ is weight-comparing iff $\exists w_m, w_i \in [0, 1], w_m + w_i = 1 \suchthat f = f^{w_m, w_i}$.
\end{definition}

Let $\Delta^{x, y}_\mathit{ineq} = (\max \lprof(x) - \min \lprof(x)) - (\max \lprof(y) - \min \lprof(y))$ represent the difference between $x$ and $y$ in terms of inequality of losses. 
(The quantity $\Delta^{x, y}_\mathit{ineq}$ depends on the profile, which is omitted from its notation here.)
That number is (strictly) positive iff $\max \lprof(x) - \min \lprof(x)$ is (strictly) greater than $\max \lprof(y) - \min \lprof(y)$, that is, iff the inequality of $x$ is (strictly) greater than that of $y$, thus, intuitively, iff $f^{w_m, w_i}$ considers $x$ (strictly) worst than $y$ from the point of view of inequality.
Note that $\Delta^{x, y}_\mathit{ineq} = - \Delta^{y, x}_\mathit{ineq}$.

Let $\Delta^{x, y}_\mathit{min} = \min\lprof(x) - \min\lprof(y)$ represent the difference between $x$ and $y$ in terms of their smallest losses. 
That number is (strictly) positive iff $f^{w_m, w_i}$ considers $x$ (strictly) worst than $y$ from the point of view of smallest losses.
Note that $\Delta^{x, y}_\mathit{min} = - \Delta^{y, x}_\mathit{min}$.

\begin{proposition}
	\label{th:wcDelta}
	The rule $f^{w_m, w_i}$ selects the maximal elements of $\succeq$ defined as 
	$x \succeq y ⇔ \frac{w_i}{w_m} \Delta^{y, x}_\mathit{ineq} ≥ \Delta^{x, y}_\mathit{min}$.
\end{proposition}
\begin{proof}
	Define $M_x = \max \lprof(x)$, $m_x = \min \lprof(x)$, and similarly $M_y$ and $m_y$.
	
	By definition, the rule $f^{w_m, w_i}$ selects the maximal elements of $\succeq'$ defined as: 
	\begin{align}
		x \succeq' y & ⇔ -w_m m_x - w_i (M_x - m_x) ≥ -w_m m_y - w_i (M_y - m_y)\\
		& ⇔ w_i [(M_y - m_y) - (M_x - m_x)] ≥ w_m (m_x - m_y)\\
		& ⇔ \frac{w_i}{w_m} \Delta^{y, x}_\mathit{ineq} ≥ \Delta^{x, y}_\mathit{min}.
	\end{align}
	Thus, $\mathbin{\succeq'} = \mathbin{\succeq}$.
\end{proof}

\begin{remark}
	Those rules can also be characterized using only one parameter, as follows: $f^{w_m}(\prof) = f^{w_m, (1 - w_m)}(\prof)$; or as $f^{w_i}(\prof) = f^{(1 - w_i), w_i}(\prof)$.
	
	Such a rule is equally characterized by the tradeoff it is willing to tolerate between $\Delta_\mathit{min}$ and $\Delta_\mathit{ineq}$. E.g. for $w_m = 1/2$, the tradeoff is $\frac{w_m}{1 - w_m} = 1$: one $\Delta_\mathit{min}$ is indifferent to one $\Delta_\mathit{ineq}$.
\end{remark}

\commentMN{I added a proof that Borda is weight-comparing. Are other scoring rules representable via a weight-comparing rule? Should we use weight-comparing or other name?}
\commentOC{Thank you for the proof; it suggests that perhaps it will be good (also wrt your previous remark) to state this “change of basis” argument more generally, as I need it as well for the argument related to Pareto (\cref{th:paretoIneq}). I have attempted this below (\cref{rq:basis}).}
\commentOC{I think (without proof and far from being sure) that most other scoring rules are not w-c rules, as they do not treat losing ranks uniformly across the ranks. Tops is one of the exceptions, see below, perhaps the sole other exception; and this is specific to two individuals as only in this case does Tops equal Plurality.}
\commentOC{The LVC property (\cref{def:lvc}) plus anonymity, perhaps, capture the scoring rules feature (not sure at all); adding an “equal consideration for tradeoffs across ranks” condition (something like Shifting, \cref{def:shifting}) might yield w-c.}
\commentOC{I’m not good with names, I agree that weight comparing is not a good name.}

\begin{remark}[Alternative basis]
	\label{rq:basis}
	If $f$ is a weight-comparing rule with weights $(w_m, w_i)$, it is equally described as a rule that weights minimal and maximal losses with weights $(w'_m, w'_M)$ where $w'_m = w_m - w_i$ and $w'_M = w_i$.
	Indeed, such a rule selects the alternatives that minimize $-w_m (l_m) - w_i (l_M - l_m)$, which equals $-(w_m - w_i) (l_m) - w_i (l_M)$.
	
	However, the relationship and the bounds on those alternate weights for an equivalent class of rules are less nice. Consider the egalitarian rule, \cref{rq:egalitarianism}, for an illustration.
	We have that $w'_M = 1 - \frac{w'_m + 1}{2}$, and the bounds are $w'_m \in [-1, 1]$ and $w'_M \in [0, 1]$.
\end{remark}

\begin{remark}
	Borda satisfies Paretian and weight-comparing and fails SPEL. It uses  $w_m = 2/3$ and $w_i = 1/3$. Indeed, by definition,
	\[\textsc{Borda}(\prof) = \argmin_{\allalts} \sum_N \lprof = \set{x \in \allalts \suchthat \forall y \in \allalts: \sum_N \lprof(x) ≤ \sum_N \lprof(y)}.\]
	
	This makes sense since the loss of an alternative for a individual is equal to $m-$(ranking of the alternative in the preference profile), wheres the usual definition of the Borda score assigned by a individual is that this score equals the ranking of the alternative in the individual's profile. Hence, they are equivalent.
	
	It is Paretian by definition since an alternative $x$ ranked higher by both individuals than an alternative $y$ obtains a higher Borda score. It fails SPEL due to the same example as the one in Proposition 4. To see why Borda rule is a weight-comparing rule, take some $f$ with weights $w_i = 1/3$ and $w_m = 2/3$ so that:
\begin{align}f(\prof) &= \argmax_{x \in \allalts} -(2/3)(\min\lprof(x)) - (1/3)(\max \lprof(x) - \min \lprof(x))\\
&\equiv \argmax_{x \in \allalts} -(1/3)(\min\lprof(x)+\max \lprof(x))\\
&\equiv \argmax_{x \in \allalts} -(\sum\lprof(x))\\
&\equiv \argmin_{x \in \allalts} (\sum\lprof(x))=\textsc{Borda}(\prof).\\
\end{align}
\end{remark}

\begin{remark}[Egalitarianism]
	\label{rq:egalitarianism}
	The egalitarian rule that minimizes inequality of losses (disregarding efficiency) satisfies SPEL and weight-comparing and fails Paretianism. It uses $w_i = 1$.
\end{remark}

\commentMN{When $w_m=1$, how is $f$? It selects the max of the minimum loss so quite uninteresting, right?}
\commentOC{Such a rule is extremely averse to min losses, and does not care about inequality. In other words, it minimizes the minimum loss. Thus, it is the rule “Top”, that selects the top alternatives.}

\begin{theorem}[Under w-c, SPEL equivalent to care for equality]
	\label{th:spelEquiv}
	Consider $f$ on variable number of candidates, weight-comparing. Then, $f$ is SPEL iff $\exists w_i ≥ w_m \in [0, 1], w_m + w_i = 1 \suchthat f = f^{w_m, w_i}$.
\end{theorem}
\begin{proof}
	We will need the following fact.
	Given $m \in \N$, $m ≥ 3$ odd, with $\allalts = \set{x, y, a_1, …, a_{m - 2}}$: 
	\begin{equation}
		\label{eq:bigIneq}
		\exists \prof \suchthat x, y \in \PE \land \lprof[\prof](x) = \left(\frac{m - 1}{2}, \frac{m - 1}{2}\right) \land \lprof[\prof](y) = \left(0, \frac{m + 1}{2}\right). 
	\end{equation}
	Letting $S_1$ denote the sequence of $\frac{m - 1}{2}$ alternatives $(y, a_1, …, a_{\frac{m - 1}{2} - 1})$ and $S_2$, the sequence of $\frac{m - 1}{2}$ alternatives $(a_{\frac{m - 1}{2}}, …, a_{m - 2})$, such a profile can be defined as the rankings $(S_1, x, S_2)$ and $(S_2, x, S_1)$. Here is a visual description of it.
	\begin{equation}
		\begin{array}{lllll lllll ll}
			y&a_1&…&a_{\frac{m - 1}{2} - 1}&x&a_{\frac{m - 1}{2}}&a_{\frac{m - 1}{2} + 1}&…&a_{m - 2}\\
			a_{\frac{m - 1}{2}}&a_{\frac{m - 1}{2} + 1}&…&a_{m - 2}&x&y&a_1&…&a_{\frac{m - 1}{2} - 1}\\
		\end{array}.
	\end{equation}
	
	Back to our main claim, given $m \in \N$ and $k \in \intvl{0, m - 1}$, let $\mathcal{P}^{x, k}$ denote the set of profiles $\prof$ where an alternative $x \in \PE$ has losses $(k, k)$. By definition, a rule $f$ is SPEL iff
	$\forall m \in \N, k \in \intvl{0, m - 1}, \prof \in \mathcal{P}^{x, k}$: 
	\begin{equation}
		\label{eq:condNotInF}
		\forall y ≠ x \in \PE: y \notin f(\prof).
	\end{equation}
	When $f$ is weight-comparing, \eqref{eq:condNotInF} is equivalent to:
	\begin{equation}
		\forall y  \in \PE\setminus \{x\}: [-w_m \min \lprof(y) - w_i (\max \lprof(y) - \min \lprof(y)) < -w_m k],
	\end{equation}
	itself equivalent to
	\begin{equation}
		\label{eq:wNotInF}
		\forall y \in \PE\setminus \{x\}: [w_m (k - \min \lprof(y)) < w_i (\max \lprof(y) - \min \lprof(y))].
	\end{equation}
	To conclude, we show that $w_m ≤ w_i$ implies \eqref{eq:wNotInF} and that \eqref{eq:wNotInF} implies $w_m ≤ w_i$.
	
	Observe first that $\forall y,x\in \PE$ with $y ≠ x$,  $k < \max \lprof(y)$ (otherwise, $y$ is Pareto-dominated by $x$); equivalently, $k - \min \lprof(y) < \max \lprof(y) - \min \lprof(y)$. Thus, $w_m ≤ w_i$ implies \eqref{eq:wNotInF}.
	
	Second, considering any $\prof$ satisfying \eqref{eq:bigIneq} and $k = \frac{m - 1}{2}$. The inequality \eqref{eq:wNotInF} implies that any alternative $y$ that satisfies \eqref{eq:bigIneq},  $w_m \left(\frac{m - 1}{2} - 0\right) < w_i \left(\frac{m + 1}{2} - 0\right)$, equivalently, $w_m < w_i \frac{m + 1}{m - 1}$, and because this must be true for all $m$, this implies $w_m ≤ w_i$.
\end{proof}

\begin{theorem}[Paretianism limits inequality consideration]
	\label{th:paretoIneq}
	Consider $f$ on variable number of candidates, weight-comparing. Then, $f$ is Paretian iff $\exists w_m ≥ w_i \in [0, 1], w_m + w_i = 1 \suchthat f = f^{w_m, w_i}$.
\end{theorem}
\begin{proof}
	For the forward direction, consider the losses $(100, 100)$ versus $(0, 99)$, thus in basis (min, ineq), $(100, 0)$ versus $(0, 99)$. The first one will be picked if $-100 w_m > -99 w_i$, which will happen if $w_i > 100/99 w_m$. (The argument can and should be completed by defining a profile where only $a$ and $b$ with losses $(0, 99)$ and $(99, 0)$ are pareto-dominant: on a smaller scale, this is $a, c, d, e, b, x, f, g, h$ and $b, f, g, h, a, x, c, d, e$.)
	
	The backwards direction follows from \cref{rq:basis}: from $w_m ≥ w_i$ follows that $w'_m ≥ 0$, and as $w'_M ≥ 0$ anyway, $f$ will respect dominance of losses (formally, $\lprof(y) > \lprof(x) ⇒ y \notin f(\prof)$).
\end{proof}

\begin{theorem}[FB as weighter]
	\label{th:fbW}
	FB = $f^{\frac{1}{2}, \frac{1}{2}}$.
\end{theorem}
\begin{proof}
	FB selects the alternatives that minimize $\max \lprof$ while $f^{\frac{1}{2}, \frac{1}{2}}$ selects those that maximize $-\frac{1}{2} \min \lprof - \frac{1}{2} (\max \lprof - \min \lprof)$, thus, that maximize $- \max \lprof$.
\end{proof}

\begin{theorem}[FB caract]
	The rule $f$ defined on variable number of candidates is Paretian, SPEL and weight-comparing iff it is FB.
\end{theorem}
\begin{proof}
	We show that the rule $f$ defined on variable number of candidates is Paretian, SPEL and weight-comparing iff it is $f^{\frac{1}{2}, \frac{1}{2}}$, as \cref{th:fbW} then brings the result.
	
	Considering a rule $f$ defined on variable number of candidates and weight-comparing, \cref{th:spelEquiv,th:paretoIneq} apply.
	From \cref{th:spelEquiv}, and assuming $f$ is also SPEL, $\exists w_i ≥ w_m \in [0, 1], w_m + w_i = 1 \suchthat f = f^{w_m, w_i}$.
	From \cref{th:paretoIneq}, and assuming $f$ is also Paretian, $\exists w_m ≥ w_i \in [0, 1], w_m + w_i = 1 \suchthat f = f^{w_m, w_i}$.
	It follows that $f = f^{\frac{1}{2}, \frac{1}{2}}$.
	
	\Cref{th:spelEquiv,th:paretoIneq} also show that the converse holds.
\end{proof}

\begin{remark}
	The rule $f$ with weights $w_{max} = \frac{m - 1}{m}$ and $w_i = \frac{1}{m}$ is, I believe, the subcorrespondence of FB that lexicographically first minimizes the worst loss, then minimizes the best loss. It is weight-comparing when defined on a fixed $m$, but not when defined on variable number of candidates. I think it would be interesting to define a weaker version of weight-comparing to allow for this.\commentMN{I agree} Perhaps some difficulty will stem from the fact that, from LVC, Pareto and a principle of indifference, FB will perhaps prefer the losses (4, 4) to (2, 5) because (4, 4) indifferent to (1, 4), itself Pareto dominating (2, 5).
\end{remark}

\subsection{A characterisation via marginal tradeoffs}
We will need the concept of a tradeoff function, intuitively, a function of the form $t(a, b)$, where $a$ is the ineq level and $b$ is the delta min, which indicates the delta ineq that compensates that delta min.
\begin{definition}[tradeoff function]
	A tradeoff function is a function of the form $t: \N^2 → \N \cup (\N + \frac{1}{2})$, with $t(a, 0) = 0$.
\end{definition}
Tradeoff functions are used to determine tradeoff preferences. To define this concept we first need to generalize the notion of Pareto dominance.
\begin{definition}[min-ineq Pareto relation]
	The min-ineq Pareto relation is a binary relation $R \subseteq \N^2 × \N^2$, defined on the loss tuples using minimal loss and inequality as a basis (from now on called the min-ineq losses), as follows. The min-ineq loss $(m, i)$ wealky Pareto-dominates the min-ineq loss $(m', i')$ iff it has a not worst min loss (that is, $m ≤ m'$) and a not worst ineq (that is, $i ≤ i'$). In other words, the min-ineq Pareto relation coincides with the $≤$ relation over ordered pairs of natural numbers.
\end{definition}
\begin{definition}[tradeoff preference]
	A tradeoff function $t$ determines a corresponding tradeoff preference, $\mathbin{\succeq^t} \subseteq \N^2 × \N^2$, defined on the min-ineq losses, as follows.
	Intuitively, given $d_m \in \N$ (a delta min), the function $t$ indicates that $(m, i)$ is indifferent to $(m, i) + (d_m, t(i, d_m))$.
	Considering in supplement the min-ineq Pareto relation, we obtain our formal definition: $(m, i) \succeq^t (m, i) + (d_m, d_i)$ iff $d_i ≥ t(i, d_m)$.
\end{definition}

\begin{conjecture}[Requirement on $t$]
	The preference $\succeq^t$ corresponding to a given tradeoff function $t$ is acyclic and complete iff $t$ satisfies: $t(i, d_m) < 0 ⇔ d_m > 0$.
\end{conjecture}
\begin{proof}[Intuition for the proof]
	The requirement is a generalization of the requirement that $(1, 3) \succeq (2, 3)$, very intuitively speaking.
	Consider $d_m > 0$.
	$(1, 3) \succeq (1, 3) + (d_m, 0)$ iff $t(i, d_m) < 0 ⇔ d_m > 0$.
\end{proof}

\begin{definition}[tradeoff rule]
	A tradeoff rule $f^t$ is a rule parameterized by a tradeoff function $t$, defined as $f^t(\prof)$ selecting maximal elements according to the relation $\succeq^t$ determined by the tradeoff function. Formally, given $l$ a loss tuple (in the usual basis), define $i(l) = \max l - \min l$ as the inequality of that loss tuple, and given $x, x' \in \allalts$, say that $f^t(\prof)$ wealy prefers $x$ to $x'$ iff $(\min \lprof(x), i(\lprof(x))) \succeq^t (\min \lprof(x), i(\lprof(x)))$, and define $f^t(\prof)$ as the maximal elements of that preference relation.
\end{definition}

These definitions thus lead naturally to the following conjecture.
\begin{conjecture}
	$f^t$ satisfies the natural [min, ineq] Pareto relation (lower is better) iff $t$ satisfies $t(i, d_m) < 0 ⇔ d_m > 0$.
\end{conjecture}
%For this reason, for now on we consider $t$ satisfying $t(i, d_m) < 0 ⇔ d_m > 0$.

\begin{definition}[Shifting]
	\label{def:shifting}
	From $\prof$, define $\prof'$ by adding a new alternative $z$ on top of $i$’s ranking and a new alternative $w$ at bottom of $i$’s ranking, and adding $w$ on top of $\ibar$’s ranking and $z$ at bottom of $\ibar$’s ranking, thereby shifting every previous alternatives one rank down. Shifting requires that $\forall \prof: f(\prof) = f(\prof')$ (unless $z$ or $w$ is taken -- details should be worked up).
\end{definition}

\begin{definition}[Shift up]
	\label{def:shiftUp}
	Consider $\prof$ and let $\set{w, z}$ designate the bottom alternatives (first individual $w$, second one $z$, with possibly $w = z$).
	Define $\prof'$ as $\prof$ where all alternatives are shifted one rank downwards, except that for the first individual, $w$ goes first, and for the last individual, $z$ goes first.
	Then, $[f(\prof') \cap \set{w, z} = \emptyset] ⇒ f(\prof') = f(\prof)$.
\end{definition}

\begin{definition}[Shift down]
	\label{def:shiftDown}
	Consider $\prof$ where the top alternatives $\set{w, z}$ (first individual $w$, second one $z$, with possibly $w = z$) do not win ($f(\prof) \cap \set{w, z} = \emptyset$). 
	Define $\prof'$ as $\prof$ where all alternatives are shifted one rank upwards, except that for the first individual, $w$ goes last, and for the last individual, $z$ goes last.
	Then, $f(\prof') = f(\prof)$.
\end{definition}

\begin{conjecture}
	Shift up implies Shift down.
\end{conjecture}

\begin{definition}[Finite shifting]
	\label{def:finiteShifting}
	$f$ satisfies finite shifting iff it satisfies Shift up (hence, iff it satisfies Shift down and Shift up).
\end{definition}

\begin{definition}[Half shift]
	Consider $\prof$ and let $w$ designate the bottom alternative of individual $i$.
	Define $\prof'$ as $\prof$ except that all alternatives of individual $i$ are shifted one rank downwards, except that $w$ goes first.
	Then, $[f(\prof') \cap \set{w} = \emptyset] ⇒ f(\prof') = f(\prof)$.
\end{definition}

\begin{definition}[Loss vector comparing (LVC)]
	\label{def:lvc}
        Let $\succeq$ be a weak order on $\lvs$. The rule $f^\succeq$ selects all alternatives whose loss vectors are minimal elements of $\restr{{\succeq}}{\lprof(\alts)}$. A rule satisfies LVC iff $\exists {\succeq} \suchthat f = f^\succeq$. (This property is called preorder based in Cailloux and Endriss 2014.)
\end{definition}

\begin{conjecture}
	LVC and shifting implies $\exists t \suchthat f = f^t$.
\end{conjecture}
\begin{proof}[Intuition for the proof]
	Assume that $x$ with losses $(2, 5)$ is preferred (according to $\succeq$) to $y$ with losses $(1, 6)$.
	Thus, in min-ineq basis: $(2, 3) \succeq (1, 5)$.
	We have to prove that then, $(1, 3) \succeq (0, 5)$ [m-i].
	Thus, that $(1, 4) \succeq (0, 5)$ [losses].
	This is clear from shifting.
\end{proof}

\begin{conjecture}
	LVC and finite shifting and Half shift equivalent to w-c.
\end{conjecture}

\begin{conjecture}
	w-c implies finite shifting.
\end{conjecture}
\begin{proof}
	Seems to follow from \cref{th:wcDelta}.
\end{proof}

\begin{conjecture}
	LVC and finite shifting and min-ineq Pareto implies $\exists t \suchthat f = f^t$.
\end{conjecture}
\begin{proof}[Rough sketch of a proof]
	Assume that $x$ with losses $(2, 2)$ is preferred (according to $\succeq$) to $y$ with losses $(1, 3)$.
	Thus, in min-ineq basis: $(2, 0) \succeq (1, 2)$.
	We have to prove that then, $(1, 0) \succeq (0, 2)$ [m-i].
	Thus, that $(1, 1) \succeq (0, 2)$ [losses].
	This is clear from shifting.
	
	More generally.
	Define $x = (2, 0)$ [m-i], and let $y = (1, b)$ be indifferent to it. Then define $t_2$ accordingly. Proceed similarly for other vectors $(2, a)$ indifferent to $(1, b)$, which defines $t_2$ completely. Proceed similarly for $t_m$, for other values of min $m$. Then prove that $t_m$ are equal for all $m$.
\end{proof}

\begin{conjecture}
	Consider $f^t$, SPEL. Then, $t(0, d_m) = inf$. BUT NOT $t(i, d_m) = d_i ≤ d_m$. (Intuitively, inequality matters more than min.)
\end{conjecture}
\begin{conjecture}
	Consider $f^t$, Paretian. Then, $t(i, d_m) = d_i ≥ d_m$. (Intuitively, inequality matters less than min.)
\end{conjecture}

\begin{conjecture}
	LVC and shifting do not imply w-c.
\end{conjecture}
\begin{proof}[Idea of a proof]
%	Assume that $x$ with losses $(2, 5)$ is preferred (according to $\succeq$) to $y$ with losses $(1, 6)$.
%	Then, any pair with $\Delta^{y, x}_\mathit{ineq} = 2$ and $\Delta^{x, y}_\mathit{min} = 1$ will have $x \succeq y$.
%	FURTHERMORE, any pair with $\Delta^{y, x}_\mathit{ineq} = 4$ and $\Delta^{x, y}_\mathit{min} = 2$ will have $x \succeq y$?
%	In fact, any trade of 2 $\Delta^{y, x}_\mathit{ineq}$ for 1 $\Delta^{x, y}_\mathit{min}$ will be accepted.
%	
%	Consider $x$ with losses $(1, 5)$ and $y$ with losses $(0, 6)$. Ineq $x$ is $4$ and ineq $y$ is 6.
%	to prove: $x$ is preferred to $y$.
	
	We could have $(1, 4) > (0, 5) > (0, 6) > (1, 5)$.
\end{proof}

\subsection{Axiomatization of rules other than FB}
$\PVe$ is equivalent to MR

VR is "Borda constrained to MR". Olivier claims that $\VR$ is the biggest rule satisfying both MR and the condition: $x \in f(\prof) ⇒ \sum(\lprof(x)) ≤ \min_{y \in H(\prof)} \sum(\lprof(y))$.

Short listing should be something around “maximax within the first half.” Again a claim of Olivier: Another condition requires that if $\exists y \in H(\prof) \suchthat \min(\lprof(y)) < \min(\lprof(x))$, then $x$ loses. Equivalently, $x \in f(\prof) ⇒ \min(\lprof(x)) ≤ \min_{y \in H(\prof)} \min(\lprof(y))$. The rule $\SL$ is the biggest rule satisfying both MR and that condition. (Note that these two conditions are independent.)

\commentOC{Here is the description of $\SL$, which can be suitably adapted to other rules.} The $\SL$ rule is the rule that selects any alternative that is MR and that is not dominated by an MR alternative in terms of minimal welfare loses. More precisely, define $\prec$ as the following partial order on the set of loss vectors: $(a, b) \prec (c, d) ⇔ [(a, b) ≤^\text{MR} (c, d) \land \min\set{a, b} < \min\set{c, d}] \lor [(a, b) <^\text{MR} (c, d)]$, where $(a, b) ≤^\text{MR} (c, d)$ iff $(a, b)$ is first half. Then, $\SL$ selects $x$ iff it is an undominated element among the loss vectors that exist in $\prof$; formally, $f(\prof) = \set{x \in \allalts \suchthat \nexists y \in \allalts \suchthat y \prec x}$. 
Defining the uncomparability relation $\sim$ as $l_1 \sim l_2$ iff $l_1 \sim^\text{MR} l_2 \land \min l_1 = \min l_2$, we obtain the weak order ${\preceq} = {\prec} \cup {\sim}$, and $f(\prof) = \argmin^{\preceq}_{x \in \allalts} \lprof(x)$.
\commentOC{We could perhaps also describe $\prec$ as a union of relations, one of which is common to all rules.}

\commentRS{Finally can we characterize WMD as a rule?}

\bibliography{bibliototal}
\end{document}

